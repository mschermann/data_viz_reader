\documentclass[]{book}
\usepackage{lmodern}
\usepackage{amssymb,amsmath}
\usepackage{ifxetex,ifluatex}
\usepackage{fixltx2e} % provides \textsubscript
\ifnum 0\ifxetex 1\fi\ifluatex 1\fi=0 % if pdftex
  \usepackage[T1]{fontenc}
  \usepackage[utf8]{inputenc}
\else % if luatex or xelatex
  \ifxetex
    \usepackage{mathspec}
  \else
    \usepackage{fontspec}
  \fi
  \defaultfontfeatures{Ligatures=TeX,Scale=MatchLowercase}
\fi
% use upquote if available, for straight quotes in verbatim environments
\IfFileExists{upquote.sty}{\usepackage{upquote}}{}
% use microtype if available
\IfFileExists{microtype.sty}{%
\usepackage{microtype}
\UseMicrotypeSet[protrusion]{basicmath} % disable protrusion for tt fonts
}{}
\usepackage[margin=1in]{geometry}
\usepackage{hyperref}
\hypersetup{unicode=true,
            pdftitle={A Reader on Data Visualization},
            pdfauthor={MSIS 2629 Spring 2018},
            pdfborder={0 0 0},
            breaklinks=true}
\urlstyle{same}  % don't use monospace font for urls
\usepackage{natbib}
\bibliographystyle{apalike}
\usepackage{color}
\usepackage{fancyvrb}
\newcommand{\VerbBar}{|}
\newcommand{\VERB}{\Verb[commandchars=\\\{\}]}
\DefineVerbatimEnvironment{Highlighting}{Verbatim}{commandchars=\\\{\}}
% Add ',fontsize=\small' for more characters per line
\usepackage{framed}
\definecolor{shadecolor}{RGB}{248,248,248}
\newenvironment{Shaded}{\begin{snugshade}}{\end{snugshade}}
\newcommand{\KeywordTok}[1]{\textcolor[rgb]{0.13,0.29,0.53}{\textbf{#1}}}
\newcommand{\DataTypeTok}[1]{\textcolor[rgb]{0.13,0.29,0.53}{#1}}
\newcommand{\DecValTok}[1]{\textcolor[rgb]{0.00,0.00,0.81}{#1}}
\newcommand{\BaseNTok}[1]{\textcolor[rgb]{0.00,0.00,0.81}{#1}}
\newcommand{\FloatTok}[1]{\textcolor[rgb]{0.00,0.00,0.81}{#1}}
\newcommand{\ConstantTok}[1]{\textcolor[rgb]{0.00,0.00,0.00}{#1}}
\newcommand{\CharTok}[1]{\textcolor[rgb]{0.31,0.60,0.02}{#1}}
\newcommand{\SpecialCharTok}[1]{\textcolor[rgb]{0.00,0.00,0.00}{#1}}
\newcommand{\StringTok}[1]{\textcolor[rgb]{0.31,0.60,0.02}{#1}}
\newcommand{\VerbatimStringTok}[1]{\textcolor[rgb]{0.31,0.60,0.02}{#1}}
\newcommand{\SpecialStringTok}[1]{\textcolor[rgb]{0.31,0.60,0.02}{#1}}
\newcommand{\ImportTok}[1]{#1}
\newcommand{\CommentTok}[1]{\textcolor[rgb]{0.56,0.35,0.01}{\textit{#1}}}
\newcommand{\DocumentationTok}[1]{\textcolor[rgb]{0.56,0.35,0.01}{\textbf{\textit{#1}}}}
\newcommand{\AnnotationTok}[1]{\textcolor[rgb]{0.56,0.35,0.01}{\textbf{\textit{#1}}}}
\newcommand{\CommentVarTok}[1]{\textcolor[rgb]{0.56,0.35,0.01}{\textbf{\textit{#1}}}}
\newcommand{\OtherTok}[1]{\textcolor[rgb]{0.56,0.35,0.01}{#1}}
\newcommand{\FunctionTok}[1]{\textcolor[rgb]{0.00,0.00,0.00}{#1}}
\newcommand{\VariableTok}[1]{\textcolor[rgb]{0.00,0.00,0.00}{#1}}
\newcommand{\ControlFlowTok}[1]{\textcolor[rgb]{0.13,0.29,0.53}{\textbf{#1}}}
\newcommand{\OperatorTok}[1]{\textcolor[rgb]{0.81,0.36,0.00}{\textbf{#1}}}
\newcommand{\BuiltInTok}[1]{#1}
\newcommand{\ExtensionTok}[1]{#1}
\newcommand{\PreprocessorTok}[1]{\textcolor[rgb]{0.56,0.35,0.01}{\textit{#1}}}
\newcommand{\AttributeTok}[1]{\textcolor[rgb]{0.77,0.63,0.00}{#1}}
\newcommand{\RegionMarkerTok}[1]{#1}
\newcommand{\InformationTok}[1]{\textcolor[rgb]{0.56,0.35,0.01}{\textbf{\textit{#1}}}}
\newcommand{\WarningTok}[1]{\textcolor[rgb]{0.56,0.35,0.01}{\textbf{\textit{#1}}}}
\newcommand{\AlertTok}[1]{\textcolor[rgb]{0.94,0.16,0.16}{#1}}
\newcommand{\ErrorTok}[1]{\textcolor[rgb]{0.64,0.00,0.00}{\textbf{#1}}}
\newcommand{\NormalTok}[1]{#1}
\usepackage{longtable,booktabs}
\usepackage{graphicx,grffile}
\makeatletter
\def\maxwidth{\ifdim\Gin@nat@width>\linewidth\linewidth\else\Gin@nat@width\fi}
\def\maxheight{\ifdim\Gin@nat@height>\textheight\textheight\else\Gin@nat@height\fi}
\makeatother
% Scale images if necessary, so that they will not overflow the page
% margins by default, and it is still possible to overwrite the defaults
% using explicit options in \includegraphics[width, height, ...]{}
\setkeys{Gin}{width=\maxwidth,height=\maxheight,keepaspectratio}
\IfFileExists{parskip.sty}{%
\usepackage{parskip}
}{% else
\setlength{\parindent}{0pt}
\setlength{\parskip}{6pt plus 2pt minus 1pt}
}
\setlength{\emergencystretch}{3em}  % prevent overfull lines
\providecommand{\tightlist}{%
  \setlength{\itemsep}{0pt}\setlength{\parskip}{0pt}}
\setcounter{secnumdepth}{5}
% Redefines (sub)paragraphs to behave more like sections
\ifx\paragraph\undefined\else
\let\oldparagraph\paragraph
\renewcommand{\paragraph}[1]{\oldparagraph{#1}\mbox{}}
\fi
\ifx\subparagraph\undefined\else
\let\oldsubparagraph\subparagraph
\renewcommand{\subparagraph}[1]{\oldsubparagraph{#1}\mbox{}}
\fi

%%% Use protect on footnotes to avoid problems with footnotes in titles
\let\rmarkdownfootnote\footnote%
\def\footnote{\protect\rmarkdownfootnote}

%%% Change title format to be more compact
\usepackage{titling}

% Create subtitle command for use in maketitle
\newcommand{\subtitle}[1]{
  \posttitle{
    \begin{center}\large#1\end{center}
    }
}

\setlength{\droptitle}{-2em}
  \title{A Reader on Data Visualization}
  \pretitle{\vspace{\droptitle}\centering\huge}
  \posttitle{\par}
  \author{MSIS 2629 Spring 2018}
  \preauthor{\centering\large\emph}
  \postauthor{\par}
  \predate{\centering\large\emph}
  \postdate{\par}
  \date{2018-05-14}

\usepackage{booktabs}
\usepackage{amsthm}
\makeatletter
\def\thm@space@setup{%
  \thm@preskip=8pt plus 2pt minus 4pt
  \thm@postskip=\thm@preskip
}
\makeatother

\usepackage{amsthm}
\newtheorem{theorem}{Theorem}[chapter]
\newtheorem{lemma}{Lemma}[chapter]
\theoremstyle{definition}
\newtheorem{definition}{Definition}[chapter]
\newtheorem{corollary}{Corollary}[chapter]
\newtheorem{proposition}{Proposition}[chapter]
\theoremstyle{definition}
\newtheorem{example}{Example}[chapter]
\theoremstyle{definition}
\newtheorem{exercise}{Exercise}[chapter]
\theoremstyle{remark}
\newtheorem*{remark}{Remark}
\newtheorem*{solution}{Solution}
\begin{document}
\maketitle

{
\setcounter{tocdepth}{1}
\tableofcontents
}
\chapter{Preface}\label{preface}

This is a collaborative writing project as part of the course MSIS 2629
``Data Visualization'' at \href{http://www.scu.edu}{Santa Clara
University}. The purpose of the class reader is to collaboratively
engage with and reflect on data visualizations, to establish a solid
theoretical background, and to collect useful practices and showcases.
More information on the background of this project is available in the
\href{https://mschermann.github.io/msis2629spring2018}{syllabus}.

The following text serves explains how we organize ourselves.

\section{References}\label{references}

\textbf{EVERY} references must be included in the \texttt{book.bib}
file. This file uses the bibtex notation (Learn how to use bibtex
\href{http://www.bibtex.org/Using/}{here}.). Most literature search
engines allow you to export the reference information in Bibtex. For
websites we use the following minimal notation (you may add further
information - usually the more the better is a good strategy):

\begin{verbatim}
@misc{great_viz,
  author = {{A great visualizer}},
  year = {1982},
  title = {A ficticious web page title},
  howpublished = {\url{http://great_viz_org/}},
  note = {Accessed: 2018-04-26}
}
\end{verbatim}

Particularly important is the \texttt{note} field. Websites change
frequently, so links will break. If we do this correctly,
\texttt{{[}@great\_viz{]}} will produce \citep{great_viz}.

\section{Images}\label{images}

Images should not be loaded from external website because the links may
change. Instead download a version of the image and create a reference
that contains the link to the image. For example the following image is
a deceptive visualization (the bars do start at zero).

\begin{figure}
\centering
\includegraphics{images/halper_welfare.jpg}
\caption{An Example of a deceptive visualization}
\end{figure}

Source: \citep{halper_2012} referenced in \citep{andale_2014}

The citation for the image looks like this.

\begin{verbatim}
@misc{halper_2012,
  author={Halper, Daniel},
  year={2012},
  title = {Over 100 Million Now Receiving Federal Welfare},
  url={https://www.weeklystandard.com/daniel-halper/over-100-million-now-receiving-federal-welfare},
  note = {Accessed: 2018-04-26}
}
\end{verbatim}

You have probably found this image through a different website that
explains the visualization. For example the following website explains
some problematic aspects of this visualization:

\begin{verbatim}
@misc{andale_2014,
  author={Andalde, Stephanie},
  year={2014},
  title = {Misleading Graphs: Real Life Examples},
  url={http://www.statisticshowto.com/misleading-graphs/},
  note = {Accessed: 2018-04-26}
}
\end{verbatim}

\chapter{Introduction}\label{introduction}

\section{1.1 What is Data
Visualization?}\label{what-is-data-visualization}

Data visualization refers to representing data in a visual context to
help people understand the significance of that data. A way so that
information, numbers, and measurements makes sense is a form of art --
the art of data visualization. Graphs do that for us.

According to Friedman (2008) the ``main goal of data visualization is to
communicate information clearly and effectively through graphical means.
It doesn't mean that data visualization needs to look boring to be
functional or extremely sophisticated to look beautiful. To convey ideas
effectively, both aesthetic form and functionality need to go hand in
hand, providing insights into a rather sparse and complex data set by
communicating its key-aspects in a more intuitive way.''

\citep{viz}

\section{Information Visualization}\label{information-visualization}

\subsection{Information
Visualization}\label{information-visualization-1}

Information visualization, the art of representing data in a way that it
is easy to understand and to manipulate, can help us make sense of
information and thus make it useful in our lives. From business decision
making to simple route navigation -- there's a huge (and growing) need
for data to be presented so that it delivers value.

This article is a brief introduction to Information Visualization. It
explains briefly how information visualization helps to make sense of
data, how it helps to find relationships between data and confirm ideas.
Some examples and common uses of information visualization is discussed
below:

\citep{info_viz}

\section{1.2 Why is Data Visualization
Important?}\label{why-is-data-visualization-important}

Today more than ever, Data Visualization represents a simple,
user-friendly approach to understand data and make business decisions
quickly. Here are some articles on why it is important today.

\textbf{Chris Pittenturf's article on the importance of data
visualization to businesses}

The article , written by Chris Pittenturf, VP-Data \& Analytics, Palace
Sports \& Entertainment, talks about what data visualization is and its
importance to the businesses today.

The article begins with a definition of data visualization in simple
terms and goes on to explain how a good data visualization should be
visually engaging to the reader. Chris goes on to explain the basic
criterias that a data visualization should satisfy to be an effective
visualization. These criterias and their brief meanings are as follows:
1. Informative: The visualization should be able to convey the
information of the data to the reader 2. Efficient: The visualization
should not be ambiguous. 3. Appealing: The visualization should be
captivating and visually pleasing. 4. (Optional) Interactive and
Predictive: The visualizations can contain variables and filters for the
users to interact with the visualizations in order to predict results of
different scenarios.

Pittenturf goes on to give various day-to-day examples where
visualization gives a better understanding of the data. One extremely
simple example used by Pittenturf is that of an energy bill. Pittenturf
states that as a consumer, when we receive an energy bill, we normally
look at the graph in the bill first before proceeding to read the text
in the bill. Pittenturf states that consumers are more likely to analyze
and understand the visualizations before reading further along. The
article ends with Pittenturf emphasizing the importance of data
visualizations in our businesses as well as in our daily lives.It gives
a simple, short and crisp understanding of what data visualization is
and how it is relevant to everyone. Data visualization is an aid to get
a better understanding of the complex insights that any business data
provides. Most of the data used by the businesses is highly unstructured
and these businesses can get a better understanding of their businesses
by visualizing their data.

The article, written by Chris Pittenturf, VP-Data \& Analytics, Palace
Sports \& Entertainment, talks about what data visualization is and its
importance to the businesses today.

The article begins with a definition of data visualization in simple
terms and goes on to explain how a good data visualization should be
visually engaging to the reader. Pittenturf goes on to explain the basic
criterias that a data visualization should satisfy to be an effective
visualization. The criterias and their brief meanings are as follows:

\begin{enumerate}
\def\labelenumi{\arabic{enumi}.}
\tightlist
\item
  Informative: The visualization should be able to convey the
  information of the data to the reader
\item
  Efficient: The visualization should not be ambiguous.
\item
  Appealing: The visualization should be captivating and visually
  pleasing.
\item
  (Optional) Interactive and Predictive: The visualizations can contain
  variables and filters for the users to interact with the
  visualizations in order to predict results of different scenarios.
\end{enumerate}

Pittenturf goes on to give various day-to-day examples where
visualization gives a better understanding of the data.

One extremely simple example used by Pittenturf is that of an energy
bill. Pittenturf states that as a consumer, when we receive an energy
bill, we normally look at the graph in the bill first before proceeding
to read the text in the bill. Consumers are more likely to analyze and
understand the visualizations before reading further along.

The article ends with Pittenturf emphasizing the importance of data
visualizations in our businesses as well as in our daily lives.

This article gives a simple, short and crisp understanding of what data
visualization is and how it is relevant to everyone. It shows that data
visualization is an aid to get a better understanding of the complex
insights that any business data provides. Most of the data used by the
businesses is highly unstructured and these businesses can get a better
understanding of their businesses by visualizing the data.

\citep{viz_importance}

\textbf{About David McCandless's TED talk on data visualization:}

Visuals help us understand concepts that would otherwise be difficult to
contextualize---for example, expenditures or valuations of extremely
large amounts of money are represented in the billion dollar-o-gram by
color-coded, relatively-sized boxes. Furthermore, it allows synthesis of
a breadth of information to be delivered in a small, easily-digestible,
aesthetically pleasing way. Visuals serve as a sort of map for a vast
landscape of information---they direct your eyes to the important places
and details. And the eye, as McCandless notes, is uniquely suited among
our senses to process large amounts of information and detect patterns.

The billion dollar-o-gram is extremely readable and rather pretty, but
it seems a bit dubious to compare the predicted Iraq War cost to the
``mushroomed'' actual cost of Iraq and Afghanistan wars, since its
purpose seems only to conflate two wars for dramatic effect.

Beyond its ability to make information from several different sources
and in large amounts more quickly and easily understood, data
visualization can also reveal smaller interesting patterns---allowing us
to play the ``data detective'' as McCandless calls it. In other words,
as we have already discussed, data visualization can not only be
extremely effective in a declarative manner, but can also be used as an
exploratory tool.

McCandless also postulates that we all have a latent ``design literacy''
that is being developed every day as we are constantly bombarded with
visuals, and that our minds and our eyes are taking in this information
and processing it so that we all have an intuitive sense of design, and
have actually begun to demand a visual aspect to our information. This
is an interesting perspective, since everyone does seem to have a sense
of visual aspects---space, color, etc., but of course the time-honored
adage tells us that beauty is in the eye of the beholder. So while it
might be whimsical to claim that we are all designers, there is still,
of course, great value in learning formal principles of design.

\citep{viz_ted}

\section{1.3 Key Figures in the History of Data
Visualization}\label{key-figures-in-the-history-of-data-visualization}

The history of data visualization is full of incredible stories marked
by major events, led by a few key players. This article introduces some
of the amazing men and women who paved the way by combining art,
science, and statistics.And one of them is Charles Joseph Minard, whose
most famous work is the map of Napoleon's Russian campaign of 1812
displayed in our class. Below we have some figures names with their
famous works, and you can find other stories in the article.

\textbf{William Playfair (1759--1823)}

William Playfair is considered the father of statistical graphics,
having invented the line and bar chart we use so often today. He is also
credited with having created the area and pie chart. Playfair was a
Scottish engineer and political economist who published The Commercial
and Political Atlas in 1786.

This book featured a variety of graphs including the image below. In
this famous example, he compares exports from England with imports into
England from Denmark and Norway from 1700 to 1780.

\begin{figure}
\centering
\includegraphics{images/Playfair.png}
\caption{}
\end{figure}

\textbf{John Snow (1813--1858)}

In 1854, a cholera epidemic spread quickly through Soho in London. The
Broad Street area had seen over 600 dead, and the remaining residents
and business owners had largely fled the terrible disease.

Physician John Snow plotted the locations of cholera deaths on a map.
The surviving maps of his work show a method of tallying the death
counts, drawn as lines parallel to the street, at the appropriate
addresses. Snow's research revealed a pattern. He saw a clear
concentration around the water pump on Broad Street, helping to find the
cause of the infection.

\begin{figure}
\centering
\includegraphics{images/Snow.png}
\caption{}
\end{figure}

\textbf{Charles Joseph Minard (1781--1870)}

Charles Joseph Minard was a French civil engineer famous for his
representation of numerical data on maps. His most famous work is the
map of Napoleon's Russian campaign of 1812 displaying the dramatic loss
of his army over the advance on Moscow and the following retreat.

You can see how many soldiers are still marching and how many died.
Drawn in 1869, it is described by many as the best statistical graphic
ever drawn. It represents the earliest beginnings of data journalism.

\begin{figure}
\centering
\includegraphics{images/Minard.png}
\caption{}
\end{figure}

\citep{history_viz}

\section{1.4 Useful Links on Data Visualization Trends, Tutorials and
Research
Papers}\label{useful-links-on-data-visualization-trends-tutorials-and-research-papers}

\citep{charts_viz} - You can find different types of plots used in data
visualization at \href{https://datavizcatalogue.com/search.html}{Data
Catalogue}.

\citep{eagereyes_viz} - Robert Kosara's website which contains recent
developments happening in visualization and are likely to have an
impact.

\citep{research_viz} - About Robert Kosara and his reserach papers.

\citep{twitter_Kosara} - Robert Kosara's twitter handle.

\citep{flowingdata} - Website which offers courses, tutorials and
happenings in viz.

\citep{infogram} - An infogram helps a user making different types of
plots and learning the art of visualization. Engaging infographics,
reports, charts, dashboards and maps can be easily created in minutes
with it.

\begin{Shaded}
\begin{Highlighting}[]
\KeywordTok{par}\NormalTok{(}\DataTypeTok{mar =} \KeywordTok{c}\NormalTok{(}\DecValTok{4}\NormalTok{, }\DecValTok{4}\NormalTok{, .}\DecValTok{1}\NormalTok{, .}\DecValTok{1}\NormalTok{))}
\KeywordTok{plot}\NormalTok{(pressure, }\DataTypeTok{type =} \StringTok{'b'}\NormalTok{, }\DataTypeTok{pch =} \DecValTok{19}\NormalTok{)}
\end{Highlighting}
\end{Shaded}

\begin{figure}

{\centering \includegraphics[width=0.8\linewidth]{Data_Viz_Reader_files/figure-latex/nice-fig-1} 

}

\caption{Here is a nice figure!}\label{fig:nice-fig}
\end{figure}

\chapter{Fundamentals}\label{fundamentals}

Reference\_clean\_w4 + Theoretical background of data visualization

\section{Theoretical Background of Bata
Visualization}\label{theoretical-background-of-bata-visualization}

\subsection{A Brief History of Data
Visualization}\label{a-brief-history-of-data-visualization}

``The only new thing in the world is the history you don't know.'' ---
Harry S Truman

This paper provides an overview of the intellectual history of data
visualization from medieval to modern times, it describes and
illustrates some significant advances along the way.

\textbf{1. Data Visualization: Modern Product?}

It is common to think of statistical graphics and data visualization as
relatively modern developments in statistics. In fact, the graphic
representation of quantitative information has deep roots.These roots
reach into the histories of the earliest map-making and visual
depiction, and later into thematic cartography, statistics and
statistical graphics, medicine, and other fields.

Developments in technologies (printing, reproduction) mathematical
theory and practice, and empirical observation and recording, enabled
the wider use of graphics and new advances in form and content.

\textbf{2. Milestones Tour}

\textbf{2.1 Pre-17th Century: Early maps and diagrams}

Data visualization has comes a long way. Prior to the 17th century, data
visualization already exists. Though display in other format such as
maps, the content are much similar to today's visualization, which
mostly presented geologic, economic, and medical data. Here is useful
link: {[}data\_viz\_history{]}

The earliest seeds of visualization arose in geometric diagrams, in
tables of the positions of stars and other celestial bodies, and in the
making of maps to aid in navigation and exploration.

\textbf{2.2 1600-1699: Measurement and theory}

Among the most important problems of the 17th century were those
concerned with physical measurement of time, distance, and space for
astronomy, surveying, map making, navigation and territorial expansion.
This century also saw great new growth in theory and the dawn of
practical application.

\textbf{2.3 1700-1799: New graphic forms}

With some rudiments of statistical theory, data of interest and
importance, and the idea of graphic representation at least somewhat
established, the 18th century witnessed the expansion of these aspects
to new domains and new graphic forms.

\textbf{2.4 1800-1850: Beginnings of modern graphics}

With the fertilization provided by the previous innovations of design
and technique, the first half of the 19th century witnessed explosive
growth in statistical graphics and thematic mapping, at a rate which
would not be equalled until modern times.

\textbf{2.5 1850--1900: The Golden Age of statistical graphics}

By the mid-1800s, all the conditions for the rapid growth of
visualization had been established a ``perfect storm'' for data
graphics. Official state statistical offices were established throughout
Europe, in recognition of the growing importance of numerical
information for social planning,industrialization, commerce, and
transportation.

2.5.1 Escaping flatland\\
2.5.2 Graphical innovations\\
2.5.3 Galton's contributions\\
2.5.4 Statistical Atlases

\textbf{2.6 1900-1950: The modern dark ages}

If the late 1800s were the ``golden age'' of statistical graphics and
thematic cartography, the early 1900s can be called the ``modern dark
ages'' of visualization. There were few graphical innovations, and, by
the mid-1930s, the enthusiasm for visualization which characterized the
late 1800s had been supplanted by the rise of quantification and formal,
often statistical, models in the social sciences.

\textbf{2.7 1950--1975: Re-birth of data visualization}

Still under the influence of the formal and numerical zeitgeist from the
mid-1930s on, data visualization began to rise from dormancy in the mid
1960s.

\textbf{2.8 1975--present: High-D, interactive and dynamic data
visualization}

During the last quarter of the 20th century data visualization has
blossomed into a mature, vibrant and multi disciplinary research area,
as may be seen in this Handbook, and software tools for a wide range of
visualization methods and data types are available for every desktop
computer.

\section{\texorpdfstring{\textbf{Practitioners Guide to Best Practices
in Data
Visualization}}{Practitioners Guide to Best Practices in Data Visualization}}\label{practitioners-guide-to-best-practices-in-data-visualization}

\citep{best-practice}

These are the best practices of data visualization. Anticipate in
advance what kind of questions the viewers will ask and then focus your
visualization with respect to those questions.

Brain processes stimuli from our environment to process what is
important in 2 ways �? unconscious (System 1 represents uncontrolled
functions such as facial expressions, reactions) and conscious (System 2
�? represents controlled function such as solving math problems). Data
Visualization leverage attributes of System -1 which has can have quick
and correct impact in a most efficient manner. The three best practices
of data visualization are as follows: -

\begin{enumerate}
\def\labelenumi{\arabic{enumi}.}
\tightlist
\item
  Design and layout matter The design and layout should facilitate ease
  of understanding to convey your message to the viewer.
\item
  Avoid Clutter Keep it simple. To implement this always keep into
  account the data-ink ratio �? the ratio of ink required to convey the
  intended meaning to the total amount of ink used in the table or chart
  should be as close to 1 as possible. That means, avoid ink which do
  not add any information.
\item
  Use color purposely and effectively Use of color may be prettier and
  attractive but can be distractive too. Thus, color should be used only
  if it assists in conveying your message. The above three principles
  are illustrated with the help of scenarios and examples which helps to
  comprehend the topic in more meaningful and practical way in the
  article. It also gives various advantages of using the above
  principles.And the above best practices could be applied to all the 3
  types of analytics: descriptive, predictive and prescriptive.
\end{enumerate}

reference-fundamentals-Tableau-Groups

\begin{itemize}
\tightlist
\item
  Theoretical background of data visualization\\
\item
  Contemporary research results
\end{itemize}

Now that we have all learnt the basics of working with Tableau, we may
think of following groups or joining communities to explore tableau
further.

The benefits:

\begin{itemize}
\tightlist
\item
  It will help us enhance our learning
\item
  Get answers for most of your doubts In tableau
\item
  Post new questions and crowd source answers
\item
  Attend events, seminars and join conferences conducted locally/
  globally
\item
  Give back to the community once you become an expert in that field
\end{itemize}

Some useful communities for Tableau users:

\subsection{Tableau Community}\label{tableau-community}

Ref: \citep{Tableau_Community}

\subsection{Fundamental Components of
Design}\label{fundamental-components-of-design}

\begin{enumerate}
\def\labelenumi{\arabic{enumi}.}
\tightlist
\item
  The dashboard should read left to right.
\item
  Group related information together.
\item
  Find relationships between seemingly unrelated areas and display
  visuals together to show the relationship.
\end{enumerate}

Blogs : Here is a list of the top 10 blogs that Tableau itself suggests
following:

\textbf{1.Balance:}\\
A design is said to be balanced if key visual elements such as color,
shape, texture, and negative spaceare uniformly distributed.

\textbf{2.Emphasis:}\\
Draw viewers attention towards important data by using key visual
elements.

\textbf{3.Movement:}\\
Ideally movement should mimic the way people usually read, starting at
the top of the page, movingacross it, and then down. Movement can also
be created by using complimentary colors to pull the user's
attentionacross the page.

\textbf{4.Pattern:}\\
Patterns are ideal for displaying similar sets of information, or for
sets of data that equal in value. Disrupting the pattern can also be
effective in drawing viewers attention; it naturally draws curiosity.

\textbf{5.Repetition:}\\
Relationships between sets of data can be communicated by repeating
chart types, shapes, or colors.

\textbf{6.Proportion:}\\
If a person is portrayed next to a house, the house is going look
bigger. In data visualization, proportion can indicate the importance of
data sets, along with the actual relationship between numbers.

\textbf{7.Rhythm:}\\
A design has proper rhythm when the design elements create movement that
is pleasing to the eye. If thedesign is not able to do so, rearranging
visual elements may help.

\textbf{8.Variety:}\\
Variety in color, shape, and chart-type draws and keeps users engaged
with data. Including more variety can increase information retention by
the viewer. But when there is too much variety, important details can be
overlooked.

\textbf{9.Unity:}\\
Unity across design will happen naturally if all other design principles
are implemented.

Artists use balance, emphasis, movement, pattern, repetition,
proportion, rhythm, variety, and unity as the design foundation of any
work. If you want to take your data visualization from an everyday
dashboard to a compelling data story, incorporate the 9 principles of
design from graphic designer Melissa Anderson's
article:\citep{design_principles}

Balance doesn't mean that each side of the visualization needs perfect
symmetry, but it is important to have the elements of the
dashboard/visualization distributed evenly. And it is important to
remember the non-data elements, such as a logo, title, caption, etc.
that can affect the balance of the display.

Another closely related component to balance is variety, which could
seem counter to balance, but when done correctly, variety can help
increase the recall of information. However if overdone, too much
variety can feel cluttered and blur together the images and data in the
mind of the viewer.

Arguably the most critical of the components is proportion. Proportion
can be subtle but it can go a long way to enhancing a viewer's
experience and understanding of the data. The danger of proportion
though is that it can be easy to deceive people subconsciously.
Naturally images will have a greater impact on how our brains perceive
the dashboard or visualization. For example, someone can change the
scale of a graph or images to inflate their results and even if they
write the numbers next to it, the shortcut many people will take is to
interpret the data based on the image. This is why it is important we
take care to accurately reflect proportion in our data visualization and
remain critical of how others use proportion in their visualization.

Emphasis was the component that I most related to when reading through
the nine principles of design in this article. From prior experience
with art through photography, I understand it is the key to be concious
of what I am drawing the viewers attention to in my art. When thinking
about the art design of data visualization it is also very important to
remain keen on the main point of your story and how the entire
visualization is either drawing the viewer to that point of emphasis or
how they are being distracted or drawn elsewhere.

\citep{Top_10_Blogs}

\begin{enumerate}
\def\labelenumi{\arabic{enumi}.}
\tightlist
\item
  Storytelling with Data
\item
  Information is Beautiful
\item
  Flowing Data
\item
  Visualising Data
\item
  Junk Charts
\item
  The Pudding
\item
  The Atlas
\item
  Graphic Detail
\item
  US Census and FEMA
\item
  Tableau Blog
\end{enumerate}

Tableau Social Media Groups: Some of the biggest and the most active
groups

Tableau Enthusiasts: Linkedin Group (19K members)

Tableau Software Fans \& Friends: LinkedIn Group (45kK members)

\citep{LinkedIn_Groups}

\url{https://www.educba.com/data-mining-vs-data-visualization/}

This article gives me a clear understanding of data mining and data
visualization.

In Data Mining, there are different processes involve carrying out the
data mining process such as data extraction, data management, data
transformations, data pre-processing, etc. In Data Visualization, the
primary goal is to convey the information efficiently and clearly
without any deviations or complexities in the form of statistical
graphs, information graphs, and plots. Also, the author listed the top 7
comparisons between data mining and data visualization, and 12 key
differences between data mining and data visualization. After reading
the article, you will have a very clear understanding of what are data
mining and data visualization and the characters for those two
techniques.

\section{1. Theoretical background of data
visualization}\label{theoretical-background-of-data-visualization}

1.1 History Data visualization has comes a long way. Prior to the 17th
century, data visualization already exists. Though display in other
format such as maps, the content are much similar to today's
visualization, which mostly presented geologic, economic, and medical
data.

1.2 Current research: Deceptive visualizations Data visualization is a
powerful communication tool to support arguments with numbers in a way
that is accessible and engaging. More people than ever before are making
their own charts and infographics, which is presenting a unique problem.
Despite the availability of some great charting resources, we are
witnessing an influx of poorly-designed misleading or downright
deceptive data visualizations.

\section{2. A Brief History of Data
Visualization}\label{a-brief-history-of-data-visualization-1}

The only new thing in the world is the history you don't know. �? Harry
S Truman

2.1. Data Visualization: modern product?

It is common to think of statistical graphics and data visualization as
relatively modern developments in statistics. In fact, the graphic
representation of quantitative information has deep roots.These roots
reach into the histories of the earliest map-making and visual
depiction, and later into thematic cartography, statistics and
statistical graphics, medicine, and other fields.Developments in
technologies (printing, reproduction) mathematical theory and practice,
and empirical observation and recording, enabled the wider use of
graphics and new advances in form and content.

\subsection{Practitioners' Guide to Best Practices in Data
Visualization}\label{practitioners-guide-to-best-practices-in-data-visualization-1}

These are the best practices of data visualization. Anticipate in
advance what kind of questions the viewers will ask and then focus your
visualization with respect to those questions.

The brain processes stimuli from our environment to process what is
important in 2 ways -- unconscious (System 1 represents uncontrolled
functions such as facial expressions, reactions) and conscious (System 2
-- represents controlled function such as solving math problems). Data
Visualization leverage attributes of System 1 which can have a quick and
correct impact in a most efficient manner. The three best practices of
data visualization are as follows:

\textbf{1. Design and layout matter}\\
The design and layout should facilitate ease of understanding to convey
your message to the viewer.

\textbf{2. Avoid Clutter}\\
Keep it simple. To implement this always keep into account the data-ink
ratio -- the ratio of ink required to convey the intended meaning to the
total amount of ink used in the table or chart should be as close to 1
as possible. That means, avoid ink which do not add any information.

\textbf{3. Use color purposely and effectively}\\
Use of color may be prettier and attractive but can be distractive too.
Thus, color should be used only if it assists in conveying your message.

The above three principles are illustrated with the help of scenarios
and examples which helps to comprehend the topic in more meaningful and
practical way in the article. It also gives various advantages of using
the above principles. And the above best practices could be applied to
all the 3 types of analytics: descriptive, predictive and prescriptive.

\textbf{Reference} \citep{practitioners_guide}

\subsection{Visualization and Graphics Principles to Refocus and Guide
You}\label{visualization-and-graphics-principles-to-refocus-and-guide-you}

Jonathon Corum is a graphics designer for The New York Times and he
provided a very informative talk to a strictly scientific audience on
how to create and design visualizations that explain material originally
created for a certain audience, i.e.~the scienctific community, but now
is to be related to a different audience, (in his case, the readership
of the Times or maybe the public at large). The talk is filled with
examples and break downs of how he has moved from his base content to
the final product, all of which are illuminative examples by themselves.
There is also great power in the broader themes that he is trying to
convey.

First, of course is knowing the audience that you are producing the work
for, but even in this step, do not lose sight of the ultimate goal of
conveying understanding, of explaining a concept. You are searching for
a visual idea in your content that can be communicated to your audience.
Some of the main highlights to help you make this connection with your
audience involve:

\textbf{Focusing the attention:}\\
What can be removed? Realize that consistency can help eliminate
unecessary distractions. There may be a trade off between losing
information but conveying the ultimate meaning more clearly. Label
important things rather than relying on a legend, which requires the
viewer to hold on to too much information at once.

\textbf{Involving your audience:}\\
Give them opportunities to connect their own general knowledge on the
topic. Use real world comparisons or examples to help build and relate
context. Encourage comparisons and make this easy for the viewer to
process and see.

\textbf{Explaining why:}\\
Providing context, adding time sequence details, showing movement,
change and mechanism will all guide your audience in connecting the dots
and understanding the significance of what you are trying to
communicate. 2.2. Milestones Tour

2.2.1 Pre-17th Century: Early maps and diagrams

The earliest seeds of visualization arose in geometric diagrams, in
tables of the positions of stars and other celestial bodies, and in the
making of maps to aid in navigation and exploration.

2.2.2 1600-1699: Measurement and theory

Among the most important problems of the 17th century were those
concerned with physical measurement�? of time,distance,and space�? for
astronomy, surveying, map making, navigation and territorial expansion.
This century also saw great new growth in theory and the dawnof
practical application.

2.2.3 1700-1799: New graphic forms

With some rudiments of statistical theory, data of interest and
importance, and the idea of graphic representation at least somewhat
established, the 18th century witnessed the expansion of these aspects
to new domains and new graphic forms.

2.2.4 1800-1850: Beginnings of modern graphics

With the fertilization provided by the previous innovations of design
and technique, the first half of the 19th century witnessed explosive
growth in statistical graphics and thematic mapping, at a rate which
would not be equalled until modern times.

2.2.5 1850�?1900: The Golden Age of statistical graphics

By the mid-1800s, all the conditions for the rapid growth of
visualization had been established�? a ``perfect storm�? for data
graphics. Official state statistical offices were established throughout
Europe, in recognition of the growing importance of numerical
information for social planning,industrialization, commerce, and
transportation.

2.2.5.1 Escaping flatland 2.2.5.2 Graphical innovations 2.2.5.3 Galton's
contributions 2.2.5.4 Statistical Atlases

2.2.6 1900-1950: The modern dark ages

If the late 1800s were the ``golden age�? of statistical graphics and
thematic cartography, the early 1900s can be called the ``modern dark
ages�? of visualization. There were few graphical innovations, and, by
the mid-1930s, the enthusiasm for visualization which characterized the
late 1800s had been supplanted by the rise of quantification and formal,
often statistical, models in the social sciences.

2.2.7 1950�?1975: Re-birth of data visualization

Still under the influence of the formal and numerical zeitgeist from the
mid-1930s on, data visualization began to rise from dormancy in the mid
1960s.

2.2.8 1975--present: High-D, interactive and dynamic data visualization

During the last quarter of the 20th century data visualization has
blossomed into a mature, vibrant and multi disciplinary research area,
as may be seen in this Handbook, and software tools for a wide range of
visualization methods and data types are available for every desktop
computer.

\section{3. Fundamental Components of
Design}\label{fundamental-components-of-design-1}

Artists use balance, emphasis, movement, pattern, repetition,
proportion, rhythm, variety, and unity as the design foundation of any
work. To take the data visualization from an everyday dashboard to a
compelling data story, we must incorporate the 9 principles of design
from graphic designer Melissa Anderson's article.

Balance doesn't mean that each side of the visualization needs perfect
symmetry, but it is important to have the elements of the
dashboard/visualiaztion distributed evenly. And it important to remember
the non-data elements, such as a logo, title,caption, etc., that can
affect the balance of the display.

Another closely related component to balance is variety which could seem
counter to balance, but when done correctly,variety can help increase
the recall of information. However if overdone, too much variety can
feel cluttered and blur together the images and data in the mind of the
viewer.Arguably the most critical of the components is proportion.
Proportion can be subtle but it can go a long way to enhancing a
viewer's experience and understanding of the data. The danger of
proportion though is that it can be easy to deceive people
subconsciously. Naturally images will have a greater impact on how our
brains perceive the dashboard or visualization. For example, someone can
change the scale of a graph or images to inflate their results and even
if they write the numbers next to it, the shortcut many people will take
is to interpret the data based on the image. This is why it is important
we take care to accurately reflect proportion in our data visualization
and remain critical of how others use proportion in their visualization.

Emphasis was the component that I most related to when reading through
the nine principles of design in this article. From prior experience
with art through photography I understand it is key to be concious of
what I am drawing the viewer attention to in my art. When thinking about
the art design of data visualization it is also very important to remain
keen on the main point of your story and how the entire visualization is
either drawing the viewer to that point of emphasis or how they are
being distracted or drawn elsewhere.

\section{4. Guide to Best Practices in Data
Visualization}\label{guide-to-best-practices-in-data-visualization}

These are the best practices of data visualization. Anticipate in
advance what kind of questions the viewers will ask and then focus your
visualization with respect to those questions.

The brain processes stimuli from our environment to process what is
important in 2 ways �? unconscious (System 1 represents uncontrolled
functions such as facial expressions, reactions) and conscious (System 2
�? represents controlled function such as solving math problems). Data
Visualization leverage attributes of System -1 which can have a quick
and correct impact in a most efficient manner. The three best practices
of data visualization are as follows:

4.1. Design and layout matter

The design and layout should facilitate ease of understanding to convey
your message to the viewer.

4.2. Avoid Clutter

Keep it simple. To implement this always keep into account the data-ink
ratio -- the ratio of ink required to convey the intended meaning to the
total amount of ink used in the table or chart should be as close to 1
as possible. That means, avoid ink which do not add any information.

4.3. Use color purposely and effectively

Use of color may be prettier and attractive but can be distractive too.
Thus, color should be used only if it assist in conveying your message.
The above three principles are illustrated with the help of scenarios
and examples which helps to comprehend the topic in more meaningful and
practical way in the article. It also gives various advantages of using
the above principles.And the above best practices could be applied to
all the 3 types of analytics: descriptive, predictive and prescriptive.
The design and layout should facilitate ease of understanding to convey
the message to the viewer.

Keep it simple. To implement this always keep into account the data-ink
ratio �? the ratio of ink required to conveythe intended meaning to the
total amount of ink used in the table or chart should be as close to 1
as possible. That means, avoid ink which do not add any information. **
2.Avoid Clutter \textbf{ Keep it simple. To implement this always keep
into account the data-ink ratio -- the ratio of ink required to convey
the intended meaning to the total amount of ink used in the table or
chart should be as close to 1 as possible. That means, avoid ink which
do not add any information. } 3.Use color purposely and effectively **
Use of color may be prettier and attractive but can be distractive too.
Thus, color should be used only if it assists in conveying your message.
The above three principles are illustrated with the help of scenarios
and examples which helps to comprehend the topic in more meaningful and
practical way in the article. It also gives various advantages of using
the above principles.And the above best practices could be applied to
all the 3 types of analytics: descriptive, predictive and prescriptive.

\textbf{Reference} Jeffrey D. Camm, Michael J. Fry, Jeffrey Shaffer
(2017) A Practitioner's Guide to Best Practices in Data
Visualization.Interfaces 47(6):473-488.
\url{https://doi.org/10.1287/inte.2017.0916}

\section{5. Data Visualization Tools}\label{data-visualization-tools}

Due to the rise of big data analytics, there has been an increased need
for data visualization tools to help understand the data. Besides
Tableau, there are several other software tools one can use for data
visualization like Sisense, Plotly, FusionCharts, Highcharts,
Datawrapper, and Qlikview. This article is from forbes and has a brief,
clear introduction about these 7 powerful software options for data
visualization. This could be helpful for future reference because for
different purposes I may need to use different tools. Each option has
its advantages and disadvantages and this article helps highlight them.

\textbf{Tableau} is the most popular of the group and has many users. It
is simple to use, making it easy to learn and can handle large datasets.
Tableau can handle big data thanks to integration with database handling
applications such as MySQL, Hadoop, and Amazon AWS.

\textbf{Qlikview} is the main competitor to Tableau and is also quite
popular. Qlikview is customizable and has a wide range of features which
can be a double-edged sword. These features take more time to learn and
get acquianted with. However, once one gets past the learning curve,
they have a powerful tool at their disposal.

The distinctive aspect of \textbf{FusionCharts} is that graphics do not
have to be created from scratch. Users can start with a template and
insert their own data from their project.

\subsection{Survey of Popular Data Visualization
Tools}\label{survey-of-popular-data-visualization-tools}

Due to the rise of big data analytics, there has been an increased need
for data visualization tools to help understand the data. Besides
Tableau, there are several other software tools one can use for data
visualization like Sisense, Plotly, FusionCharts, Highcharts,
Datawrapper, and Qlikview. This article from forbes has a brief, clear
introduction about these 7 powerful software options for data
visualization. This could be helpful for future reference because for
different purposes I may need to use different tools. Each option has
its advantages and disadvantages and this article helps highlight them.

\textbf{1. Tableau:}\\
The most popular of the group and has many users. It is simple to use,
making it easy to learn and can handle large datasets. Tableau can
handle big data thanks to integration with database handling
applications such as MySQL, Hadoop, and Amazon AWS.

\textbf{2. Qlikview:}\\
The main competitor to Tableau and is also quite popular. Qlikview is
customizable and has a wide range of features which can be a
double-edged sword. These features take more time to learn and get
acquianted with. However, once one gets past the learning curve, they
have a powerful tool at their disposal.

\textbf{3. FusionCharts:}\\
The distinctive aspect of it is that graphics do not have to be created
from scratch. Users can start with a template and insert their own data
from their project.

\textbf{4. Highcharts:}\\
It proudly claims to be used by 72\% of the 100 biggest companies in the
world. It is a simple tool that does not require specialized training
and quickly generates the desired output. Unlike some tools, Highcharts
focuses on cross-browser support, allowing for greater access and use.

\textbf{5. Datawrapper:}\\
It is making a name for itself in the media industry. It has a simple
user interface making it easy to generate charts and embed into reports.

\textbf{6. Plotly:}\\
It can create more sophisticated visuals thanks to integration with
programming languages such as Python and R. The danger is creating
something more complicated than necessary. The whole point of data
visualization is to quickly and clearly convey information.

\textbf{7. Sisense:}\\
It can bring together multiple sources of data for easier access. It can
even work with large datasets. Sisense makes it easy to share finished
products across departments, ensuring everyone can get the information
they need.

\textbf{Highcharts} proudly claims to be used by 72\% of the 100 biggest
companies in the world. It is a simple tool that does not require
specialized training and quickly generates the desired output. Unlike
some tools, Highcharts focuses on cross- browser support, allowing for
greater access and use.

\textbf{Datawrapper} is making a name for itself in the media industry.
It has a simple user interface making it easy to generate charts and
embed into reports.

\textbf{Plotly} can create more sophisticated visuals thanks to
integration with programming languages such as Python and R. The danger
is creating something more complicated than necessary. The whole point
of data visualization is to quickly and clearly convey information.

\textbf{Sisense} can bring together multiple sources of data for easier
access. It can even work with large datasets. Sisense makes it easy to
share finished products across departments, ensuring everyone can get
the information they need.

Lisa Rost's article ``What I learned recreating one chart using 24
tools'' describes lessons learned from recreating one chart using many
different data visualization tools. The author used apps Excel, Plotly,
Easycharts, Google Sheets, Lyra, Highcharts, Tableau, Polestar,
Quadrigram, Illustrator, RAW, and NodeBox, as well as charting libraries
ggvis, Bokeh, Highcharts, ggplot2, Processing, NVD3, Seaborn, Vega, D3,
matplotlib, Vega-Lite, and R. She links her github page on the project
which details the dataset she used, containing the health expectancy in
years as well as GDP per capita and population for about 200 countries
in the year 2015, as well has her process and results of visualizing the
data using each tool. However, in the article, she focuses on the main
takeaways from the exercise, which was especially interesting in the
context of our class discussion on different types of tools and their
respective strengths. She also provides her own graphics to help
illustrate her lessons learned.

Rost's first takeaway: \textbf{``There Are No Perfect Tools, Just Good
Tools for People with Certain Goals''}

Since data visualization is necessary in many spheres, from science to
journalism, data visualization projects will often have quite disparate
objectives, and the people working on them will have different
requirements. And as the author aptly points out, it is impossible for
one tool to satisfy the needs of every data visualizer; so there will
necessarily be tools better suited to specific situations.

For example, does the user need a tool for exploratory visualization of
the data, or does the user seek to create graphs and charts to show the
public or a specific audience something?

\begin{figure}
\centering
\includegraphics{images/analysis_spectrum.png}
\caption{}
\end{figure}

The author also notes that the flexibility of a tool is a sticking point
as well---if you need to change your data while developing a data
visualization, certain apps like Illustrator will not be ideal because
changing the data even slightly requires you to build the graph again
from scratch. Another thing to think about is the type of chart you are
trying to create---is a basic, canned bar or line graph all you need (in
which case something like Excel will do the trick), or does your project
necessitate a more innovative or custom chart (like something possible
in D3.js)? Interactivity is another big question---only certain tools
will make this possible.

\subsection{Pick the Right Chart Type!}\label{pick-the-right-chart-type}

Data visualization is a combination of art and science. When it comes to
the artistic aspect, there are no correct answers for doing the
visualization. There are many ways to present the data. However, when
making sense of facts, numbers, and measurement, better understanding is
promoted by a logical path to follow.

To determine the best type of chart is hard for those new to data
visulization. Most people learn it by referring to other people's work
without understanding the underlying logic, so they don't have the
theory in their mind to make the judgement. Here, I will introduce some
guidance on how to select the best chart for the objective.

When we are choosing the type of chart, we need to answer some
questions:\\
- How many features would you like to show in a chart?\\
- how many data points do you want to display for each variable?\\
- Will you display time serious data or among items or groups.

\begin{figure}
\centering
\includegraphics{images/interactivity.png}
\caption{}
\end{figure}

Rost's next takeaway: \textbf{``There Are No Perfect Tools, Just Good
Tools for People with Certain Mindsets''}

This section of the article is all about the difference in people's
preferences and opinions; from the people who build the tools to the
users, everyone thinks differently. Therefore, certain tools will be
inherently more intuitive to use for different people.

Let's review the most commonly used chart types and explain the ideal
circumstances for different types of charts and the pros and cons of
each. Before introducing different types of charts, you can use the
following reference to familiarize yourself with different types of
charts. \citep{charts_viz}

\textbf{Type 1 Column Charts}\\
This should be the most popular chart type. This chart is good to do
comparison between different values when specific values are important.
TBD Rost's third lesson: \textbf{`` We Still Live in an `Apps Are for
the Easy Stuff, Code Is for the Good Stuff'~World''} Basically, writing
code can be scary for anyone without a coding background, but it
provides more flexibility, and, as mentioned in class, code is perfectly
reproducible. On the other hand, apps are much more user-friendly for
the less computer science-savvy.

\begin{figure}
\centering
\includegraphics{images/apps_vs_code.png}
\caption{}
\end{figure}

Rost's final lesson: \textbf{``\,`Every Tool Forces You Down a
Path'\,''}

Rost quotes her former NPR Visuals teammate for the final lesson header,
pointing out that tools themselves influence the development of a data
visualization with their respective features, strengths, and
limitations.

\begin{figure}
\centering
\includegraphics{images/tools_force_paths.png}
\caption{}
\end{figure}

reference: \citep{different_tools}

\subsection{Three Rules to Follow in order to Develop Intuitive
Dashboards:}\label{three-rules-to-follow-in-order-to-develop-intuitive-dashboards}

Often a designer can become too concerned with coming up with a visual
that is too intricate and overly complicated. A dashboard should be
appealing but also easy to understand. Following these rules will lead
to effective presentation of the data.

\textbf{1. The dashboard should read left to right}\\
Because we read from top to bottom and left to right, a reader's eyes
will naturally look in the upper left of a page. The content should
therefore flow like words in a book. It is important to note that the
information at the top of the page does not always have to be the most
important. Annual data is usually more important to a business but daily
or weekly data could be used more often for day to day work. This should
be kept in mind when designing a dashboard as dashboards are often used
as a quick convenient way to look up data.

\textbf{2. Group related information together}\\
Grouping related data together is an intuitive way to help the flow of
the visual. It does not make sense for a user to have to search in
different areas to find the information they need.

\textbf{3. Find relationships between seemingly unrelated areas and
display visuals together to show the relationship.}\\
Grouping unrelated data seems contradictory to the second rule, but the
important thing is to tell a story not previously observed. Data
analytics is all about finding stories the data are trying to tell. Once
they are discovered, the stories need to be presented in an effective
manner. Grouping unrelated data together makes it easier to see how they
change together.

\textbf{Reference} \citep{intuitive-dash}

\subsection{Definions of Date Deception and Graphic
Integrity}\label{definions-of-date-deception-and-graphic-integrity}

Data visualization is becoming more and more popular to communicate and
support arguments nowdays. There are lots of great resources online to
create and design amazing data products, but at the same time, there are
some poorly-designed, misleading deceptive, data visualizations.

So what does \textbf{data deception} mean? Data deception, defined by
School of Law at the New York University, as ``a graphical depiction of
information, designed with or without an intent to deceive, that may
create a belief about the message and/or its components, which varies
from the actual message.''

In reality, decades ago, Edward Tufte already introduced the concept of
graphical intergrity in his book and presented six principles of graphic
integrity. Here are the principles from book:

\begin{enumerate}
\def\labelenumi{\arabic{enumi}.}
\item
  The representation of numbers, as physically measured on the surface
  of the graphic itself, should be directly proportional to the
  numerical quantities measured.
\item
  Clear, detailed, and thorough labeling should be used to defeat
  graphical distortion and ambiguity. Write out explanations of the data
  on the graphic itself. Label important events in the data.
\end{enumerate}

\section{6. Pick the Right Chart
Type}\label{pick-the-right-chart-type-1}

Data divusalization is combining the art and science. As for the art, we
can say there are no correct answers for doing the visualization. There
are many ways to present the data. However, how to making sense of
facts, numbers and measurement for better understanding is still have a
logical path to follow.

To determine which kind of chart is hard for those people new to data
visulization. Most people learn it by refering some other people's work
without understanding the logic behind. So they don't have the theory in
their mind to make the judgement. Here , I will introduce some guidance
to choose the charts.

When we about to choose the type of chart, we need to answer some
questions. - How many features would you like to show in a chart? - how
many data points do you want to display for each variable? - Will you
display time serious data or among items or groups.

After answered this question, you shoul able to get a better imagenation
of your ideal graph. The simple guidance for using different type of
chart is line charts for tracking trends over time, bar charts to
compare quantities, scatter plots for joint variation of two data items,
bubble charts showing joint variation of three data items, and pie
charts to compare parts of a whole.

Let's review the most commonly used chart types and expalin what
circumstance should better use typical chart and the pros and conts of
each type of chart. Before introduce differnt types of charts, you can
use the following website to familiar with different types of charts.

\textbf{Type 1 Column Charts.} This should be the most popular chart
type. This chart is good to do comparison between different values when
specific values are important. TBD

Still have hard time to choose? There are many resources on line can
help you do the decision. For example, Dr.~Andre Abela create a chart
selection diagram that is helpful to pick the right chart depends on the
data type. The link of website is

\section{7. Guide for Developing
Dashboards}\label{guide-for-developing-dashboards}

Three rules to follow in order to develop intuitive dashboards:

\begin{verbatim}
1. the dashboard should read left to right
2. group related information together
3. find relationships between seemingly unrelated areas and display visuals together to show the relationship.
\end{verbatim}

Often a designer can become too concerned with coming up with a visual
that is too intricate and overly complicated. A dashboard should be
appealing but also easy to understand. Following these rules will lead
to effective presentation of the data.

Because we read from top to bottom and left to right, a reader's eyes
will naturally look in the upper left of a page. The content should
therefore flow like words in a book. It is important to note that the
information at the top of the page does not always have to be the most
important. Annual data is usually more important to a business but daily
or weekly data could be used more often for day to day work. This should
be kept in mind when designing a dashboard as dashboards are often used
as a quick convenient way to look up data.

Grouping related data together is an intuitive way to help the flow of
the visual. It does not make sense for a user to have to search in
different areas to find the information they need.

Grouping unrelated data seems contradictory to the second rule, but the
important thing is to tell a story not previously observed. Data
analytics is all about finding stories the data is trying to tell. Once
they are discovered, the stories need to be presented in an effective
manner. Grouping unrelated data together makes it easier to see how they
change together.

\section{8. Definions of Date Deception and Graphic
Integrity}\label{definions-of-date-deception-and-graphic-integrity-1}

Data visualization becomes more and more pupular to communicate and
support arguments nowdays. There are lots of great resources online to
create and design amazing data products, in the same time, there are
some poorly-designed misleading deceptive data visualizations.

So what does \textbf{data deception} mean? Data deception, defined by
School of Law at the New York University, as ``a graphical depiction of
information, designed with or without an intent to deceive, that may
create a belief about the message and/or its components, which varies
from the actual message.�?

In reality, decades ago, Edward Tufte already introduced the concept of
graphical intergrity in his book and presented six principles of graphic
integrity. Here are the principles from book:

\begin{enumerate}
\def\labelenumi{\arabic{enumi}.}
\setcounter{enumi}{3}
\item
  In time-series displays of money, deflated and standardized units of
  monetary measurement are nearly always better than nominal units.
\item
  The number of information-carrying (variable) dimensions depicted
  should not exceed the number of dimensions in the data.

  \begin{enumerate}
  \def\labelenumii{\arabic{enumii}.}
  \item
    The representation of numbers, as physically measured on the surface
    of the graphic itself, should be directly proportional to the
    numerical quantities measured.
  \item
    Clear, detailed, and thorough labeling should be used to defeat
    graphical distortion and ambiguity. Write out explanations of the
    data on the graphic itself. Label important events in the data.
  \item
    Show data variation, not design variation.
  \item
    In time-series displays of money, deflated and standardized units of
    monetary measurement are nearly always better than nominal units.
  \item
    The number of information-carrying (variable) dimensions depicted
    should not exceed the number of dimensions in the data.
  \end{enumerate}
\end{enumerate}

\subsection{Misleading Graphs}\label{misleading-graphs}

Misleading graphs or distorted graphs, are graphs created which skews
the data, intentionally or unintentionally, resulting in a
representation of incorrect conclusions.

There are some ways in which distorted graphs can be created:

\textbf{1. Improper scaling of y axis:}\\
This is one of the classic misleading graphs. Instead of scale starting
from zero or a baseline, y axis is scaled conveniently to highlight the
differences among bins.

\textbf{2. Improper labelling of graphs:}\\
Lack of labels make the graph hard to interpret for the reader and lead
to wrong conclusions.

\textbf{3. Paired graphs on different scale:}\\
It is not a fair comparison if two elements are plotted side-by-side, on
a different scale and compared. This makes one graph look better than
the other, even when it is not.

\textbf{4. Dual axis with different scales:}\\
If we are plotting two elements on the same graph with different scales,
even if the axes are properly labeled, it is assumed that both axes are
on the same scale.

**5. Incomplete \url{data:**}\\
Short-term graphs are made to manipulate the trend, which will not be
seen otherwise. Time-series data are cut intentionally to just show a
trend within a particular period to create a more favorable visual
impression.

Please find the references below. \citep{evil_axes}
\citep{mislead_graph_ex}

\subsection{Current Research: Deceptive
Disualizations}\label{current-research-deceptive-disualizations}

Data visualization is a powerful communication tool to support arguments
with numbers in a way that is accessible and engaging. More people than
ever before are making their own charts and infographics, which is
presenting a unique problem. Despite the availability of some great
charting resources, we are witnessing an influx of poorly-designed
misleading or downright deceptive data visualizations. Here are
additional references on this topic: \citep{decept_study}
\citep{rose_tint}

\textbf{Reference} \citep{visual-lies}

The article focuses on a few methods that data visualizers utilize to
mislead users about research findings. For each method, the author has
highlighted the signifiers that are manipulated to promote an
unrealistic understanding of the visualized data. The author has
concentrated on examples of three areas to create deceptive data
visualization: size, segmentation, and graph type.

** Size **\\
Size signifies quantity, volume or degree of variables within a data. In
first figure the y-axis from the graph to the right is cut when
transcribed onto the graph on the left. Here both the graphs show the
same data but the one on the left represents the data in a misleading
fashion because of the way the axis is cut, and the result is that
interest rates have increased drastically from 2008 to 2012 -- a
misinterpretation that is avoided in the graph on the right.

Figure 1:\\
\includegraphics{images/Size1.png}

Quantity is the measure of size. When depicting points on a scatter
plot, the author suggest that it is helpful to manipulate the size the
points to represent differing values of a variable that is not
represented on the x and y axes. Following graph shows quantity as a two
completely different measure. One chart uses quantity as Area and other
uses it as radius. The result is that the differences in quantity
between points on such a scatter plot would appear more dramatic than
they should be.

Figure 2:\\
\includegraphics{images/Quantity1.png}

** Segmentation **\\
Figure shows an example of this with a deceptive instance of binning
given in the legend on the left. Segmentation can be used to show
category, parts, domains or ranges within a chart. The author states
that correct use of segmentation can enable users to enhance
understanding and if used incorrectly can be deceptive. It is shown here
binning is different in both and since in the left figure binning is not
done appropriately it is difficult to come up with actual values of the
data.

Figure 3:\\
\includegraphics{images/Segmentation 1.png}

** Graph **\\
Two graphs that are most often misrepresented are pie-charts and maps.
The author explains that in the following figure Pie charts can't be
compared accurately to one another. When striving for an accurate
portrayal of values, they should be avoided. The author further states
that it would be difficult to understand the pie-charts had the numbers
weren't given.

Figure 4:\\
\includegraphics{images/PieCharts.png}

The author then states that when showing spatial data analysis always
show population density when visualizing values that are
person-dependent. On a heat map where color signifies quantity, The
author suggests that a user will be drawn to the colors that a legend
indicates are most extreme.

In following figure, areas that are darkest are simply the most
population-dense regions of the United States. Without accounting for
population density, the newly created map may look the same as hundreds
of maps bearing a striking resemblance to the figure, which are falsely
considered informative and are regularly shared across social media
sites.

Figure 5:\\
\includegraphics{images/Maps1.png}

The above pointers are very helpful when creating a deceptive version of
a data product. However, as data visualizers we carefully need to draw
the line between creating misleading graphs that tells a different story
and deceptive version which is meant for exaggeration. The above can be
applied in our projects and can also be used to enhance our
understanding of great data visualization product.

\subsection{Typography and Data
Visualization}\label{typography-and-data-visualization}

This article discusses less common applications of typography in data
visualization. While data components such as quantitative or categorical
data are commonly represented by visual features like colors, sizes or
shapes, utilization of boldface, font variation, and other typographic
elements in data visualization are less prevalent.

Highlighted in the article are preattentive visual attributes;
preattentive attributes are those that perceptual psychologists have
determined to be easily recognized by the human brain irrespective of
how many items are displayed. Therefore, ``preattentive visual
attributes are desirable in data visualization as they can demand
attention only when a target is present, can be difficult to ignore, and
are virtually unaffected by load.'' Examples of preattentive attributes
are size/area, hue, and curvature.

This brings us to the disparateness of the popularity of visual aspects
like color and size and typographic aspects such as font variation,
capitalization and bold. The authors present several possible reasons
for this, beginning with the preattentiveness of visual attributes like
size and hue.However, some typographic attributes such as line width or
size, intensity, or font weight (a combination of the two) are
considered preattentive as well.

Furthermore, these visual attributes are inherently more viscerally
powerful, and they are easy to code in a variety of programming
languages. Technology has also perhaps previously limited the use of
typographic attributes, for only recently have fine details such as
serifs, italics, etc. been made readily visible to the audiences of data
visualizations by technological advances.

Lastly, the authors remark that it is possible the lack of variety of
typographic elements used in data visualizations is due to the limited
knowledge of computer scientists and other individuals pursuing data
visualization in how to apply these elements effectively. While the
first few proposed explanations make sense from personal experience with
technology and exposure to data visualizations and design in general,
the hypothesis that lack of knowledge of typographic elements in data
visualization seems more plausible if it was being applied to a small
group of people rather than all of the data visualization design
community. I would say that it is more likely that the use of
typographic elements in data visualization is less popular because there
are fewer instances in which it can be used appropriately, or a status
quo bias---if current visual attributes are received well, the
prevailing attitude may be not to fix what is not broken. However, the
authors also point out that despite the dearth of typographic attributes
in data visualization, other spheres like typography, cartography,
mathematics, chemistry, and programming ``have a rich history with type
and font attributes that informs the scope of the parameter space.''

The authors continue by pointing out some tips for using typographic
attributes to encode different data types, since certain attributes may
be suited to particular purposes. For example, font weight (size and
intensity) is ideal for representing quantitative or ordered data, and
font type (shape) is better suited to denote categories in the data.

Furthermore, as in typography and cartography, use of typographic
attributes in data visualization raises concerns of legibility, the
ability to understand both individual characters and commonalities that
identify a font family, and readability, the ability to read lines and
blocks of words.Often, interactivity of a visualization will not only
improve functionality, but also provide a solution to readability issues
by providing a means to zoom in on small text.

\begin{verbatim}
6.Graphics must not quote data out of context.
\end{verbatim}

\textbf{Misleading graphs:}

Misleading graphs or distorted graphs, are graphs created which skews
the data, intentionally or unintentionally, resulting in a
representation of incorrect conclusions.

There are some ways in which distorted graphs can be created: 1.
Improper scaling of y axis: This is one of the classic misleading
graphs. Instead of scale starting from zero or a baseline, y axis is
scaled conveniently to highlight the differences among bins. 2. Improper
labelling of graphs: Lack of labels make the graph hard to interpret for
the reader and lead to wrong conclusions. 3. Paired graphs on different
scale: It is not a fair comparison if two elements are plotted
side-by-side, on a different scale and compared. This makes one graph
look better than the other, even when it is not. 4. Dual axis with
different scales: If we are plotting two elements on the same graph with
different scales, even if the axes are properly labeled, it is assumed
that both axes are on the same scale. 5. Incomplete data: Short-term
graphs are made to manipulate the trend, which will not be seen
otherwise. Time-series data are cut intentionally to just show a trend
within a particular period to create a more favorable visual impression.

Please find the references below.

\section{9. Contemporary Research Results \& What's
Next}\label{contemporary-research-results-whats-next}

Next Steps for Data Visualization Research

With the development, studies and new tools applied in data
visualization, more people understand it matters. But given its youth
and interdisciplinary nature, research methods and training in the field
of data visualization are still developing. So, we asked ourselves: what
steps might help accelerate the development of the field? Based on a
group brainstorm and discussion, this article shares some of the
proposals of ongoing discussion and experiment with new approaches:

\begin{enumerate}
\def\labelenumi{\arabic{enumi}.}
\item
  Adapting the Publication and Review Process As the article states,
  ``both `good' and `bad' reviews could serve as valuable guides'', so
  providing reviewer guidelines could be helpful for fledgling
  practitioners in the field.
\item
  Promoting Discussion and Accretion Discussion of research papers
  actively occurs at conferences, on social media, and within research
  groups. Much of this discussion is either ephemeral or non-public. So
  ongoing discussion might explicitly transition to the online forum.
\item
  Research Methods Training Developing a core curriculum for data
  visualization research might help both cases, guiding students and
  instructors alike. For example, recognizing that empirical methods
  were critical to multiple areas of computer science, Stanford CS
  faculty organized a new course on Designing Computer Science
  Experiments. Also, online resources could be reinforced with a catalog
  of learning resources, ranging from tutorials and self-guided study to
  online courses. Useful examples include Jake Wobbrock's Practical
  Statistics for HCI and Pierre Dragicevic's resources for reforming
  statistical practice.
\end{enumerate}

\section{10. Typography and Data
Visualization}\label{typography-and-data-visualization-1}

This article discusses less common applications of typography in data
visualization. While data components such as quantitative or categorical
data are commonly represented by visual features like colors, sizes or
shapes, utilization of boldface, font variation, and other typographic
elements in data visualization are less prevalent.

Highlighted in the article are preattentive visual attributes;
preattentive attributes are those that perceptual psychologists have
determined to be easily recognized by the human brain irrespective of
how many items are displayed. Therefore, ``preattentive visual
attributes are desirable in data visualization as they can demand
attention only when a target is present, can be difficult to ignore, and
are virtually unaffected by load.�? Examples of preattentive attributes
are size/area, hue, and curvature.

This brings us to the disparateness of the popularity of visual aspects
like color and size and typographic aspects such as font variation,
capitalization and bold. The authors present several possible reasons
for this, beginning with the preattentiveness of visual attributes like
size and hue. However, some typographic attributes such as line width or
size, intensity, or font weight (a combination of the two) are
considered preattentive as well.\\
Furthermore, these visual attributes are inherently more viscerally
powerful, and they are easy to code in a variety of programming
languages. Technology has also perhaps previously limited the use of
typographic attributes, for only recently have fine details such as
serifs, italics, etc. been made readily visible to the audiences of data
visualizations by technological advances.

Lastly, the authors remark that it is possible the lack of variety of
typographic elements used in data visualizations is due to the limited
knowledge of computer scientists and other individuals pursuing data
visualization in how to apply these elements effectively. While the
first few proposed explanations make sense from personal experience with
technology and exposure to data visualizations and design in general,
the hypothesis that lack of knowledge of typographic elements in data
visualization seems more plausible if it was being applied to a small
group of people rather than all of the data visualization design
community. I would say that it is more likely that the use of
typographic elements in data visualization is less popular because there
are fewer instances in which it can be used appropriately, or a status
quo bias---if current visual attributes are received well, the
prevailing attitude may be not to fix what is not broken.\\
However, the authors also point out that despite the dearth of
typographic attributes in data visualization, other spheres like
typography, cartography, mathematics, chemistry, and programming ``have
a rich history with type and font attributes that informs the scope of
the parameter space.�?

The authors continue by pointing out some tips for using typographic
attributes to encode different data types, since certain attributes may
be suited to particular purposes. For example, font weight (size and
intensity) is ideal for representing quantitative or ordered data, and
font type (shape) is better suited to denote categories in the data.\\
Furthermore, as in typography and cartography, use of typographic
attributes in data visualization raises concerns of legibility, the
ability to understand both individual characters and commonalities that
identify a font family, and readability, the ability to read lines and
blocks of words. Often, interactivity of a visualization will not only
improve functionality, but also provide a solution to readability issues
by providing a means to zoom in on small text.

There are a few examples of unusual/innovative use of typography for
data visualization in the article, not all of which I agree are made
more effective by the interesting utilization of typographic attributes,
but the ``Who Survived the Titanic�? visualization's use of typographic
attributes allowed it to not only answer macro-questions very quickly,
such as if women and children were actually first to be evacuated across
classes, but also to provide answers to micro-questions, like whether or
not the Astors survived. It used common visual elements like color and
area to indicate whether or not a person survived and number/proportion
of people, as well as typographic aspects like italic and simple text
replacement to indicate gender and the passengers�? names.

The authors round out the article by addressing the most common
criticisms of typography in data visualization, the foremost one being
whether or not text should even be considered an element of data
visualization, since visualization connotes preattentive visual encoding
of information, and text or sequential information necessitates more
investment of attention to understand. Another criticism is that textual
representations are not as visually appealing even when used
effectively. However, the authors counter that ``this criticism
indicates both the strength and weakness of type,�? that while text may
not be suited for adding style or drama to a visualization, it can be
particularly powerful in situations where a finer level of detail is
needed, without sacrificing representation of higher level patterns.
Lastly, a label length problem is common when using text in
visualizations; differing lengths of names or labels may skew perception
so that longer labels seem more important than shorter labels. This
problem was encountered in the Titanic visualization with the varying
lengths representations of passengers�? names, and was corrected by only
including a given name and a surname, the length of which could only
vary so much.

The authors round out the article by addressing the most common
criticisms of typography in data visualization, the foremost one being
whether or not text should even be considered an element of data
visualization, since visualization connotes preattentive visual encoding
of information, and text or sequential information necessitates more
investment of attention to understand.Another criticism is that textual
representations are not as visually appealing even when used
effectively. However, the authors counter that ``this criticism
indicates both the strength and weakness of type,'' that while text may
not be suited for adding style or drama to a visualization, it can be
particularly powerful in situations where a finer level of detail is
needed, without sacrificing representation of higher level
patterns.Lastly, a label length problem is common when using text in
visualizations; differing lengths of names or labels may skew perception
so that longer labels seem more important than shorter labels. This
problem was encountered in the Titanic visualization with the varying
lengths representations of passengers' names, and was corrected by only
including a given name and a surname, the length of which could only
vary so much. The authors round out the article by addressing the most
common criticisms of typography in data visualization, the foremost one
being whether or not text should even be considered an element of data
visualization, since visualization connotes preattentive visual encoding
of information, and text or sequential information necessitates more
investment of attention to understand. Another criticism is that textual
representations are not as visually appealing even when used
effectively. However, the authors counter that ``this criticism
indicates both the strength and weakness of type,'' that while text may
not be suited for adding style or drama to a visualization, it can be
particularly powerful in situations where a finer level of detail is
needed, without sacrificing representation of higher level patterns.
Lastly, a label length problem is common when using text in
visualizations; differing lengths of names or labels may skew perception
so that longer labels seem more important than shorter labels. This
problem was encountered in the Titanic visualization with the varying
lengths representations of passengers' names, and was corrected by only
including a given name and a surname, the length of which could only
vary so much.

All in all, this article has an interesting take on a somewhat less
fashionable tool and puts forth the idea that text and typographic
attributes can convey additional important information in data
visualizations when used innovatively and correctly.

\section{11. Data Insights}\label{data-insights}

\begin{enumerate}
\def\labelenumi{\arabic{enumi}.}
\tightlist
\item
  \textbf{Balance}: A design is said to be balanced if key visual
  elements such as color, shape, texture, and negative space are
  uniformly distributed.
\end{enumerate}

\subsection{Using Data Visualization to Find Insights in
Data}\label{using-data-visualization-to-find-insights-in-data}

\begin{enumerate}
\def\labelenumi{\arabic{enumi}.}
\setcounter{enumi}{1}
\item
  \textbf{Emphasis}: Draw viewers attention towards important data by
  using key visual elements.
\item
  \textbf{Movement}: Ideally movement should mimic the way people
  usually read, starting at the top of the page, moving across it, and
  then down. Movement can also be created by using complimentary colors
  to pull the user's attention across the page.
\item
  \textbf{Pattern}: Patterns are ideal for displaying similar sets of
  information, or for sets of data that equal in value.Disrupting the
  pattern can also be effective in drawing viewers attention; it
  naturally draws curiosity.
\item
  \textbf{Repetition}: Relationships between sets of data can be
  communicated by repeating chart types, shapes, or colors.
\item
  \textbf{Proportion}: If a person is portrayed next to a house, the
  house is going look bigger. In data visualization, proportion can
  indicate the importance of data sets, along with the actual
  relationship between numbers.
\end{enumerate}

In Data Mining, there are different processes involve carrying out the
data mining process such as data extraction, data management, data
transformations, data pre-processing, etc.

In Data Visualization, the primary goal is to convey the information
efficiently and clearly without any deviations or complexities in the
form of statistical graphs, information graphs, and plots.

Also, the author listed the top 7 comparisons between data mining and
data visualization, and 12 key differences between data mining and data
visualization. After reading the article, you will have a very clear
understanding of what are data mining and data visualization and the
characters for those two techniques.

\begin{enumerate}
\def\labelenumi{\arabic{enumi}.}
\setcounter{enumi}{6}
\item
  \textbf{Rhythm}: A design has proper rhythm when the design elements
  create movement that is pleasing to the eye. If the design is not able
  to do so, rearranging visual elements may help.
\item
  \textbf{Variety}: Variety in color, shape, and chart-type draws and
  keeps users engaged with data. Including more variety can increase
  information retention by the viewer. But when there is too much
  variety, important details can be overlooked.
\item
  \textbf{Unity}: Unity across design will happen naturally if all other
  design principles are implemented.
\end{enumerate}

\subsection{Data Journalism}\label{data-journalism}

\section{12. Using Data Visualization to find insights in
data}\label{using-data-visualization-to-find-insights-in-data-1}

This article is extracted from a book known as Data Journalism Handbook
and this is one of the chapters of the book. The author starts the
article by introducing a very simple idea that loading any dataset into
a spreadsheet can also be a form of visualization as an invisible data
becomes visible in a picture form into a table. Hence the focus should
not be whether we need data visualization or not but should be on which
form of data visualization is best in which situation.

The author then proceeds by stating that data visualization will not
always unleash a readymade story on its own. Sometimes the insights are
known before the visualization and sometimes an insight can be
completely new. The author has given a process for finding insights in
the following way:

This article is extracted from a book known as Data Journalism Handbook
and this is one of the chapters of the book. The author starts the
article by introducing a very simple idea that loading any dataset into
a spreadsheet can also be a form of visualization as an invisible data
becomes visible in a picture form into a table. Hence the focus should
not be whether we need data visualization or not but should be on which
form of data visualization is best in which situation.

The author then proceeds by stating that data visualization will not
always unleash a readymade story on its own. Sometimes the insights are
known before the visualization and sometimes an insight can be
completely new. The author has given a process for finding insights in
the following way:

Visualize Data-\textgreater{} Analyze -\textgreater{} Document Insights
-\textgreater{} Transform Datasets -\textgreater{}Visualize Data

Each stage is explained in-depth further. Data Visualization can be done
in many ways such as tables which are great for one dimensional data
however they are bad for multi-dimensional data. Then he goes further to
explain the situation where each type of visualization such as bar
charts, maps, scatterplots, graphs, etc. are used. This gives a thorough
understanding of when to use which type of visualization. Once we
visualize the data we need to ask the following questions:

1 What can I see in this image? Is it what I expected? 2 Are there any
interesting patterns? 3 What does this mean in the context of the data?

The basic question answer format gives an idea to the viewers about what
kind of perspectives can we look at the data. Sometimes we discover
something and sometimes we don't. But the author mentions that we always
learn something from the visualization. Once we document the data
insights based on the above question we need to have the following
points into consideration:

1 Why have I created this chart? 2 What have I done to the data to
create it? 3 What does this chart tell me? Each stage is explained
in-depth further. Data Visualization can be done in many ways such as
tables which are great for one dimensional data however they are bad for
multi-dimensional data. Then he goes further to explain the situation
where each type of visualization such as bar charts, maps, scatterplots,
graphs, etc. are used. This gives a thorough understanding of when to
use which type of visualization. Once we visualize the data we need to
ask the following questions:

\begin{enumerate}
\def\labelenumi{\arabic{enumi}.}
\tightlist
\item
  What can I see in this image? Is it what I expected?\\
\item
  Are there any interesting patterns?\\
\item
  What does this mean in the context of the data?
\end{enumerate}

1 What can I see in this image? Is it what I expected? 2 Are there any
interesting patterns? 3 What does this mean in the context of the data?

The basic question answer format gives an idea to the viewers about what
kind of perspectives can we look at the data. Sometimes we discover
something and sometimes we don't. But the author mentions that we always
learn something from the visualization. Once we document the data
insights based on the above question we need to have the following
points into consideration:

\begin{enumerate}
\def\labelenumi{\arabic{enumi}.}
\tightlist
\item
  Why have I created this chart?\\
\item
  What have I done to the data to create it?\\
\item
  What does this chart tell me?
\end{enumerate}

1 Why have I created this chart? 2 What have I done to the data to
create it? 3 What does this chart tell me?

The above question answer format compels the viewers to think deeper
about what exactly we are trying to find. Because many times the viewers
are simply too overwhelmed with the size of data that they lose the
basic idea. Hence this kind of approach help to stay focused. The author
then mentions that based on the above insights we might have some idea
about some interesting patterns. Since we already have an idea we might
want to see it in more detail and hence we transform data in more
details such as Zooming, Filtering, Outlier Removal. The author then
explains how transformed data can help us to see a more detailed view of
our insights.

Further the author gives a detailed explanation of which data
visualization tool to use based on the situation. The entire process
given above is explained in depth with the help of examples. The
technical approach listed above is practical and can be implemented
easily on our data visualization projects. I liked the author's approach
because he has cleverly integrated the step-by-step process of finding
insights with the technical way of handling datasets using tools such as
Tableau, Python, etc. And the process can be repeated many times till we
find the insights we are looking for.

\textbf{Reference} \citep{data_journ}

\subsection{Contemporary Research Results \& What's
Next}\label{contemporary-research-results-whats-next-1}

Next Steps for Data Visualization Research:

With the development, studies and new tools applied in data
visualization, more people understand it matters. But given its youth
and interdisciplinary nature, research methods and training in the field
of data visualization are still developing. So, we asked ourselves: what
steps might help accelerate the development of the field? Based on a
group brainstorm and discussion, this article shares some of the
proposals of ongoing discussion and experiment with new approaches:

\textbf{1. Adapting the Publication and Review Process}\\
As the article states, ``both `good' and `bad' reviews could serve as
valuable guides'', so providing reviewer guidelines could be helpful for
fledgling practitioners in the field.

\textbf{2. Promoting Discussion and Accretion}\\
Discussion of research papers actively occurs at conferences, on social
media, and within research groups. Much of this discussion is either
ephemeral or non-public. So ongoing discussion might explicitly
transition to the online forum.

\textbf{3. Research Methods Training}\\
Developing a core curriculum for data visualization research might help
both cases, guiding students and instructors alike. For example,
recognizing that empirical methods were critical to multiple areas of
computer science, Stanford CS faculty organized a new course on
Designing Computer Science
Experiments(\url{http://sing.stanford.edu/cs303-sp11/}). Also, online
resources could be reinforced with a catalog of learning resources,
ranging from tutorials and self-guided study to online courses. Useful
examples include Jake Wobbrock's Practical Statistics for HCI and Pierre
Dragicevic's resources for reforming statistical practice.

\textbf{Reference} \citep{next_steps}

\textbf{Building advanced analytics application with TabPy}\\
\citep{TabPy}

Imagine a scenario where we can just enter some x values in a dashboard
form, and the visualization would predict the y variable!!!

Here is a link that shows how to integrate and visualize data from
Python in Tableau. This is especially relevant to all data science
students, as this is one of the tools used for visualizing advanced
analytics.

The author here has given an example using data from Seattle's police
department's 911 calls and he tries to identify criminal hotspots in the
area.The author uses machine learning (spatial clustering) and creates a
great interactive visualization, where you can click on the type of
criminal activity and the graph will show various clusters.

There are other examples and use cases that may be downloaded, and the
scripts are also given by the author for anyone who is interested in
trying it out.

\textbf{Some best practices for visualization:}\\
\citep{DataVizBestPrac}

Here is free pdf to some best practices in visual analysis. It talks
about the right charts to be used for various kinds of analysis. It is
very relevant for data science students as we would be interested in
presenting our analysis using simple and effective visualizations that
tell the complete story.

Some of key areas for which the author highlights some best practices
are for visualizing trends over time, comparison and ranking,
correlation, distribution, geographical data etc.

The author gives examples on how simple graphs can also become more
effective by just adding a few more elements or some simple adjustments.

I feel this is a great starting point to create effective charts and we
may use these principles also when we start doing advanced analytics.

\textbf{Avoiding Common Mistakes with Time Series}\\
\url{https://www.svds.com/avoiding-common-mistakes-with-time-series/}

This article explains how time series data visualization can sometimes
be deceptive.It first takes an example of two random time series data
and plots them on a graph which gives an impression that the two are
strongly correlated. But if we do some statistical testing the two do
not show any relationship, this is an example of ``correlation does not
necessary mean causation''.In another set of examples author has taken
trending two random time series data and shown how even statistical
tests can give a wrong interpretation. The article then explains using
visualization how a general trended time series can be different than a
more controlled and measured trending time series.

\subsection{Corporate Scorecards and Data
Visualization}\label{corporate-scorecards-and-data-visualization}

\begin{itemize}
\tightlist
\item
  (reference:
  \url{http://www.boostlabs.com/corporate-scorecards-data-visualization/})
\end{itemize}

Corporate transparency, flat organizations, open book policies, etc. are
terms executives and entrepreneurs learn about all the time. As the
corporate world shifts towards a more open culture, the demand for open
data and insights have increased dramatically. This shift has helped the
overall corporate strategic planning and management process--easing the
alignment of business activities towards a series of goals. Being
transparent top down aligns the culture to sail towards the same North
Star.

The growth of corporate transparency is not only important internally,
but externally as well. Corporate certifications like B Corporations
certifications (B Corp), require companies to provide a transparent view
on their social conscious efforts to the general public. Achieving the
certification is one step of the process; the true goal is to show the
world how and why the certification is truly deserved.

How does data visualization play a part? Data Visualization helps reveal
insights and patterns that aren't immediately visible in the raw data.
Here's the process on how to get it done:

\textbf{Step 1: Perform Data Discovery and Determine The Story}\\
Before this step it is easy to underestimate the effort level it takes
to pull the best insights from the data. Data manipulation products like
Tableau, Domo, Pentaho, IBM's Many Eyes, and R, among others, make
insight extraction that much easier to gain understanding of data using
a visual medium.

The key is to start with a simple portion of your data and to start
pulling basic insights to visualize and correlate with each other. This
process leads towards a compound series of questions, which helps
provide an overall vision to the end product. We see the effect during
our discovery process, which leads to unforeseen avenues for data
intelligence.

\textbf{Step 2: Data Infrastructure Setup}\\
Data infrastructures can be simple or complex depending what the end
goal is. Many clients prefer to go the route of complete data
integration in order to centralize their data repositories. Technologies
such as Hadoop have helped by unifying disparate data sources, but other
options such as data cloud environments can help produce API's for
future product deployments. Why is this important? Accessibility of data
is an important foundation not only within the context of dashboards,
but also the possibility of branching out to other products.

\textbf{Step 3: Product Design \& Development}\\
Wireframing, prototyping, and application development are the main
engines to transform an idea into a final product. Products can range
from static presentations/reports to full interactive applications.
Mobile, tablet, TV, and workstation platforms can all be mediums to help
deliver the final product. The secret to a great end product is how well
the data story is conceptualized. If the story is weak then the end
product will also suffer.

\textbf{Step 4: QA \& Product Release}\\
The best part of any project is to get it finalized and released for all
to see. All data gets verified for accuracy, functionality testing (if
applicable), application flow (if applicable), design testing, and
remaining items are all completed. The end result is an engaging visual
product for all intended audiences to see and use.

\subsection{Data Visualization Tools}\label{data-visualization-tools-1}

Lisa Rost's article ``What I learned recreating one chart using 24
tools'' describes lessons learned from recreating one chart using many
different data visualization tools.The author used apps Excel, Plotly,
Easycharts, Google Sheets, Lyra, Highcharts, Tableau, Polestar,
Quadrigram, Illustrator, RAW, and NodeBox, as well as charting libraries
ggvis, Bokeh, Highcharts, ggplot2, Processing, NVD3, Seaborn, Vega, D3,
matplotlib, Vega-Lite, and R.She links her github page on the project
which details the dataset she used, containing the health expectancy in
years as well as GDP per capita and population for about 200 countries
in the year 2015, as well has her process and results of visualizing the
data using each tool.However, in the article, she focuses on the main
takeaways from the exercise, which was especially interesting in the
context of our class discussion on different types of tools and their
respective strengths.She also provides her own graphics to help
illustrate her lessons learned.

Rost's first takeaway: \textbf{``There Are No Perfect Tools, Just Good
Tools for People with Certain Goals''}

Since data visualization is necessary in many spheres, from science to
journalism, data visualization projects will often have quite disparate
objectives, and the people working on them will have different
requirements.And as the author aptly points out, it is impossible for
one tool to satisfy the needs of every data visualizer; so there will
necessarily be tools better suited to specific situations.

For example, does the user need a tool for exploratory visualization of
the data, or does the user seek to create graphs and charts to show the
public or a specific audience something?

\begin{figure}
\centering
\includegraphics{images/analysis_spectrum.png}
\caption{}
\end{figure}

The author also notes that the flexibility of a tool is a sticking point
as well---if you need to change your data while developing a data
visualization, certain apps like Illustrator will not be ideal because
changing the data even slightly requires you to build the graph again
from scratch. Another thing to think about is the type of chart you are
trying to create---is a basic, canned bar or line graph all you need (in
which case something like Excel will do the trick), or does your project
necessitate a more innovative or custom chart (like something possible
in D3.js)? Interactivity is another big question---only certain tools
will make this possible.

Rost's next takeaway: \textbf{``There Are No Perfect Tools, Just Good
Tools for People with Certain Mindsets''}

This section of the article is all about the difference in people's
preferences and opinions; from the people who build the tools to the
users, everyone thinks differently.Therefore, certain tools will be
inherently more intuitive to use for different people.

Rost's third lesson: \textbf{`` We Still Live in an `Apps Are for the
Easy Stuff, Code Is for the Good Stuff' World''}

Basically, writing code can be scary for anyone without a coding
background, but it provides more flexibility, and, as mentioned in
class, code is perfectly reproducible.On the other hand, apps are much
more user-friendly for the less computer science-savvy.

Rost's final lesson: \textbf{``\,`Every Tool Forces You Down a
Path'\,''}

The above question answer format compels the viewers to think deeper
about what exactly we are trying to find. Because many times the viewers
are simply too overwhelmed with the size of data that they lose the
basic idea. Hence this kind of approach help to stay focused. The author
then mentions that based on the above insights we might have some idea
about some interesting patterns. Since we already have an idea we might
want to see it in more detail and hence we transform data in more
details such as Zooming, Filtering, Outlier Removal. The author then
explains how transformed data can help us to see a more detailed view of
our insights.

Further the author gives a detailed explanation of which data
visualization tool to use based on the situation. The entire process
given above is explained in depth with the help of examples. The
technical approach listed above is practical and can be implemented
easily on our data visualization projects. I liked the author's approach
because he has cleverly integrated the step-by-step process of finding
insights with the technical way of handling datasets using tools such as
Tableau, Python, etc. And the process can be repeated many times till we
find the insights we are looking for.

\citep{great_viz} Imagine a scenario where we can just enter some x
values in a dashboard form, and the visualization would predict the y
variable!!!

\begin{itemize}
\tightlist
\item
  Theoretical background of data visualization
\item
  Contemporary research results
\end{itemize}

\citep{LinkedIn_Groups}

** Data Mining v.s. Data Visualization ** Reference:
\url{https://www.educba.com/data-mining-vs-data-visualization/} (March
30, 2018)

This article gives me a clear understanding of data mining and data
visualization.

\subsection{Comparison of Different
Tools}\label{comparison-of-different-tools}

\textbf{Interactive Data Visualization}

In Data Mining, there are different processes involve carrying out the
data mining process such as data extraction, data management, data
transformations, data pre-processing, etc. In Data Visualization, the
primary goal is to convey the information efficiently and clearly
without any deviations or complexities in the form of statistical
graphs, information graphs, and plots. Also, the author listed the top 7
comparisons between data mining and data visualization, and 12 key
differences between data mining and data visualization. After reading
the article, you will have a very clear understanding of what are data
mining and data visualization and the characters for those two
techniques.

\section{13.Interactive Data
Visualization}\label{interactive-data-visualization}

Interactive or Dynamic data visualization delivers today's complex sea
of data in a graphically compelling and an easy-to-understand way. It
enables direct actions on a plot to change elements and link between
multiple plots. It enables users to accomplish traditional data
exploration tasks by making charts interactive.

Benefits of Interactive Data Visualization Software: 13.1 Benefits of
Interactive Data Visualization Software:

\begin{enumerate}
\def\labelenumi{\arabic{enumi}.}
\tightlist
\item
  Absorb information in constructive ways: With the volume and velocity
  of data created everyday, dynamic data viz enables enhanced process
  optimization, insight discovery and decision making.\\
\item
  Visualize relationships and patterns: Helps inbetter understanding of
  correlations among operational data and business performance.\\
\item
  Identify and act on emerging trends faster: Helps decision makers to
  grasp shifts in behaviors and trends across multiple data sets much
  more quickly.\\
\item
  Manipulate and interact directly with data: Enables users to engage
  data more frequently.\\
\item
  Foster a new business language : Ability to tell a story through data
  that instantly relates the performance of a business and its assets.
\end{enumerate}

\citep{benefits_interactive_viz}

\textbf{There are multiple ways by which interactive data visualizations
can be developed:}

\textbf{D3.js:}\\
Visualize the Big Data in literally any way you need.

D3.js stands for Data Driven Document, a JS library for interactive Big
Data visualization in literally ANY way required real-time. This is not
a tool, mind you, so a user should have a solid understanding of
Javascript to work with the data and present it in a
humanly-understandable form. To say more, this library renders the data
into SVG and HTML5 formats, so older browsers like IE7 and 8 cannot
leverage D3.js capabilities.

The data gathered from disparate sources like huge-scale data sets is
binded in real-time with DOMto produce interactive animations ( 2D and
3D alike) in an extremely rapid way. The D3 architecture allows the
users to intensively reuse the codes across a variety of add-ons and
plug-ins. 1. Absorb information in constructive ways: With the volume
and velocity of data created everyday, dynamic data viz enables enhanced
process optimization, insight discovery and decision making. 2.
Visualize relationships and patterns: Helps in better understanding of
correlations among operational data and business performance. 3.
Identify and act on emerging trends faster: Helps decision makers to
grasp shifts in behaviors and trends across multiple data sets much more
quickly. 4. Manipulate and interact directly with data: Enables users to
engage data more frequently. 5. Foster a new business language : Ability
to tell a story through data that instantly relates the performance of a
business and its assets.

\citep{benefits_interactive_viz}

There are multiple ways by which interactive data visualizations can be
developed: 1. D3.js 2.Tableau 3.R shiny

13.2 D3.js:

D3.js (or just D3 for Data-Driven Documents) is a JavaScript library for
producing dynamic, interactive data visualizations in web browsers(From
Wikipedia). It is highly functional, meaning you can reuse the code and
add functions relevant to your project. Embedded within an HTML webpage,
the JavaScript D3.js library uses pre-built JavaScript functions to
select elements, create SVG objects, style them, or add transitions,
dynamic effects or tooltips to them.

Some of the key advantages are: It is dynamic, free and open source and
very flexible with all web technologies, the abiity to handle big data
and the functional style allows to reuse the codes.

\citep{d3_interactive_viz}

\textbf{Tableau:}\\
The best solution for visualizing AI, Big Data and Machine Learning
apps.

Tableau is amidst the market leaders for the Big Data visualization,
especially efficient for delivering interactive data visualization for
the results derived from Big Data operations, deep learning algorithms
and multiple types of AI-driven apps.

Tableau can be integrated with Amazon AWS, MySQL, Hadoop, Teradata and
SAP, making this solution a versatile tool for creating detailed graphs
and intuitive data representation. This way the C-suite and middle-chain
managers are able to make grounded decisions based on informative and
easily-readable Tableau graphs.

13.3 Tableau:

Tableau is business intelligence (BI) and analytics platform created for
the purposes of helping people see, understand, and make decisions with
data. It is the industry leader in interactive data visualization tools,
offering a broad range of maps, charts, graphs, and more graphical data
presentations. It is a painless option when cost is not a concern and
you do not need advanced and complex analysis.The application is very
handy for quickly visualizing trends in data, connecting to a variety of
data sources, and mapping cities/regions and their associated data.

Tableau is business intelligence (BI) and analytics platform created for
the purposes of helping people see, understand, and make decisions with
data. It is the industry leader in interactive data visualization tools,
offering a broad range of maps, charts, graphs, and more graphical data
presentations. It is a painless option when cost is not a concern and
you do not need advanced and complex analysis.The application is very
handy for quickly visualizing trends in data, connecting to a variety of
data sources, and mapping cities/regions and their associated data.

The key advantages are: It provides non technical user the ability to
build complex reports and dashboard with zero coding skills. Using
drag-n-drop functionalities of Tableau, user can create a very
interactive visuals within minutes. It can handle millions of rows of
data with ease and users can make live to connections to different data
sources like SQL etc.

\citep{tableau_interactive_viz}

Now that we have all learnt the basics of working with Tableau, we may
think of following groups or joining communities to explore tableau
further.

The benefits:

-It will help us enhance our learning\\
-Get answers for most of your doubts In tableau\\
-Post new questions and crowd source answers\\
-Attend events, seminars and join conferences conducted locally/
globally\\
-Give back to the community once you become an expert in that field

Some useful communities for Tableau users:

Tableau Community - \citep{Tableau_Community}

Blogs : Here is a list of the top 10 blogs that Tableau itself suggests
following:\\
\citep{Top_10_Blogs}

1.Storytelling with Data\\
2.Information is Beautiful\\
3.Flowing Data\\
4.Visualising Data\\
5.Junk Charts\\
6.The Pudding\\
7.The Atlas\\
8.Graphic Detail\\
9.US Census and FEMA\\
10.Tableau Blog

Tableau Social Media Groups: Some of the biggest and the most active
groups

Tableau Enthusiasts: Linkedin Group (19K members)

Tableau Software Fans \& Friends: LinkedIn Group (45kK members)

\citep{LinkedIn_Groups}

\textbf{R Shiny :} 13.4 R Shiny :

R Shiny enables us to produce interactive data visualizations with a
minimum knowledge of HTML, CSS, or Java using a simple web application
framework that runs under the R statistical platform. Standalone apps
can be hosted on a webpage or embedded in R Markdown documents and
dashboards can be built using R shiny. It combines the computational
power of R with the interactivity of the modern web.

The main advantages of using R Shiny are : Its flexibility of pulling in
whatever package in R that you want to solve your problem, reaping the
benefits of an open source ecosystem for R and Javascript visualization
libraries, thereby allowing to create highly custom applications and
enabling timely, high quality interactive data experience without (or
with much less) web development and without the limitations or cost of
proprietary BI tools.

\citep{shiny_interactive_viz}

\textbf{Jupyter:}

\section{14. Avoiding Common Mistakes with Time
Series}\label{avoiding-common-mistakes-with-time-series}

This article explains how time series data visualization can sometimes
be deceptive. It first takes an example of two random time series data
and plots them on a graph which gives an impression that the two are
strongly correlated. But if we do some statistical testing the two do
not show any relationship, this is an example of ``correlation does not
necessary mean causation''. In another set of examples author has taken
trending two random time series data and shown how even statistical
tests can give a wrong interpretation. The article then explains using
visualization how a general trended time series can be different than a
more controlled and measured trending time series. master

reference-fundamentals-Tableau-Groups

\begin{itemize}
\tightlist
\item
  Theoretical background of data visualization\\
\item
  Contemporary research results
\end{itemize}

Now that we have all learnt the basics of working with Tableau, we may
think of following groups or joining communities to explore tableau
further.

The benefits:

\begin{itemize}
\tightlist
\item
  It will help us enhance our learning
\item
  Get answers for most of your doubts In tableau
\item
  Post new questions and crowd source answers
\item
  Attend events, seminars and join conferences conducted locally/
  globally
\item
  Give back to the community once you become an expert in that field
\end{itemize}

Some useful communities for Tableau users:

Tableau Community - \citep{Tableau_Community}

Blogs : Here is a list of the top 10 blogs that Tableau itself suggests
following:

\citep{Top_10_Blogs}

\begin{enumerate}
\def\labelenumi{\arabic{enumi}.}
\tightlist
\item
  Storytelling with Data
\item
  Information is Beautiful
\item
  Flowing Data
\item
  Visualising Data
\item
  Junk Charts
\item
  The Pudding
\item
  The Atlas
\item
  Graphic Detail
\item
  US Census and FEMA
\item
  Tableau Blog
\end{enumerate}

Tableau Social Media Groups: Some of the biggest and the most active
groups

Tableau Enthusiasts: Linkedin Group (19K members)

Tableau Software Fans \& Friends: LinkedIn Group (45kK members)

\citep{LinkedIn_Groups}

\url{https://www.educba.com/data-mining-vs-data-visualization/}

This article gives me a clear understanding of data mining and data
visualization.

In Data Mining, there are different processes involve carrying out the
data mining process such as data extraction, data management, data
transformations, data pre-processing, etc. In Data Visualization, the
primary goal is to convey the information efficiently and clearly
without any deviations or complexities in the form of statistical
graphs, information graphs, and plots. Also, the author listed the top 7
comparisons between data mining and data visualization, and 12 key
differences between data mining and data visualization. After reading
the article, you will have a very clear understanding of what are data
mining and data visualization and the characters for those two
techniques.

This article explains how time series data visualization can sometimes
be deceptive. It first takes an example of two random time series data
and plots them on a graph which gives an impression that the two are
strongly correlated. But if we do some statistical testing the two do
not show any relationship, this is an example of ``correlation does not
necessary mean causation''. In another set of examples author has taken
trending two random time series data and shown how even statistical
tests can give a wrong interpretation. The article then explains using
visualization how a general trended time series can be different than a
more controlled and measured trending time series. master

\section{15.The Hitchhiker��s Guide to
d3.js}\label{the-hitchhikers-guide-to-d3.js}

This is a wonderful guide for self-teaching d3.js. This guide is meant
to prepare readers mentally as well as give readers some fruitful
directions to pursue. There is a lot to learn besides the d3.js API,
both technical knowledge around web standards like HTML, SVG, CSS and
JavaScript as well as communication concepts and data visualization
principles. Chances are you know something about some of those things,
so this guide will attempt to give you good starting points for the
things you want to learn more about.

It starts from the insights of learning d3.js by showing interviews with
those top visualization practitioners. Then the author gives key
concepts and useful features for learning visualization like d3-shape,
d3 selection, d3-collection, ds-hierarchy, ds-zoom as well as d3-force.

My favorite part of this guide is it lists a lot of useful resources
links for learning d3.js. For example, it recommends d3 API Reference,
2000+ d3 case studies and tutorials for d3. I did my exploratory
analysis version of group project on d3. And I found this guide helpful
during the progress. It also includes some meetup groups here in the bay
area. So, maybe we can meet data friends through the group.

This article explains how time series data visualization can sometimes
be deceptive. It first takes an example of two random time series data
and plots them on a graph which gives an impression that the two are
strongly correlated. But if we do some statistical testing the two do
not show any relationship, this is an example of ``correlation does not
necessary mean causation''. In another set of examples author has taken
trending two random time series data and shown how even statistical
tests can give a wrong interpretation. The article then explains using
visualization how a general trended time series can be different than a
more controlled and measured trending time series.

\section{15. Corporate Scorecards and Data
Visualization}\label{corporate-scorecards-and-data-visualization-1}

\citep{SCORECARDS}

Corporate transparency, flat organizations, open book policies, etc. are
terms executives and entrepreneurs learn about all the time. As the
corporate world shifts towards a more open culture, the demand for open
data and insights have increased dramatically. This shift has helped the
overall corporate strategic planning and management process--easing the
alignment of business activities towards a series of goals. Being
transparent top down aligns the culture to sail towards the same North
Star.

The growth of corporate transparency is not only important internally,
but externally as well. Corporate certifications like B Corporations
certifications (B Corp), require companies to provide a transparent view
on their social conscious efforts to the general public. Achieving the
certification is one step of the process; the true goal is to show the
world how and why the certification is truly deserved.

\textbf{Story Telling with Data}

Story telling is an essential part of data visualization. It is
extremely important to effectively communicate information through the
visualization. Stikeleather (2013)'s article talked about how a visual
designer tells a story with a visualization. Mainly there are five
strategies that can be applied during visualization:

\begin{enumerate}
\def\labelenumi{\arabic{enumi}.}
\tightlist
\item
  Find the compelling narrative.\\
\item
  Think about the audience (e.g., novice, generalist, managerial,
  export, exectitve)\\
\item
  Be objective and offer balance\\
\item
  Don't censor\\
\item
  Finally, Edit, Edit, Edit.
\end{enumerate}

How does data visualization play a part? Data Visualization helps reveal
insights and patterns that aren't immediately visible in the raw data.
Here's the process on how to get it done:

\textbf{Step 1: Perform Data Discovery and Determine The Story} Before
this step it is easy to underestimate the effort level it takes to pull
the best insights from the data. Data manipulation products like
Tableau, Domo, Pentaho, IBM's Many Eyes, and R, among others, make
insight extraction that much easier to gain understanding of data using
a visual medium.

The key is to start with a simple portion of your data and to start
pulling basic insights to visualize and correlate with each other. This
process leads towards a compound series of questions, which helps
provide an overall vision to the end product. We see the effect during
our discovery process, which leads to unforeseen avenues for data
intelligence.

\textbf{Step 2: Data Infrastructure Setup} Data infrastructures can be
simple or complex depending what the end goal is. Many clients prefer to
go the route of complete data integration in order to centralize their
data repositories. Technologies such as Hadoop have helped by unifying
disparate data sources, but other options such as data cloud
environments can help produce API's for future product deployments. Why
is this important? Accessibility of data is an important foundation not
only within the context of dashboards, but also the possibility of
branching out to other products.

\textbf{Step 3: Product Design \& Development} Wireframing, prototyping,
and application development are the main engines to transform an idea
into a final product. Products can range from static
presentations/reports to full interactive applications. Mobile, tablet,
TV, and workstation platforms can all be mediums to help deliver the
final product. The secret to a great end product is how well the data
story is conceptualized. If the story is weak then the end product will
also suffer.

\textbf{Step 4: QA \& Product Release} The best part of any project is
to get it finalized and released for all to see. All data gets verified
for accuracy, functionality testing (if applicable), application flow
(if applicable), design testing, and remaining items are all completed.
The end result is an engaging visual product for all intended audiences
to see and use.

\textbf{Google chart:}\\
A free and powerful integration of all Google power.

\section{16 Visual Lies: Usability in Deceptive Data
Visualizations}\label{visual-lies-usability-in-deceptive-data-visualizations}

Reference - \citep{visual-lies}

The article focuses on a few methods that data visualizers utilize to
mislead users about research findings. For each method, the author has
highlighted the signifiers that are manipulated to promote an
unrealistic understanding of the visualized data. The author has
concentrated on examples of three areas to create deceptive data
visualization: size, segmentation, and graph type.

** Size ** Size signifies quantity, volume or degree of variables within
a data. In first figure the y-axis from the graph to the right is cut
when transcribed onto the graph on the left. Here both the graphs show
the same data but the one on the left represents the data in a
misleading fashion because of the way the axis is cut, and the result is
that interest rates have increased drastically from 2008 to 2012 -- a
misinterpretation that is avoided in the graph on the right.

Figure 1: \includegraphics{images/Size1.png}

The tool is rendering the resulting charts to HTML5/SVG, so they are
compatible with any browser. Support for VML ensures compatibility with
older IE versions, and the charts can be ported to the latest releases
of Android and iOS. What's even more important, Google chart combines
the data from multiple Google services like Google Maps. This results in
producing interactive charts that absorb data real-time and can be
controlled using an interactive dashboard.\\
{[}\url{https://towardsdatascience.com/top-4-popular-big-data-visualization-tools-4ee945fe207d}{]}
Quantity is the measure of size. When depicting points on a scatter
plot, the author suggest that it is helpful to manipulate the size the
points to represent differing values of a variable that is not
represented on the x and y axes. Following graph shows quantity as a two
completely different measure. One chart uses quantity as Area and other
uses it as radius. The result is that the differences in quantity
between points on such a scatter plot would appear more dramatic than
they should be.

Figure 2:

\subsection{Tufte's Design Principles}\label{tuftes-design-principles}

** Segmentation **\\
Figure shows an example of this with a deceptive instance of binning
given in the legend on the left. Segmentation can be used to show
category, parts, domains or ranges within a chart. The author states
that correct use of segmentation can enable users to enhance
understanding and if used incorrectly can be deceptive. It is shown here
binning is different in both and since in the left figure binning is not
done appropriately it is difficult to come up with actual values of the
data. \includegraphics{images/Quantity1.png} ``

** Segmentation ** Figure shows an example of this with a deceptive
instance of binning given in the legend on the left. Segmentation can be
used to show category, parts, domains or ranges within a chart. The
author states that correct use of segmentation can enable users to
enhance understanding and if used incorrectly can be deceptive. It is
shown here binning is different in both and since in the left figure
binning is not done appropriately it is difficult to come up with actual
values of the data.

Figure 3: \includegraphics{images/Segmentation 1.png}

** Graph **\\
Two graphs that are most often misrepresented are pie-charts and maps.
The author explains that in the following figure Pie charts can't be
compared accurately to one another. When striving for an accurate
portrayal of values, they should be avoided. The author further states
that it would be difficult to understand the pie-charts had the numbers
weren't given.

Figure 4: \includegraphics{images/PieCharts.png}

The author then states that when showing spatial data analysis always
show population density when visualizing values that are
person-dependent. On a heat map where color signifies quantity, The
author suggests that a user will be drawn to the colors that a legend
indicates are most extreme.

In following figure, areas that are darkest are simply the most
population-dense regions of the United States. Without accounting for
population density, the newly created map may look the same as hundreds
of maps bearing a striking resemblance to the figure, which are falsely
considered informative and are regularly shared across social media
sites.

Figure 5:

The above pointers are very helpful when creating a deceptive version of
a data product. However, as data visualizers we carefully need to draw
the line between creating misleading graphs that tells a different story
and deceptive version which is meant for exaggeration. The above can be
applied in our projects and can also be used to enhance our
understanding of great data visualization product. The above pointers
are very helpful when creating a deceptive version of a data product.
However, as data visualizers we carefully need to draw the line between
creating misleading graphs that tells a different story and deceptive
version which is meant for exaggeration. The above can be applied in our
projects and can also be used to enhance our understanding of great data
visualization product.

\section{17. Guide to Best Practices in Data
Visualization}\label{guide-to-best-practices-in-data-visualization-1}

\url{http://paristech.com/blog/data-visualization-best-practices/}\\
\url{http://extremepresentation.typepad.com/blog/2015/01/announcing-the-slide-chooser.html}

A graph should be impressive and can obtain audience's attention. How
can we achieve this? We must consider several aspects:
\textbf{efficiency, complexity, structure, density and beauty}. We also
should consider the audience whether they will be confused about the
design.

\textbf{• Design principle of graphical excellence}\\
Statistical graphics are kind of graphics we use a lot for data analysis
so I want to summarize some principles mentioned in this Tufte's great
book.Here we must follow some principles for statistical graphics:

\textbf{Principle 1: Maximizing the data-ink ratio, within reason.}

Data-ink is the non-erasable core of a graphic, the non-redundant
inkarranged in response to variation in the numbers represented.

Data-ink ratio = data-ink/total ink used to print the graphic =
proportion of graphic's ink devoted to the non-redundant display of
data-information

This basic principle follows by two principles:\\
1. Erase non-data-ink, within reason.\\
2. Erase redundant data-ink, within reason.\\
3. always revise and edit.

Examples:\\
\textbf{1.Erase non-data-ink and redundant data-ink.}\\
\includegraphics{images/Tufte_figure1.png} (source:\citep{Tufte_2001})

\textbf{2. Erase non-data-ink and redundant data-ink.}\\
\includegraphics{images/Tufte_figure2.png} (source: \citep{appli_2017})

\includegraphics{images/Tufte_figure3.png} (source: \citep{appli_2017})

\textbf{3. always revise and edit.}\\
\includegraphics{images/Tufte_figure4.png} (source:\citep{Tufte_2001})

\textbf{Conclusion}\\
1. The graphs will be better for more information per unit of space an d
per unit of ink is displayed.

\begin{enumerate}
\def\labelenumi{\arabic{enumi}.}
\setcounter{enumi}{1}
\item
  Graphics are almost always going to improve as they go through editing
  ,revision, and testing against differernt design options.
\item
  Try to figure out whehter the audience looking at the new designs be
  confused? Nothing is lost to those puzzled by the frame of dashes,and
  something is gained by those who do understand. We can also assume
  that if you understand the statistical graphics, most other readers
  will, too because it is a frequent mistake in thinking about
  statistical graphics to underestimate the audience.
\item
  Some of the new designs may appear odd, but this is probably because
  we have not seen them before.
\end{enumerate}

\textbf{Principle 2: Mobilize every graphical element, perhaps several
times over, to show the data.}

The danger of multifunctioning elements is that they tend to generate
graphical puzzles, with encodings that can only be broken by their
inventor.Thus design techniques for enhancing graphical clarity in the
face of complexity must be developed along with multifunctioning
elements.

In other words, we should try to make all present graphical elements
data encoding elements.We must make every graphical element effective.

Example:\\
\includegraphics{images/Tufte_figure6.png} (source:\citep{Tufte_2001})

\textbf{Principle 3: maximize data density and the size of the data
matrix, within reason.}

High performation graphics should be design with special care. As volume
of data increases, data measures must shrink (smaller dots ofr
scatters,thinner lines for busy time-series). \citep{best-practice}

Data Density = \# entries in data matrix /area of data graphic

\textbf{• Conclusion: composition of design principles}

\textbf{Principle 1: Escape flatland -- small multiples, parallel
sequencing.}

Data is multivariate. Doesn't necessarily mean 3D projection. How can we
enhance mulitvariate data on inherently 2D surfaces?

\begin{enumerate}
\def\labelenumi{\arabic{enumi}.}
\item
  Example for small multiples.\\
  \includegraphics{images/Tufte_figure8.png} (source:\citep{Tufte_2001})
\item
  Example for parallel sequencing\\
  \includegraphics{images/Tufte_figure7.png} (source:\citep{Tufte_2001})
\end{enumerate}

\textbf{Principle 2: Macro/Micro-Provide the user with both views
(overview and detail).}

Carefully designed view can show a macro structure (overview) as well as
micro structure (detail) in one space.

Example:\\
\includegraphics{images/Tufte_figure9.png} (source:\citep{Tufte_2001})

\textbf{Principle 3: Utilize Layering \& Separation.}

Supported by Gestalt laws (The principles of grouping):\\
1. Grouping with colors\\
2. Using Color to separate\\
3. 1+1 = 3 (clutter)

Example:\\
\includegraphics{images/Tufte_figure10.png} (source:\citep{Tufte_2001})

\textbf{Principle 4: Utilize narratives of space and time.}

Tell a story of position and chronology through visual elements.

Example:\\
\includegraphics{images/Tufte_figure11.png} (source:
\citep{narratives_2017})\\
\includegraphics{images/Tufte_figure12.png} (source:
\citep{narratives_2017})

Ref:\url{https://www.tableau.com/about/blog/2016/5/5-tips-effective-visual-data-communication-54174}

We want visualizations to speak about the data. This article is about
some tips that can help visualizations to speak:

Keep it simple:Keep charts simple and easy to interpret. Instead of
overloading peoples brain with lots of information, keep only the
necessary things in the chart and help the audience understand quickly
what's going on.

Pretty doesn't mean effective: There is a misconception that
aesthetically pleasing visualization is more effective.To draw
attention, sometimes we want them to be pretty and eye-catching. But if
it fails to communicate that data properly, then you'll lose people's
interest as quickly as you gained it.

Color for Numerical Scales: Color for numerical scales should be used
with caution.The way you interpret a shade depends on the colors around
it and sometimes it can lead to false conclusions.For data with
geographical fields, it may be tempting to use maps. But maps may not be
very effective for the following reasons:

As mentioned above, color perception can be tricky.

Choosing the right charts type is very critical. For comparison based on
numerical scale bar chart can be more effective than maps.

Leverage Color Associations: When we say strawberries we associate red
color with it. If we can leverage the how people associate different
colors for different things, we will not even need legend to interpret
things. Color can be used to leverage long-term memory very quickly.

Use Bright Colors to Highlight: To attract attention to a certain part
of data, bright colors can be used. Alarm colors draw the eye quickly to
areas that need attention and help get that message across.

Maps: Use of maps can be tricky. Geographical data doesn't imply a map.
Maps can be useful for application where proximity matters they can be
great for applications where proximity matters, but for straight ``what
is higher'' type comparisons, they're not very effective as large
regions will draw attention easier than smaller regions due to more
concentrated color.

5 Second Rule: Research shows that on average modern attention span for
looking at things online is less than 5 seconds. So if you can't grab
attention within 5 minutes, you've likely lost your viewer

Includes clear titles and instructions and tell people succinctly what
the visualization shows and how to interact with it.

These are the best practices of data visualization. Anticipate in
advance what kind of questions the viewers will ask and then focus your
visualization with respect to those questions.

Brain processes stimuli from our environment to process what is
important in 2 ways -- unconscious (System 1 represents uncontrolled
functions such as facial expressions, reactions) and conscious (System 2
-- represents controlled function such as solving math problems). Data
Visualization leverage attributes of System -1 which has can have quick
and correct impact in a most efficient manner. The three best practices
of data visualization are as follows: -

** 1. Design and layout matter \textbf{ The design and layout should
facilitate ease of understanding to convey your message to the viewer. }
2. Avoid Clutter \textbf{ Keep it simple. To implement this always keep
into account the data-ink ratio -- the ratio of ink required to convey
the intended meaning to the total amount of ink used in the table or
chart should be as close to 1 as possible. That means, avoid ink which
do not add any information. } 3. Use color purposely and effectively **
Use of color may be prettier and attractive but can be distractive too.
Thus, color should be used only if it assists in conveying your message.
The above three principles are illustrated with the help of scenarios
and examples which helps to comprehend the topic in more meaningful and
practical way in the article. It also gives various advantages of using
the above principles.And the above best practices could be applied to
all the 3 types of analytics: descriptive, predictive and prescriptive.

\textbf{Reference} Jeffrey D. Camm, Michael J. Fry, Jeffrey Shaffer
(2017) A Practitioner's Guide to Best Practices in Data
Visualization.Interfaces 47(6):473-488.
\url{https://doi.org/10.1287/inte.2017.0916}

\section{Gestalt Principles for Data
Viz}\label{gestalt-principles-for-data-viz}

\citep{principles-fusioncharts}

This is a pretty detailed white paper PDF from FusionCharts that
explores key aspects of effective data visualization in the business
world, from goals to preattentive attributes to applying Gestalt
Principles. Because some of the aspects such as using color and design
effectively have been covered earlier on in the Fundamentals section,
this summary will mainly focus on the Gestalt Principles.

Data is simply a collection of many individual elements (ie,
observations, typically represented as rows in a data table). In data
viz, our goal is usually to group these elements together in a
meaningful way to highlight patterns and anomalies. Described this way,
it makes sense that Gestalt Principles are a good set of guidelines for
data viz, because these principles describe how we assemble different
elements into groups.

\textbf{Gestalt Principles include the following:} 1. Proximity - white
space can be used to group elements together and separate others 2.
Similarity - objects that look similar are instinctively grouped
together in our minds 3. Enclosure - helps distinguish between groups 4.
Symmetry 5. Closure - we tend to complete shapes and paths even if part
of them is missing 6. Continuity - similar to closure 7. Connection -
helps group elements together as well 8. Figure and ground - we
typically notice only one of several main visual aspects of a graph;
what we do notice becomes the figure, and everything else becomes the
``background''. This one is especially interesting because it is not as
obvious as some of the others, but is really important in matching a
data viz design to its purpose.

Examples of each of the above are presented on page 11 of the PDF.

The principles are quite theoretical and in practice, preattentive
attributes such as spatial position, size, and color are means of
applying the principles.

\chapter{Case Studies}\label{case-studies}

10 Best Data Visualization Projects of 2015 \citep{10_BEST} The author
picked top 10 projects for the best data visualization of 2015, for each
pick, the author showed the project plot and also described the reason
why he chose. So after reading this article, I have a basic
understanding of what kind of characters should include in a good
visualization project.

\citep{10_BEST} Let us have a look at some good examples of graphs,
visuals and data products that make a claim and proceed to tell a story
that is wholly contained within the example itself.

\section{\texorpdfstring{1. 15 Cool Information Graphics and Data Viz
from 2016**
\citep{cool_data}}{1. 15 Cool Information Graphics and Data Viz from 2016** {[}@cool\_data{]}}}\label{cool-information-graphics-and-data-viz-from-2016-cool_data}

\textbf{Description and Replication of Great Examples of Data
Visualization} \textbf{Collections of Great Examples of Data
Visualization}

\section{\texorpdfstring{2. 16 Captivating Data Visualization Examples**
\citep{int_viz_capt}}{2. 16 Captivating Data Visualization Examples** {[}@int\_viz\_capt{]}}}\label{captivating-data-visualization-examples-int_viz_capt}

\section{\texorpdfstring{3. 15 Data Visualizations That Explain Trump,
the White Oscars and Other Crazy Current Events**
\citep{int_viz_2}}{3. 15 Data Visualizations That Explain Trump, the White Oscars and Other Crazy Current Events** {[}@int\_viz\_2{]}}}\label{data-visualizations-that-explain-trump-the-white-oscars-and-other-crazy-current-events-int_viz_2}

15 Cool Information Graphics and Data Viz from 2016
reference:\citep{cool_data}

Case studies contain valuable information about development records. The
evaluation and study of case study helps show that the new design is
just as usable as existing techniques, making it suitable for future
development. This chapter contains some very useful case studies. Many
of the case studies below come from the following articles:

Visualization is like art. It speaks where words fail. There are
phenomenas like the Syrian war, the number flights during Thanksgiving
in the USA, the understanding of depths for developing perspective about
the range of the issue, the controversy of `\#OscarsSoWhite', etc. on
which we can write bundles of paragraphs, but they might still have
scope for ambiguity. The links show some intricate visualizations of the
topics like those mentioned above, and speak volumes without requiring
paragraphs to explain what is going on within these visualizations.
According to me, it is really interesting to see that almost anything in
this world can be explained by visualizations. Visualizations are not
just limited to businesses and their analytics. Wars, rescue operations,
etc. can also be visualized to get a clear idea of all the details of
the issues.

\section{\texorpdfstring{4. 10 Best Data Visualization Projects of
2015**
\citep{10_BEST}}{4. 10 Best Data Visualization Projects of 2015** {[}@10\_BEST{]}}}\label{best-data-visualization-projects-of-2015-10_best}

\section{\texorpdfstring{1. 15 Cool Information Graphics and Data Viz
from 2016
\citep{cool_data}}{1. 15 Cool Information Graphics and Data Viz from 2016 {[}@cool\_data{]}}}\label{cool-information-graphics-and-data-viz-from-2016-cool_data-1}

The author chose fifteen of the best infographics and data
visualizations from 2016 and described why they think these are the
best.And the following six examples are from the articles:

\section{5. Connecting the Dots Behind the
Election}\label{connecting-the-dots-behind-the-election}

\citep{campaign_staff} referenced in \citep{cool_data} \#\# 1.1
Connecting the Dots Behind the Election

This article by the New York Times lists several different candidates
and creates compelling visuals that link their campaigns to previous
ones.

Each visual contains several different-sized dots that represent a
specific campaign, administration, or other governmental organization
related to the candidate's current campaign, which are then connected by
arrows.

Hovering over a specific dot highlights the connections between the
groups. The visual is a great way to put what would otherwise be a long
slog through years of information into an easily accessible, easily
viewable format so that voters can figure out where the candidates'
experiences lie.

\begin{figure}
\centering
\includegraphics{images/clinton_campaign.png}
\caption{Clinton 2016 Campaign Staff}
\end{figure}

\section{6. Spies in the Skies}\label{spies-in-the-skies}

\citep{spies_sky} referenced in \citep{cool_data}

Source: \citep{campaign_staff} referenced in \citep{cool_data}

\subsection{1.2 Spies in the Skies}\label{spies-in-the-skies-1}

The map is filled with red and blue lines (representing FBI and DHS
aircraft, respectively) which illustrate the flight paths of the planes.
When planes circle an area more than once, the circles become darker.
The circles change in accordance to day and time, and individual cities
can be typed into a search bar to see the flight patterns over them.

The visualization, rather creatively, almost looks like a hand-drawn
map. While presenting a normally uncomfortable topic, this allows
individuals to check things for themselves, hopefully providing some
peace of mind.

\begin{figure}
\centering
\includegraphics{images/NYCflights.png}
\caption{New York Flight Patterns}
\end{figure}

\section{7. Green Honey}\label{green-honey}

\citep{green_honey} referenced in \citep{cool_data} Source:
\citep{spies_sky} referenced in \citep{cool_data}

\subsection{1.3 Green Honey}\label{green-honey-1}

The visualization spans a webpage. As you scroll down, the text changes,
as do many colored dots that move over the white background. The dots
are used to represent not only each colors' hue, but the numbers that
fall into each category---for example, what colors are the most popular
``base'' colors for English and Chinese.

The continuous flow of this visualization helps really bring it
together, allowing users to scroll through the information at their own
pace, but also creating a seamless, creative work.

\begin{figure}
\centering
\includegraphics{images/colorwords.png}
\caption{}
\end{figure}

\section{8. How People Like You Spend Their
Time}\label{how-people-like-you-spend-their-time}

\citep{spendingtime} referenced in \citep{cool_data} \#\#\# 1.4 How
People Like You Spend Their Time

The visual lists several categories along one side of a graph---such as
``personal care'' and ``work''---with a line illustrating the amount of
time the average person in a certain demographic spends on each subject.
Entering different statistics at the top---such as changing gender or
age---causes the lines to shift to feature that demographic.

The simplicity of this visualization really helps the information get
across and avoids bogging down the statistics. Sometimes, less is more.

\section{6.Is it Better to Rent or
Buy?}\label{is-it-better-to-rent-or-buy}

reference: \citep{rent_or_buy} \includegraphics{images/SpendingTime.png}

\section{9. Is it Better to Rent or
Buy?}\label{is-it-better-to-rent-or-buy-1}

\citep{rent_or_buy} referenced in \citep{cool_data} Source:
\citep{spendingtime} referenced in \citep{cool_data}

\subsection{1.5 Is it Better to Rent or
Buy?}\label{is-it-better-to-rent-or-buy-2}

The calculator includes several sloping charts. Each chart includes a
factor that'll affect how much you'll have to pay, such as the
individual cost of your home and your mortgage rates. A movable scale
along the bottom of each chart allows you to enter different data,
changing the ``cost of rent per month'' on the side. If you can find a
similar house to rent for that much per month or less, it's more cost
effective to just rent the home. This visualization is incredibly
thorough and a useful tool for homeowners of any age and status.

\begin{figure}
\centering
\includegraphics{images/rentcalc.png}
\caption{}
\end{figure}

\section{10. Two Centuries of U.S.
Immigration}\label{two-centuries-of-u.s.-immigration}

\citep{Immigration} referenced in \citep{cool_data}

\subsection{1.6 Two Centuries of U.S.
Immigration}\label{two-centuries-of-u.s.-immigration-1}

The interactive map shows the rate of immigration into the U.S. from
other countries over the last 200 years in 10-year segments. Colored
dots represent 10,000 people coming from the specified country.
Countries then light up when they have one of the highest rates of
migration. What makes this a good visualization is that it is engaging
and easy to read and interpret. The movement of the dots draws the
reader's attention while the brightly lit countries make it easy to pick
out the highest total migrations.

\begin{figure}
\centering
\includegraphics{images/immigration.png}
\caption{US Immigration}
\end{figure}

\section{11. What's really warming the
world?}\label{whats-really-warming-the-world}

\citep{world_warming} referenced in \citep{int_viz_1}

Source: \citep{Immigration} referenced in \citep{cool_data}

\section{8.What's really warming the
world?}\label{whats-really-warming-the-world-1}

reference:\citep{world_warming}

\section{\texorpdfstring{2. 16 Captivating Data Visualization Examples
\citep{int_viz_1}}{2. 16 Captivating Data Visualization Examples {[}@int\_viz\_1{]}}}\label{captivating-data-visualization-examples-int_viz_1}

\subsection{2.1 What's really warming the
world?}\label{whats-really-warming-the-world-2}

In this case study, it first claimed the background story and the
analytical questions clearly. Then it analyzed each different factor
separately using both verbal explanations and dynamic graphics to
compare with the observed temperature movements, and then grouped
related factors into Natural factors category or Human factors category.
After that, it combined all the dynamic graphics into one and made the
results more straightforward in terms of comparisons. In the end, the
authors also provided more detailed explanations with dataset sources to
support the results shown above.

Overall, this case study is straightforward, easy to understand but also
with enough information shown on each graphics.

Source: \citep{world_warming} referenced in \citep{int_viz_1}

\section{12. The Strengths of Animated Data
Visualization}\label{the-strengths-of-animated-data-visualization}

\citep{American_life} referenced in \citep{int_viz_2} \#\# 3. 15 Data
Visualizations That Explain Trump, the White Oscars and Other Crazy
Current Events** \citep{int_viz_2}

Visualization is like art. It speaks where words fail. There are
phenomenas like the Syrian war, the number flights during Thanksgiving
in the USA, the understanding of depths for developing perspective about
the range of the issue, the controversy of `\#OscarsSoWhite', etc. on
which we can write bundles of paragraphs, but they might still have
scope for ambiguity.

The links show some intricate visualizations of the topics like those
mentioned above, and speak volumes without requiring paragraphs to
explain what is going on within these visualizations.

\section{9.The Strengths of Animated Data
Visualization}\label{the-strengths-of-animated-data-visualization-1}

reference:\citep{American_life}

According to me, it is really interesting to see that almost anything in
this world can be explained by visualizations. Visualizations are not
just limited to businesses and their analytics. Wars, rescue operations,
etc. can also be visualized to get a clear idea of all the details of
the issues.

\subsection{3.1 The Strengths of Animated Data
Visualization}\label{the-strengths-of-animated-data-visualization-2}

The page linked above includes a great example of animated data
visualization showing the time people spend on daily activities
throughout the day. The plot is simple and easy to interpret, but it
also includes a good number of variables including time, activity type,
number of people doing each activity, and the order in which activities
are done.

One of the plot's biggest strengths is that by using one dot to
represent each person in the study and using animation, we can actually
drill down to each individual and follow them throughout the day. The
accumulation of dots for each particular activity also gives us an
aggregate-level view of the same data, so we get both an individual and
aggregate insights.

A drawback of the plot is that it is hard for our eyes to keep track of
1000 simultaneously moving dots. The author of the post addresses this
by creating subsequent plots with stationary lines at key times of the
day. This represents people's movements from one activity to another
without overwhelming the reader.

Overall, this is an engaging, informative, and fun animated plot that
has relevance and tells a story.

Source: \citep{American_life} referenced in \citep{int_viz_2}

\section{13. An Aging Nation: Projected Number of Children and Older
Adults}\label{an-aging-nation-projected-number-of-children-and-older-adults}

\citep{aging_nation}

Aging population is always a hot topic in social economics and politics.
I collected several different data visualizations that show the aging
population in the world. They are good examples to learn and apply to
census data.

\begin{figure}
\centering
\includegraphics{images/aging_nation.jpg}
\caption{}
\end{figure}

13.1: This one includes bar chart and line graph to demonstrate the
aging population compared with population of children. The good things
about this visualization: simple to see and compare, color to
differentiate the category, highlight the intersection point.

13.2 From Pyramid to Pillar: A Century of Change, Population of the U.S.
\citep{population_pyramid}

\begin{figure}
\centering
\includegraphics{images/Pyramid.jpg}
\caption{}
\end{figure}

\section{10.An Aging Nation: Projected Number of Children and Older
Adults}\label{an-aging-nation-projected-number-of-children-and-older-adults-1}

reference:\citep{pyramid}

Aging population is always a hot topic in social economics and politics.
I collect several different data visualizations that show aging
population in the world. They are good examples to learn and apply to
census data.

\section{\texorpdfstring{4. 10 Best Data Visualization Projects of 2015
\citep{10_BEST}}{4. 10 Best Data Visualization Projects of 2015 {[}@10\_BEST{]}}}\label{best-data-visualization-projects-of-2015-10_best-1}

The author picked top 10 projects for the best data visualization of
2015, for each pick, the author showed the project plot and also
described the reason why he chose. So after reading this article, I have
a basic understanding of what kind of characters should include in a
good visualization project.

\section{5. An Aging Nation: Projected Number of Children and Older
Adults}\label{an-aging-nation-projected-number-of-children-and-older-adults-2}

Aging population is always a hot topic in social economics and politics.
I collected several different data visualizations that show the aging
population in the world. They are good examples to learn and apply to
census data.

\subsection{5.1 An Aging Nation: Projected Number of Children and Older
Adults}\label{an-aging-nation-projected-number-of-children-and-older-adults-3}

\begin{figure}
\centering
\includegraphics{images/aging_nation.jpg}
\caption{}
\end{figure}

Reference: \citep{aging_nation}

This one includes bar chart and line graph to demonstrate the aging
population compared with population of children. The good things about
this visualization: simple to see and compare, color to differentiate
the category, highlight the intersection point.

9.2 From Pyramid to Pillar: A Century of Change, Population of the U.S.
reference:\citep{pyramid}

\includegraphics{images/Pyramid.jpg} \#\#\# 5.2 From Pyramid to Pillar:
A Century of Change, Population of the U.S.

\begin{figure}
\centering
\includegraphics{images/Pyramid.jpg}
\caption{}
\end{figure}

Reference: \citep{population_pyramid}

This is a \textbf{population pyramid}. ``A \textbf{population pyramid}
is a pair of back-to to histograms for each sex that displays the
distribution of a population in all age groups and in gender''.

It is a good candidate to compare changes in population distributions
(sex, age, year). Also the shape of pyramid is used to interpret a
population. To illustrate, A pyramid with a very wide base and a narrow
top section suggests a population with both high fertility and death
rates. It is a useful tool in the census data.

13.3 Animated pyramid \citep{animated_pyramid} \#\#\# 5.3 Animated
pyramid

\begin{figure}
\centering
\includegraphics{images/3_1.png}
\caption{}
\end{figure}

\begin{figure}
\centering
\includegraphics{images/3_1.png}
\caption{}
\end{figure}

This is an animated and multiple population pyramids. It used to compare
different patterns across countries. One additional benefit for the
interactive population pyramid is that it shows the shape changes year
by year, which is useful for countinous time-series comparison.

Similar projected with R code is provided for references:

\href{https://www.r-bloggers.com/who-is-old-visualizing-the-concept-of-prospective-ageing-with-animated-population-pyramids/}{link}

\section{6. A guide to Who is Fighting Whom in
Syria}\label{a-guide-to-who-is-fighting-whom-in-syria}

\section{14. A guide to Who is Fighting Whom in
Syria}\label{a-guide-to-who-is-fighting-whom-in-syria-1}

Picking up from one of the charts shown in the above mentioned link
\citep{int_viz_1}, the visualization of `A guide to Who is Fighting Whom
in Syria' is one of the most interesting charts in the list. The
visualization and its report can be seen at \citep{syria_chart}

Picking up from one of the charts shown in the above mentioned link
\citep{int_viz_1}, the visualization of `A guide to Who is Fighting Whom
in Syria' is one of the most interesting charts in the list. The
visualization and its report can be seen at

\begin{figure}
\centering
\includegraphics{images/img_syria_summary.PNG}
\caption{Who is Fighting Whom in Syria}
\end{figure}

This visualization makes an extremely complicated topic like the Syrian
War easily understandable. It consists of 3 different emojis in three
different colours, with each (colour+facial expression) combination
showing the relationship between the various groups involved in the
Syrian War. When you click on each of the emoji, a small dialogue box
pops up which explains the relationship between the various countries
and rebel groups involved in the war. This is not only easy to
understand, but it is also pleasing to the eyes.

\begin{figure}
\centering
\includegraphics{images/img_syria_friendly.PNG}
\caption{Green emoji shows `Friendly' relationship}
\end{figure}

\begin{figure}
\centering
\includegraphics{images/img_syria_enemies.PNG}
\caption{Red emoji shows the `Enemies' relationship}
\end{figure}

\begin{figure}
\centering
\includegraphics{images/img_syria_complicated.PNG}
\caption{Yellow emoji shows `Complicated' relationship}
\end{figure}

\section{15. Adding up the White Oscars
Winners}\label{adding-up-the-white-oscars-winners}

\citep{oscars_sowhite_chart} referenced in \citep{int_viz_2}

\section{7. Adding up the White Oscars
Winners}\label{adding-up-the-white-oscars-winners-1}

A visualization of all previous winners of the Best Actor/Actress Oscar
winners can be seen here \citep{oscars_sowhite_chart} in an article by
Bloomberg. The writers of this article developed the attributes of the
future winners of Oscars by taking up the attributes of the past
winners. It is extremely interesting to see how the article shows the
features of the Best Actress, Actor, movies, etc. in a simple and
captivating visual. The visualization is interactive and we can click on
each attribute like `Hair Color', `Eye Color', etc. to see what are the
features of the actors and actresses who are more likely to win the
Oscars.

\begin{figure}
\centering
\includegraphics{images/img_oscars_actors.PNG}
\caption{Best Actor and Best Actress}
\end{figure}

Source: \citep{oscars_sowhite_chart} referenced in \citep{int_viz_2}

Similarly, the visualization gives information about the different
aspects of movies that are more likely to win, like `Length', `Month',
`Budget', etc.

\begin{figure}
\centering
\includegraphics{images/img_oscars_pic.PNG}
\caption{Best Picture}
\end{figure}

\section{16. Young voters, class and turnout: how Britain voted in
2017}\label{young-voters-class-and-turnout-how-britain-voted-in-2017}

\citep{UKvotes2017}

\section{8. Young voters, class and turnout: how Britain voted in
2017}\label{young-voters-class-and-turnout-how-britain-voted-in-2017-1}

The article's goal is to convey the change in party votes in the 2017 UK
general election compared to votes in 2015. The change in party votes
was shown with regards to three demographic factors: age, class, and
ethnicity. For each factor, there are four graphs (one per political
party), each illustrated in their party's standard color. The change in
percent of votes is shown as an arrow where the arrow's shaft is the
length of the difference from 2015 to 2017 while the x-axis is the
demographic factor split into different bins. What makes this a good
visualization is that it is very easy to read and interpret. The
color-coding of the arrows and party name makes it easy to pick out the
different parties and the arrow lengths highlight just how large of a
change happened. For example, in the Age section, it is easy to see the
pattern between the Labour party gaining many voters ages 18 to 44 and
the Conservative party gaining voters ages 45 and up.

\begin{figure}
\centering
\includegraphics{images/Party_Votes_by_Age.png}
\caption{UK Party Votes by Age}
\end{figure}

\section{17. Uber: Crafting Data-Driven
Maps}\label{uber-crafting-data-driven-maps}

\citep{uber_maps} This is a blog about Tableau based data visualization.
The author is Andy Kriebel who is a famous Tableau Zen Master. I would
like to recommend this blog because it is not only practical, but also
full of insights.

My favorite part of this blog is so called ``Makeover Monday'', which
will develop a new visualization based on an original one. For example,
the author re-designed ``The Seasonality of Confirmed Malaria Cases in
Zambia Southern Province'' by pointing out ``what works well'', ``what
could be improved'' and also his goals for the new visualization (ref:
\url{http://www.vizwiz.com/2018/04/malaria.html}) That's how you can
learn all the insight and reason behind a good visualization.

Besides, this blog also includes great tips and showcases for Tableau.

\section{16.Uber: Crafting Data-Driven
Maps}\label{uber-crafting-data-driven-maps-1}

Reference: \citep{UKvotes2017}

\section{9. Uber: Crafting Data-Driven
Maps}\label{uber-crafting-data-driven-maps-2}

Map visualization is very important for companies like Uber that needs
to track metrics using geo space points. In this article, the designer
from Uber talks about the challenges of design such visualization and
their solutions. While a lot of the problems are related to the large
scale of the data, there are some insights on using scatter plots and
hex bins, adding trip lines and making custom tools to help make
decisions. The visualization in this article is beneficial for
developing geo spatial graphics.

\section{18. Linguistic Concepts}\label{linguistic-concepts}

\citep{lingui_data}

Following the idea behind this article, it helped understand the case
study and its importance. \citep{article_case}. The case study follows.
This case study is about the linguistic concepts usage. How the data is
being used and how visual graphics is used to deliver the insight. It
presents an educational tool that integrates computational linguistics
resources for use in non-technical undergraduate language science
courses. By using the tool in conjunction with case studies, it provides
opportunities for students to gain an understanding of linguistic
concepts and analysis through the lens of realistic problems in feasible
ways.

\section{19. Kissmetrics blog: visualization of
metrics}\label{kissmetrics-blog-visualization-of-metrics}

\citep{facebook_organic}

\section{10. Linguistic Concepts}\label{linguistic-concepts-1}

\textbf{Case Study on computational linguistics}

Following the idea behind this article, it helped understand the case
study and its importance. \citep{article_case}. The case study follows.
This case study is about the linguistic concepts usage. How the data is
being used and how visual graphics is used to deliver the insight. It
presents an educational tool that integrates computational linguistics
resources for use in non-technical undergraduate language science
courses. By using the tool in conjunction with case studies, it provides
opportunities for students to gain an understanding of linguistic
concepts and analysis through the lens of realistic problems in feasible
ways.

Reference: \citep{lingui_data}

\section{11. Kissmetrics blog: visualization of
metrics}\label{kissmetrics-blog-visualization-of-metrics-1}

Kissmetrics blog is a place where people talk about analytics, marketing
and testing through narratives and metrics visualization. Metrics are
important in real-life world especially when developing/promoting
products. Visualization of metrics are also essential so that
stakeholders can monitor performance, identify problems and deep dive
into potential issues.

A good example from the Kissmetrics blog is about Facebook's Organic
Reach. One important point in the blog discussed whether the Facebook's
organic reach is decreasing drastically. The general trend shows that
there is a huge decline in Facebook's page organic reach.

The following graphs show that the engagement is actually increasing,
meaning while the quantity of content is decreasing, the quality is
increasing.

\begin{figure}
\centering
\includegraphics{images/average-facebook-reach.png}
\caption{}
\end{figure}

\begin{figure}
\centering
\includegraphics{images/average-facebook-daily-reach.png}
\caption{}
\end{figure}

This resonates with what we have learnt at class in terms of how
different perspectives of interpreting data can lead to different
conclusions.

\section{20. How the Recession Reshaped the Economy, in 255
Charts}\label{how-the-recession-reshaped-the-economy-in-255-charts}

\citep{recession_economy}

\section{12. How the Recession Reshaped the Economy, in 255
Charts}\label{how-the-recession-reshaped-the-economy-in-255-charts-1}

The first large graph contains 255 lines to show how the number of jobs
has changed for every industry in America. Using color to highlight the
lines lets viewers see the specifics for each industry. By hovering over
a line, the detailed information of that industry's job trend will show
up. Keeping this extra data hidden until needed makes it easier for
readers to absorb information from this otherwise huge data
visualization. Below the overall chart on top are subsets categorized by
job sector and sub-industries. Readers can choose the industry or sector
they are interested in and, like in the first graph, view the more
detailed information by hovering over a line.

\section{21. Vizwiz blog: case studies about how to improve your
visualizations}\label{vizwiz-blog-case-studies-about-how-to-improve-your-visualizations}

vizwiz.com

Reference: \citep{recession_economy}

\section{13. Vizwiz blog: case studies about how to improve your
visualizations}\label{vizwiz-blog-case-studies-about-how-to-improve-your-visualizations-1}

This is a blog about Tableau based data visualization. The author is
Andy Kriebel who is a famous Tableau Zen Master. I would like to
recommend this blog because it is not only practical, but also full of
insights.

An intersting case study could be \citep{case_thesis}. It is a thesis
but it has intersting insights about visualization using mobile data.

My favorite part of this blog is so called ``Makeover Monday'', which
will develop a new visualization based on an original one. For example,
the author re-designed ``The Seasonality of Confirmed Malaria Cases in
Zambia Southern Province'' by pointing out ``what works well'', ``what
could be improved'' and also his goals for the new visualization

That's how you can learn all the insight and reason behind a good
visualization. Besides, this blog also includes great tips and showcases
for Tableau.

\section{22. 15 Data Visualizations That Will Blow Your
Mind}\label{data-visualizations-that-will-blow-your-mind}

\citep{15_mindblowing}

Reference: \citep{vizwiz_malaria}

\section{14. 15 Data Visualizations That Will Blow Your
Mind}\label{data-visualizations-that-will-blow-your-mind-1}

``If a picture is worth a thousand words, a data visualization is worth
at least a million. As inspiration for your own work with data, check
out these 15 data visualizations that will wow you. Taken together, this
roundup is an at-a-glance representation of the range of uses data
analysis has, from pop culture to public good.''

As inspiration for your own work with data, check out these 15 data
visualizations that will wow you. Taken together, this roundup is an
at-a-glance representation of the range of uses data analysis has, from
pop culture to public good."

22.1. Every Satellite Orbiting Earth \citep{Satellite}

Reference: \citep{15_mindblowing}

\subsection{14.1. Every Satellite Orbiting
Earth}\label{every-satellite-orbiting-earth}

By David Yanofsky and Tim Fernholz, Published:Nov17,2014

Reference: \citep{Satellite}

This interactive graph, built using a database from the Union of
Concerned Scientists, displays the trajectories of the 1,300 active
satellites orbiting the Earth as you read this. Each satellite is
represented by a circular icon, color-coded by country and sized
according to launch mass.

\subsection{14.2. Simpson's Paradox}\label{simpsons-paradox}

\begin{itemize}
\tightlist
\item
  \url{http://vudlab.com/simpsons/}
\end{itemize}

The Visualizing Urban Data Idealab (VUDlab) out of the University of
California-Berkeley put together this visual look at data that disproves
the claim in a 1973 suit that charged the school with sex
discrimination. Though the graduate schools had accepted 44\% of male
applicants but only 35\% of female applicants, researchers later
uncovered that if the data were properly pooled, there was actually a
small but statistically significant bias in favor of women. That's
called a Simpson's Paradox.

\subsection{14.3. Charles Minard's Visualization of Napoleon's 1812
March}\label{charles-minards-visualization-of-napoleons-1812-march}

\begin{itemize}
\tightlist
\item
  \url{https://www.edwardtufte.com/tufte/minard}
\end{itemize}

This classic lithograph dates back to 1869, displaying the number of men
in Napoleon's 1812 Russian army, their movements, and the temperatures
they encountered along their way. It's been called one of the ``best
statistical drawings ever created.'' The work is an important reminder
that the fundamentals of data visualization lie in a nuanced
understanding of the many dimensions of data. Tools like D3.js and HTML
are no good without a firm grasp of your dataset and sharp communication
skills.

\subsection{14.4. Hans Rosling's 200 Countries, 200 Years, 4
Minutes}\label{hans-roslings-200-countries-200-years-4-minutes}

reference\citep{hans_rosling}

Global health data expert Hans Rosling's famous statistical documentary
The Joy of Stats aired on BBC in 2010, but it's still turning heads. One
segment in particular is pretty mind-blowing. In ``200 Countries, 200
Years, 4 Minutes,'' Rosling uses augmented reality to explore public
health data in 200 countries over 200 years using 120,000 numbers, in
just four minutes.

\subsection{14.5. Music Timeline}\label{music-timeline}

\begin{itemize}
\tightlist
\item
  \url{https://research.google.com/bigpicture/music/}
\end{itemize}

Google's Music Timeline illustrates a variety of music genres waxing and
waning in popularity from 2010 to present day, based on how many Google
Play Music users have an artist or album in their library, and other
data such as album release dates.

22.6. State of the Union 2014 Minute by Minute on Twitter

\subsection{14.6. State of the Union 2014 Minute by Minute on
Twitter}\label{state-of-the-union-2014-minute-by-minute-on-twitter}

\begin{itemize}
\tightlist
\item
  \url{http://twitter.github.io/interactive/sotu2014/\#p1}
\end{itemize}

Twitter's data team assembled an impressive interactive data hub that
depicts how Twitter users across the globe reacted to each paragraph of
President Obama's 2014 State of the Union address. You can slice and
dice the data by topic hashtag (for example, \#budget, \#defense, or
\#education) and state. Pretty powerful.

22.7. An Interactive Visualization of NYC Street Trees \citep{trees}
\#\#\# 14.7. NYC Street Trees

Using data from NYC Open Data, this interactive visualization shows the
variety and quantity of street trees planted across the five New York
City boroughs.

\subsection{14.8. Millennial Generation
Diversity}\label{millennial-generation-diversity}

22.8. Millennial generation is bigger, more diverse than boomers
\citep{age_groups}

CNNMoney's interactive chart showing the size and diversity of the
millennial generation compared to baby boomers was built using U.S.
Census Data. It turns dry numbers into an intriguing story, illustrating
the racial makeup of different age groups from 1913 to present.

22.9. Goldilocks Exoplanets \#\#\# 14.9. Goldilocks Exoplanets

\begin{itemize}
\tightlist
\item
  \url{https://news.nationalgeographic.com/news/2014/04/140417-exoplanet-interactive/}
\end{itemize}

Using data from the Planetary Habitability Laboratory at the University
of Puerto Rico, the interactive graph plots planetary mass, atmospheric
pressure, and temperature to determine what exoplanets might be home, or
have been home at one point, to living beings.

22.10. Washington Wizards' Shooting Stars \#\#\# 14.10 Washington
Wizards' Shooting Stars

\citep{basketball}

This detailed data visualization demonstrates D.C.'s basketball team's
shooting success during the 2013 season. Using stats released by the
NBA, the visualization lets you examine data for each of 15 players. See
how successful each person was at a variety of types of shots from a
range of spots on the court, compared with others in the league.

22.11. U.S. Migration Patterns \#\#\# 14.11 U.S. Migration Patterns

\citep{migration}

The New York Times data team mapped out Americans' moving patterns from
1900 to present, and the results are fascinating to play around with.
You can see where people living in each state were born, and to what
states people move from others.

22.12. Selfie City \#\#\# 14.12 Selfie City

\citep{selfie}

Selfie City, a detailed multi-component visual exploration of 3,200
selfies from five major cities around the world, offers a close look at
the demographics and trends of selfies. The team behind the project
collected and filtered the data using Instagram and Mechanical Turk.
Explore the differences between selfies snapped in, say, New York and
Berlin, as well as those between men and women across the world.

22.13. The American Workday

\subsection{14.13 Global Carbon
Emissions}\label{global-carbon-emissions}

\citep{CO2_emission}

NPR tapped into American Time Use Survey data to ascertain the share of
workers in a wide range of industries who are at work at any given time.
The chart overlays the traditional 9 AM-5 PM standard over the graph for
a reference point, helping you draw interesting conclusions.

22.14. Global Carbon Emissions +
\url{https://www.theguardian.com/environment/ng-interactive/2014/dec/01/carbon-emissions-past-present-and-future-interactive}

This data visualization, based on data from the World Resource
Institute's Climate Analysis Indicators Tool and the Intergovernmental
Panel on Climate Change, shows how national CO₂ emissions have
transformed over the last 150 years and what the future might hold.
Explore emissions by country for a range of different scenarios.

\section{15. Other sources of great
visualization:}\label{other-sources-of-great-visualization}

\section{Other sources of great
visualization:}\label{other-sources-of-great-visualization-1}

\section{23. Tableau: Viz of the Day}\label{tableau-viz-of-the-day}

NPR tapped into American Time Use Survey data to ascertain the share of
workers in a wide range of industries who are at work at any given time.
The chart overlays the traditional 9 AM-5 PM standard over the graph for
a reference point, helping you draw interesting conclusions.

\textbf{Tableau: Viz of the Day}

Tableau has a gallery called Viz of the Day
(\url{https://public.tableau.com/en-us/s/gallery}) that displays great
data visualization examples created by Tableau. It is cool to see how
people are using all kinds of data to create informative yet fun data
visuals. Data being used is also attached so we can try to mimic what
other people did as well.

\textbf{Some examples from Tableau Gallery:}

Describe Artists with Emoji
(\url{https://public.tableau.com/en-us/s/gallery/what-emoji-say-about-music?gallery=featured}).
Using the data from Spotify, the author listed the 10 most distinctive
emoji used in the playlists related to popular artists. The table being
used in this visual is very straight forward to link artist to the
emojis and is very easy to compare among artists. When you hover over
the emoji, further information is presented.

\section{24. Deceptive data graphs
examples}\label{deceptive-data-graphs-examples}

\section{16. Deceptive data graphs
examples}\label{deceptive-data-graphs-examples-1}

Reference: \url{http://www.statisticshowto.com/misleading-graphs/}
(Stephanie, Jan 24, 2014)

Misleading graphs are sometimes deliberately misleading and sometimes
it's just a case of people not understanding the data behind the graph
they create. But some real life misleading graphs go above and beyond
the classic types. Some are intended to mislead, others are intended to
shock. The ``classic'' types of misleading graphs include cases where:

\begin{itemize}
\tightlist
\item
  \textbf{24.1 The Missing Baseline.} \#\#\# 16.1 The Missing Baseline
\end{itemize}

For example, the Vertical scale is too big or too small, or skips
numbers, or doesn't start at zero, like the graph below:

You might be thinking that the graph on the right shows The Times makes
double the sales of The Daily Telegraph. But take a closer look at the
scale and you'll see although The Times does make more sales, it's only
beating the competition by about 10\%.

\begin{itemize}
\tightlist
\item
  \textbf{24.2 The graph isn't labeled properly.}
\end{itemize}

\subsection{16.2 The graph isn't labeled
properly}\label{the-graph-isnt-labeled-properly}

Graghs can have the correct figures, but still can mislead you.

This one used a BIG HEADLINE makes you think that 5.3\% of children get
spinal cord injuries which is a pretty scary statistic for parents. But
the real figure is about .0000003\% (based on 2000 injuries per year out
of a population of around 74,000,000).

And for the figure 1 used in this article: Misleading Graphs: Displaying
a Change in One Variable Using Area or Volume \citep{scaling_issues},
the label for the smaller triangle in this graph says \$26.4 while the
label for the larger triangle says \$114.6. \$114.6 is 4.34 times
\$26.4. It certainly looks to me as if more than 4.34 smaller triangles
will fit in the larger triangle. It is the altitudes of the triangles
that are proportional to the numbers in the labels.

\subsection{16.3 Data is left out}\label{data-is-left-out}

\begin{itemize}
\tightlist
\item
  \textbf{24.3 Data is left out.}
\end{itemize}

Only includes part of the data like the following graph which uses
temperatures of the first half of the year to prove it was rising
dramatically.

For more examples and inspirations on misleading graphs or deceptive
graphs refer the following articles :

\begin{enumerate}
\def\labelenumi{\alph{enumi})}
\tightlist
\item
  Bar charts without zero \& evenly spaced tick marks for uneven
  intervals: \citep{whats_wrong}
\end{enumerate}

\begin{itemize}
\item
  graphs not drawn to scale: \citep{scaling_issues}
\item
  \textbf{24.4 Treating correlation as causation.}
\end{itemize}

\begin{enumerate}
\def\labelenumi{\alph{enumi})}
\setcounter{enumi}{1}
\tightlist
\item
  Graphs not drawn to scale:\citep{scaling_issues}
\end{enumerate}

\subsection{16.4 Treating correlation as
causation}\label{treating-correlation-as-causation}

Even if the labels and data in your graph is correct, it does not mean
that the conclusion is logically correct. A correlation between X and Y
does not automatically indicate that the change in one variable is
caused by the change in the values of the other one, whereas the
causation means that one event is the result of the occurrence of the
other event. From the graph, we should bear in mind that it only
presents the correlation between ice cream sold and murders, rather than
causation.

\begin{figure}
\includegraphics[width=0.7\linewidth]{images/harlin-ice-cream} \caption{A strange correlation between ice cream sales and murders (Source: [@harlin-coorelation])}\label{fig:harlin-ice-cream}
\end{figure}

\section{25. Application of Data
Visualization}\label{application-of-data-visualization}

\citep{outliar}

\section{17. Application of Data
Visualization}\label{application-of-data-visualization-1}

There are ways to use data visualization at every level of an
organization. These applications lets us quickly create insightful
visualizations, in minutes. It allows users to visualize data and
explore the vast domain interactively. Ref: \citep{app1} Some of them
are mentioned below:

\textbf{Data Preprocessing}

Data preprocessing can greatly improve the quality of data mining
results, no matter whether an algorithmic or a visual approach is
considered. There are four different aspects of data preprocessing: Data
cleaning, data integration, data transformation and data reduction.

\section{26. Data Augmentation}\label{data-augmentation}

\citep{ref_pdf_ar}

Computer interfacing is changing everyday, it is important for our
clients to adapt the technology. The language of communicating data in
3D is explored to understand ways to take advantage of all dimensions in
augmented reality and virtual reality to deliver information based on
the user's perspective, interest, and urgency.

Creating a mechanism to become aware of the user's intention by
analyzing the gaze through reactive design, we achieved developing a
complex system for demonstrating massive amount of data and organizing
it in a spatial system. The user could walk through and explore the data
and interact with different data visualizations. Moving through space is
used to provide different levels of detail for specific data through Z
axis. \textbf{Data Augmentation} \citep{outliar}

Computer interfacing is changing everyday, it is important for our
clients to adapt the technology. The language of communicating data in
3D is explored to understand ways to take advantage of all dimensions in
augmented reality and virtual reality to deliver information based on
the user's perspective, interest, and urgency.

Creating a mechanism to become aware of the user's intention by
analyzing the gaze through reactive design, we achieved developing a
complex system for demonstrating massive amount of data and organizing
it in a spatial system. The user could walk through and explore the data
and interact with different data visualizations. Moving through space is
used to provide different levels of detail for specific data through Z
axis.

Reference:\citep{ref_pdf_ar}

Analytical engineer Steluta Iordache states virtual reality is changing
the environment of data analysis. It has long been predicted that
augmented reality (AR) and virtual reality (VR) will, sooner rather than
later, dive head first into the mainstream of public consciousness. Now,
expectations are to meeting reality, and heavy investment from tech
giants such as Facebook, Samsung, and Google, this seems inevitable.
However, placing the headsets and gaming -- the industry most experts
believe AR and VR will most dynamically disrupt -- to one side, these
nascent technologies can be used by corporate organisations, too.

By using proper visualization, it is possible to discover a solution
more easily. By using proper visualisation, it is possible to simplify
understanding of a problem and discover a solution more easily. Using VR
and AR you could build a more efficient visualisation of the data.
Recently we have seen data integrated in the real world and users have
been able to interact with that data, which is not possible with
traditional methods such as plots and charts. We believe AR and VR can
build the presentation of the data and show more information at the same
time, and it can allow the viewer to explore the data by interacting
with it. But when we analyse data it can be difficult to see the big
picture while also having access to the detail. So the question is: how
can AR and VR be used to understand complex data by interacting with it
within a virtual environment? You can find the answer
here\citep{vr_education}

\section{18. Outlier Detection}\label{outlier-detection}

We use data visualization for outliar detection in the dataset.
Different methods for outlier detection in functional data have been
developed during the years. Among them, several rely on different
notions of functional depth , on robust principal components, or on
random projections of infinite-dimensional data into R. Also, some
distributional approaches have been considered (Gervini, 2009). In
functional data analysis, we observe curves defined over a given real
interval and shape outliers may be defined as those curves that exhibit
a different shape from the rest of the sample. Whereas magnitude
outliers, that is, curves that lie outside the range of the majority of
the data, are in general easy to identify, shape outliers are often
masked among the rest of the curves and thus difficult to detect.
Ref:\citep{outliar}. Outlier treatment is important because, it can
drastically bias/change the fit estimates and predictions.

\section{27. Outlier Detection}\label{outlier-detection-1}

A simple example is mentioned below. Outlier treatment is important
because, it can drastically bias/change the fit estimates and
predictions. Illustration:

\begin{Shaded}
\begin{Highlighting}[]
\CommentTok{# Inject outliers into data.}
\NormalTok{cars1 <-}\StringTok{ }\NormalTok{cars[}\DecValTok{1}\OperatorTok{:}\DecValTok{30}\NormalTok{, ]  }\CommentTok{# original data}
\NormalTok{cars_outliers <-}\StringTok{ }\KeywordTok{data.frame}\NormalTok{(}\DataTypeTok{speed=}\KeywordTok{c}\NormalTok{(}\DecValTok{19}\NormalTok{,}\DecValTok{19}\NormalTok{,}\DecValTok{20}\NormalTok{,}\DecValTok{20}\NormalTok{,}\DecValTok{20}\NormalTok{), }\DataTypeTok{dist=}\KeywordTok{c}\NormalTok{(}\DecValTok{190}\NormalTok{, }\DecValTok{186}\NormalTok{, }\DecValTok{210}\NormalTok{, }\DecValTok{220}\NormalTok{, }\DecValTok{218}\NormalTok{))  }\CommentTok{# introduce outliers.}
\NormalTok{cars2 <-}\StringTok{ }\KeywordTok{rbind}\NormalTok{(cars1, cars_outliers)  }\CommentTok{# data with outliers.}

\CommentTok{# Plot of data with outliers.}
\KeywordTok{par}\NormalTok{(}\DataTypeTok{mfrow=}\KeywordTok{c}\NormalTok{(}\DecValTok{1}\NormalTok{, }\DecValTok{2}\NormalTok{))}
\KeywordTok{plot}\NormalTok{(cars2}\OperatorTok{$}\NormalTok{speed, cars2}\OperatorTok{$}\NormalTok{dist, }\DataTypeTok{xlim=}\KeywordTok{c}\NormalTok{(}\DecValTok{0}\NormalTok{, }\DecValTok{28}\NormalTok{), }\DataTypeTok{ylim=}\KeywordTok{c}\NormalTok{(}\DecValTok{0}\NormalTok{, }\DecValTok{230}\NormalTok{), }\DataTypeTok{main=}\StringTok{"With Outliers"}\NormalTok{, }\DataTypeTok{xlab=}\StringTok{"speed"}\NormalTok{, }\DataTypeTok{ylab=}\StringTok{"dist"}\NormalTok{, }\DataTypeTok{pch=}\StringTok{"*"}\NormalTok{, }\DataTypeTok{col=}\StringTok{"red"}\NormalTok{, }\DataTypeTok{cex=}\DecValTok{2}\NormalTok{)}
\KeywordTok{plot}\NormalTok{(cars2}\OperatorTok{$}\NormalTok{dist,cars2}\OperatorTok{$}\NormalTok{speed)}
\end{Highlighting}
\end{Shaded}

\includegraphics{Data_Viz_Reader_files/figure-latex/unnamed-chunk-1-1.pdf}

\begin{Shaded}
\begin{Highlighting}[]
\CommentTok{# Plot of original data without outliers. Note the change in slope (angle) of best fit line.}
\KeywordTok{plot}\NormalTok{(cars1}\OperatorTok{$}\NormalTok{speed, cars1}\OperatorTok{$}\NormalTok{dist, }\DataTypeTok{xlim=}\KeywordTok{c}\NormalTok{(}\DecValTok{0}\NormalTok{, }\DecValTok{28}\NormalTok{), }\DataTypeTok{ylim=}\KeywordTok{c}\NormalTok{(}\DecValTok{0}\NormalTok{, }\DecValTok{230}\NormalTok{), }\DataTypeTok{main=}\StringTok{"Outliers removed }\CharTok{\textbackslash{}n}\StringTok{ A much better fit!"}\NormalTok{, }\DataTypeTok{xlab=}\StringTok{"speed"}\NormalTok{, }\DataTypeTok{ylab=}\StringTok{"dist"}\NormalTok{, }\DataTypeTok{pch=}\StringTok{"*"}\NormalTok{, }\DataTypeTok{col=}\StringTok{"red"}\NormalTok{, }\DataTypeTok{cex=}\DecValTok{2}\NormalTok{)}
\end{Highlighting}
\end{Shaded}

\includegraphics{Data_Viz_Reader_files/figure-latex/unnamed-chunk-1-2.pdf}
Detection of Outliers is prformed using:

\begin{itemize}
\tightlist
\item
  Univariate approach
\item
  Multivariate approach
\item
  Multivariate Model Approach
\end{itemize}

\section{28. Genetic Network
Reconstruction}\label{genetic-network-reconstruction}

Data visualization techniques are used to reconstruct genetic networks
from genomics data. Reconstructed genetic networks are predicted
interactions among genes of interest and these interactions are inferred
from genomics data,microarray data or DNA sequence. Genomics data are
generally contaminated and high-dimensional. It is important to examine
and clean data carefully to attain meaningful inferences. Thus
visualization tools that are used in the preprocessing of data
associated with genetic network reconstruction are also reviewed and
chosen wisely.

\section{29. Two Awesome Visualists}\label{two-awesome-visualists}

29.1. \textbf{DAVID McCANDLESS} David McCandless is a British
data-journalist and his blog \emph{``Information is Beautiful''}
\citep{info_beautiful} hosts some of the most visually stunning graphs,
charts and maps on a wide range of topics like science, food, dogs and
countries.

\section{19. Genetic Network
Reconstruction}\label{genetic-network-reconstruction-1}

Data visualization techniques are used to reconstruct genetic networks
from genomics data. Reconstructed genetic networks are predicted
interactions among genes of interest and these interactions are inferred
from genomics data,microarray data or DNA sequence. Genomics data are
generally contaminated and high-dimensional. It is important to examine
and clean data carefully to attain meaningful inferences. Thus
visualization tools that are used in the preprocessing of data
associated with genetic network reconstruction are also reviewed and
chosen wisely.

\section{20. Two Awesome Visualists}\label{two-awesome-visualists-1}

\subsection{20.1. DAVID McCANDLESS}\label{david-mccandless}

David McCandless is a British data-journalist and his blog ``Information
is Beautiful'' \citep{info_beautiful} hosts some of the most visually
stunning graphs, charts and maps on a wide range of topics like science,
food, dogs and countries. A chart on this blog,''International Number
Ones: Because every country is good at something (according to data)''
is an interesting and captivating work that shows which country is No.1
in what. \citep{country_chart}

Some of the interesting findings are as follows:

\textbf{`Country' : `No.1 in' } Canada : Doughnuts USA : Spam Emails
India : Bananas Norway : Pizza Eaters Togo : Unhappiness Colombia :
Happiness

The visualizations on this website are updated and revised whenever new
data is available. The original version of the above mentioned graph can
be seen here: \citep{country_original}

29.2. \textbf{HANS ROSLING} Hans Rosling took his interest in Global
Health and developed stunning visualizations about it using statistical
methods and data from the UN. He was a noted TED speaker and one of his
most interesting TED talks is \emph{``Asia's Rise: How and When''}
\citep{hans}. In this, Hans shows trends of the Western countries vs
Developing countries like India and China and makes predictions using
stunning visualizations like the Bubble chart. In this video, he also
predicts the exact date on which India and China will move ahead of USA
as strong economic forces.

Hans was the co-founder and developer of the foundation
``Gapminder''\citep{gapminder} which develops tools to help the people
make sense of global data. One of the most important goals of Gapminder
foundation is to end ignorance in the world by developing fact-based
visualizations to show how the world really is.

\section{30. Using Shapes as Filters in Tableau When Your Fields Are
Measures}\label{using-shapes-as-filters-in-tableau-when-your-fields-are-measures}

Reference: \citep{interworks}

\subsection{20.2. HANS ROSLING}\label{hans-rosling}

Hans Rosling took his interest in Global Health and developed stunning
visualizations about it using statistical methods and data from the UN.
He was a noted TED speaker and one of his most interesting TED talks is
``Asia's Rise: How and When'' \citep{hans}. In this, Hans shows trends
of the Western countries vs Developing countries like India and China
and makes predictions using stunning visualizations like the Bubble
chart. In this video, he also predicts the exact date on which India and
China will move ahead of USA as strong economic forces.

Hans was the co-founder and developer of the foundation
``Gapminder''\citep{gapminder} which develops tools to help the people
make sense of global data. One of the most important goals of Gapminder
foundation is to end ignorance in the world by developing fact-based
visualizations to show how the world really is.

\section{21. Using Shapes as Filters in Tableau When Your Fields Are
Measures}\label{using-shapes-as-filters-in-tableau-when-your-fields-are-measures-1}

Reference: \citep{interworks}

I found this article quite useful for my individual project. This
article introduces the methodologies on how to use shapes as filters in
Tableau when your fields Are Measures. Basically, it teaches you how to
load custom shape as action filters and use them for showing different
graphs with those filters which can make your visualization more
interesting and interactive. You can also download the tableau file for
practice. This article is very useful to analyze and redesign the
different graphs presented in the article ``America's unique gun
violence problem, explained in 17 maps and charts''.
Ref:\citep{gunviolence}. This article introduces the methodologies on
how to use shapes as filters in Tableau when the fields are Measures.
Basically, it teaches us how to load custom shape as action filters and
use them for showing different graphs with those filters which can make
the visualization more interesting and interactive. You can also
download the tableau file for practice.

Case studies document the development record of a project.They provide
the user with an insight into what occurred and relevant details of the
process. A person can gain valuable knowledge that can be reused in
their own projects and allow their own system to be better simply by
learning from what others have done.

This article explains how data visualization can enhance awareness of
the data available and its importance in business decisions. The Author
explains a situation where poor data visualization led to bad decisions
and the impact that these decisions had.

\section{22. Visualization of big data security: a case study on the
KDD99 cup data
set}\label{visualization-of-big-data-security-a-case-study-on-the-kdd99-cup-data-set}

This paper utilized visualization algorithm together with big data
analysis in order to gain better insights into the KDD99 data set:

\textbf{Abstract}

Cyber security has been thrust into the limelight in the modern
technological era because of an array of attacks often bypassing
untrained intrusion detection systems (IDSs). Therefore, greater
attention has been directed on being able deciphering better methods for
identifying attack types to train IDSs more effectively. Keycyber-attack
insights exist in big data; however, an efficient approach is required
to determine strong attack types to train IDSs to become more effective
in key areas. Despite the rising growth in IDS research, there is a lack
of studies involving big data visualization, which is key. The KDD99
data set has served as a strong benchmark since 1999; therefore, we
utilized this data set in our experiment. In this study, we utilized
hash algorithm, a weight table, and sampling method to deal with the
inherent problems caused by analyzing big data; volume, variety, and
velocity. By utilizing a visualization algorithm, we were able to gain
insights into the KDD99 data set with a clear identification of
``normal'' clusters and described distinct clusters of effective
attacks.

To read the full paper, please follow the reference link: \citep{KDD99}

\section{28. Britain's diet in data}\label{britains-diet-in-data}

This is a very good example about how to present a large amount of
comprehensive data - distributed across different categories and
measured in different metrics - in a simple yet effective manner, while
still making it interesting to look at. The data product attempts to
show how the average Briton's diet has changed over the last 4 decades
for the better \citep{britain_diet_2016}. It does this by displaying
simple trend lines that show that more harmful and rich foods are being
consumed less and the healthier and leaner foods are being consumed
more. It further breaks down every major food categories into tens of
its constituent products, and in both the overview and deep-dive
versions, provides further levers to toggle and change to massage more
meaning out of the data. It also shows how the contribution of different
foods to the typical diet has changed over the years. Here, we can
toggle the year to see exactly how much of each food was consumed, again
with another deep-dive into the constituents of every major food group.

\includegraphics{images/britain-diet-data-trends.PNG} Source:
\citep{britain-diet-data-trends} referenced in \citep{britain_diet_2016}
\includegraphics{images/britain-diet-data-typical_diet.png} Source:
\citep{britain-diet-data-typical_diet} referenced in
\citep{britain_diet_2016}

Such a visualization is ideal for the layman wanting to walk away with a
basic but accurate understanding of the dietary changes, but also
provides plenty for the more discerning viewer who might have more time
and inclination to dissect and parse through the graphs. It is very
difficult to use the same visual/data product to cater to both types of
viewers in such a satisfactory capacity, which is what makes this
particular data product so alluring and effective. It satisfies the
principles of graphical excellence as stated by Edward Tufte
\citep{visual_display} - \textgreater{} Graphical excellence is that
which gives to the viewer the greatest number of ideas in the shortest
time with the least ink in the smallest space.

\section{Visualization of big data security: a case study on the KDD99
cup data
set}\label{visualization-of-big-data-security-a-case-study-on-the-kdd99-cup-data-set-1}

This paper utilized visualization algorithm together with big data
analysis in order to gain better insights into the KDD99 data set:

\textbf{Abstract}\\
Cyber security has been thrust into the limelight in the modern
technological era because of an array of attacks often bypassing
untrained intrusion detection systems (IDSs). Therefore, greater
attention has been directed on being able deciphering better methods for
identifying attack types to train IDSs more effectively. Keycyber-attack
insights exist in big data; however, an efficient approach is required
to determine strong attack types to train IDSs to become more effective
in key areas. Despite the rising growth in IDS research, there is a lack
of studies involving big data visualization, which is key. The KDD99
data set has served as a strong benchmark since 1999; therefore, we
utilized this data set in our experiment. In this study, we utilized
hash algorithm, a weight table, and sampling method to deal with the
inherent problems caused by analyzing big data; volume, variety, and
velocity. By utilizing a visualization algorithm, we were able to gain
insights into the KDD99 data set with a clear identification of
``normal'' clusters and described distinct clusters of effective
attacks.

To read the full paper, please follow the reference link: \citep{KDD99}

\chapter{Patterns}\label{patterns}

\begin{itemize}
\tightlist
\item
  Reusable solutions to everyday data visualization questions
\item
  Applied by multiple members of the course
\end{itemize}

\section{1. Why pie chart is bad: a comparison with bar
chart}\label{why-pie-chart-is-bad-a-comparison-with-bar-chart}

Using pie chart is usually considered as a bad idea when it comes to
data visualization. But why? Here, we explore some cons of using pie
chart to convey information and compare its effectiveness to bar chart
\citep{hickey-pie-worst} \citep{henry-defense-pie} \citep{quach-penny}.

\begin{enumerate}
\def\labelenumi{\arabic{enumi}.}
\tightlist
\item
  Some information may look nearly identical in pie chart. But if the
  data is presented with bar charts, the story is different. See figure
  \ref{fig:hickey-before} and \ref{fig:hickey-after} for examples.
\end{enumerate}

\includegraphics{images/hickey-before.jpg} Source:
\citep{hickey-pie-worst}

\includegraphics{images/hickey-after.jpg} Source:
\citep{hickey-pie-worst}

\begin{enumerate}
\def\labelenumi{\arabic{enumi}.}
\setcounter{enumi}{1}
\tightlist
\item
  It is difficult to compare the slices of a circle to figure out the
  distinctions in size between each pie slice, especially when there are
  a lot of categories. ** See figure \ref{fig:hickey-breakdown} for
  example.
\end{enumerate}

\begin{figure}
\centering
\includegraphics{images/hickey-breakdown.jpg}
\caption{}
\end{figure}

(Source: \citep{hickey-pie-worst})

\begin{enumerate}
\def\labelenumi{\arabic{enumi}.}
\setcounter{enumi}{2}
\tightlist
\item
  Pie chart is easy to be manipulated (e.g.~using a 3D pie chart). See
  figure \ref{fig:hickey-3D} for example.
\end{enumerate}

\includegraphics{images/hickey-3D.jpg} Source: \citep{hickey-pie-worst}

\begin{enumerate}
\def\labelenumi{\arabic{enumi}.}
\setcounter{enumi}{3}
\tightlist
\item
  Pie chart may be useful when comparing 2 different categories with
  different amounts of information. Specifically, it does a better job
  to distinguish two parts with a 25:75 split or one that is not 50:50
  as people are sensitive to a right angle or a dividing line that is
  not straight. But this could be simply done by showing two numbers!
  See figure \ref{fig:henry-quarter} and \ref{fig:henry-half} for
  examples.
\end{enumerate}

\includegraphics{images/henry-quarter.png} (Source:
\citep{henry-defense-pie})

\includegraphics{images/henry-half.png} (Source:
\citep{henry-defense-pie})

\section{2. Chose the right baseline in data
visualization}\label{chose-the-right-baseline-in-data-visualization}

Baseline is very important to data visualization. If baseline is
different, the meanning will change a lot. Now here is a case study to
show the importance of baseline and how to use it in different ways.

Here I use the same method for a new dataset to .

\begin{Shaded}
\begin{Highlighting}[]
\CommentTok{# Create the data.}
\NormalTok{a <-}\KeywordTok{rep}\NormalTok{(}\KeywordTok{c}\NormalTok{(}\DecValTok{2010}\NormalTok{,}\DecValTok{2011}\NormalTok{,}\DecValTok{2012}\NormalTok{,}\DecValTok{2013}\NormalTok{,}\DecValTok{2014}\NormalTok{,}\DecValTok{2015}\NormalTok{),}\DataTypeTok{each =} \DecValTok{4}\NormalTok{)}
\NormalTok{b <-}\StringTok{ }\KeywordTok{seq}\NormalTok{(}\DecValTok{1}\OperatorTok{:}\DecValTok{24}\NormalTok{)}
\NormalTok{c <-}\StringTok{ }\KeywordTok{c}\NormalTok{(}\FloatTok{64.9}\NormalTok{,}\FloatTok{65.33}\NormalTok{,}\FloatTok{71.67}\NormalTok{,}\FloatTok{79.17}\NormalTok{,}\FloatTok{68.78}\NormalTok{,}\FloatTok{69.83}\NormalTok{,}\FloatTok{78.61}\NormalTok{,}\FloatTok{92.68}\NormalTok{,}\FloatTok{89.28}\NormalTok{,}\FloatTok{90.43}\NormalTok{,}\FloatTok{97.96}\NormalTok{,}\FloatTok{106.96}\NormalTok{,}\FloatTok{100.66}\NormalTok{,}\FloatTok{107.53}\NormalTok{,}\FloatTok{117.06}\NormalTok{,}\FloatTok{119.21}\NormalTok{,}\FloatTok{110.05}\NormalTok{,}\FloatTok{97.42}\NormalTok{,}\FloatTok{93.62}\NormalTok{,}\FloatTok{97.99}\NormalTok{,}\DecValTok{80}\NormalTok{,}\FloatTok{88.74}\NormalTok{,}\FloatTok{102.06}\NormalTok{,}\DecValTok{83}\NormalTok{)}
\NormalTok{data <-}\StringTok{ }\KeywordTok{as.data.frame}\NormalTok{(}\KeywordTok{cbind}\NormalTok{(a,b,c))}
\KeywordTok{colnames}\NormalTok{(data) <-}\KeywordTok{c}\NormalTok{(}\StringTok{"year"}\NormalTok{,}\StringTok{"quater"}\NormalTok{,}\StringTok{"sales"}\NormalTok{)}
\end{Highlighting}
\end{Shaded}

\begin{enumerate}
\def\labelenumi{\arabic{enumi}.}
\tightlist
\item
  Regular quaterly sales. We can see sales decreased a lot around 2014.
  \textbf{The baseline here is historical sales.}
\end{enumerate}

\begin{Shaded}
\begin{Highlighting}[]
\CommentTok{# Regular time series for sales}
\KeywordTok{par}\NormalTok{(}\DataTypeTok{cex.axis=}\FloatTok{0.7}\NormalTok{)}
\NormalTok{data.ts <-}\StringTok{ }\KeywordTok{ts}\NormalTok{(data}\OperatorTok{$}\NormalTok{sales, }\DataTypeTok{start=}\KeywordTok{c}\NormalTok{(}\DecValTok{2010}\NormalTok{, }\DecValTok{1}\NormalTok{), }\DataTypeTok{frequency=}\DecValTok{4}\NormalTok{)}
\KeywordTok{plot}\NormalTok{(data.ts, }\DataTypeTok{xlab=}\StringTok{""}\NormalTok{, }\DataTypeTok{ylab=}\StringTok{""}\NormalTok{, }\DataTypeTok{main=}\StringTok{"sales per quater"}\NormalTok{, }\DataTypeTok{las=}\DecValTok{1}\NormalTok{, }\DataTypeTok{bty=}\StringTok{"n"}\NormalTok{)}
\end{Highlighting}
\end{Shaded}

\includegraphics{Data_Viz_Reader_files/figure-latex/unnamed-chunk-3-1.pdf}

\begin{enumerate}
\def\labelenumi{\arabic{enumi}.}
\setcounter{enumi}{1}
\tightlist
\item
  Quaterly and yearly change sales. \textbf{The baseline here is zero
  and look at the percentage changes.}
\end{enumerate}

\begin{Shaded}
\begin{Highlighting}[]
\CommentTok{# Quaterly change}
\NormalTok{curr <-}\StringTok{ }\KeywordTok{as.numeric}\NormalTok{(data}\OperatorTok{$}\NormalTok{sales[}\OperatorTok{-}\DecValTok{1}\NormalTok{])}
\NormalTok{prev <-}\StringTok{ }\KeywordTok{as.numeric}\NormalTok{(data}\OperatorTok{$}\NormalTok{sales[}\DecValTok{1}\OperatorTok{:}\NormalTok{(}\KeywordTok{length}\NormalTok{(data}\OperatorTok{$}\NormalTok{sales)}\OperatorTok{-}\DecValTok{1}\NormalTok{)])}
\NormalTok{quaChange <-}\StringTok{ }\DecValTok{100} \OperatorTok{*}\StringTok{ }\KeywordTok{round}\NormalTok{( (curr}\OperatorTok{-}\NormalTok{prev) }\OperatorTok{/}\StringTok{ }\NormalTok{prev, }\DecValTok{2}\NormalTok{ )}
\NormalTok{barCols <-}\StringTok{ }\KeywordTok{sapply}\NormalTok{(quaChange, }
    \ControlFlowTok{function}\NormalTok{(x) \{ }
        \ControlFlowTok{if}\NormalTok{ (x }\OperatorTok{<}\StringTok{ }\DecValTok{0}\NormalTok{) \{}
            \KeywordTok{return}\NormalTok{(}\StringTok{"#2cbd25"}\NormalTok{)}
\NormalTok{        \} }\ControlFlowTok{else}\NormalTok{ \{}
            \KeywordTok{return}\NormalTok{(}\StringTok{"gray"}\NormalTok{)}
\NormalTok{        \}}
\NormalTok{    \})}

\KeywordTok{barplot}\NormalTok{(quaChange, }\DataTypeTok{border=}\OtherTok{NA}\NormalTok{, }\DataTypeTok{space=}\DecValTok{0}\NormalTok{, }\DataTypeTok{las=}\DecValTok{1}\NormalTok{, }\DataTypeTok{col=}\NormalTok{barCols, }\DataTypeTok{main=}\StringTok{"% change, quaterly"}\NormalTok{)}
\end{Highlighting}
\end{Shaded}

\includegraphics{Data_Viz_Reader_files/figure-latex/unnamed-chunk-4-1.pdf}

\begin{Shaded}
\begin{Highlighting}[]
\CommentTok{# Year-over-year change}
\NormalTok{curr <-}\StringTok{ }\KeywordTok{as.numeric}\NormalTok{(data}\OperatorTok{$}\NormalTok{sales[}\OperatorTok{-}\NormalTok{(}\DecValTok{1}\OperatorTok{:}\DecValTok{4}\NormalTok{)])}
\NormalTok{prev <-}\StringTok{ }\KeywordTok{as.numeric}\NormalTok{(data}\OperatorTok{$}\NormalTok{sales[}\DecValTok{1}\OperatorTok{:}\NormalTok{(}\KeywordTok{length}\NormalTok{(data}\OperatorTok{$}\NormalTok{sales)}\OperatorTok{-}\DecValTok{4}\NormalTok{)])}
\NormalTok{annChange <-}\StringTok{ }\DecValTok{100} \OperatorTok{*}\StringTok{ }\KeywordTok{round}\NormalTok{( (curr}\OperatorTok{-}\NormalTok{prev) }\OperatorTok{/}\StringTok{ }\NormalTok{prev, }\DecValTok{2}\NormalTok{ )}
\NormalTok{barCols <-}\StringTok{ }\KeywordTok{sapply}\NormalTok{(annChange, }
    \ControlFlowTok{function}\NormalTok{(x) \{ }
        \ControlFlowTok{if}\NormalTok{ (x }\OperatorTok{<}\StringTok{ }\DecValTok{0}\NormalTok{) \{}
            \KeywordTok{return}\NormalTok{(}\StringTok{"#2cbd25"}\NormalTok{)}
\NormalTok{        \} }\ControlFlowTok{else}\NormalTok{ \{}
            \KeywordTok{return}\NormalTok{(}\StringTok{"gray"}\NormalTok{)}
\NormalTok{        \}}
\NormalTok{    \})}

\KeywordTok{barplot}\NormalTok{(annChange, }\DataTypeTok{border=}\OtherTok{NA}\NormalTok{, }\DataTypeTok{space=}\DecValTok{0}\NormalTok{, }\DataTypeTok{las=}\DecValTok{1}\NormalTok{, }\DataTypeTok{col=}\NormalTok{barCols, }\DataTypeTok{main=}\StringTok{"% change, annual"}\NormalTok{)}
\end{Highlighting}
\end{Shaded}

\includegraphics{Data_Viz_Reader_files/figure-latex/unnamed-chunk-5-1.pdf}
From this plot, it is very clear that the magnitude of drops in sales
for some quaters.

\begin{enumerate}
\def\labelenumi{\arabic{enumi}.}
\setcounter{enumi}{2}
\tightlist
\item
  The sales difference compare to now. \textbf{The baseline here is the
  current sales.}
\end{enumerate}

\begin{Shaded}
\begin{Highlighting}[]
\CommentTok{# Relative to current 2015}
\NormalTok{curr <-}\StringTok{ }\KeywordTok{as.numeric}\NormalTok{(data}\OperatorTok{$}\NormalTok{sales[}\KeywordTok{length}\NormalTok{(data}\OperatorTok{$}\NormalTok{sales)])}
\NormalTok{salesDiff <-}\StringTok{ }\KeywordTok{as.numeric}\NormalTok{(data}\OperatorTok{$}\NormalTok{sales) }\OperatorTok{-}\StringTok{ }\NormalTok{curr}
\NormalTok{barCols.diff <-}\StringTok{ }\KeywordTok{sapply}\NormalTok{(salesDiff,}
    \ControlFlowTok{function}\NormalTok{(x) \{}
        \ControlFlowTok{if}\NormalTok{ (x }\OperatorTok{<}\StringTok{ }\DecValTok{0}\NormalTok{) \{}
            \KeywordTok{return}\NormalTok{(}\StringTok{"gray"}\NormalTok{)}
\NormalTok{        \} }\ControlFlowTok{else}\NormalTok{ \{}
            \KeywordTok{return}\NormalTok{(}\StringTok{"black"}\NormalTok{)}
\NormalTok{        \}}
\NormalTok{    \}}
\NormalTok{)}
\KeywordTok{barplot}\NormalTok{(salesDiff, }\DataTypeTok{border=}\OtherTok{NA}\NormalTok{, }\DataTypeTok{space=}\DecValTok{0}\NormalTok{, }\DataTypeTok{las=}\DecValTok{1}\NormalTok{, }\DataTypeTok{col=}\NormalTok{barCols.diff, }\DataTypeTok{main=}\StringTok{"Sales difference from last quater 2015"}\NormalTok{)}
\end{Highlighting}
\end{Shaded}

\includegraphics{Data_Viz_Reader_files/figure-latex/unnamed-chunk-6-1.pdf}

\begin{enumerate}
\def\labelenumi{\arabic{enumi}.}
\setcounter{enumi}{3}
\tightlist
\item
  Sales difference compared to the first quater. ** The baseline here is
  the first quater sales.**
\end{enumerate}

\begin{Shaded}
\begin{Highlighting}[]
\CommentTok{# Relative to first quater}
\NormalTok{ori <-}\StringTok{ }\KeywordTok{as.numeric}\NormalTok{(data}\OperatorTok{$}\NormalTok{sales[}\DecValTok{1}\NormalTok{])}
\NormalTok{salesDiff <-}\StringTok{ }\KeywordTok{as.numeric}\NormalTok{(data}\OperatorTok{$}\NormalTok{sales) }\OperatorTok{-}\StringTok{ }\NormalTok{ori}
\NormalTok{barCols.diff <-}\StringTok{ }\KeywordTok{sapply}\NormalTok{(salesDiff,}
    \ControlFlowTok{function}\NormalTok{(x) \{}
        \ControlFlowTok{if}\NormalTok{ (x }\OperatorTok{<}\StringTok{ }\DecValTok{0}\NormalTok{) \{}
            \KeywordTok{return}\NormalTok{(}\StringTok{"gray"}\NormalTok{)}
\NormalTok{        \} }\ControlFlowTok{else}\NormalTok{ \{}
            \KeywordTok{return}\NormalTok{(}\StringTok{"black"}\NormalTok{)}
\NormalTok{        \}}
\NormalTok{    \}}
\NormalTok{)}
\KeywordTok{barplot}\NormalTok{(salesDiff, }\DataTypeTok{border=}\OtherTok{NA}\NormalTok{, }\DataTypeTok{space=}\DecValTok{0}\NormalTok{, }\DataTypeTok{las=}\DecValTok{1}\NormalTok{, }\DataTypeTok{col=}\NormalTok{barCols.diff, }\DataTypeTok{main=}\StringTok{"Sales difference from first quater 2010"}\NormalTok{)}
\end{Highlighting}
\end{Shaded}

\includegraphics{Data_Viz_Reader_files/figure-latex/unnamed-chunk-7-1.pdf}

\begin{enumerate}
\def\labelenumi{\arabic{enumi}.}
\setcounter{enumi}{4}
\tightlist
\item
  The difference between quater sales and mean. ** The baseline is mean
  now.**
\end{enumerate}

\begin{Shaded}
\begin{Highlighting}[]
\CommentTok{# difference from the mean}
\NormalTok{mean <-}\StringTok{ }\KeywordTok{mean}\NormalTok{(}\KeywordTok{as.numeric}\NormalTok{(data}\OperatorTok{$}\NormalTok{sales))}
\NormalTok{salesDiff <-}\StringTok{ }\KeywordTok{as.numeric}\NormalTok{(data}\OperatorTok{$}\NormalTok{sales) }\OperatorTok{-}\StringTok{ }\NormalTok{mean}
\NormalTok{barCols.diff <-}\StringTok{ }\KeywordTok{sapply}\NormalTok{(salesDiff,}
    \ControlFlowTok{function}\NormalTok{(x) \{}
        \ControlFlowTok{if}\NormalTok{ (x }\OperatorTok{<}\StringTok{ }\DecValTok{0}\NormalTok{) \{}
            \KeywordTok{return}\NormalTok{(}\StringTok{"gray"}\NormalTok{)}
\NormalTok{        \} }\ControlFlowTok{else}\NormalTok{ \{}
            \KeywordTok{return}\NormalTok{(}\StringTok{"black"}\NormalTok{)}
\NormalTok{        \}}
\NormalTok{    \}}
\NormalTok{)}
\KeywordTok{barplot}\NormalTok{(salesDiff, }\DataTypeTok{border=}\OtherTok{NA}\NormalTok{, }\DataTypeTok{space=}\DecValTok{0}\NormalTok{, }\DataTypeTok{las=}\DecValTok{1}\NormalTok{, }\DataTypeTok{col=}\NormalTok{barCols.diff, }\DataTypeTok{main=}\StringTok{"Sales difference from mean"}\NormalTok{)}
\end{Highlighting}
\end{Shaded}

\includegraphics{Data_Viz_Reader_files/figure-latex/unnamed-chunk-8-1.pdf}

So before we start to plot, we should decide the baseline we want to
use. Different baseline will lead to totally different graphs.

So before we start to plot, we should decide the baseline we want to
use. Different baseline will lead to totally different
graphs.\citep{baseline}

\section{3. Using design patterns to find greater meaning in your
data}\label{using-design-patterns-to-find-greater-meaning-in-your-data}

Visualizations that show comparisons, connections, and conclusions offer
analytical clarity.

Patterns based on function can help you see differences and similarities
more clearly, understand relationships and behaviors more intimately,
and predict future results with a greater level of certainty. When these
patterns are presented as visualizations, they help you 1) see
comparisons, 2) make connections, and 3) draw conclusions from your data
sets. The major functions can be described as:

** 3.1 Comparisons**

As shown in Figure 1, the bar chart with sparkline enables you to review
the data at two different levels: a high-level assessment of the
short-term three-month returns is represented with the bar chart, while
the sparkline (the line chart below the bar) provides the details of the
historical returns. Quickly and concisely, the sparkline shows you the
path that has led up to the most recent returns. You can then assess
that a narrow path provides consistent returns across the years while a
wide path provides varied returns. Side-by-side comparisons of funds
organized into two columns---\% Returns and \% Ahead of
Benchmark---enables peer comparisons and fund-specific benchmark
comparisons. Hence, you can see that not only has Global Large Cap Core
provided positive returns, it has also provided the best and most
consistent returns when compared to the benchmark.

\textbf{3.2 Connections} The string of charts in Figure 2 shows 10-year
to year-to-date (YTD) performance returns, which can be interpreted as
individual charts or a group of category charts.

Similar to sounds waves, the symmetrical area charts grow equidistant
from the source (the zero line) at each time interval to accentuate the
returns even further. Here, the y-axis is shown in percentage. Instead
of using the zero line to indicate positive or negative returns, it uses
color to denote if the category returns are positive (black) or negative
(red). For example, Multi Cap Russell 3000 Growth produced 20\% positive
returns within the one-year time period and is shown with color fill in
both directions from the zero line to purposefully duplicate the large
gains and specifically uses black color fill to indicate the returns are
positive. As evident from the name, the symmetrical chart doubles the
returns to emphasize the amount with color fill.

What else can you derive from organizing the information in a spectrum
of negative to positive returns? Based on this organization, three
groups of categories have resulted in straight losses (red), heavy gains
(black), or a mix of gains and losses across a decade of returns. The
string of charts makes it easier for you to see these three groups of
categories to assess their distribution. Just like sound waves, each
chart is a sound bite that streams the returns for each category with a
``scream'' announcing a huge gain (e.g., Multi-Cap Russel 3000 Growth)
or loss (e.g., Mid Cap Russel Mid Cap Growth). In some cases (e.g.,
Large Cap S\&P 500), the chart quietly announces mixed returns to
adequately demand less attention.

Next, you might wonder how you would have fared if you had invested in
certain funds. You might ask: if I had purchased this fund five years
ago, what would my return be? And what about the YTD returns? Since
market timing is key to investment choices, the following presentation
of hypothetical investments represents a range of results.

\textbf{3.3 Conclusions}

In Figure 3, varied performance results become clear with a layered
approach to show five potential entry points (10-year, 5-year, 3-year,
1-year, YTD) into an investment. For example, the International Large
Cap Core fund provided 27\% YTD returns, which contrast the negative
returns you would have received had you invested in the fund 1, 5, or 10
years ago. Here, conclusions are derived based on known inputs with a
divided review of positive or negative outcomes (shown on the y-axis).

The line weights help to identify each entry point and show the range of
differences between the entry points. Accordingly so, resulting returns
are shown with simplified curves that connect the inputs and outputs. In
this case, the chart has been customized to show an instance in which
the user has opted to see the YTD return values as percentages listed to
the right of each resulting output.\citep{data_meaning}

\section{4. Example Visualizations of Time Series
Data}\label{example-visualizations-of-time-series-data}

Reference: \citep{aya-time-series}

What are some of the most common data visualizations you see in
newspapers, textbooks, and corporate annual reports? Graphs showing a
country's GDP growth trends or charts capturing a company's sales growth
in the last 4 quarters would be high up on the list. Essentially, these
are visualizations that track time series data -- the performance of an
indicator over a period of time -- also known as temporal
visualizations.

Temporal visualizations are one of the simplest, quickest ways to
represent important time series data. There are 7 handy temporal
visualization styles for your time series data.

\begin{enumerate}
\def\labelenumi{\arabic{enumi}.}
\item
  \emph{Line Graph}. A line graph is the simplest way to represent time
  series data. It is intuitive, easy to create, and helps the viewer get
  a quick sense of how something has changed over time.
\item
  \emph{Stacked Area Chart} is an area chart similar to a line chart. In
  an area chart, multiple variables are ``stacked'' on top of each
  other, and the area below each line is colored to represent each
  variable.
\end{enumerate}

Figure \ref{fig:aya-stacked} is a stacked area chart showing time series
data:

\begin{figure}

{\centering \includegraphics[width=0.6\linewidth]{images/aya-stacked} 

}

\caption{Student enrollments in India from 2001-10. (Source: [@aya-time-series])}\label{fig:aya-stacked}
\end{figure}

Stacked area charts are useful to show how both a cumulative total and
individual components of that total changed over time.

The order in which we stack the variables is crucial because there can
sometimes be a difference in the actual plot versus human perception.
The chart plots the value vertically whereas we perceive the value to be
at right angles to the general direction of the chart. For instance, in
the figure below, a bar graph would be a cleaner alternative.

\begin{figure}

{\centering \includegraphics[width=0.3\linewidth]{images/aya-stacked-perception} 

}

\caption{Human perception vs actual value. (Source: [@aya-time-series])}\label{fig:aya-stacked-perception}
\end{figure}

\begin{enumerate}
\def\labelenumi{\arabic{enumi}.}
\setcounter{enumi}{2}
\tightlist
\item
  \emph{Bar Charts} represent data as horizontal or vertical bars. The
  length of each bar is proportional to the value of the variable at
  that point in time. A bar chart is the right choice for you when you
  wish to look at how the variable moved over time or when you wish to
  compare variables versus each other. Grouped or stacked bar charts
  help you combine both these purposes in one chart while keeping your
  visualization simple and intuitive.
\end{enumerate}

For instance, this grouped bar chart in this interactive visualization
of number of deaths by disease type in India not only lets you compare
the deaths due to diarrhea, malaria, and acute respiratory disease
across time, but also lets you compare the number of deaths by these
three diseases in a given year. By switching to the stacked bar chart
view, you get an intuitive sense of the proportion of deaths caused by
each disease.

\begin{figure}

{\centering \includegraphics[width=0.4\linewidth]{images/aya-bar1} \includegraphics[width=0.4\linewidth]{images/aya-bar2} 

}

\caption{Two different bar charts to represent time series data. (Source: [@aya-time-series])}\label{fig:aya-bar}
\end{figure}

To avoid clutter and confusion, make sure to not use more than 3
variables in a stacked or group bar chart. It is also a good practice to
use consistent bold colors and leave appropriate space between two bars
in a bar chart. Also, check out our blog on 5 common mistakes that lead
to bad data visualization to learn why the base axis for your bar charts
should start from zero.

\begin{enumerate}
\def\labelenumi{\arabic{enumi}.}
\setcounter{enumi}{3}
\tightlist
\item
  A \emph{Gantt Chart} is a horizontal bar chart showing work completed
  in a certain period of time with respect to the time allocated for
  that particular task. It is named after the American engineer and
  management consultant Henry Gantt who extensively used this framework
  for project management.
\end{enumerate}

\begin{figure}

{\centering \includegraphics[width=0.5\linewidth]{images/aya-gantt} 

}

\caption{A typical Gantt chart. (Source: [@aya-time-series])}\label{fig:aya-gantt}
\end{figure}

Assume you're planning the logistics for a dance concert. There are lots
of activities to be completed, some of which will take place
simultaneously while some can be done only after another activity has
been completed. For instance, the choreographers, soundtrack, and
dancers need to be finalized before the choreography can begin. However,
the costumes, props, and stage decor can be planned at the same time as
the choreography. With careful preparation, Gantt charts can help you
plan for complex, long-term projects that are likely to undergo several
revisions and have various resource and task dependencies.

Gantt charts are a popular project management tool since they present a
concise snapshot of various tasks spread across various phases of the
project. You can show additional information such as the correlation
between individual tasks, resources used in each task, overlapping
resources, etc., by the use of colors and placement of bars in a Gantt
chart.

\begin{enumerate}
\def\labelenumi{\arabic{enumi}.}
\setcounter{enumi}{4}
\tightlist
\item
  A \emph{Stream Graph} is essentially a stacked area graph, but
  displaced around a central horizontal axis. The stream graph looks
  like flowing liquid, hence the name.
\end{enumerate}

\begin{figure}

{\centering \includegraphics[width=0.6\linewidth]{images/aya-stream} 

}

\caption{A stream graph showing a randomly chosen listener's last.fm music-listening habits over time. (Source: [@aya-time-series])}\label{fig:aya-stream}
\end{figure}

Stream graphs are great to represent and compare time series data for
multiple variables. Stream graphs are, thus, apt for large data sets.
Remember that choice of colors is very important, especially when there
are lots of variables. Variables that do not have significantly high
values might tend to get drowned out in the visualization if the colors
are not chosen well.

\begin{enumerate}
\def\labelenumi{\arabic{enumi}.}
\setcounter{enumi}{5}
\tightlist
\item
  \emph{Heat Map} Geospatial visualizations often use heat maps since
  they quickly help identify ``Hot spots'' or regions of high
  concentrations of a given variable. When adapted to temporal
  visualizations, heat maps can help us explore two levels of time in a
  2D array.
\end{enumerate}

This heat map visualizes birthdays for babies born in the United States
between 1973 and 1999. The vertical axis represents the 31 days in a
month while the horizontal axis represents the 12 months in a year. This
chart quickly helps us identify that a large number of babies were born
in the later half of July, August, and September.

\begin{figure}

{\centering \includegraphics[width=0.4\linewidth]{images/aya-heat-map} 

}

\caption{Heat map can be useful to present 2-D time data. (Source: [@aya-time-series])}\label{fig:aya-heat-map}
\end{figure}

Heat maps are perfect for a two-tiered time frame -- for instance, 7
days of the week spread across 52 weeks in the year, or 24 hours in a
day spread across 30 days of the month, and so on. The limitation,
though, is that only one variable can be visualized in a heat map.
Comparison between two or more variables is very difficult to represent.

\begin{enumerate}
\def\labelenumi{\arabic{enumi}.}
\setcounter{enumi}{6}
\tightlist
\item
  \emph{Polar Area Diagram}. Think beyond the straight line! Sometimes,
  time series data can be cyclical -- a season in a year, time of the
  day, and so on. Polar area diagrams help represent the cyclical nature
  time series data cleanly. A polar diagram looks like a traditional pie
  chart, but the sectors differ from each other not by the size of their
  angles but by how far they extend out from the centre of the circle.
\end{enumerate}

This popular polar area diagram created by Florence Nightingale shows
causes of mortality among British troops in the Crimean War. Each color
in the diagram represents a different cause of death. (Check out the the
text legend for more details.)

\begin{figure}

{\centering \includegraphics[width=0.5\linewidth]{images/aya-polar} 

}

\caption{Source: [@aya-time-series]}\label{fig:aya-polar}
\end{figure}

Polar area diagrams are useful for representing seasonal or cyclical
time series data, such as climate or seasonal crop data. Multiple
variables can be neatly stacked in the various sectors of the pie.

It is crucial to clarify whether the variable is proportional to the
area or radius of the sector. It is a good practice to have the area of
the sectors proportional to the value being represented. In that case,
the radius should be proportional to the square root of the value of the
variable (since area of a circle is proportional to the square of the
radius).

\section{5 Tips to improve Data
Visualization}\label{tips-to-improve-data-visualization}

** 5.1 Comparison \textbf{ Include a zero baseline if possibleAlthough a
line chart does not have to start at a zero baseline, it should be
included if it gives more context for comparison. If relatively small
fluctuations in data are meaningful (e.g., in stock market data), you
may truncate the scale to showcase these variances; Always choose the
most efficient visualization; Watch your placement You may have two nice
stacked bar charts that are meant to let your reader compare points, but
if they're placed too far apart to ``get'' the comparison, you've
already lost; Tell the whole story Maybe you had a 30\% sales increase
in Q4. Exciting! But what's more exciting? Showing that you've actually
had a 100\% sales increase since Q1. } 5.2 Copy \textbf{ Don't over
explain If the copy already mentions a fact, the subhead, callout, and
chart header don't have to reiterate it; Keep chart and graph headers
simple and to the point There's no need to get clever, verbose, or
pun-tastic. Keep any descriptive text above the chart brief and directly
related to the chart underneath. Remember: Focus on the quickest path to
comprehension; Use callouts wisely Callouts are not there to fill space.
They should be used intentionally to highlight relevant information or
provide additional context; Don't use distracting fonts or elements
Sometimes you do need to emphasize a point. If so, only use bold or
italic text to emphasize a point---and don't use them both at the same
time. } 5.3 Color \textbf{ Use a single color to represent the same type
of data; Watch out for positive and negative numbers Don't use red for
positive numbers or green for negative numbers. Those color associations
are so strong it will automatically flip the meaning in the viewer's
mind; Make sure there is sufficient contrast between colors; Avoid
patterns Stripes and polka dots sound fun, but they can be incredibly
distracting. If you are trying to differentiate, say, on a map, use
different saturations of the same color. On that note, only use
solid-colored lines (not dashes); Select colors appropriately; Don't use
more than 6 colors in a single layout. } 5.4 Ordering \textbf{ Order
data intuitively There should be a logical hierarchy. Order categories
alphabetically, sequentially, or by value; Order consistently; Order
evenly Use natural increments on your axes (0, 5, 10, 15, 20) instead of
awkward or uneven increments (0, 3, 5, 16, 50). } 5.5 Audience
perspective \textbf{ Let the users lead;Know your audience,Designers
should consider the way users prefer to understand information, even in
choosing basic analytic approaches. For users to feel comfortable
adopting and sharing insights from analytics, they must be able to
explain and defend the data. } 5.6 Use layers to tell a story \textbf{
While style is one form of customization, layering unique data sets on a
single visualization can tell a richer narrative and connect users to
the data without getting too crowded. On a map, this can be as simple as
zooming in and out, but it can also involve drill-downs (choosing a data
point and expanding it to show more detail), links and other shortcuts.
} 5.7 Keep it simple ** Analytic results shouldn't be presented to 10
decimal places when the user doesn't need that level of precision to
make a decision or understand a concept. Effective visual interfaces
avoid 3-D effects or ornate gauge designs (a.k.a. ``chart junk'') when
simple numbers, maps or graphs will do.

+reference: \citep{French} +reference: \citep{Steier}

\section{6. More ways to improve your visualization
design}\label{more-ways-to-improve-your-visualization-design}

\subsection{Free eBooks All Designers Should
Read}\label{free-ebooks-all-designers-should-read}

From online surveys to beefed-up analytics, we're able to gather and
analyze more data than ever before. But how do you turn your findings
from a dense spreadsheet into something that really makes your point?
Good information design is the key.

There's a wealth of free resources out there in the form of handy little
design ebooks.

\begin{itemize}
\tightlist
\item
  \textbf{Design's Iron Fist} --- Jarrod Drysdale
\end{itemize}

The free ebook, Design's Iron Fist, is a collection of Drysdale's
previous work all wrapped up in one neat little package. Aside from
practical tutorials and processes, this book also offers help on how to
get into the mindset of being a truly great designer.

\begin{itemize}
\tightlist
\item
  \textbf{The Creative Aid Handbook} --- Kooroo Kooroo
\end{itemize}

Creativity doesn't just happen overnight. It's something that each and
every designer has to work at on a day-to-day basis. If you find that
your innovative juices are running dry, The Creative Aid Handbook could
be the answer. The helpful guide looks at how you can boost your
intellect, foster your well-being, and, most importantly, become more
creative.

\begin{itemize}
\tightlist
\item
  \textbf{Designbetter.co} --- InVision
\end{itemize}

InVision released three fantastic design books that are available for
free. Each book discusses various aspects of design like design process,
management, and business. Moreover, some of the materials are available
in audio format.

\begin{itemize}
\tightlist
\item
  \textbf{Type Classification}
\end{itemize}

Type Classification is a helpful beginner's guide to typography. It
should give you the foundations you need to not only start classifying
various forms of type but also understanding when and how to use them to
alarmingly great effect. It covers a history of each of the type forms
and the basic facts you need know about them.\citep{design_ebooks}

\section{7.Tips for Tableau}\label{tips-for-tableau}

\begin{enumerate}
\def\labelenumi{\arabic{enumi}.}
\tightlist
\item
  Running totals
\item
  Common Baseline
\item
  Weighted averages
\item
  Moving average
\item
  Grouping by aggregates
\item
  Different years comparison
\item
  Appending excel sheets
\item
  Bar chart totals
\item
  Fixed axis when re-drawing charts
\item
  Auto-fitting screen behavior depending on data selection
\end{enumerate}

\section{8.Word Cloud}\label{word-cloud}

A Word Cloud or Tag Cloud is a visual representation of text data in the
form of tags, which are typically single words whose importance is
visualized by way of their size and color. It displays how frequently
words appear in a given body of text, by making the size of each word
proportional to its frequency.

Word clouds can add clarity during text analysis in order to effectively
communicate your data results.It is an effective tool for Q researchers,
marketers, Non-profits, Human resources ,Educators, Politicians and
journalists.

** 8.1 Pros of Word Clouds **

\begin{enumerate}
\def\labelenumi{\arabic{enumi}.}
\tightlist
\item
  It is easy to understand and make an impact.
\item
  It can easily be shared.
\item
  It is visually engaging than a table data.
\item
  It is fast and reveals the essential.
\item
  They delight and provide emotional connection..
\end{enumerate}

** 8.2 Cons of Word Clouds **

\begin{enumerate}
\def\labelenumi{\arabic{enumi}.}
\tightlist
\item
  Emphasis based on length of the words.
\item
  Words whose letters contain many ascenders and descenders may receive
  more attention.
\item
  They're not very accurate.
\item
  Lot of data cleaning required before generating word cloud.
\item
  Context is lost.
\end{enumerate}

\citep{wordcloud}

** 8.3 Ways of generating a word cloud **

\begin{itemize}
\item
  R +The procedure of creating word clouds is very simple in R with text
  mining package (tm) and the word cloud generator package. The major
  steps involved are :text mining which involves text cleaning and
  transformation, building term-document matrix and generating word
  cloud. \citep{r}
\item
  Wordle +Wordle is a toy for generating ``word clouds'' from text that
  you provide.It is very popular, free and easy to use. You do need Java
  though Chrome. In Wordle, you generate word clouds from text you
  input. Clouds can be tweaked with different color schemes, layouts,
  and fonts. Images created from this tool can be saved and reused.
  \citep{wordle}
\end{itemize}

Other popular tools include ABCya, Tagul, Tag Crowd, CloudArt.

\subsection{Calendar View}\label{calendar-view}

(Redproducable code for reference)

\citep{Calendar_Layout} We have all seen the calendar views in the
various data products that we worked on. Please find below an open
source code that I found, this will help you replicate and create your
own calendar: \citep{CalendarView}

This example demonstrates loading of CSV data, which is then quantized
into a diverging color scale. The values are visualized as colored cells
per day. Days are arranged into columns by week, then grouped by month
and years.

\section{\texorpdfstring{11. An example to back some of our theories on
`how to tell stories using data visualization' / `exploratory data
visualization'}{11. An example to back some of our theories on how to tell stories using data visualization / exploratory data visualization}}\label{an-example-to-back-some-of-our-theories-on-how-to-tell-stories-using-data-visualization-exploratory-data-visualization}

DATA USA : \citep{DataUSA} MIT Media Lab in collaboration with Deloitte
has created a new visualization tool, that aggregates US government open
source data from various sources and mines information to generate
trends and stories about cities, jobs, industries etc. to the common
man. Just looking at any of the open data sources would give us an idea
about the vastness (breadth and depth) of the available data. It is
amazing to see how they have brought it all together on a single
platform in a very easy to decipher format. What caught my attention
here is the categorization of Information on the website that enables
the following - 1. Easy browsing of various categories of information
available at a single glance 2. Easy search on any topic of interest and
get deeper information on each 3. Logical construction of information
using data and visuals under each category 4. Comparative Analysis of
cities 5. Variety of exploratory visualizations to learn from 6. Most
important - Stories that these data tell For e.g.~Evolution of the
American Worker, Poverty is bad for your health, Men still dominate in
the highest paying industries, Opioid addiction damage and so many
others.

We think of a topic, and its possible it's there! Value add to students,
organizations, governments etc. is better understanding of your
consumers, talent pool, jobs, climate, and what not, that just improves
our decision-making ability manifold by spending just a few seconds on
the website And for this class, the best part is that the data is also
available for download. So, we can easily download this data, replicate
the visuals and try to redesign and tell our own stories with this data.
There is also other similar websites, that has some good visualizations
on census data: \citep{CensusDataViz}

Do explore!

Ref:

\url{https://flowingdata.com/2017/04/07/automatic-visualization-is-a-bad-idea-generally-speaking/}

\url{https://yseop.com/blog/automation-making-data-visualization-smarter/}

\url{http://blog.avenuecode.com/a-ui-engineers-thoughts-on-data-visualization-tools}

\url{https://www.quora.com/What-is-the-future-of-data-visualization}

Plug in any dataset into a magic box and it spits out a lovely
visualization you can show all of your co-workers, friends, and family.
That's the promise of a lot of startups, but it doesn't quite work that
way. The goal of data visualization tools was to make understanding data
easier, but more often than not it doesn't quite go to plan. The problem
is that graphics alone don't fully explain data, and so we are inundated
with queries: why did the numbers fall in whatever month? Data
visualization can't explain data, leaving room for interpretation.

Although simple visualizations such as standard chart types (bar chart,
line chart etc.) are already automated to a certain extend in Microsoft
Office tools and other software available in the market, but full on
automation where insight fountains out from any dataset is farfetched at
this point, because this requires automatic analysis. Automated analysis
here means that the tool or algorithm has to understand the context and
also select the best visualization.

The focus in today's world has been on open source tools and
technologies and these tools although being free for most part require
more effort to seamlessly integrate to the current visualization
workflow. As mentioned in one of the articles about D3.js:

D3.js is one of the first data visualization tools that comes to mind
when talking about free, open-source alternatives. It's a JavaScript
based library for creating web visualization and displays the results on
the web page. However, with great power comes great responsibility.
D3.js is extremely powerful and flexible, because it allows you to build
amazing things with it, but as a trade-off, it's not the easiest tool to
use, so you might need to spend some time going through the helpful
library documentation.

At the end its not only about the tool its more about what you are
trying to do; what your professor, client, business or whatever needs.

\section{12. Reusable Data Visualization Code in
R}\label{reusable-data-visualization-code-in-r}

\citep{viz_R}

This site includes full sets of R code to generate specific types of
graphs in ggplot2. Plots in ggplot2 are created by using ``layering''.
There is a base plot and then other aspects of the plot such as
aesthetics, titles and labels are added on using extra code. For those
who favor Python for data viz, this layering approach in R is actually
similar to the syntax in Python's matplotlib library, in which
set\_style and specifying the axes labels and title are done separately
from the code that generates the plot itself.

To provide an example the ``layering'' mentioned above, here is a
generic snippet of code for creating a scatterplot with ggplot2 and the
mtcars dataset in R base, using this website's code as a template:

\begin{Shaded}
\begin{Highlighting}[]
\KeywordTok{library}\NormalTok{(ggplot2)}

\KeywordTok{theme_set}\NormalTok{(}\KeywordTok{theme_bw}\NormalTok{())  }\CommentTok{#set background theme}

\NormalTok{plot1 <-}\StringTok{ }\KeywordTok{ggplot}\NormalTok{(mtcars, }\KeywordTok{aes}\NormalTok{(}\DataTypeTok{x =}\NormalTok{ hp, }\DataTypeTok{y =}\NormalTok{ mpg)) }\OperatorTok{+}\StringTok{ }\KeywordTok{geom_point}\NormalTok{(}\KeywordTok{aes}\NormalTok{(}\DataTypeTok{col=}\KeywordTok{factor}\NormalTok{(vs), }\DataTypeTok{size =} \DecValTok{2}\NormalTok{)) }\OperatorTok{+}\StringTok{ }\KeywordTok{geom_smooth}\NormalTok{(}\DataTypeTok{method =} \StringTok{"loess"}\NormalTok{, }\DataTypeTok{se =}\NormalTok{ F) }\OperatorTok{+}\StringTok{ }\KeywordTok{xlim}\NormalTok{(}\KeywordTok{c}\NormalTok{(}\DecValTok{0}\NormalTok{, }\DecValTok{400}\NormalTok{)) }\OperatorTok{+}\StringTok{ }\KeywordTok{ylim}\NormalTok{(}\KeywordTok{c}\NormalTok{(}\DecValTok{0}\NormalTok{, }\DecValTok{40}\NormalTok{)) }\OperatorTok{+}\StringTok{ }\KeywordTok{labs}\NormalTok{(}\DataTypeTok{title =} \StringTok{"Horsepower vs. MPG"}\NormalTok{, }\DataTypeTok{y =} \StringTok{"Miles Per Gallon"}\NormalTok{, }\DataTypeTok{x =} \StringTok{"Horsepower"}\NormalTok{)}

\KeywordTok{plot}\NormalTok{(plot1)  }\CommentTok{#we have to actually call the plot() function on the plot object we created}
\end{Highlighting}
\end{Shaded}

\includegraphics{Data_Viz_Reader_files/figure-latex/unnamed-chunk-9-1.pdf}
The ggplot2 package allows R users to go beyond the simple and often
rudimentary-looking graphs in R and offers many ways of customizing data
visualizations. In a way, the layering technique also makes it easier to
remember the code to generate these plots, since geom functions for the
layers remain constant and they are all included in a single line of
code.

\chapter{Ethics}\label{ethics}

\section{5.1 Importance of Ethics in
Visualization}\label{importance-of-ethics-in-visualization}

\citep{poli_social_science}

Over the years, researchers and lawyers have come up with rules and
practices for proper data collection and utilization, with particular
attention on human subject research. Consent of the subjects to use
their data, evaluation of any risk with use or collection of data, and
protecting anonymity of data are some of the rules that must be
considered for ethical research methods. Under U.S. law, research
institutions receiving federal funding, must consider ethical aspects of
their research. These rules continue to evolve.

Data presented in charts can persuade viewers on the subject matter,
even if viewers do not support the idea presented. This means that
visualizations can also be used to deceive and there are many techniques
for this leading viewers to wrong conclusions. Misleading,
incomprehensible, or incredible data visualization can jeopardize
people's trust, goodwill, or faith in research and advocacy on vital
human rights issues. Its ethical responsibility to create visualizations
to give correct and faithful representation of data and subjects.

\subsubsection{\texorpdfstring{\textbf{Ethical Theory and Practice from
Journalism and
Engineering}}{Ethical Theory and Practice from Journalism and Engineering}}\label{ethical-theory-and-practice-from-journalism-and-engineering}

\textbf{1. Data visualization in political and social sciences} -
(Reference:
\url{https://github.com/mschermann/data_viz_reader/files/1933699/Zinovyev_Data_Visualization.pdf})
\citep{ethical_infographics}

Alberto Cairo addresses the ethical `why' of data visualization in this
article, while still grounding the discussion in straightforward
analysis of what to do and what not to do. He emphasizes that the
effectiveness of the communicative display is as important as the
information itself. This makes intuitive sense because useful
information is rendered utterly useless if no one can understand it.

Cairo sees data visualization as a harmonization of journalism and
engineering. From these two disciplines, he takes the journalist ethos
of truth-telling and honesty and combines this with an engineering focus
on efficacy and efficiency. The result is a data visualization that
contains accurate and relevant information which is clearly and
concisely conveyed.

Cairo describes himself as a ``rule utilitarian'' and uses this to
explain why it is ethical or, in his words, ``morally right,'' to create
graphics in this way. Here, it very useful to review his blogpost
introducing the article. Essentially, the goal is to create the most
good while doing the least harm. As such, conveying truthful and honest
relevant information increases a persons understanding. Increased
understanding and knowledge positively correlates with personal
well-being.

The information presented must be accurate and relevant. Cairo briefly
addresses guidelines for this which are applicable in all information
gathering fields: beware of selection bias when choosing preexisting
datasets, validate the data, and include important context. False or
irrelevant information doesn't improve anyone's decision-making
capacity, so it cannot enhance well-being.

Even if the information is both accurate and relevant, moral engineering
pitfalls may remain. To avoid the unethical trap of inscrutable (or
misleading) graphics, Cairo exhorts us to take an evidenced based
approach when possible. The purpose of the graphic dictates the form it
takes; aesthetic preferences should never override clarity.

Again, since the ethical purpose is to improve well-being through
understanding, a graphic which is confusing or misleading is unethical,
regardless of intent, since it actually creates misunderstanding for the
audience. While it can be a bit jarring to think of a poorly designed
graphic as ``morally wrong'', it is important to think of the unintended
consequences of visuals which have a powerful impact on their viewers.

\section{\texorpdfstring{5.2 Data Visualization in \textbf{Political and
Social
Sciences}}{5.2 Data Visualization in Political and Social Sciences}}\label{data-visualization-in-political-and-social-sciences}

\citep{poli_social_science}

The basic objective of data visualization is to provide an efficient
graphical display for summarizing and reasoning about quantitative
information. And during the last decades, political science has
accumulated a large corpus of various kinds of data, which makes it
gradually become a more quantitative scientific field and requires using
quantitative information in the analysis and reasoning.

Data visualization plays several important roles in it:

\begin{enumerate}
\def\labelenumi{\arabic{enumi}.}
\tightlist
\item
  it helps create informative illustrations of the data, recapitulating
  large amount of quantitative information on a diagram;
\item
  it helps formulate new or supporting existing hypotheses from
  quantitative data;
\item
  it guides a statistical analysis of data and checks its validity.
\end{enumerate}

Some useful visualization methods are:

\begin{enumerate}
\def\labelenumi{\arabic{enumi}.}
\tightlist
\item
  Statistical graphics and infographics;
\item
  Geographical information systems (GIS);
\item
  Graph visualization or network maps;
\item
  Data cartography.
\end{enumerate}

\section{\texorpdfstring{5.3 Data Visualization in
\textbf{Business}}{5.3 Data Visualization in Business}}\label{data-visualization-in-business}

\citep{biz_strategy}

According to an Experian report, 95\% of U.S. organizations say that
they use data to power business opportunities, and another 84 percent
believe data is an integral part of forming a business strategy.
Visualization helps data impact business in following ways:

\textbf{1. Cleaning}

The simplest way to explain the importance of visualization is to look
at visualization as the means to making sense of data. Even the most
basic, widely-used data visualization tools that combine simple pie
charts and bar graphs help people comprehend large amounts of
information fast and easily, compared to paper reports and spreadsheets.

In other words, visualization is the initial filter for the quality of
data streams. Combining data from various sources, visualization tools
perform preliminary standardization, shape data in a unified way and
create easy-to-verify visual objects. As a result, these tools become
indispensable for data cleansing and vetting and help companies prepare
quality assets to derive valuable insights.

\textbf{2. Extracting}

Known versatile tools for data visualization and analytics -- Elastic
Stack, Tableau, Highcharts, and more complex database solutions like
Hadoop, Amazon AWS and Teradata, have wide applications in business,
from monitoring performance to improving customer experience on mobile
tools. New generation of data visualization based on AR and VR
technology, however, provides formerly infeasible advantages in terms of
identifying patterns and drawing insights from various data streams.

Building 3D data visualization spaces, companies can create an intuitive
environment that helps data scientists grasp and analyze more data
streams at the same time, observe data points from multiple dimensions,
identify previously unavailable dependencies and manipulate data by
naturally moving objects, zooming, and focusing on more granulated
areas. Moreover, these tools allow us to expand the capabilities of data
visualization by creating collaborative 3D environments for teams. As a
result, new technology helps extract more valuable insights from the
same volume of data.

\textbf{3. Strategizing}

As the amount of data grows, it becomes harder to catch up with it.
Therefore, data strategy becomes the necessary part of the success in
applying data to business. Then how data visualization become an
important tool in your strategic kit? First, it helps you cleanse your
data. Secondly, it allows you to identify and extract meaningful
information from it. Finally, data visualization tools enable continuous
real-time monitoring of how your strategy and now data-driven decisions
influence performance and business outcomes. In other words, these tools
visualize not only the data, but also the results, and help correct and
optimize strategy on the go.

Data visualization is one of the initial steps made to derive value from
data. It's also one of the most important steps, as it determines how
efficiently analysts can work with data assets, what insights they are
able to extract and how their data strategy will develop over time.

Therefore, the quality and capabilities of data visualization directly
influence how data impacts your business strategy and what benefits data
applications can bring to the companies and their industries.\_

\section{5.4 Implications of (Good/Bad) Data
Visualization}\label{implications-of-goodbad-data-visualization}

Raw data is often meaningless or their meaning is not easily concluded.
When people face a large set of measurements they are unable or
unwilling to spend the time required to process it. Our modern living
contributes to an ever-growing pool of ``big data'' and our ability to
collect this type of information becomes easier and easier. Thus
filtering, visualization, and interpretation of data become increasingly
important.

We should understand what to do with data, but first we should
understand why their presentation in graphical format is so powerful.

\textbf{1. Easy Recall}

People can process images more quickly than words. When data is
transformed into images, the readability and cognition of the content
greatly improves. While people can only remember just 10\% of what they
hear and 20\% of what they read, retention jumps up to 80\% when they
for visual information with interaction.

\textbf{2. Providing Window for Perspective}

With infographics you can pack a lot of information into a small space.
Colors, shape, movement, contrast in scale and weight, and even sound
can be used to denote different aspects of the data allowing for
multi-layered understanding. Below is an example for a good graph:
Reference: \citep{image_good}

\textbf{3. Enable Qualitative Analysis}

Color, shape, sounds, and size can make evident relationships within
data very intuitive. When data points are represented as images or
components of an entire scene, readers are able to see the big picture
and understand how the information fits within a larger context.

\textbf{4. Increase in User Participation}

Interactive infographics can substantially increase the amount of time
someone will spend with the content.

Because of their impact, infographics are widely used nowadays. A quick
google will produce a huge array of great examples --- as well as poor
ones. Because while people recognize the value of information graphic
design, and a number of tools are available today that make the creation
of them possible for the layperson, it doesn't mean that they're all
successful or even necessary.

In the example below, the information would be better presented does not
easily answer the simple question: How many airplane seats are left
empty each year? It could have been more clear with the numbers and
comparisons. Reference: \citep{image_bad}

\subsubsection{\texorpdfstring{\textbf{Misrepresentation through Data
Visualization}}{Misrepresentation through Data Visualization}}\label{misrepresentation-through-data-visualization}

While the ideal purpose of data visualization is to improve others'
understanding of the data presented, visualization can also be used to
mislead. Some of the main methods of doing so are omitting baselines,
axis manipulation, omitting data, and going against graphing convention.

Omitting baselines is used to imply a greater difference between two
categories, such as in poll results comparing political parties. Axis
manipulation by increasing the highest value on the y-axis affects the
visibility of a slope, making data with an otherwise visible trend
appear flat. Omitting selected data points or narrowing the window of a
graph is used to hide an overall trend, such as a graph of a stock only
showing a current trend and hiding previous bubbles. Graphs can also be
designed to subvert convention so that at first glance the graph is
conveying the opposite message, for example, by using the reader's
associations of colors and temperature to create a graph where hot is
blue and cold is red.

\section{5.5 General Guidelines to Ethical
Visuals}\label{general-guidelines-to-ethical-visuals}

\citep{ethics_code}

Data visualization is an up and coming field that currently doesn't have
many regulations. This makes it easy to manipulate readers without
technically reporting false information. However, certain standards
should be followed in order to generate meaningful visuals. The process
can be broken down into three steps, each with its own set of guiding
rules.

\textbf{1. Data Collection}

The first step in any project is gathering the data. This is relatively
simple and does not offer much of an opportunity to introduce confusion.
The one thing to remember is to always get data from a reliable source.
The data provides the foundation for the entire project and must
therefore be trustworthy and verifiable.

\textbf{2. Data Analysis}

This is the stage where the discoveries are made and provides the first
opportunity to manipulate the story for good or bad. There is usually a
lot of data cleaning to do before creating a visual representation, but
all manipulation should make sense. Code should be shared so anyone can
follow the entire process. It is also important to explicitly state any
assumptions taken, though these should be kept to a minimum.Here it is
important to look at what the source data actually shows.its ethical
responsibility of presenters for careful analysis of the data and find
true stories from them.

\textbf{3. Design}

Once a story is found, it must be presented in an honest way. This is
where deceptive techniques could be tempting to make a stronger
argument. An experienced individual will know how to spot these
deceptions and disregard any findings. This ultimately hurts the
credibility of the author and anyone else involved in the publication.

Visualization should not be used to intentionally hide or confuse the
truth, it should not mislead the uninformed. Visualization has got great
power and so does lots of responsibility.

\chapter{Conclusion}\label{conclusion}

Reflection, Key Learnings, Outlook

\chapter*{References}\label{references-1}
\addcontentsline{toc}{chapter}{References}

\section{Chapter 2-fundamentals}\label{chapter-2-fundamentals}

\textbf{2.1 The History of Data Visualization} \textbf{Author: Dashboard
Insight, Dashboard Insight, 2013} URL:
\url{http://www.dashboardinsight.com/news/news-articles/the-history-of-data-visualization.aspx}

\citep{The_History_of_Data_Visualization}

\begin{verbatim}
@misc{The_History_of_Data_Visualization,
  author = {{Dashboard Insight}},
  year = {2013},
  title = The History of Data Visualization},
  howpublished = {\url{http://www.dashboardinsight.com/news/news-articles/the-history-of-data-visualization.aspx}},
  note = {Accessed: 2018-05-12}
}
\end{verbatim}

\textbf{2.2 Current research: Deceptive visualizations} \textbf{Author:
Infogram, 2016}
URL:\url{https://medium.com/@Infogram/study-asks-how-deceptive-are-deceptive-visualizations-8ff52fd81239}
\citep{Deceptive_visualization}

\begin{verbatim}
@misc{Deceptive_visualizations,
  author = {{Infogram}},
  year = {2016},
  title = {Deceptive visualizations},
  howpublished = {\url{https://medium.com/@Infogram/study-asks-how-deceptive-are-deceptive-visualizations-8ff52fd81239}},
  note = {Accessed: 2018-05-12}
}
\end{verbatim}

\textbf{Author: Agata Kwapien in Data Visualization, 2015} URL:
\url{https://www.datapine.com/blog/misleading-data-visualization-examples/}
\citep{Data_Visualization}

\begin{verbatim}
@misc{Data_Visualization,
  author = {{Agata kwapien}},
  year = {2015},
  title = {Agata Kwapien in Data Visualization},
  howpublished = {\url{https://www.datapine.com/blog/misleading-data-visualization-examples/}},
  note = {Accessed: 2018-05-12}
}
\end{verbatim}

\textbf{2.3 A Brief History of Data Visualization,York University.}
\textbf{Auhtor: Michael Friendly, 2006}
URL:\url{http://www.datavis.ca/papers/hbook.pdf}
\citep{A_Brief_History_of_Data_Visualization}

\begin{verbatim}
@misc{A_Brief_History_of_Data_Visualization,
  author = {{Micharl Friendly}},
  year = {2006},
  title = {A Brief History of Data Visualization,York University},
  howpublished = {\url{http://www.datavis.ca/papers/hbook.pdf}},
  note = {Accessed: 2018-05-12}
}
\end{verbatim}

\emph{Summary: } This paper provides an overview of the intellectual
history of data visualization from medieval to modern times,describing
and illustrating some significant advances along the way.

** 2.4 Data Visualization and the 9 Fundamental Design
Principles\textbf{ }Auhthor: Melissa Anderson, 2017**
URL:\url{https://www.idashboards.com/blog/2017/07/26/data-visualization-and-the-9-fundamental-design-principles/}
\citep{Data_Visualization_and_the_9_Fundamental_Design_Principles}

\begin{verbatim}
@misc{ta_Visualization_and_the_9_Fundamental_Design_Principles,
  author = {{Melissa Anderson}},
  year = {2017},
  title = {a_Visualization_and_the_9_Fundamental_Design_Principles},
  howpublished = {\url{https://www.idashboards.com/blog/2017/07/26/data-visualization-and-the-9-fundamental-design-principles/}},
  note = {Accessed: 2018-05-12}
}
\end{verbatim}

** 2.5 A Practitioner's Guide to Best Practices in Data
Visualization.Interfaces 47(6):473-488.\textbf{ } Auhtor: Jeffrey D.
Camm, Michael J. Fry, Jeffrey Shaffer, 2017 ** URL:
\url{https://doi.org/10.1287/inte.2017.0916}

\citep{A_Practitioner_Guide_to_Best_Practices_in_Data_Visualization}

\begin{verbatim}
@misc{A_Practitioner_Guide_to_Best_Practices_in_Data_Visualization,
  author = {{Jeffrey D. Camm, Michael J. Fry, Jeffrey Shaffer}},
  year = {2017},
  title = {A_Practitioner_Guide_to_Best_Practices_in_Data_Visualization},
  howpublished = {\url{Jeffrey D. Camm, Michael J. Fry, Jeffrey Shaffer}},
  note = {Accessed: 2018-05-12}
}
\end{verbatim}

** 2.6 The 7 Best Data Visualization Tools In 2017\textbf{ } Author:
Bernard Marr, 2017** URL:
\url{https://www.forbes.com/sites/bernardmarr/2017/07/20/the-7-best-data-visualization-tools-in-2017/\#3a12b8ea6c30}

\citep{The_7_Best_Data_Visualization_Tools_In_2017}

\begin{verbatim}
@misc{The_7_Best_Data_Visualization_Tools_In_2017,
  author = {{Bernard Marr}},
  year = {2017},
  title = {The_7_Best_Data_Visualization_Tools_In_2017},
  howpublished = {\url{https://www.forbes.com/sites/bernardmarr/2017/07/20/the-7-best-data-visualization-tools-in-2017/#3a12b8ea6c30}},
  note = {Accessed: 2018-05-12}
}
\end{verbatim}

** 2.7 The Data Visualisation Catalogue**

\textbf{2.7 The Data Visualisation Catalogue}

URL: \url{https://datavizcatalogue.com}
\citep{The_Data_Visualisation_Catalogue}

\begin{verbatim}
@misc{he_Data_Visualisation_Catalogue,
  howpublished = {\url{https://datavizcatalogue.com}},
  note = {Accessed: 2018-05-12}
}
\end{verbatim}

** 2.8 The Extreme Presentation(tm) Method \textbf{ } Aurthor:
Dr.~Abela, 2015 ** URL:
\url{http://extremepresentation.typepad.com/blog/2015/01/announcing-the-slide-chooser.html}
\citep{The_Extreme_Presentation}

\begin{verbatim}
@misc{The_Extreme_Presentation,
  author = {{Dr. Abela}},
  year = {2015},
  title = {The_Extreme_Presentation},
  howpublished = {\url{The_Extreme_Presentation}},
  note = {Accessed: 2018-05-12}
}
\end{verbatim}

** 2.9 Data Visualization -- How to Pick the Right Chart Type? **

\textbf{Jānis Gulbis, 2016}
\url{https://eazybi.com/blog/data_visualization_and_chart_types/}
\citep{How_to_Pick_the_Right_Chart_Type}

\begin{verbatim}
@misc{How_to_Pick_the_Right_Chart_Type,
  author = {{Jānis Gulbis}},
  year = {2016},
  title = {How_to_Pick_the_Right_Chart_Type?},
  howpublished = {\url{https://eazybi.com/blog/data_visualization_and_chart_types/}},
  note = {Accessed: 2018-05-12}
}
\end{verbatim}

** Author: AJānis Gulbis, 2016 ** URL:
\url{https://eazybi.com/blog/data_visualization_and_chart_types/}

** 2.10 Data Visualization Best Practices\textbf{ } Author:
melindasantos, 2017** URL:
\url{http://paristech.com/blog/data-visualization-best-practices/}
\citep{Data_Visualization_Best_Practices}

\begin{verbatim}
@misc{Data_Visualization_Best_Practices,
  author = {{melindasantos}},
  year = {2017},
  title = {Data_Visualization_Best_Practices},
  howpublished = {\url{http://paristech.com/blog/data-visualization-best-practices/
}},
  note = {Accessed: 2018-05-12}
}
\end{verbatim}

\url{http://extremepresentation.typepad.com/blog/2015/01/announcing-the-slide-chooser.html}
【\citet{extreme_presentation}{]}

\begin{verbatim}
@misc{extreme_presentation,
  author = {{Dr. Abela}},
  year = {2015},
  title = {extreme_presentation},
  howpublished = {\url{http://extremepresentation.typepad.com/blog/2015/01/announcing-the-slide-chooser.html}},
  note = {Accessed: 2018-05-12}
}
\end{verbatim}

** 2.11 3 simple rules for intuitive dashboard design\textbf{ } Author:
Happy Dashboarding, 2017 ** URL:
\url{https://www.klipfolio.com/blog/intuitive-dashboard-design}
\citep{3_simple_rules_for_intuitive_dashboard_design}

\begin{verbatim}
@misc{3_simple_rules_for_intuitive_dashboard_design,
  author = {{Happy Dashboarding}},
  year = {2017},
  title = {3_simple_rules_for_intuitive_dashboard_design},
  howpublished = {\url{https://www.klipfolio.com/blog/intuitive-dashboard-design}},
  note = {Accessed: 2018-05-12}
}
\end{verbatim}

** 2.12 How deceptive are deceptive visualizations?\textbf{ } Author:
Pandey, A. V., Rall, K., Satterthwaite, M. L., Nov, O., \& Bertini, E.
,2015 **

** 2.13 An empirical analysis of common distortion techniques\textbf{ }
Author: Anshul Vikram Pandey, 2015 **

\textbf{2.14 Factors in Computing Systems: Crossings (Vol. 2015-April,
pp.~1469-1478). Association for Computing Machinery. DOI:
10.1145/2702123.2702608 (2) Tufte, E. R., and Graves-Morris, P. The
visual display of quantitative information, vol.~2. Graphics press
Cheshire, CT,1983.}

** 2.15 Axes of evil: How to lie with graphs**

\textbf{ANDREA ROBERTSON}
\url{http://hypsypops.com/axes-evil-lie-graphs/}
\citep{how_to_lie_with_graphs}

\begin{verbatim}
@misc{how_to_lie_with_graphs,
  author = {{Andrea Robertson}},
  title = {how_to_lie_with_graphs},
  howpublished = {\url{http://hypsypops.com/axes-evil-lie-graphs/}},
  note = {Accessed: 2018-05-12}
}
\end{verbatim}

\textbf{Author: ANDREA ROBERTSON} URL:
\url{http://hypsypops.com/axes-evil-lie-graphs/}

** 2.16 Misleading Graphs: Real Life Examples\textbf{ }Author:
Stephanie, February 28th, 2016** URL:
\url{http://www.statisticshowto.com/misleading-graphs/}

\citep{Misleading_Graphs}

\begin{verbatim}
@misc{Misleading_Graphs,
  author = {{Stephanie}},
  year = {2018},
  title = {Misleading_Graphs},
  howpublished = {\url{http://www.statisticshowto.com/misleading-graphs/ }},
  note = {Accessed: 2018-05-12}
}
\end{verbatim}

** 2.17 Next Steps for Data Visualization Research**

** UW Interactive Data Lab,2015**
\url{https://medium.com/@uwdata/next-steps-for-data-visualization-research-3ef5e1a5e349}
\citep{17_Next_Steps_for_Data_Visualization_Research}

\begin{verbatim}
@misc{17_Next_Steps_for_Data_Visualization_Research,
  author = {{ UW Interactive Data Lab}},
  year = {2015},
  title = {17_Next_Steps_for_Data_Visualization_Research},
  howpublished = {\url{https://medium.com/@uwdata/next-steps-for-data-visualization-research-3ef5e1a5e349}},
  note = {Accessed: 2018-05-12}
}
\end{verbatim}

** 2.18 Using Typography to Expand the Design Space of Data
Visualization. She Ji: The Journal of Design, Economics and Innovation,
2(1), pp 59-87. **

** 2.19 Using Typography to Expand the Design Space of Data
Visualization\textbf{ }Banissi, Ebad, \& Brath, Richard. (2016). **
\url{https://www.sciencedirect.com/science/article/pii/S2405872616300107}.
\citep{Using_Typography_to_Expand_the_Design_Space_of_Data_Visualization}

\begin{verbatim}
@misc{Using_Typography_to_Expand_the_Design_Space_of_Data_Visualization,
  author = {{Banissi, Ebad, & Brath, Richard}},
  year = {2016},
  title = {Using_Typography_to_Expand_the_Design_Space_of_Data_Visualization},
  howpublished = {\url{https://www.sciencedirect.com/science/article/pii/S2405872616300107}},
  note = {Accessed: 2018-05-12}
}
\end{verbatim}

** 2.20 Using Data Visualization to Find Insights in Data\textbf{
}Gregor Aisch, Open Knowledge Foundation**
\url{http://datajournalismhandbook.org/1.0/en/understanding_data_7.html}
\citep{Using_Data_Visualization_to_Find_Insights_in_Data}

\begin{verbatim}
@misc{Using_Data_Visualization_to_Find_Insights_in_Data,
  author = {{Gregor Aisch}},
  title = {Using_Data_Visualization_to_Find_Insights_in_Data},
  howpublished = {\url{http://datajournalismhandbook.org/1.0/en/understanding_data_7.html }},
  note = {Accessed: 2018-05-12}
}
\end{verbatim}

** 2.21 Building advanced analytics application with TabPy**
\url{https://www.tableau.com/about/blog/2017/1/building-advanced-analytics-applications-tabpy-64916}
\citep{Building_advanced_analytics_application_with_TabPy}

\begin{verbatim}
@misc{Building_advanced_analytics_application_with_TabPy,
  author = {{BORA BERAN}},
  year = {2017},
  title = {Building_advanced_analytics_application_with_TabPye},
  howpublished = {\url{https://www.tableau.com/about/blog/2017/1/building-advanced-analytics-applications-tabpy-64916}},
  note = {Accessed: 2018-05-12}
}
\end{verbatim}

** 2.22 Some best practices for visualization:**
\url{http://www.dataplusscience.com/files/visual-analysis-guidebook.pdf}
\citep{Visual_Analysis_Best_Practices}

\begin{verbatim}
@misc{Visual_Analysis_Best_Practices,
  author = {{tableau}},
  title = {AVisual_Analysis_Best_Practices},
  howpublished = {\url{http://www.dataplusscience.com/files/visual-analysis-guidebook.pdf}},
  note = {Accessed: 2018-05-12}
}
\end{verbatim}

** 2.23 Avoiding Common Mistakes with Time Series\textbf{ } TOM
FAWCETT,2015**
\url{https://www.svds.com/avoiding-common-mistakes-with-time-series/}
\citep{Avoiding_Common_Mistakes_with_Time_Series}

\begin{verbatim}
@misc{Avoiding_Common_Mistakes_with_Time_Series,
  author = {{Tom fawcett}},
  year = {2015},
  title = {Avoiding_Common_Mistakes_with_Time_Series},
  howpublished = {\url{https://www.svds.com/avoiding-common-mistakes-with-time-series/}},
  note = {Accessed: 2018-05-12}
}
\end{verbatim}

\section{Chapter 4 patterns}\label{chapter-4-patterns}

** 4.1 The Baseline and Working with Time Series in R\textbf{ } Nathan
Yau, 2013** \url{https://flowingdata.com/2013/11/26/the-baseline/}
\citep{The_Baseline_and_Working_with_Time_Series_in_R}

\begin{verbatim}
@misc{he_Baseline_and_Working_with_Time_Series_in_R,
  author = {{Nathan Yau}},
  year = {2013},
  title = {he_Baseline_and_Working_with_Time_Series_in_R},
  howpublished = {\url{https://flowingdata.com/2013/11/26/the-baseline/}},
  note = {Accessed: 2018-05-12}
}
\end{verbatim}

** 4.2 Using design patterns to find greater meaning in your
data\textbf{ }Julie RodriguezPiotr Kaczmarek May 11, 2016**
\url{https://www.oreilly.com/ideas/using-design-patterns-to-find-greater-meaning-in-your-data}
\citep{Using_design_patterns_to_find_greater_meaning_in_your_data}

\begin{verbatim}
@misc{Using_design_patterns_to_find_greater_meaning_in_your_data,
  author = {{Julie RodriguezPiotr Kaczmarek}},
  year = {2016},
  title = {Using_design_patterns_to_find_greater_meaning_in_your_data},
  howpublished = {\url{https://www.oreilly.com/ideas/using-design-patterns-to-find-greater-meaning-in-your-data}},
  note = {Accessed: 2018-05-12}
}
\end{verbatim}

\textbf{4.3Design's Iron Fist} \textbf{Jarrod Drysdale}
\url{https://studiofellow.com/newsletter/} \citep{Design_Iron_Fist}

\begin{verbatim}
@misc{Design_Iron_Fist,
  author = {{Jarrod Drysdale}},
  title = {Design_Iron_Fist},
  howpublished = {\url{https://studiofellow.com/newsletter/}},
  note = {Accessed: 2018-05-12}
}
\end{verbatim}

\textbf{4.4The Creative Aid Handbook} \textbf{Kooroo Kooroo}
\url{https://issuu.com/koorookooroo/docs/kooroo_kooroo_creative_aid}
\citep{The_Creative_Aid_Handbook}

\begin{verbatim}
@misc{the_Creative_Aid_Handbook,
  author = {{Kooroo Kooroo}},
  title = {the_Creative_Aid_Handbook},
  howpublished = {\url{https://issuu.com/koorookooroo/docs/kooroo_kooroo_creative_aid}},
  note = {Accessed: 2018-05-12}
}
\end{verbatim}

\textbf{4.5Designbetter.co} \textbf{InVision}
\url{https://www.designbetter.co/} \citep{Designbetter}

\begin{verbatim}
@misc{Designbetter,
  title = {Designbetter},
  howpublished = {\url{https://www.designbetter.co/}},
  note = {Accessed: 2018-05-12}
}
\end{verbatim}

\textbf{4.6 Type Classification}
\url{http://justcreative.com/featured-articles/type-classification-ebook/}
\citep{Classification_type}

\begin{verbatim}
@misc{Classification_type,
  title = {Classification_type},
  howpublished = {\url{http://justcreative.com/featured-articles/type-classification-ebook/}},
  note = {Accessed: 2018-05-12}
}
\end{verbatim}

** 4.7 Free eBooks All Designers Should Read\textbf{ } Michael
Abehsera,2018**
\url{https://uxplanet.org/free-ebooks-all-designers-should-read-60dbb63f762}
\citep{Free_eBooks_All_Designers_Should_Read}

\begin{verbatim}
@misc{Free_eBooks_All_Designers_Should_Read,
  author = {{Michael Abehsera}},
  year = {2018},
  title = {Free_eBooks_All_Designers_Should_Read},
  howpublished = {\url{https://uxplanet.org/free-ebooks-all-designers-should-read-60dbb63f762}},
  note = {Accessed: 2018-05-12}
}
\end{verbatim}

\textbf{4.8} \textbf{Javier Guillen}\\
\url{http://cdn2.hubspot.net/hubfs/257922/Docs/BlueGranite_whitepaper_10useful.pdf}
\citep{USEFUL_TABLEAU_TECHNIQUES}

\begin{verbatim}
@misc{USEFUL_TABLEAU_TECHNIQUES,
  author = {{Javier Guillen}},
  title = {USEFUL_TABLEAU_TECHNIQUES},
  howpublished = {\url{http://cdn2.hubspot.net/hubfs/257922/Docs/BlueGranite_whitepaper_10useful.pdf}},
  note = {Accessed: 2018-05-12}
}
\end{verbatim}

** Author: UW Interactive Data Lab, 2015** URL:
\url{https://medium.com/@uwdata/next-steps-for-data-visualization-research-3ef5e1a5e349}

\textbf{2.18 Using Typography to Expand the Design Space of Data
Visualization. She Ji: The Journal of Design, Economics and Innovation,
2(1), pp 59-87.}

** 2.19 Using Typography to Expand the Design Space of Data
Visualization\textbf{ }Banissi, Ebad, \& Brath, Richard. (2016). ** URL:
\url{https://www.sciencedirect.com/science/article/pii/S2405872616300107}.

** 2.20 Using Data Visualization to Find Insights in Data** URL:
\url{http://datajournalismhandbook.org/1.0/en/understanding_data_7.html}

** 2.21 Building advanced analytics application with TabPy** URL:
\url{https://www.tableau.com/about/blog/2017/1/building-advanced-analytics-applications-tabpy-64916}

** 2.22 Some best practices for visualization:** URL:
\url{http://www.dataplusscience.com/files/visual-analysis-guidebook.pdf}

** 2.23 Avoiding Common Mistakes with Time Series\textbf{ }Author: TOM
FAWCETT,2015** URL:
\url{https://www.svds.com/avoiding-common-mistakes-with-time-series/}

\section{Chapter 4 patterns}\label{chapter-4-patterns-1}

** 4.1 The Baseline and Working with Time Series in R\textbf{ }Author:
Nathan Yau, 2013** URL:
\url{https://flowingdata.com/2013/11/26/the-baseline/}

** 4.2 Using design patterns to find greater meaning in your
data\textbf{ }Author: Julie RodriguezPiotr Kaczmarek May 11, 2016** URL:
\url{https://www.oreilly.com/ideas/using-design-patterns-to-find-greater-meaning-in-your-data}

\textbf{4.3Design's Iron Fist} \textbf{Author: Jarrod Drysdale} URL:
\url{https://studiofellow.com/newsletter/}

\textbf{4.4The Creative Aid Handbook} URL:
\url{https://issuu.com/koorookooroo/docs/kooroo_kooroo_creative_aid}

Another trick of misleading graph is axis change: Changing thy y-axis
maximum afftect how the graph look like. A higher maximum will make the
grpha to appear less volatiliy ,less strrp than a lower maximum. The
other way of axis change is changing the ratio of a graph's dimensions.
This way will affect how the graph appears. We demostrate chaning the
ratio of graph dimension for below graphs.

\begin{figure}
\centering
\includegraphics{images/Line_graph1.svg.png}
\caption{}
\end{figure}

\begin{figure}
\centering
\includegraphics{images/175px-Line_graph1-3.svg.png}
\caption{}
\end{figure}

\begin{figure}
\centering
\includegraphics{images/200px-Line_graph1-4.svg.png}
\caption{}
\end{figure}

\textbf{4.5Designbetter.co} URL: \url{https://www.designbetter.co/}

\textbf{4.6 Type Classification} URL:
\url{http://justcreative.com/featured-articles/type-classification-ebook/}

** 4.7 Free eBooks All Designers Should Read\textbf{ } Author: Michael
Abehsera, 2018** URL:
\url{https://uxplanet.org/free-ebooks-all-designers-should-read-60dbb63f762}

It is not technically wrong but it is definitely misleading.This is
often called improper extraction or tactic omitting data, when only a
certain chunk of data is included.This is more common in graphs that
have time as one of their axis. Here is the graph to show what it is.

\begin{figure}
\centering
\includegraphics{images/Bad_graph_extraction.png}
\caption{}
\end{figure}

\begin{figure}
\centering
\includegraphics{images/Good_graph_extraction.png}
\caption{}
\end{figure}

\begin{enumerate}
\def\labelenumi{\arabic{enumi}.}
\setcounter{enumi}{3}
\tightlist
\item
  Using The Wrong Graphs So far, we have intentional misinformation
  tactics that writers use to push their agendas.
\end{enumerate}

\textbf{4.8} \textbf{Author: Javier Guillen}\\
URL:
\url{http://cdn2.hubspot.net/hubfs/257922/Docs/BlueGranite_whitepaper_10useful.pdf}

\section{Chapter 5}\label{chapter-5}

\textbf{5.1 Importance of Ethics in Visualization} Data visualization in
political and social sciences, author: Andrei Zinovyev, year: N/A,
\url{https://github.com/mschermann/data_viz_reader/files/1933699/Zinovyev_Data_Visualization.pdf}

Ethical Infographics: In data visualization, journalism meets
engineering, author: Alberto Cairo, year: 2014,
\url{http://www.thefunctionalart.com/2014/06/infographics-data-and-visualization.html}

\textbf{5.2 Data Visualization in~Political and Social Sciences} Data
visualization in political and social sciences, author: Andrei Zinovyev,
year: N/A,
\url{https://github.com/mschermann/data_viz_reader/files/1933699/Zinovyev_Data_Visualization.pdf}

\textbf{5.3 Data Visualization in~Business} How Data Visualization
Impacts Your Business Strategy, author: Katherine Lazarevich, year:
2018,
\url{https://www.iotforall.com/data-visualization-strategy-for-business/}

\textbf{5.4 Implications of (Good/Bad) Data Visualization} How Writers
Use Misleading Graphs to Manipulate You, author: Ryan McCready, year:
2017, \url{https://venngage.com/blog/misleading-graphs/}

\textbf{5.5 General Guidelines to Ethical Visuals} A Code of Ethics for
Data Visualization Professionals, author: Drew Skau, year: 2012,
\url{https://visual.ly/blog/a-code-of-ethics-for-data-visualization-professionals/}

\section{Others}\label{others}

\textbf{1. 3 Expert Data Visualization Tips for Grabbing Readers'
Attention} \citep{attention_grabbers}

This article found on Medium explores three important aspects to focus
on when creating a data visualization. The importance of each aspect is
explained along with helpful questions to ask and to help you evaluate
your visualization to ensure it caters to your audience. Although it
primarily focuses on the appearance of visuals, it also discusses the
psychology of the reader as they're looking at a data visual, which
offers a unique and useful perspective.

\textbf{Author: Payman Taei,2017} URL:
\url{https://towardsdatascience.com/3-expert-data-visualization-tips-for-grabbing-readers-attention-206d8c4621bf}
\citep{3_Expert_Data_Visualization_Tips}

\begin{verbatim}
@misc{3_Expert_Data_Visualization_Tips,
  author = {{Payman Taei}},
  year = {2017},
  title = {3_Expert_Data_Visualization_Tips},
  howpublished = {\url{https://towardsdatascience.com/3-expert-data-visualization-tips-for-grabbing-readers-attention-206d8c4621bf}},
  note = {Accessed: 2018-05-12}
}
\end{verbatim}

\emph{Summary}: This article found on Medium explores three important
aspects to focus on when creating a data visualization. The importance
of each aspect is explained along with helpful questions to ask and to
help you evaluate your visualization to ensure it caters to your
audience. Although it primarily focuses on the appearance of visuals, it
also discusses the psychology of the reader as they're looking at a data
visual, which offers a unique and useful perspective.

Here is an outline of each of the 3 aspects: 1. Know what you really
want to say. We want to share patterns, trends, anomalies, etc. with
others through data visuals but we must find the right things to
represent.

\begin{enumerate}
\def\labelenumi{\arabic{enumi}.}
\setcounter{enumi}{1}
\item
  Design. Visuals should be kept as simple as possible without leaving
  out key points. This makes sense because then the audience can focus
  on what's really important.
\item
  Labeling. This section of the article shows a nice comparison of
  before and after removing labels from a chart, and the `after' chart
  looks much cleaner and easier to interpret.
\end{enumerate}

I think often when working with data, we tend to gravitate toward
including more information in a visual, so an important takeaway for me
is that less is more, and not everything we want to show has to be
crammed into one big-picture visual.

\textbf{2. Choose best colors for cartography visualization in a
professional manner} \textbf{Author: © Cynthia Brewer, Mark Harrower and
The Pennsylvania State University} URL: \url{http://colorbrewer.org}
\citep{best_colors_for_cartography_visualization}

\begin{verbatim}
@misc{best_colors_for_cartography_visualization,
  author = {{Cynthia Brewer, Mark Harrower and The Pennsylvania State University}},
  title = {best_colors_for_cartography_visualization},
  howpublished = {\url{http://colorbrewer.org}},
  note = {Accessed: 2018-05-12}
}
\end{verbatim}

\emph{Summary}: It has been carefully designed to be a diagnostic tool
for evaluating the robustness of individual color schemes. Full use of
this tool will benefit your map designs because colors (even very
similar colors) are easy to differentiate when they appear in a nicely
ordered sequence (such as a legend). The task of differentiating the
colors, however, becomes much harder when the patterns on the map are
complex, such as in the lower left corner of the diagnostic map.

It will automatically recommend the color schemes in the following
aspects:

\begin{enumerate}
\def\labelenumi{\arabic{enumi}.}
\item
  Can you easily distinguish every color in the random section of the
  map (the lower left)? If you have a ten-class map, you should be able
  to see clearly ten unique colors.
\item
  Within each large band of color on the map, we placed several polygons
  filled with each map color (`outliers'). For example, if you have a
  seven-class map, there will be six outlier colors per band,
  demonstrating the appearance of all map colors with each as a
  surrounding color. Can you see each outlier clearly? Do all pairs of
  outliers in the band look different? If not, perhaps you should choose
  a different scheme or fewer classes.
\item
  You can also change the settings to colorblind-friendly on this site.
\end{enumerate}

\textbf{3. Visualization Tools: An introduction to tools for creating
infographics, timelines and other data visualizations.}
\citep{viz_tools}

\textbf{Author:Jess Rios, 2017}
URL:\url{https://guides.library.harvard.edu/c.php?g=310952\&p=2073191}
\citep{Visualization_Tools}

\begin{verbatim}
@misc{Visualization_Tools,
  author = {{Jess Rios}},
  title = {Visualization_Tools},
  howpublished = {\url{https://guides.library.harvard.edu/c.php?g=310952&p=2073191}},
  note = {Accessed: 2018-05-12}
}
\end{verbatim}

This website lists lots of tools to do different types of
visualizations.

\textbf{4. Visual Capitalist} \citep{visual_cap}

This company/website creates visual contents in the field of business
and marketing.

\textbf{5. Misleading Graphs}

5.1: \citep{venngage}

5.2: \citep{wiki_mislead}

5.3: \citep{data_analysis_deception}

\emph{Summary}: As a student to learn how to be a good data scientis or
business analytics professional, it is important to learn how to read
the chart and interpret the statistic. Graphs can be one of the best
ways to present statistical information, but they can also be one of the
most misleading, even when they are completely accurate. Here, I would
like to share how to detect misleading graphs.Furthermore, we can learn
how to improve our data visualization skills.

5.1. Omitting Baselines

In the data visulization terms, we call it truncated graph. A truncated
graph (also known as a torn graph) has a y axis that does not start at
0. These graphs can create the impression of important change where
there is relatively little change.Truncated graphs are useful in
illustrating small differences.{[}16{]} Graphs may also be truncated to
save space. Commercial software such as MS Excel will tend to truncate
graphs by default if the values are all within a narrow range.
Truncating graphs make the readers to change their judement for
something that is not significant looks like a huge differece.

A example of using good data in a misleading graph to fool readers comes
from Fox News. \includegraphics{images/1.png}

5.2. Axis Manipulation

Another trick of misleading graph is axis change: Changing thy y-axis
maximum afftect how the graph look like. A higher maximum will make the
grpha to appear less volatiliy ,less strrp than a lower maximum. The
other way of axis change is changing the ratio of a graph's dimensions.
This way will affect how the graph appears. We demostrate chaning the
ratio of graph dimension for below graphs.
\includegraphics{images/Line_graph1.svg.png}
\includegraphics{images/175px-Line_graph1-3.svg.png}
\includegraphics{images/200px-Line_graph1-4.svg.png}

Axis manipulation is the opposite of truncating data, because they
include the axis and baselines but change them so much that they lose
meaning. This type of graph manipulation can be used to push a false
narrative.

5.3.Cherry Picking Data This is to pick the data that shows a typical
viewpoint. For example, we know house price in Bay area kept increasing
since 2011. However, for those house agencies who want convince buyers
that house prices has decreased , they might select some areas in
typical months that house prices happend to decreased.

It is not technically wrong but it is definitely misleading.This is
often called improper extraction or tactic omitting data, when only a
certain chunk of data is included.This is more common in graphs that
have time as one of their axis. Here is the graph to show what it is.
\includegraphics{images/Bad_graph_extraction.png}
\includegraphics{images/Good_graph_extraction.png}

\textbf{6. The Year in Visual Stories and Graphics: New York Times}
URL:\url{http://www.visualcapitalist.com/category/politics/}
\citep{Visual_Capitalist}

\begin{verbatim}
@misc{Visual_Capitalist,
  title = {Visual_Capitalist},
  howpublished = {\url{Visual Capitalist}},
  note = {Accessed: 2018-05-12}
}
\end{verbatim}

This company/website creates visual contents in the field of business
and marketing.

\textbf{5. The Year in Visual Stories and Graphics: New York Times}
\textbf{Author:The New York Times, 2013} URL:
\url{http://www.nytimes.com/newsgraphics/2013/12/30/year-in-interactive-storytelling/index.html\#dataviz}
\citep{The_Year_in_Visual_Stories_and_Graphics}

\begin{verbatim}
@misc{The_Year_in_Visual_Stories_and_Graphics,
  author = {{The New York Times}},
  year = {2013},
  title = {The_Year_in_Visual_Stories_and_Graphics},
  howpublished = {\url{http://www.nytimes.com/newsgraphics/2013/12/30/year-in-interactive-storytelling/index.html#dataviz}},
  note = {Accessed: 2018-05-12}
}
\end{verbatim}

\emph{Summary: }Every year, New York Times will select a collection of
the year's best storyteller visualizations, which includes different
forms of state-of-the-art news visualizations. Personally, I think it
can be a good inspiration when we feel like ``I don't know how to be
creative on this''.

\bibliography{book.bib,packages.bib}


\end{document}
