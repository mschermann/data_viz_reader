\documentclass[]{book}
\usepackage{lmodern}
\usepackage{amssymb,amsmath}
\usepackage{ifxetex,ifluatex}
\usepackage{fixltx2e} % provides \textsubscript
\ifnum 0\ifxetex 1\fi\ifluatex 1\fi=0 % if pdftex
  \usepackage[T1]{fontenc}
  \usepackage[utf8]{inputenc}
\else % if luatex or xelatex
  \ifxetex
    \usepackage{mathspec}
  \else
    \usepackage{fontspec}
  \fi
  \defaultfontfeatures{Ligatures=TeX,Scale=MatchLowercase}
\fi
% use upquote if available, for straight quotes in verbatim environments
\IfFileExists{upquote.sty}{\usepackage{upquote}}{}
% use microtype if available
\IfFileExists{microtype.sty}{%
\usepackage{microtype}
\UseMicrotypeSet[protrusion]{basicmath} % disable protrusion for tt fonts
}{}
\usepackage[margin=1in]{geometry}
\usepackage{hyperref}
\hypersetup{unicode=true,
            pdftitle={A Reader on Data Visualization},
            pdfauthor={MSIS 2629 Spring 2018},
            pdfborder={0 0 0},
            breaklinks=true}
\urlstyle{same}  % don't use monospace font for urls
\usepackage{natbib}
\bibliographystyle{apalike}
\usepackage{color}
\usepackage{fancyvrb}
\newcommand{\VerbBar}{|}
\newcommand{\VERB}{\Verb[commandchars=\\\{\}]}
\DefineVerbatimEnvironment{Highlighting}{Verbatim}{commandchars=\\\{\}}
% Add ',fontsize=\small' for more characters per line
\usepackage{framed}
\definecolor{shadecolor}{RGB}{248,248,248}
\newenvironment{Shaded}{\begin{snugshade}}{\end{snugshade}}
\newcommand{\KeywordTok}[1]{\textcolor[rgb]{0.13,0.29,0.53}{\textbf{#1}}}
\newcommand{\DataTypeTok}[1]{\textcolor[rgb]{0.13,0.29,0.53}{#1}}
\newcommand{\DecValTok}[1]{\textcolor[rgb]{0.00,0.00,0.81}{#1}}
\newcommand{\BaseNTok}[1]{\textcolor[rgb]{0.00,0.00,0.81}{#1}}
\newcommand{\FloatTok}[1]{\textcolor[rgb]{0.00,0.00,0.81}{#1}}
\newcommand{\ConstantTok}[1]{\textcolor[rgb]{0.00,0.00,0.00}{#1}}
\newcommand{\CharTok}[1]{\textcolor[rgb]{0.31,0.60,0.02}{#1}}
\newcommand{\SpecialCharTok}[1]{\textcolor[rgb]{0.00,0.00,0.00}{#1}}
\newcommand{\StringTok}[1]{\textcolor[rgb]{0.31,0.60,0.02}{#1}}
\newcommand{\VerbatimStringTok}[1]{\textcolor[rgb]{0.31,0.60,0.02}{#1}}
\newcommand{\SpecialStringTok}[1]{\textcolor[rgb]{0.31,0.60,0.02}{#1}}
\newcommand{\ImportTok}[1]{#1}
\newcommand{\CommentTok}[1]{\textcolor[rgb]{0.56,0.35,0.01}{\textit{#1}}}
\newcommand{\DocumentationTok}[1]{\textcolor[rgb]{0.56,0.35,0.01}{\textbf{\textit{#1}}}}
\newcommand{\AnnotationTok}[1]{\textcolor[rgb]{0.56,0.35,0.01}{\textbf{\textit{#1}}}}
\newcommand{\CommentVarTok}[1]{\textcolor[rgb]{0.56,0.35,0.01}{\textbf{\textit{#1}}}}
\newcommand{\OtherTok}[1]{\textcolor[rgb]{0.56,0.35,0.01}{#1}}
\newcommand{\FunctionTok}[1]{\textcolor[rgb]{0.00,0.00,0.00}{#1}}
\newcommand{\VariableTok}[1]{\textcolor[rgb]{0.00,0.00,0.00}{#1}}
\newcommand{\ControlFlowTok}[1]{\textcolor[rgb]{0.13,0.29,0.53}{\textbf{#1}}}
\newcommand{\OperatorTok}[1]{\textcolor[rgb]{0.81,0.36,0.00}{\textbf{#1}}}
\newcommand{\BuiltInTok}[1]{#1}
\newcommand{\ExtensionTok}[1]{#1}
\newcommand{\PreprocessorTok}[1]{\textcolor[rgb]{0.56,0.35,0.01}{\textit{#1}}}
\newcommand{\AttributeTok}[1]{\textcolor[rgb]{0.77,0.63,0.00}{#1}}
\newcommand{\RegionMarkerTok}[1]{#1}
\newcommand{\InformationTok}[1]{\textcolor[rgb]{0.56,0.35,0.01}{\textbf{\textit{#1}}}}
\newcommand{\WarningTok}[1]{\textcolor[rgb]{0.56,0.35,0.01}{\textbf{\textit{#1}}}}
\newcommand{\AlertTok}[1]{\textcolor[rgb]{0.94,0.16,0.16}{#1}}
\newcommand{\ErrorTok}[1]{\textcolor[rgb]{0.64,0.00,0.00}{\textbf{#1}}}
\newcommand{\NormalTok}[1]{#1}
\usepackage{longtable,booktabs}
\usepackage{graphicx,grffile}
\makeatletter
\def\maxwidth{\ifdim\Gin@nat@width>\linewidth\linewidth\else\Gin@nat@width\fi}
\def\maxheight{\ifdim\Gin@nat@height>\textheight\textheight\else\Gin@nat@height\fi}
\makeatother
% Scale images if necessary, so that they will not overflow the page
% margins by default, and it is still possible to overwrite the defaults
% using explicit options in \includegraphics[width, height, ...]{}
\setkeys{Gin}{width=\maxwidth,height=\maxheight,keepaspectratio}
\IfFileExists{parskip.sty}{%
\usepackage{parskip}
}{% else
\setlength{\parindent}{0pt}
\setlength{\parskip}{6pt plus 2pt minus 1pt}
}
\setlength{\emergencystretch}{3em}  % prevent overfull lines
\providecommand{\tightlist}{%
  \setlength{\itemsep}{0pt}\setlength{\parskip}{0pt}}
\setcounter{secnumdepth}{5}
% Redefines (sub)paragraphs to behave more like sections
\ifx\paragraph\undefined\else
\let\oldparagraph\paragraph
\renewcommand{\paragraph}[1]{\oldparagraph{#1}\mbox{}}
\fi
\ifx\subparagraph\undefined\else
\let\oldsubparagraph\subparagraph
\renewcommand{\subparagraph}[1]{\oldsubparagraph{#1}\mbox{}}
\fi

%%% Use protect on footnotes to avoid problems with footnotes in titles
\let\rmarkdownfootnote\footnote%
\def\footnote{\protect\rmarkdownfootnote}

%%% Change title format to be more compact
\usepackage{titling}

% Create subtitle command for use in maketitle
\newcommand{\subtitle}[1]{
  \posttitle{
    \begin{center}\large#1\end{center}
    }
}

\setlength{\droptitle}{-2em}
  \title{A Reader on Data Visualization}
  \pretitle{\vspace{\droptitle}\centering\huge}
  \posttitle{\par}
  \author{MSIS 2629 Spring 2018}
  \preauthor{\centering\large\emph}
  \postauthor{\par}
  \predate{\centering\large\emph}
  \postdate{\par}
  \date{2018-04-30}

\usepackage{booktabs}
\usepackage{amsthm}
\makeatletter
\def\thm@space@setup{%
  \thm@preskip=8pt plus 2pt minus 4pt
  \thm@postskip=\thm@preskip
}
\makeatother

\usepackage{amsthm}
\newtheorem{theorem}{Theorem}[chapter]
\newtheorem{lemma}{Lemma}[chapter]
\theoremstyle{definition}
\newtheorem{definition}{Definition}[chapter]
\newtheorem{corollary}{Corollary}[chapter]
\newtheorem{proposition}{Proposition}[chapter]
\theoremstyle{definition}
\newtheorem{example}{Example}[chapter]
\theoremstyle{definition}
\newtheorem{exercise}{Exercise}[chapter]
\theoremstyle{remark}
\newtheorem*{remark}{Remark}
\newtheorem*{solution}{Solution}
\begin{document}
\maketitle

{
\setcounter{tocdepth}{1}
\tableofcontents
}
\chapter{Preface}\label{preface}

This is a collaborative writing project as part of the course MSIS 2629
``Data Visualization'' at \href{http://www.scu.edu}{Santa Clara
University}. The purpose of the class reader is to collaboratively
engage with and reflect on data visualizations, to establish a solid
theoretical background, and to collect useful practices and showcases.
More information on the background of this project is available in the
\href{https://mschermann.github.io/msis2629spring2018}{syllabus}.

The following text serves explains how we organize ourselves.

\section{References}\label{references}

\textbf{EVERY} references must be included in the \texttt{book.bib}
file. This file uses the bibtex notation (Learn how to use bibtex
\href{http://www.bibtex.org/Using/}{here}.). Most literature search
engines allow you to export the reference information in Bibtex. For
websites we use the following minimal notation (you may add further
information - usually the more the better is a good strategy):

\begin{verbatim}
@misc{great_viz,
  author = {{A great visualizer}},
  year = {1982},
  title = {A ficticious web page title},
  howpublished = {\url{http://great_viz_org/}},
  note = {Accessed: 2018-04-26}
}
\end{verbatim}

Particularly important is the \texttt{note} field. Websites change
frequently, so links will break. If we do this correctly,
\texttt{{[}@great\_viz{]}} will produce \citep{great_viz}.

\section{Images}\label{images}

Images should not be loaded from external website because the links may
change. Instead download a version of the image and create a reference
that contains the link to the image. For example the following image is
a deceptive visualization (the bars do start at zero).

\begin{figure}
\centering
\includegraphics{images/halper_welfare.jpg}
\caption{An Example of a deceptive visualization}
\end{figure}

Source: \citep{halper_2012} referenced in \citep{andale_2014}

The citation for the image looks like this.

\begin{verbatim}
@misc{halper_2012,
  author={Halper, Daniel},
  year={2012},
  title = {Over 100 Million Now Receiving Federal Welfare},
  url={https://www.weeklystandard.com/daniel-halper/over-100-million-now-receiving-federal-welfare},
  note = {Accessed: 2018-04-26}
}
\end{verbatim}

You have probably found this image through a different website that
explains the visualization. For example the following website explains
some problematic aspects of this visualization:

\begin{verbatim}
@misc{andale_2014,
  author={Andalde, Stephanie},
  year={2014},
  title = {Misleading Graphs: Real Life Examples},
  url={http://www.statisticshowto.com/misleading-graphs/},
  note = {Accessed: 2018-04-26}
}
\end{verbatim}

\chapter{Introduction}\label{intro}

\textbf{Data Visualization} Data visualization refers to representing
data in a visual context to help people understand the significance of
that data. A way so that information, numbers, and measurements makes
sense is a form of art -- the art of data visualization. Graphs do that
for us.Below is a link for different types of plots used in data
visualization.

Plot links: \url{https://datavizcatalogue.com/search.html}

Infogram helps a user make different types of plots and learn the art of
visualization. Below is the link:
\url{https://infogram.com/page/data-visualization}

Some useful links are mentioned below:

\url{https://eagereyes.org/}
\url{https://research.tableau.com/user/robert-kosara}
\url{https://twitter.com/eagereyes?ref_src=twsrc\%5Egoogle\%7Ctwcamp\%5Eserp\%7Ctwgr\%5Eauthor}
\url{http://flowingdata.com/}

\textbf{Why is data visualization important?}

You can label chapter and section titles using \texttt{\{\#label\}}
after them, e.g., we can reference Chapter \ref{intro}. If you do not
manually label them, there will be automatic labels anyway, e.g.,
Chapter \ref{methods}.
\url{https://research.tableau.com/user/robert-kosara}
\url{https://twitter.com/eagereyes?ref_src=twsrc\%5Egoogle\%7Ctwcamp\%5Eserp\%7Ctwgr\%5Eauthor}

\url{https://data-visualization.cioreview.com/cxoinsight/what-is-data-visualization-and-why-is-it-important-nid-11806-cid-163.html}

The article , written by Chris Pittenturf, VP-Data \& Analytics, Palace
Sports \& Entertainment, talks about what data visualization is and its
importance to the businesses today. The article begins with a definition
of data visualization in simple terms and goes on to explain how a good
data visualization should be visually engaging to the reader. Chris goes
on to explain the basic criterias that a data visualization should
satisfy to be an effective visualization. These criterias and their
brief meanings are as follows: 1. Informative: The visualization should
be able to convey the information of the data to the reader 2.
Efficient: The visualization should not be ambiguous. 3. Appealing: The
visualization should be captivating and visually pleasing. 4. (Optional)
Interactive and Predictive: The visualizations can contain variables and
filters for the users to interact with the visualizations in order to
predict results of different scenarios.

Chris goes on to give various day-to-day examples where visualization
gives a better understanding of the data. One extremely simple example
used by Chris is that of an energy bill. Chris states that as a
consumer, when we receive an energy bill, we normally look at the graph
in the bill first before proceeding to read the text in the bill. Chris
states that consumers are more likely to analyze and understand the
visualizations before reading further along. The article ends with Chris
emphasizing the importance of data visualizations in our businesses as
well as in our daily lives. According to me, the article gives a simple,
short and crisp understanding of what data visualization is and how it
is relevant to everyone. It shows that data visualization is an aid to
get a better understanding of the complex insights that any business
data provides. Most of the data used by the businesses is highly
unstructured and these businesses can get a better understanding of
their businesses by visualizing their data.

\url{https://www.interaction-design.org/literature/article/information-visualization-a-brief-introduction}
This article is a brief introduction to Information Visualization. It
explains briefly how information visualization helps to make sense of
data, how it helps to find relationships between data and confirm ideas.

About David McCandless's TED talk on data visualization:
\url{https://www.ted.com/talks/david_mccandless_the_beauty_of_data_visualization}

Visuals help us understand concepts that would otherwise be difficult to
contextualize---for example, expenditures or valuations of extremely
large amounts of money are represented in the billion dollar-o-gram by
color-coded, relatively-sized boxes. Furthermore, it allows synthesis of
a breadth of information to be delivered in a small, easily-digestible,
aesthetically pleasing way. Visuals serve as a sort of map for a vast
landscape of information---they direct your eyes to the important places
and details. And the eye, as McCandless notes, is uniquely suited among
our senses to process large amounts of information and detect patterns.

The billion dollar-o-gram is extremely readable and rather pretty, but
it seems a bit dubious to compare the predicted Iraq War cost to the
``mushroomed'' actual cost of Iraq and Afghanistan wars, since its
purpose seems only to conflate two wars for dramatic effect.

Beyond its ability to make information from several different sources
and in large amounts more quickly and easily understood, data
visualization can also reveal smaller interesting patterns---allowing us
to play the ``data detective'' as McCandless calls it. In other words,
as we have already discussed, data visualization can not only be
extremely effective in a declarative manner, but can also be used as an
exploratory tool.

McCandless also postulates that we all have a latent ``design literacy''
that is being developed every day as we are constantly bombarded with
visuals, and that our minds and our eyes are taking in this information
and processing it so that we all have an intuitive sense of design, and
have actually begun to demand a visual aspect to our information. This
is an interesting perspective, since everyone does seem to have a sense
of visual aspects---space, color, etc., but of course the time-honored
adage tells us that beauty is in the eye of the beholder. So while it
might be whimsical to claim that we are all designers, there is still,
of course, great value in learning formal principles of design.

Basic Guidelines: Figures and tables with captions will be placed in
\texttt{figure} and \texttt{table} environments, respectively.

\textbf{Key Figures in the History of Data Visualization}

\textbf{Reference}

Infogram,Jun 14,2016.Key Figures in the History of Data
Visualization,Medium.
\url{https://medium.com/@Infogram/key-figures-in-the-history-of-data-visualization-30486681844c}

The history of data visualization is full of incredible stories marked
by major events, led by a few key players. This article introduces some
of the amazing men and women who paved the way by combining art,
science, and statistics.And one of them is Charles Joseph Minard whose
most famous work is the map of Napoleon's Russian campaign of 1812
displayed in our class. I present several figures names with their
famous works and you can find other stories in the article.

\textbf{William Playfair (1759--1823)}

William Playfair is considered the father of statistical graphics,
having invented the line and bar chart we use so often today. He is also
credited with having created the area and pie chart. Playfair was a
Scottish engineer and political economist who published The Commercial
and Political Atlas in 1786.

This book featured a variety of graphs including the image below. In
this famous example, he compares exports from England with imports into
England from Denmark and Norway from 1700 to 1780.

\textbf{John Snow (1813--1858)}

In 1854, a cholera epidemic spread quickly through Soho in London. The
Broad Street area had seen over 600 dead, and the remaining residents
and business owners had largely fled the terrible disease.

Physician John Snow plotted the locations of cholera deaths on a map.
The surviving maps of his work show a method of tallying the death
counts, drawn as lines parallel to the street, at the appropriate
addresses. Snow's research revealed a pattern. He saw a clear
concentration around the water pump on Broad Street, helping to find the
cause of the infection.

\textbf{Charles Joseph Minard (1781--1870)}

Charles Joseph Minard was a French civil engineer famous for his
representation of numerical data on maps. His most famous work is the
map of Napoleon's Russian campaign of 1812 displaying the dramatic loss
of his army over the advance on Moscow and the following retreat.

You can see how many soldiers are still marching and how many died.
Drawn in 1869, it is described by many as the best statistical graphic
ever drawn. It represents the earliest beginnings of data journalism.

\begin{Shaded}
\begin{Highlighting}[]
\KeywordTok{par}\NormalTok{(}\DataTypeTok{mar =} \KeywordTok{c}\NormalTok{(}\DecValTok{4}\NormalTok{, }\DecValTok{4}\NormalTok{, .}\DecValTok{1}\NormalTok{, .}\DecValTok{1}\NormalTok{))}
\KeywordTok{plot}\NormalTok{(pressure, }\DataTypeTok{type =} \StringTok{'b'}\NormalTok{, }\DataTypeTok{pch =} \DecValTok{19}\NormalTok{)}
\end{Highlighting}
\end{Shaded}

\begin{figure}

{\centering \includegraphics[width=0.8\linewidth]{Data_Viz_Reader_files/figure-latex/nice-fig-1} 

}

\caption{Here is a nice figure!}\label{fig:nice-fig}
\end{figure}

Reference a figure by its code chunk label with the \texttt{fig:}
prefix, e.g., see Figure \ref{fig:nice-fig}. Similarly, you can
reference tables generated from \texttt{knitr::kable()}, e.g., see Table
\ref{tab:nice-tab}.

\begin{Shaded}
\begin{Highlighting}[]
\NormalTok{knitr}\OperatorTok{::}\KeywordTok{kable}\NormalTok{(}
  \KeywordTok{head}\NormalTok{(iris, }\DecValTok{20}\NormalTok{), }\DataTypeTok{caption =} \StringTok{'Here is a nice table!'}\NormalTok{,}
  \DataTypeTok{booktabs =} \OtherTok{TRUE}
\NormalTok{)}
\end{Highlighting}
\end{Shaded}

\begin{table}

\caption{\label{tab:nice-tab}Here is a nice table!}
\centering
\begin{tabular}[t]{rrrrl}
\toprule
Sepal.Length & Sepal.Width & Petal.Length & Petal.Width & Species\\
\midrule
5.1 & 3.5 & 1.4 & 0.2 & setosa\\
4.9 & 3.0 & 1.4 & 0.2 & setosa\\
4.7 & 3.2 & 1.3 & 0.2 & setosa\\
4.6 & 3.1 & 1.5 & 0.2 & setosa\\
5.0 & 3.6 & 1.4 & 0.2 & setosa\\
\addlinespace
5.4 & 3.9 & 1.7 & 0.4 & setosa\\
4.6 & 3.4 & 1.4 & 0.3 & setosa\\
5.0 & 3.4 & 1.5 & 0.2 & setosa\\
4.4 & 2.9 & 1.4 & 0.2 & setosa\\
4.9 & 3.1 & 1.5 & 0.1 & setosa\\
\addlinespace
5.4 & 3.7 & 1.5 & 0.2 & setosa\\
4.8 & 3.4 & 1.6 & 0.2 & setosa\\
4.8 & 3.0 & 1.4 & 0.1 & setosa\\
4.3 & 3.0 & 1.1 & 0.1 & setosa\\
5.8 & 4.0 & 1.2 & 0.2 & setosa\\
\addlinespace
5.7 & 4.4 & 1.5 & 0.4 & setosa\\
5.4 & 3.9 & 1.3 & 0.4 & setosa\\
5.1 & 3.5 & 1.4 & 0.3 & setosa\\
5.7 & 3.8 & 1.7 & 0.3 & setosa\\
5.1 & 3.8 & 1.5 & 0.3 & setosa\\
\bottomrule
\end{tabular}
\end{table}

You can write citations, too. For example, we are using the
\textbf{bookdown} package \citep{R-bookdown} in this sample book, which
was built on top of R Markdown and \textbf{knitr} \citep{xie2015}.

\chapter{Fundamentals}\label{fundamentals}

\url{https://www.educba.com/data-mining-vs-data-visualization/}

This article gives me a clear understanding of data mining and data
visualization.

\url{https://www.educba.com/data-mining-vs-data-visualization/}

This article gives me a clear understanding of data mining and data
visualization.

In Data Mining, there are different processes involve carrying out the
data mining process such as data extraction, data management, data
transformations, data pre-processing, etc. In Data Visualization, the
primary goal is to convey the information efficiently and clearly
without any deviations or complexities in the form of statistical
graphs, information graphs, and plots. Also, the author listed the top 7
comparisons between data mining and data visualization, and 12 key
differences between data mining and data visualization. After reading
the article, you will have a very clear understanding of what are data
mining and data visualization and the characters for those two
techniques.

In Data Mining, there are different processes involve carrying out the
data mining process such as data extraction, data management, data
transformations, data pre-processing, etc. In Data Visualization, the
primary goal is to convey the information efficiently and clearly
without any deviations or complexities in the form of statistical
graphs, information graphs, and plots. Also, the author listed the top 7
comparisons between data mining and data visualization, and 12 key
differences between data mining and data visualization. After reading
the article, you will have a very clear understanding of what are data
mining and data visualization and the characters for those two
techniques.

\begin{itemize}
\tightlist
\item
  Theoretical background of data visualization
\end{itemize}

History Data visualization has comes a long way. Prior to the 17th
century, data visualization already exists. Though display in other
format such as maps, the content are much similar to today's
visualization, which mostly presented geologic, economic, and medical
data. Here is useful link:
\url{http://www.dashboardinsight.com/news/news-articles/the-history-of-data-visualization.aspx}

Current research: Deceptive visualizations Data visualization is a
powerful communication tool to support arguments with numbers in a way
that is accessible and engaging. More people than ever before are making
their own charts and infographics, which is presenting a unique problem.
Despite the availability of some great charting resources, we are
witnessing an influx of poorly-designed misleading or downright
deceptive data visualizations. Here are useful links:
\url{https://medium.com/@Infogram/study-asks-how-deceptive-are-deceptive-visualizations-8ff52fd81239}
\url{https://www.datapine.com/blog/misleading-data-visualization-examples/}

\section{1. A Brief History of Data
Visualization}\label{a-brief-history-of-data-visualization}

Michael Friendly,2006,A Brief History of Data Visualization,York
University.\url{http://www.datavis.ca/papers/hbook.pdf}

\begin{verbatim}
The only new thing in the world is the history you don’t know. — Harry S Truman

This paper provides an overview of the intellectual history of data visualization from medieval to modern times,
describing and illustrating some significant advances along the way.
\end{verbatim}

\begin{enumerate}
\def\labelenumi{\arabic{enumi}.}
\item
  Data Visualization: modern product?

  It is common to think of statistical graphics and data visualization
  as relatively modern developments in statistics. In fact, the graphic
  representation of quantitative information has deep roots.These roots
  reach into the histories of the earliest map-making and visual
  depiction, and later into thematic cartography, statistics and
  statistical graphics, medicine, and other fields.

  Developments in technologies (printing, reproduction) mathematical
  theory and practice, and empirical observation and\\
  recording, enabled the wider use of graphics and new advances in form
  and content.
\item
  Milestones Tour

  2.1 Pre-17th Century: Early maps and diagrams

\begin{verbatim}
  The earliest seeds of visualization arose in geometric diagrams, in tables of the positions of stars and other
  celestial bodies, and in the making of maps to aid in navigation and exploration. 
\end{verbatim}

  2.2 1600-1699: Measurement and theory

\begin{verbatim}
  Among the most important problems of the 17th century were those concerned with physical measurement— of time,
  distance,and space— for astronomy, surveying, map making, navigation and territorial expansion. This century also
  saw great new growth in theory and the dawnof practical application.
\end{verbatim}

  2.3 1700-1799: New graphic forms

\begin{verbatim}
  With some rudiments of statistical theory, data of interest and importance, and the idea of graphic representation
  at least somewhat established, the 18th century witnessed the expansion of these aspects to new domains and new
  graphic forms. 
\end{verbatim}

  2.4 1800-1850: Beginnings of modern graphics

\begin{verbatim}
  With the fertilization provided by the previous innovations of design and technique, the first half of the 19th
  century witnessed explosive growth in statistical graphics and thematic mapping, at a rate which would not be
  equalled until modern times.
\end{verbatim}

  2.5 1850--1900: The Golden Age of statistical graphics

\begin{verbatim}
  By the mid-1800s, all the conditions for the rapid growth of visualization had been established— a “perfect storm”
  for data graphics. Official state statistical offices were established throughout Europe, in recognition of the
  growing importance of numerical information for social planning,industrialization, commerce, and transportation. 

   2.5.1 Escaping flatland
   2.5.2 Graphical innovations
   2.5.3 Galton’s contributions
   2.5.4 Statistical Atlases
\end{verbatim}

  2.6 1900-1950: The modern dark ages

\begin{verbatim}
  If the late 1800s were the “golden age” of statistical graphics and thematic cartography, the early 1900s can be
  called the “modern dark ages” of visualization. There were few graphical innovations, and, by the mid-1930s, the
  enthusiasm for visualization which characterized the late 1800s had been supplanted by the rise of quantification
  and formal, often statistical, models in the social sciences.
\end{verbatim}

  2.7 1950--1975: Re-birth of data visualization

\begin{verbatim}
  Still under the influence of the formal and numerical zeitgeist from the mid-1930s on, data visualization began to
  rise from dormancy in the mid 1960s. 
\end{verbatim}

  2.8 1975--present: High-D, interactive and dynamic data visualization

\begin{verbatim}
  During the last quarter of the 20th century data visualization has blossomed into a mature, vibrant and multi
  disciplinary research area, as may be seen in this Handbook, and software tools for a wide range of visualization
  methods and data types are available for every desktop computer.
\end{verbatim}
\end{enumerate}

\section{2. Fundamental Components of
Design}\label{fundamental-components-of-design}

Artists use balance, emphasis, movement, pattern, repetition,
proportion, rhythm, variety, and unity as the design foundation of any
work. If you want to take your data visualization from an everyday
dashboard to a compelling data story, incorporate the 9 principles of
design from graphic designer Melissa Anderson's article:
\url{https://www.idashboards.com/blog/2017/07/26/data-visualization-and-the-9-fundamental-design-principles/}

Balance doesn't mean that each side of the visualization needs perfect
symmetry, but it is important to have the elements of the
dashboard/visualiaztion distributed evenly. And it important to remember
the non-data elements, such as a logo, title, caption, etc., that can
affect the balance of the display.

Another closely related component to balance is variety which could seem
counter to balance, but when done correctly, variety can help increase
the recall of information. However if overdone, too much variety can
feel cluttered and blur together the images and data in the mind of the
viewer.

Arguably the most critical of the components is proportion. Proportion
can be subtle but it can go a long way to enhancing a viewer's
experience and understanding of the data. The danger of proportion
though is that it can be easy to deceive people subconsciously.
Naturally images will have a greater impact on how our brains perceive
the dashboard or visualization. For example, someone can change the
scale of a graph or images to inflate their results and even if they
write the numbers next to it, the shortcut many people will take is to
interpret the data based on the image. This is why it is important we
take care to accurately reflect proportion in our data visualization and
remain critical of how others use proportion in their visualization.

Emphasis was the component that I most related to when reading through
the nine principles of design in this article. From prior experience
with art through photography I understand it is key to be concious of
what I am drawing the viewers attention to in my art. When thinking
about the art design of data visualization it is also very important to
remain keen on the main point of your story and how the entire
visualization is either drawing the viewer to that point of emphasis or
how they are being distracted or drawn elsewhere.

\section{3. Guide to Best Practices in Data
Visualization}\label{guide-to-best-practices-in-data-visualization}

These are the best practices of data visualization. Anticipate in
advance what kind of questions the viewers will ask and then focus your
visualization with respect to those questions.

The brain processes stimuli from our environment to process what is
important in 2 ways -- unconscious (System 1 represents uncontrolled
functions such as facial expressions, reactions) and conscious (System 2
-- represents controlled function such as solving math problems). Data
Visualization leverage attributes of System -1 which can have a quick
and correct impact in a most efficient manner. The three best practices
of data visualization are as follows:

** 1. Design and layout matter \textbf{ The design and layout should
facilitate ease of understanding to convey your message to the viewer. }
2. Avoid Clutter \textbf{ Keep it simple. To implement this always keep
into account the data-ink ratio -- the ratio of ink required to convey
the intended meaning to the total amount of ink used in the table or
chart should be as close to 1 as possible. That means, avoid ink which
do not add any information. } 3. Use color purposely and effectively **
Use of color may be prettier and attractive but can be distractive too.
Thus, color should be used only if it assists in conveying your message.
The above three principles are illustrated with the help of scenarios
and examples which helps to comprehend the topic in more meaningful and
practical way in the article. It also gives various advantages of using
the above principles.And the above best practices could be applied to
all the 3 types of analytics: descriptive, predictive and prescriptive.

\textbf{Reference} Jeffrey D. Camm, Michael J. Fry, Jeffrey Shaffer
(2017) A Practitioner's Guide to Best Practices in Data
Visualization.Interfaces 47(6):473-488.
\url{https://doi.org/10.1287/inte.2017.0916}

\section{4. Survey of Popular Data Visualization
Tools}\label{survey-of-popular-data-visualization-tools}

Due to the rise of big data analytics, there has been an increased need
for data visualization tools to help understand the data. Besides
Tableau, there are several other software tools one can use for data
visualization like Sisense, Plotly, FusionCharts, Highcharts,
Datawrapper, and Qlikview. This article is from forbes and has a brief,
clear introduction about these 7 powerful software options for data
visualization. This could be helpful for future reference because for
different purposes I may need to use different tools. Each option has
its advantages and disadvantages and this article helps highlight them.

\textbf{Tableau} is the most popular of the group and has many users. It
is simple to use, making it easy to learn and can handle large datasets.
Tableau can handle big data thanks to integration with database handling
applications such as MySQL, Hadoop, and Amazon AWS.

\textbf{Qlikview} is the main competitor to Tableau and is also quite
popular. Qlikview is customizable and has a wide range of features which
can be a double-edged sword. These features take more time to learn and
get acquianted with. However, once one gets past the learning curve,
they have a powerful tool at their disposal.

The distinctive aspect of \textbf{FusionCharts} is that graphics do not
have to be created from scratch. Users can start with a template and
insert their own data from their project.

\textbf{Highcharts} proudly claims to be used by 72\% of the 100 biggest
companies in the world. It is a simple tool that does not require
specialized training and quickly generates the desired output. Unlike
some tools, Highcharts focuses on cross- browser support, allowing for
greater access and use.

\textbf{Datawrapper} is making a name for itself in the media industry.
It has a simple user interface making it easy to generate charts and
embed into reports.

\textbf{Plotly} can create more sophisticated visuals thanks to
integration with programming languages such as Python and R. The danger
is creating something more complicated than necessary. The whole point
of data visualization is to quickly and clearly convey information.

\textbf{Sisense} can bring together multiple sources of data for easier
access. It can even work with large datasets. Sisense makes it easy to
share finished products across departments, ensuring everyone can get
the information they need.

\url{https://www.forbes.com/sites/bernardmarr/2017/07/20/the-7-best-data-visualization-tools-in-2017/\#3a12b8ea6c30}

\section{5. Pick the Right Chart Type!}\label{pick-the-right-chart-type}

Data divusalization is combining the art and science. As for the art, we
can say there are no correct answers for doing the visualization. There
are many ways to present the data. However, how to making sense of
facts, numbers and measurement for better understanding is still have a
logical path to follow.

To determine which kind of chart is hard for those people new to data
visulization. Most people learn it by refering some other people's work
without understanding the logic behind. So they don't have the theory in
their mind to make the judgement. Here , I will introduce some guidance
to choose the charts.

When we about to choose the type of chart, we need to answer some
questions. - How many features would you like to show in a chart? - how
many data points do you want to display for each variable? - Will you
display time serious data or among items or groups.

After answered this question, you shoul able to get a better imagenation
of your ideal graph. The simple guidance for using different type of
chart is line charts for tracking trends over time, bar charts to
compare quantities, scatter plots for joint variation of two data items,
bubble charts showing joint variation of three data items, and pie
charts to compare parts of a whole.

Let's review the most commonly used chart types and expalin what
circumstance should better use typical chart and the pros and conts of
each type of chart. Before introduce differnt types of charts, you can
use the following website to familiar with different types of charts.
\href{https://datavizcatalogue.com/}{The Data Visualisation Catalogue}

\textbf{Type 1 Column Charts.} This should be the most popular chart
type. This chart is good to do comparison between different values when
specific values are important. TBD

Still have hard time to choose? There are many resources on line can
help you do the decision. For example, Dr.~Andre Abela create a chart
selection diagram that is helpful to pick the right chart depends on the
data type. The link of website is
\url{http://extremepresentation.typepad.com/blog/2015/01/announcing-the-slide-chooser.html}

Reference: Data Visualization -- How to Pick the Right Chart Type? , By
Jānis Gulbis
\url{https://eazybi.com/blog/data_visualization_and_chart_types/}

Data Visualization Best Practices by melindasantos \textbar{} Sep 19,
2017 \url{http://paristech.com/blog/data-visualization-best-practices/}

\url{http://paristech.com/blog/data-visualization-best-practices/}
\url{http://extremepresentation.typepad.com/blog/2015/01/announcing-the-slide-chooser.html}

\section{6. Guide for Developing
Dashboards}\label{guide-for-developing-dashboards}

\url{https://www.klipfolio.com/blog/intuitive-dashboard-design} Three
rules to follow in order to develop intuitive dashboards:

\begin{verbatim}
1. the dashboard should read left to right
2. group related information together
3. find relationships between seemingly unrelated areas and display visuals together to show the relationship.
\end{verbatim}

Often a designer can become too concerned with coming up with a visual
that is too intricate and overly complicated. A dashboard should be
appealing but also easy to understand. Following these rules will lead
to effective presentation of the data.

Because we read from top to bottom and left to right, a reader's eyes
will naturally look in the upper left of a page. The content should
therefore flow like words in a book. It is important to note that the
information at the top of the page does not always have to be the most
important. Annual data is usually more important to a business but daily
or weekly data could be used more often for day to day work. This should
be kept in mind when designing a dashboard as dashboards are often used
as a quick convenient way to look up data.

Grouping related data together is an intuitive way to help the flow of
the visual. It does not make sense for a user to have to search in
different areas to find the information they need.

Grouping unrelated data seems contradictory to the second rule, but the
important thing is to tell a story not previously observed. Data
analytics is all about finding stories the data is trying to tell. Once
they are discovered, the stories need to be presented in an effective
manner. Grouping unrelated data together makes it easier to see how they
change together.

\section{7. Definions of Date Deception and Graphic
Integrity}\label{definions-of-date-deception-and-graphic-integrity}

Data visualization becomes more and more pupular to communicate and
support arguments nowdays. There are lots of great resources online to
create and design amazing data products, in the same time, there are
some poorly-designed misleading deceptive data visualizations.

So what does \textbf{data deception} mean? Data deception, defined by
School of Law at the New York University, as ``a graphical depiction of
information, designed with or without an intent to deceive, that may
create a belief about the message and/or its components, which varies
from the actual message.''

In reality, decades ago, Edward Tufte already introduced the concept of
graphical intergrity in his book and presented six principles of graphic
integrity. Here are the principles from book:

\begin{verbatim}
1. The representation of numbers, as physically measured on the surface of the graphic itself, should be directly
proportional to the numerical quantities measured.

2. Clear, detailed, and thorough labeling should be used to defeat graphical distortion and ambiguity. Write out
explanations of the data on the graphic itself. Label important events in the data.

3. Show data variation, not design variation.

4. In time-series displays of money, deflated and standardized units of monetary measurement are nearly always better than
nominal units.

5. The number of information-carrying (variable) dimensions depicted should not exceed the number of dimensions in the
data.

6.Graphics must not quote data out of context.
\end{verbatim}

\textbf{Reference} (1) Pandey, A. V., Rall, K., Satterthwaite, M. L.,
Nov, O., \& Bertini, E. (2015). How deceptive are deceptive
visualizations? An empirical analysis of common distortion techniques.
In CHI 2015 - Proceedings of the 33rd Annual CHI Conference on Human
Factors in Computing Systems: Crossings (Vol. 2015-April,
pp.~1469-1478). Association for Computing Machinery. DOI:
10.1145/2702123.2702608 (2) Tufte, E. R., and Graves-Morris, P. The
visual display of quantitative information, vol.~2. Graphics press
Cheshire, CT, 1983.

\textbf{Misleading graphs:}

Misleading graphs or distorted graphs, are graphs created which skews
the data, intentionally or unintentionally, resulting in a
representation of incorrect conclusions.

There are some ways in which distorted graphs can be created: 1.
Improper scaling of y axis: This is one of the classic misleading
graphs. Instead of scale starting from zero or a baseline, y axis is
scaled conveniently to highlight the differences among bins. 2. Improper
labelling of graphs: Lack of labels make the graph hard to interpret for
the reader and lead to wrong conclusions. 3. Paired graphs on different
scale: It is not a fair comparison if two elements are plotted
side-by-side, on a different scale and compared. This makes one graph
look better than the other, even when it is not. 4. Dual axis with
different scales: If we are plotting two elements on the same graph with
different scales, even if the axes are properly labeled, it is assumed
that both axes are on the same scale. 5. Incomplete data: Short-term
graphs are made to manipulate the trend, which will not be seen
otherwise. Time-series data are cut intentionally to just show a trend
within a particular period to create a more favorable visual impression.

Please find the references below.
\url{http://hypsypops.com/axes-evil-lie-graphs/}
\url{http://www.statisticshowto.com/misleading-graphs/}

\section{8. Contemporary Research Results \& What's
Next}\label{contemporary-research-results-whats-next}

Next Steps for Data Visualization Research

With the development, studies and new tools applied in data
visualization, more people understand it matters. But given its youth
and interdisciplinary nature, research methods and training in the field
of data visualization are still developing. So, we asked ourselves: what
steps might help accelerate the development of the field? Based on a
group brainstorm and discussion, this article shares some of the
proposals of ongoing discussion and experiment with new approaches:

\begin{enumerate}
\def\labelenumi{\arabic{enumi}.}
\item
  Adapting the Publication and Review Process As the article states,
  ``both `good' and `bad' reviews could serve as valuable guides'', so
  providing reviewer guidelines could be helpful for fledgling
  practitioners in the field.
\item
  Promoting Discussion and Accretion Discussion of research papers
  actively occurs at conferences, on social media, and within research
  groups. Much of this discussion is either ephemeral or non-public. So
  ongoing discussion might explicitly transition to the online forum.
\item
  Research Methods Training Developing a core curriculum for data
  visualization research might help both cases, guiding students and
  instructors alike. For example, recognizing that empirical methods
  were critical to multiple areas of computer science, Stanford CS
  faculty organized a new course on Designing Computer Science
  Experiments(\url{http://sing.stanford.edu/cs303-sp11/}). Also, online
  resources could be reinforced with a catalog of learning resources,
  ranging from tutorials and self-guided study to online courses. Useful
  examples include Jake Wobbrock's Practical Statistics for HCI and
  Pierre Dragicevic's resources for reforming statistical practice.
\end{enumerate}

Reference:
\url{https://medium.com/@uwdata/next-steps-for-data-visualization-research-3ef5e1a5e349}

\section{Typography and Data
Visualization}\label{typography-and-data-visualization}

This article discusses less common applications of typography in data
visualization. While data components such as quantitative or categorical
data are commonly represented by visual features like colors, sizes or
shapes, utilization of boldface, font variation, and other typographic
elements in data visualization are less prevalent.

Highlighted in the article are preattentive visual attributes;
preattentive attributes are those that perceptual psychologists have
determined to be easily recognized by the human brain irrespective of
how many items are displayed. Therefore, ``preattentive visual
attributes are desirable in data visualization as they can demand
attention only when a target is present, can be difficult to ignore, and
are virtually unaffected by load.'' Examples of preattentive attributes
are size/area, hue, and curvature.

This brings us to the disparateness of the popularity of visual aspects
like color and size and typographic aspects such as font variation,
capitalization and bold. The authors present several possible reasons
for this, beginning with the preattentiveness of visual attributes like
size and hue. However, some typographic attributes such as line width or
size, intensity, or font weight (a combination of the two) are
considered preattentive as well.\\
Furthermore, these visual attributes are inherently more viscerally
powerful, and they are easy to code in a variety of programming
languages. Technology has also perhaps previously limited the use of
typographic attributes, for only recently have fine details such as
serifs, italics, etc. been made readily visible to the audiences of data
visualizations by technological advances.

Lastly, the authors remark that it is possible the lack of variety of
typographic elements used in data visualizations is due to the limited
knowledge of computer scientists and other individuals pursuing data
visualization in how to apply these elements effectively. While the
first few proposed explanations make sense from personal experience with
technology and exposure to data visualizations and design in general,
the hypothesis that lack of knowledge of typographic elements in data
visualization seems more plausible if it was being applied to a small
group of people rather than all of the data visualization design
community. I would say that it is more likely that the use of
typographic elements in data visualization is less popular because there
are fewer instances in which it can be used appropriately, or a status
quo bias---if current visual attributes are received well, the
prevailing attitude may be not to fix what is not broken.\\
However, the authors also point out that despite the dearth of
typographic attributes in data visualization, other spheres like
typography, cartography, mathematics, chemistry, and programming ``have
a rich history with type and font attributes that informs the scope of
the parameter space.''

The authors continue by pointing out some tips for using typographic
attributes to encode different data types, since certain attributes may
be suited to particular purposes. For example, font weight (size and
intensity) is ideal for representing quantitative or ordered data, and
font type (shape) is better suited to denote categories in the data.\\
Furthermore, as in typography and cartography, use of typographic
attributes in data visualization raises concerns of legibility, the
ability to understand both individual characters and commonalities that
identify a font family, and readability, the ability to read lines and
blocks of words. Often, interactivity of a visualization will not only
improve functionality, but also provide a solution to readability issues
by providing a means to zoom in on small text.

There are a few examples of unusual/innovative use of typography for
data visualization in the article, not all of which I agree are made
more effective by the interesting utilization of typographic attributes,
but the ``Who Survived the Titanic'' visualization's use of typographic
attributes allowed it to not only answer macro-questions very quickly,
such as if women and children were actually first to be evacuated across
classes, but also to provide answers to micro-questions, like whether or
not the Astors survived. It used common visual elements like color and
area to indicate whether or not a person survived and number/proportion
of people, as well as typographic aspects like italic and simple text
replacement to indicate gender and the passengers' names.

The authors round out the article by addressing the most common
criticisms of typography in data visualization, the foremost one being
whether or not text should even be considered an element of data
visualization, since visualization connotes preattentive visual encoding
of information, and text or sequential information necessitates more
investment of attention to understand. Another criticism is that textual
representations are not as visually appealing even when used
effectively. However, the authors counter that ``this criticism
indicates both the strength and weakness of type,'' that while text may
not be suited for adding style or drama to a visualization, it can be
particularly powerful in situations where a finer level of detail is
needed, without sacrificing representation of higher level patterns.
Lastly, a label length problem is common when using text in
visualizations; differing lengths of names or labels may skew perception
so that longer labels seem more important than shorter labels. This
problem was encountered in the Titanic visualization with the varying
lengths representations of passengers' names, and was corrected by only
including a given name and a surname, the length of which could only
vary so much.

All in all, this article has an interesting take on a somewhat less
fashionable tool and puts forth the idea that text and typographic
attributes can convey additional important information in data
visualizations when used innovatively and correctly.

Reference: Banissi, Ebad, \& Brath, Richard. (2016). Using Typography to
Expand the Design Space of Data Visualization. She Ji: The Journal of
Design, Economics and Innovation, 2(1), pp 59-87.
\url{https://www.sciencedirect.com/science/article/pii/S2405872616300107}.

\citep{data-insights}

\section{This article explains 9 design principles which can be used for
vizulation. These 9 design principles
are:}\label{this-article-explains-9-design-principles-which-can-be-used-for-vizulation.-these-9-design-principles-are}

\url{https://www.idashboards.com/blog/2017/07/26/data-visualization-and-the-9-fundamental-design-principles/}

\begin{enumerate}
\def\labelenumi{\arabic{enumi}.}
\item
  Balance: A design is said to be balanced if key visual elements such
  as color, shape, texture, and negative space are uniformly
  distributed.
\item
  Emphasis: Draw viewers attention towards important data by using key
  visual elements.
\item
  Movement: Ideally movement should mimic the way people usually read,
  starting at the top of the page, moving across it, and then down.
  Movement can also be created by using complimentary colors to pull the
  user's attention across the page.
\item
  Pattern: Patterns are ideal for displaying similar sets of
  information, or for sets of data that equal in value. Disrupting the
  pattern can also be effective in drawing viewers attention; it
  naturally draws curiosity.
\item
  Repetition: Relationships between sets of data can be communicated by
  repeating chart types, shapes, or colors.
\item
  Proportion: If a person is portrayed next to a house, the house is
  going look bigger. In data visualization, proportion can indicate the
  importance of data sets, along with the actual relationship between
  numbers.
\item
  Rhythm: A design has proper rhythm when the design elements create
  movement that is pleasing to the eye. If the design is not able to do
  so, rearranging visual elements may help.
\item
  Variety: Variety in color, shape, and chart-type draws and keeps users
  engaged with data. Including more variety can increase information
  retention by the viewer. But when there is too much variety, important
  details can be overlooked.
\item
  Unity: Unity across design will happen naturally if all other design
  principles are implemented.
\end{enumerate}

\textbf{Using Data Visualization to Find Insights in Data}

\textbf{Reference Link}
\url{http://datajournalismhandbook.org/1.0/en/understanding_data_7.html}

This article is extracted from a book known as Data Journalism Handbook
and this is one of the chapters of the book. The author starts the
article by introducing a very simple idea that loading any dataset into
a spreadsheet can also be a form of visualization as an invisible data
becomes visible in a picture form into a table. Hence the focus should
not be whether we need data visualization or not but should be on which
form of data visualization is best in which situation.

The author then proceeds by stating that data visualization will not
always unleash a readymade story on its own. Sometimes the insights are
known before the visualization and sometimes an insight can be
completely new. The author has given a process for finding insights in
the following way:

Visualize Data-\textgreater{} Analyze -\textgreater{} Document Insights
-\textgreater{} Transform Datasets -\textgreater{}Visualize Data

Each stage is explained in-depth further. Data Visualization can be done
in many ways such as tables which are great for one dimensional data
however they are bad for multi-dimensional data. Then he goes further to
explain the situation where each type of visualization such as bar
charts, maps, scatterplots, graphs, etc. are used. This gives a thorough
understanding of when to use which type of visualization. Once we
visualize the data we need to ask the following questions:

1 What can I see in this image? Is it what I expected? 2 Are there any
interesting patterns? 3 What does this mean in the context of the data?

The basic question answer format gives an idea to the viewers about what
kind of perspectives can we look at the data. Sometimes we discover
something and sometimes we don't. But the author mentions that we always
learn something from the visualization. Once we document the data
insights based on the above question we need to have the following
points into consideration:

1 Why have I created this chart? 2 What have I done to the data to
create it? 3 What does this chart tell me?

The above question answer format compels the viewers to think deeper
about what exactly we are trying to find. Because many times the viewers
are simply too overwhelmed with the size of data that they lose the
basic idea. Hence this kind of approach help to stay focused. The author
then mentions that based on the above insights we might have some idea
about some interesting patterns. Since we already have an idea we might
want to see it in more detail and hence we transform data in more
details such as Zooming, Filtering, Outlier Removal. The author then
explains how transformed data can help us to see a more detailed view of
our insights.

Further the author gives a detailed explanation of which data
visualization tool to use based on the situation. The entire process
given above is explained in depth with the help of examples. The
technical approach listed above is practical and can be implemented
easily on our data visualization projects. I liked the author's approach
because he has cleverly integrated the step-by-step process of finding
insights with the technical way of handling datasets using tools such as
Tableau, Python, etc. And the process can be repeated many times till we
find the insights we are looking for.

Building advanced analytics application with TabPy

\url{https://www.tableau.com/about/blog/2017/1/building-advanced-analytics-applications-tabpy-64916}

\citet{misc}\{great\_viz, author = \{\{Bora Beran\}\}, year = \{2017\},
title = \{Building advanced analytics applications with TabPy\},
howpublished =
\{\url{https://www.tableau.com/about/blog/2017/1/building-advanced-analytics-applications-tabpy-64916}\},
note = \{Accessed: 2018-04-28\} \}

Imagine a scenario where we can just enter some x values in a dashboard
form, and the visualization would predict the y variable!!! Here is a
link that shows how to integrate and visualize data from Python in
Tableau. This is especially relevant to all data science students, as
this is one of the tools used for visualizing advanced analytics. The
author here has given an example using data from Seattle's police
department's 911 calls and he tries to identify criminal hotspots in the
area. The author uses machine learning (spatial clustering) and creates
a great interactive visualization, where you can click on the type of
criminal activity and the graph will show various clusters. There are
other examples and use cases that may be downloaded, and the scripts are
also given by the author for anyone who is interested in trying it out.

\begin{verbatim}
+ Theoretical background of data visualization  
\end{verbatim}

\begin{itemize}
\tightlist
\item
  Contemporary research results
\end{itemize}

Some best practices for visualization:

\url{http://www.dataplusscience.com/files/visual-analysis-guidebook.pdf}

Here is free pdf to some best practices in visual analysis. It talks
about the right charts to be used for various kinds of analysis. It is
very relevant for data science students as we would be interested in
presenting our analysis using simple and effective visualizations that
tell the complete story.

Some of key areas for which the author highlights some best practices
are for visualizing trends over time, comparison and ranking,
correlation, distribution, geographical data etc.

The author gives examples on how simple graphs can also become more
effective by just adding a few more elements or some simple adjustments.

I feel this is a great starting point to create effective charts and we
may use these principles also when we start doing advanced analytics.

contributions

\#\#\#Interactive Data Visualization

Interactive or Dynamic data visualization delivers today's complex sea
of data in a graphically compelling and an easy-to-understand way. It
enables direct actions on a plot to change elements and link between
multiple plots. It enables users to accomplish traditional data
exploration tasks by making charts interactive.

\#\#\#\#Benefits of Interactive Data Visualization Software:

\begin{enumerate}
\def\labelenumi{\arabic{enumi}.}
\tightlist
\item
  Absorb information in constructive ways: With the volume and velocity
  of data created everyday, dynamic data viz enables enhanced process
  optimization, insight discovery and decision making.
\item
  Visualize relationships and patterns: Helps in better understanding of
  correlations among operational data and business performance.
\item
  Identify and act on emerging trends faster: Helps decision makers to
  grasp shifts in behaviors and trends across multiple data sets much
  more quickly.
\item
  Manipulate and interact directly with data: Enables users to engage
  data more frequently.
\item
  Foster a new business language : Ability to tell a story through data
  that instantly relates the performance of a business and its assets.
\end{enumerate}

\citep{benefits_interactive_viz}

There are multiple ways by which interactive data visualizations can be
developed: 1. D3.js 2.Tableau 3.R shiny

\#\#\#\#D3.js:

D3.js (or just D3 for Data-Driven Documents) is a JavaScript library for
producing dynamic, interactive data visualizations in web browsers(From
Wikipedia). It is highly functional, meaning you can reuse the code and
add functions relevant to your project. Embedded within an HTML webpage,
the JavaScript D3.js library uses pre-built JavaScript functions to
select elements, create SVG objects, style them, or add transitions,
dynamic effects or tooltips to them.

Some of the key advantages are: It is dynamic, free and open source and
very flexible with all web technologies, the abiity to handle big data
and the functional style allows to reuse the codes.

\citep{d3_interactive_viz}

\#\#\#\#Tableau:

Tableau is business intelligence (BI) and analytics platform created for
the purposes of helping people see, understand, and make decisions with
data. It is the industry leader in interactive data visualization tools,
offering a broad range of maps, charts, graphs, and more graphical data
presentations. It is a painless option when cost is not a concern and
you do not need advanced and complex analysis.The application is very
handy for quickly visualizing trends in data, connecting to a variety of
data sources, and mapping cities/regions and their associated data.

The key advantages are: It provides non technical user the ability to
build complex reports and dashboard with zero coding skills. Using
drag-n-drop functionalities of Tableau, user can create a very
interactive visuals within minutes. It can handle millions of rows of
data with ease and users can make live to connections to different data
sources like SQL etc.

\citep{tableau_interactive_viz}

\#\#\#\#R Shiny :

R Shiny enables us to produce interactive data visualizations with a
minimum knowledge of HTML, CSS, or Java using a simple web application
framework that runs under the R statistical platform. Standalone apps
can be hosted on a webpage or embedded in R Markdown documents and
dashboards can be built using R shiny. It combines the computational
power of R with the interactivity of the modern web.

The main advantages of using R Shiny are : Its flexibility of pulling in
whatever package in R that you want to solve your problem, reaping the
benefits of an open source ecosystem for R and Javascript visualization
libraries, thereby allowing to create highly custom applications and
enabling timely, high quality interactive data experience without (or
with much less) web development and without the limitations or cost of
proprietary BI tools.

\citep{shiny_interactive_viz}

\chapter{Avoiding Common Mistakes with Time
Series}\label{avoiding-common-mistakes-with-time-series}

\url{https://www.svds.com/avoiding-common-mistakes-with-time-series/}

This article explains how time series data visualization can sometimes
be deceptive. It first takes an example of two random time series data
and plots them on a graph which gives an impression that the two are
strongly correlated. But if we do some statistical testing the two do
not show any relationship, this is an example of ``correlation does not
necessary mean causation''. In another set of examples author has taken
trending two random time series data and shown how even statistical
tests can give a wrong interpretation. The article then explains using
visualization how a general trended time series can be different than a
more controlled and measured trending time series.

\chapter{Case Studies}\label{case-studies}

\begin{itemize}
\tightlist
\item
  Description and replication of great examples of data visualization
\end{itemize}

\url{http://flowingdata.com/2015/12/22/10-best-data-visualization-projects-of-2015/}

The author picked top 10 projects for the best data visualization of
2015, for each pick, the author showed the project plot and also
described the reason why he chose. So after reading this article, I have
a basic understanding of what kind of characters should include in a
good visualization project.

\section{Description and replication of great examples of data
visualization}\label{description-and-replication-of-great-examples-of-data-visualization}

reference:\url{http://blog.visme.co/best-information-graphics-2016/\#e030mFiF7wCpk7Ld.99}

\begin{enumerate}
\def\labelenumi{\arabic{enumi}.}
\tightlist
\item
  Connecting the Dots Behind the Election
\end{enumerate}

\url{https://www.nytimes.com/interactive/2015/05/17/us/elections/2016-presidential-campaigns-staff-connections-clinton-bush-cruz-paul-rubio-walker.html?_r=1}

This article by the New York Times lists several different candidates
and creates compelling visuals that link their campaigns to previous
ones.

Each visual contains several different-sized dots that represent a
specific campaign, administration, or other governmental organization
related to the candidate's current campaign, which are then connected by
arrows.

Hovering over a specific dot highlights the connections between the
groups. The visual is a great way to put what would otherwise be a long
slog through years of information into an easily accessible, easily
viewable format so that voters can figure out where the candidates'
experiences lie.

\begin{enumerate}
\def\labelenumi{\arabic{enumi}.}
\setcounter{enumi}{1}
\tightlist
\item
  Spies in the Skies
\end{enumerate}

\url{https://www.buzzfeed.com/peteraldhous/spies-in-the-skies?utm_term}=.so1GQ6ZGDo\#.ec8kL3WkZe

The map is filled with red and blue lines (representing FBI and DHS
aircraft, respectively) which illustrate the flight paths of the planes.
When planes circle an area more than once, the circles become darker.
The circles change in accordance to day and time, and individual cities
can be typed into a search bar to see the flight patterns over them.

The visualization, rather creatively, almost looks like a hand-drawn
map. While presenting a normally uncomfortable topic, this allows
individuals to check things for themselves, hopefully providing some
peace of mind.

\begin{enumerate}
\def\labelenumi{\arabic{enumi}.}
\setcounter{enumi}{2}
\tightlist
\item
  Green Honey
\end{enumerate}

\url{http://muyueh.com/greenhoney/?es_p=1228877}

The visualization spans a webpage. As you scroll down, the text changes,
as do many colored dots that move over the white background. The dots
are used to represent not only each colors' hue, but the numbers that
fall into each category---for example, what colors are the most popular
``base'' colors for English and Chinese.

The continuous flow of this visualization helps really bring it
together, allowing users to scroll through the information at their own
pace, but also creating a seamless, creative work.

\begin{enumerate}
\def\labelenumi{\arabic{enumi}.}
\setcounter{enumi}{3}
\tightlist
\item
  How People Like You Spend Their Time
\end{enumerate}

\url{http://flowingdata.com/2016/12/06/how-people-like-you-spend-their-time/}

The visual lists several categories along one side of a graph---such as
``personal care'' and ``work''---with a line illustrating the amount of
time the average person in a certain demographic spends on each subject.
Entering different statistics at the top---such as changing gender or
age---causes the lines to shift to feature that demographic.

The simplicity of this visualization really helps the information get
across and avoids bogging down the statistics. Sometimes, less is more.

\begin{enumerate}
\def\labelenumi{\arabic{enumi}.}
\setcounter{enumi}{4}
\tightlist
\item
  Is it Better to Rent or Buy?
\end{enumerate}

\url{https://www.nytimes.com/interactive/2014/upshot/buy-rent-calculator.html?_r=0}

The calculator includes several sloping charts. Each chart includes a
factor that'll affect how much you'll have to pay, such as the
individual cost of your home and your mortgage rates. A movable scale
along the bottom of each chart allows you to enter different data,
changing the ``cost of rent per month'' on the side. If you can find a
similar house to rent for that much per month or less, it's more cost
effective to just rent the home.

This visualization is incredibly thorough and a useful tool for
homeowners of any age and status.

\begin{enumerate}
\def\labelenumi{\arabic{enumi}.}
\setcounter{enumi}{5}
\tightlist
\item
  What's really warming the world?
\end{enumerate}

\url{https://www.bloomberg.com/graphics/2015-whats-warming-the-world/}

In this case study, it first claimed the background story and the
analytical questions clearly. Then it analyzed each different factors
separately using both verbal explanations and dynamic graphics to
compare with the observed temperature movements, and then grouped
related factors into Natural factors category or Human factors category.
After that, it combined all the dynamic graphics into one and made the
results more straightforward in terms of comparisons. In the end, the
authors also provided more detailed methodology explanations with
dataset sources to support the results shown above.

Overall, this case study is straightforward, easy to understand but also
with enough information shown on each graphics.

\begin{enumerate}
\def\labelenumi{\arabic{enumi}.}
\setcounter{enumi}{6}
\tightlist
\item
  The Strengths of Animated Data Visualization
\end{enumerate}

\url{http://flowingdata.com/2015/12/15/a-day-in-the-life-of-americans/?platform=hootsuite}

The page linked above includes a great example of animated data
visualization showing the time people spend on daily activities
throughout the day. The plot is simple and easy to interpret, but it
also includes a good number of variables including time, activity type,
number of people doing each activity, and the order in which activities
are done.

One of the plot's biggest strengths is that by using one dot to
represent each person in the study and using animation, we can actually
drill down to each individual and follow them throughout the day. The
accumulation of dots for each particular activity also gives us an
aggregate-level view of the same data, so we get both an individual and
aggregate insights.

A drawback of the plot is that it is hard for our eyes to keep track of
1000 simultaneously moving dots. The author of the post addresses this
by creating subsequent plots with stationary lines at key times of the
day. This represents people's movements from one activity to another
without overwhelming the reader.

Overall, this is an engaging, informative, and fun animated plot that
has relevance and tells a story.

\begin{enumerate}
\def\labelenumi{\arabic{enumi}.}
\setcounter{enumi}{7}
\tightlist
\item
  Case studies: \textbf{An Aging Population}
\end{enumerate}

Aging population is always a hot topic in social economics and politics.
I collect several different data visualizations that show aging
population in the world. They are good examples to learn and apply to
census data.

8.1 An Aging Nation: Projected Number of Children and Older Adults
\href{https://www.census.gov/library/visualizations/2018/comm/historic-first.html}{Source}

\begin{figure}
\centering
\includegraphics{images/aging_nation.jpg}
\caption{}
\end{figure}

This one includes bar chart and line graph to demonstrate the aging
population compared with population of children. The good things about
this visualization: simple to see and compare, color to differentiate
the category, highlight the intersection point.

8.2 From Pyramid to Pillar: A Century of Change, Population of the U.S.
\href{https://www.census.gov/library/visualizations/2018/comm/century-of-change.html}{Source}

\begin{figure}
\centering
\includegraphics{images/Pyramid.jpg}
\caption{}
\end{figure}

This is a \textbf{population pyramid}. ``A \textbf{population pyramid}
is a pair of back-to to histograms for each sex that displays the
distribution of a population in all age groups and in gender''.

It is a good candidate to compare changes in population distributions
(sex, age, year). Also the shape of pyramid is used to interpret a
population. To illustrate, A pyramid with a very wide base and a narrow
top section suggests a population with both high fertility and death
rates. It is a useful tool in the census data.

8.3 Animated pyramid \href{https://fathom.info/aging/}{Source}
\includegraphics{images/3_1.png}

This is an animated and multiple population pyramids. It used to compare
different patterns across countries. One additional benefit for the
interactive population pyramid is that it shows the shape changes year
by year, which is useful for countinous time-series comparison.

Similar projected with R code is provided for references:
\href{https://www.r-bloggers.com/who-is-old-visualizing-the-concept-of-prospective-ageing-with-animated-population-pyramids/}{link}

\section{Deceptive data graphs
examples}\label{deceptive-data-graphs-examples}

references: **Misleading Graphs: Real Life Examples
\url{http://www.statisticshowto.com/misleading-graphs/**}

Misleading graphs are sometimes deliberately misleading and sometimes
it's just a case of people not understanding the data behind the graph
they create. But some real life misleading graphs go above and beyond
the classic types. Some are intended to mislead, others are intended to
shock. The ``classic'' types of misleading graphs include cases where:

\begin{itemize}
\tightlist
\item
  \textbf{The Missing Baseline.}
\end{itemize}

For example, the Vertical scale is too big or too small, or skips
numbers, or doesn't start at zero, like the graph below:

You might be thinking that the graph on the right shows The Times makes
double the sales of The Daily Telegraph. But take a closer look at the
scale and you'll see although The Times does make more sales, it's only
beating the competition by about 10\%.

\begin{itemize}
\tightlist
\item
  \textbf{The graph isn't labeled properly.}
\end{itemize}

Graghs can have the correct figures, but still can mislead you.

This one used a BIG HEADLINE makes you think that 5.3\% of children get
spinal cord injuries which is a pretty scary statistic for parents. But
the real figure is about .0000003\% (based on 2000 injuries per year out
of a population of around 74,000,000).

And for the figure 1 used in this article: Misleading Graphs: Displaying
a Change in One Variable Using Area or Volume
\url{https://www.forbes.com/sites/naomirobbins/2012/02/28/misleading-graphs-displaying-a-change-in-one-variable-using-area-or-volume/\#696674551781},
the label for the smaller triangle in this graph says \$26.4 while the
label for the larger triangle says \$114.6. \$114.6 is 4.34 times
\$26.4. It certainly looks to me as if more than 4.34 smaller triangles
will fit in the larger triangle. It is the altitudes of the triangles
that are proportional to the numbers in the labels.

\begin{itemize}
\tightlist
\item
  \textbf{Data is left out.}
\end{itemize}

Only include part of the data like the following graph which using
temperatures of the first half of the year to prove it was rising
dramatically.

For more examples of misleading graphs or deceptive graphs you can read
the following articles for more inspirations:

\begin{itemize}
\item
  bar charts without zero \& evenly spaced tick marks for uneven
  intervals:
  \url{https://www.forbes.com/sites/naomirobbins/2011/11/17/whats-wrong-with-this-graph/\#502ab1a42a33}
\item
  graphs not drawn to scale:
  \url{https://www.forbes.com/sites/naomirobbins/2012/02/16/misleading-graphs-figures-not-drawn-to-scale/\#351dcf9c15ef}
\item
  \textbf{Treating correlation as causation.}
\end{itemize}

Even if the labels and data in your graph is correct, it does not mean
that the conclusion is logically correct. A correlation between X and Y
does not automatically indicate that the change in one variable is
caused by the change in the values of the other one, whereas the
causation means that one event is the result of the occurrence of the
other event. From the graph, we should bear in mind that it only
presents the correlation between ice cream sold and murders, rather than
causation.

\begin{figure}
\includegraphics[width=0.7\linewidth]{images/harlin-ice-cream} \caption{A strange correlation between ice cream sales and murders (Source: [@harlin-coorelation])}\label{fig:harlin-ice-cream}
\end{figure}

Some Interesting Visualizations:
\url{https://blog.hubspot.com/marketing/great-data-visualization-examples}
\url{http://blog.visme.co/data-visualizations-current-events/}
Visualization is like art. It speaks where words fail. There are
phenomenas like the Syrian war, the number flights during Thanksgiving
in the USA, the understanding of depths for developing perspective about
the range of the issue, the controversy of `\#OscarsSoWhite', etc. on
which we can write bundles of paragraphs, but they might still have
scope for ambiguity. The links show some intricate visualizations of the
topics like those mentioned above, and speak volumes without requiring
paragraphs to explain what is going on within these visualizations.
According to me, it is really interesting to see that almost anything in
this world can be explained by visualizations. Visualizations are not
just limited to businesses and their analytics. Wars, rescue operations,
etc. can also be visualized to get a clear idea of all the details of
the issues. 1. Picking up from one of the charts shown in the above
mentioned links, the visualization of `A guide to Who is Fighting Whom
in Syria' is one of the most interesting charts in the list. The
visualization and its report can be seen at
\url{http://www.slate.com/blogs/the_slatest/2015/10/06/syrian_conflict_relationships_explained.html}

This visualization makes an extremely complicated topic like the Syrian
War easily understandable. It consists of 3 different emojis in three
different colours, with each (colour+facial expression) combination
showing the relationship between the various groups involved in the
Syrian War. When you click on each of the emoji, a small dialogue box
pops up which explains the relationship between the various countries
and rebel groups involved in the war. This is not only easy to
understand, but it is also pleasing to the eyes.

\begin{enumerate}
\def\labelenumi{\arabic{enumi}.}
\setcounter{enumi}{1}
\tightlist
\item
  The second visualization `Adding up the White Oscars Winners' can be
  seen here
  (\url{https://www.bloomberg.com/graphics/2016-oscar-winners/}) in an
  article by Bloomberg. The writes of this article developed the
  attributes of the future winners of Oscars by taking up the attributes
  of the past winners. It is extremely interesting to see how the
  article shows the features of the Best Actress, Actor, movies, etc. in
  a simple and captivating visual. The visualization is interactive and
  we can click on each attribute like `Hair Color', `Eye Color', etc. to
  see what are the features of the actors and actresses who are more
  likely to win the Oscars.
\end{enumerate}

Similarly, the visualization gives information about the different
aspects of movies that are more likely to win, like `Length' ,'Month'
,'Budget', etc.

\section{Application of Data
Visualization}\label{application-of-data-visualization}

\textbf{Data Preprocessing}

We use data visualization for outliar detection in the dataset.
Different methods for outlier detection in functional data have been
developed during the years. Among them, several rely on different
notions of functional depth , on robust principal components, or on
random projections of infinite-dimensional data into R. Also, some
distributional approaches have been considered (Gervini, 2009). In
functional data analysis, we observe curves defined over a given real
interval and shape outliers may be defined as those curves that exhibit
a different shape from the rest of the sample. Whereas magnitude
outliers, that is, curves that lie outside the range of the majority of
the data, are in general easy to identify, shape outliers are often
masked among the rest of the curves and thus difficult to detect.

\citep{outliar}

\begin{enumerate}
\def\labelenumi{\arabic{enumi}.}
\setcounter{enumi}{7}
\tightlist
\item
  Young voters, class and turnout: how Britain voted in 2017
\end{enumerate}

Reference:
\url{https://www.theguardian.com/politics/datablog/ng-interactive/2017/jun/20/young-voters-class-and-turnout-how-britain-voted-in-2017}

The article's goal is to convey the change in party votes in the 2017 UK
general election compared to votes in 2015. The change in party votes
was shown with regards to three demographic factors: age, class, and
ethnicity. For each factor, there are four graphs (one per political
party), each illustrated in their party's standard color. The change in
percent of votes is shown as an arrow where the arrow's shaft is the
length of the difference from 2015 to 2017 while the x-axis is the
demographic factor split into different bins. What makes this a good
visualization is that it is very easy to read and interpret. The
color-coding of the arrows and party name makes it easy to pick out the
different parties and the arrow lengths highlight just how large of a
change happened. For example, in the Age section, it is easy to see the
pattern between the Labour party gaining many voters ages 18 to 44 and
the Conservative party gaining voters ages 45 and up.

\begin{enumerate}
\def\labelenumi{\arabic{enumi}.}
\setcounter{enumi}{7}
\tightlist
\item
  Vizwiz blog: casestudies about how to improve your visualizations
\end{enumerate}

\url{http://www.vizwiz.com/}

This is a blog about Tableau based data visualization. The author is
Andy Kriebel who is a famous Tableau Zen Master. I would like to
recommend this blog because it is not only practical, but also full of
insights.

My favorite part of this blog is so called ``Makeover Monday'', which
will develop a new visualization based on an original one. For example,
the author re-designed ``The Seasonality of Confirmed Malaria Cases in
Zambia Southern Province'' by pointing out ``what works well'', ``what
could be improved'' and also his goals for the new visualization (ref:
\url{http://www.vizwiz.com/2018/04/malaria.html}) That's how you can
learn all the insight and reason behind a good visualization.

Besides, this blog also includes great tips and showcases for Tableau.

\begin{enumerate}
\def\labelenumi{\arabic{enumi}.}
\setcounter{enumi}{6}
\tightlist
\item
  Uber: Crafting Data-Driven Maps
\end{enumerate}

Map visualization is very important for companies like Uber that needs
to track metrics using geo space points. In this article, the designer
from Uber talks about the challenges of design such visualization and
their solutions. While a lot of the problems are related to the large
scale of the data, there are some insights on using scatter plots and
hex bins, adding trip lines and making custom tools to help make
decisions. The visualization in this article is beneficial for
developing geo spatial graphics.

Reference:
\url{https://medium.com/uber-design/crafting-data-driven-maps-b0835b620554}

Kissmetrics blog: visualization of metrics

Kissmetrics blog is a place where people talk about analytics, marketing
and testing through narratives and metrics visualization. Metrics are
important in real-life world especially when developing/promoting
products. Visualization of metrics are also essential so that
stakeholders can monitor performance, identify problems and deep dive
into potential issues.

A good example from the Kiss metrics blog is about Facebook's Organic
Reach. One important point in the blog discussed whether the Facebook's
organic reach is decreasing drastically. The general trend shows that
there is a huge decline in Facebook's page organic reach.

The following graphs show that the engagement is actually increasing,
meaning while the quantity of content is decreasing, the quality is
increasing. \includegraphics{images/average-facebook-reach.png}

\begin{figure}
\centering
\includegraphics{images/average-facebook-daily-reach.png}
\caption{}
\end{figure}

This resonates with what we have learnt at class in terms of how
different perspectives of interpreting data can lead to different
conclusions.

Reference:\url{https://blog.kissmetrics.com/is-facebook-organic-reach-dead/}

\subsection{15 Data Visualizations That Will Blow Your
Mind}\label{data-visualizations-that-will-blow-your-mind}

\textbf{Reference}

Allison Stadd,January 21, 2015.15 Data Visualizations That Will Blow
Your Mind,Udacity.
\url{https://blog.udacity.com/2015/01/15-data-visualizations-will-blow-mind.html}

If a picture is worth a thousand words, a data visualization is worth at
least a million.

As inspiration for your own work with data, check out these 15 data
visualizations that will wow you. Taken together, this roundup is an
at-a-glance representation of the range of uses data analysis has, from
pop culture to public good.

\begin{enumerate}
\def\labelenumi{\arabic{enumi}.}
\tightlist
\item
  Every Satellite Orbiting Earth
\end{enumerate}

\begin{itemize}
\tightlist
\item
  \url{https://qz.com/296941/interactive-graphic-every-active-satellite-orbiting-earth/}
\end{itemize}

This interactive graph, built using a database from the Union of
Concerned Scientists, displays the trajectories of the 1,300 active
satellites orbiting the Earth as you read this. Each satellite is
represented by a circular icon, color-coded by country and sized
according to launch mass.

\begin{enumerate}
\def\labelenumi{\arabic{enumi}.}
\setcounter{enumi}{1}
\tightlist
\item
  Simpson's Paradox
\end{enumerate}

\begin{itemize}
\tightlist
\item
  \url{http://vudlab.com/simpsons/}
\end{itemize}

The Visualizing Urban Data Idealab (VUDlab) out of the University of
California-Berkeley put together this visual look at data that disproves
the claim in a 1973 suit that charged the school with sex
discrimination. Though the graduate schools had accepted 44\% of male
applicants but only 35\% of female applicants, researchers later
uncovered that if the data were properly pooled, there was actually a
small but statistically significant bias in favor of women. That's
called a Simpson's Paradox.

\begin{enumerate}
\def\labelenumi{\arabic{enumi}.}
\setcounter{enumi}{2}
\tightlist
\item
  Charles Minard's Visualization of Napoleon's 1812 March
\end{enumerate}

\begin{itemize}
\tightlist
\item
  \url{https://www.edwardtufte.com/tufte/minard}
\end{itemize}

This classic lithograph dates back to 1869, displaying the number of men
in Napoleon's 1812 Russian army, their movements, and the temperatures
they encountered along their way. It's been called one of the ``best
statistical drawings ever created.'' The work is an important reminder
that the fundamentals of data visualization lie in a nuanced
understanding of the many dimensions of data. Tools like D3.js and HTML
are no good without a firm grasp of your dataset and sharp communication
skills.

\begin{enumerate}
\def\labelenumi{\arabic{enumi}.}
\setcounter{enumi}{3}
\tightlist
\item
  Hans Rosling's 200 Countries, 200 Years, 4 Minutes
\end{enumerate}

\begin{itemize}
\tightlist
\item
  \url{https://www.youtube.com/watch?feature=player_embedded\&v=jbkSRLYSojo}
\end{itemize}

Global health data expert Hans Rosling's famous statistical documentary
The Joy of Stats aired on BBC in 2010, but it's still turning heads. One
segment in particular is pretty mind-blowing. In ``200 Countries, 200
Years, 4 Minutes,'' Rosling uses augmented reality to explore public
health data in 200 countries over 200 years using 120,000 numbers, in
just four minutes.

\begin{enumerate}
\def\labelenumi{\arabic{enumi}.}
\setcounter{enumi}{4}
\tightlist
\item
  Renting vs.~Buying
\end{enumerate}

\begin{itemize}
\tightlist
\item
  \url{https://www.nytimes.com/interactive/2014/upshot/buy-rent-calculator.html}
\end{itemize}

Mike Bostock, New York Times graphics department editor and inventor of
D3.js, built a complex interactive data calculator that offers a
cost/benefit analysis for prospective homebuyers. Along with his
colleagues Shan Charter and Archie Tse, Bostock tapped into everything
from home price and mortgage-interest tax deduction to property tax rate
and inflation to help you determine whether to rent or buy a home.

\begin{enumerate}
\def\labelenumi{\arabic{enumi}.}
\setcounter{enumi}{5}
\tightlist
\item
  Music Timeline
\end{enumerate}

\begin{itemize}
\tightlist
\item
  \url{https://research.google.com/bigpicture/music/}
\end{itemize}

Google's Music Timeline illustrates a variety of music genres waxing and
waning in popularity from 2010 to present day, based on how many Google
Play Music users have an artist or album in their library, and other
data such as album release dates.

\begin{enumerate}
\def\labelenumi{\arabic{enumi}.}
\setcounter{enumi}{6}
\tightlist
\item
  State of the Union 2014 Minute by Minute on Twitter
\end{enumerate}

\begin{itemize}
\tightlist
\item
  \url{http://twitter.github.io/interactive/sotu2014/\#p1}
\end{itemize}

Twitter's data team assembled an impressive interactive data hub that
depicts how Twitter users across the globe reacted to each paragraph of
President Obama's 2014 State of the Union address. You can slice and
dice the data by topic hashtag (for example, \#budget, \#defense, or
\#education) and state. Pretty powerful.

\begin{enumerate}
\def\labelenumi{\arabic{enumi}.}
\setcounter{enumi}{7}
\tightlist
\item
  NYC Street Trees
\end{enumerate}

\begin{itemize}
\tightlist
\item
  \url{https://www.cloudred.com/labprojects/nyctrees/\#about}
\end{itemize}

Using data from NYC Open Data, this interactive visualization shows the
variety and quantity of street trees planted across the five New York
City boroughs.

\begin{enumerate}
\def\labelenumi{\arabic{enumi}.}
\setcounter{enumi}{8}
\tightlist
\item
  Millennial Generation Diversity
\end{enumerate}

\begin{itemize}
\tightlist
\item
  \url{http://money.cnn.com/interactive/economy/diversity-millennials-boomers/}
\end{itemize}

CNNMoney's interactive chart showing the size and diversity of the
millennial generation compared to baby boomers was built using U.S.
Census Data. It turns dry numbers into an intriguing story, illustrating
the racial makeup of different age groups from 1913 to present.

\begin{enumerate}
\def\labelenumi{\arabic{enumi}.}
\setcounter{enumi}{9}
\tightlist
\item
  Goldilocks Exoplanets
\end{enumerate}

\begin{itemize}
\tightlist
\item
  \url{https://news.nationalgeographic.com/news/2014/04/140417-exoplanet-interactive/}
\end{itemize}

Using data from the Planetary Habitability Laboratory at the University
of Puerto Rico, the interactive graph plots planetary mass, atmospheric
pressure, and temperature to determine what exoplanets might be home, or
have been home at one point, to living beings.

\begin{enumerate}
\def\labelenumi{\arabic{enumi}.}
\setcounter{enumi}{10}
\tightlist
\item
  Washington Wizards' Shooting Stars
\end{enumerate}

\begin{itemize}
\tightlist
\item
  \url{http://www.washingtonpost.com/wp-srv/special/sports/wizards-shooting-stars/}
\end{itemize}

This detailed data visualization demonstrates D.C.'s basketball team's
shooting success during the 2013 season. Using stats released by the
NBA, the visualization lets you examine data for each of 15 players. See
how successful each person was at a variety of types of shots from a
range of spots on the court, compared with others in the league.

\begin{enumerate}
\def\labelenumi{\arabic{enumi}.}
\setcounter{enumi}{11}
\tightlist
\item
  U.S. Migration Patterns
\end{enumerate}

\begin{itemize}
\tightlist
\item
  \url{https://www.nytimes.com/interactive/2014/08/13/upshot/where-people-in-each-state-were-born.html?abt=0002\&abg=0}
\end{itemize}

The New York Times data team mapped out Americans' moving patterns from
1900 to present, and the results are fascinating to play around with.
You can see where people living in each state were born, and to what
states people move from others.

\begin{enumerate}
\def\labelenumi{\arabic{enumi}.}
\setcounter{enumi}{12}
\tightlist
\item
  Selfie City
\end{enumerate}

\begin{itemize}
\tightlist
\item
  \url{http://selfiecity.net/}
\end{itemize}

Selfie City, a detailed multi-component visual exploration of 3,200
selfies from five major cities around the world, offers a close look at
the demographics and trends of selfies. The team behind the project
collected and filtered the data using Instagram and Mechanical Turk.
Explore the differences between selfies snapped in, say, New York and
Berlin, as well as those between men and women across the world.

\begin{enumerate}
\def\labelenumi{\arabic{enumi}.}
\setcounter{enumi}{13}
\tightlist
\item
  The American Workday
\end{enumerate}

\begin{itemize}
\tightlist
\item
  \url{https://www.npr.org/sections/money/2014/08/27/343415569/whos-in-the-office-the-american-workday-in-one-graph?/templates/story/story_php}=
\end{itemize}

NPR tapped into American Time Use Survey data to ascertain the share of
workers in a wide range of industries who are at work at any given time.
The chart overlays the traditional 9 AM-5 PM standard over the graph for
a reference point, helping you draw interesting conclusions.

\begin{enumerate}
\def\labelenumi{\arabic{enumi}.}
\setcounter{enumi}{14}
\tightlist
\item
  Global Carbon Emissions
\end{enumerate}

\begin{itemize}
\tightlist
\item
  \url{https://www.theguardian.com/environment/ng-interactive/2014/dec/01/carbon-emissions-past-present-and-future-interactive}
\end{itemize}

This data visualization, based on data from the World Resource
Institute's Climate Analysis Indicators Tool and the Intergovernmental
Panel on Climate Change, shows how national CO₂ emissions have
transformed over the last 150 years and what the future might hold.
Explore emissions by country for a range of different scenarios.

\chapter{Patterns}\label{patterns}

\begin{itemize}
\tightlist
\item
  Reusable solutions to everyday data visualization questions
\item
  Applied by multiple members of the course
\end{itemize}

\section{Why pie chart is bad: a comparison with bar
chart}\label{why-pie-chart-is-bad-a-comparison-with-bar-chart}

Using pie chart is usually considered as a bad idea when it comes to
data visualization. But why? Here, we explore some cons of using pie
chart to convey information and compare its effectiveness to bar chart
\citep{hickey-pie-worst} \citep{henry-defense-pie} \citep{quach-penny}.

\begin{enumerate}
\def\labelenumi{\arabic{enumi}.}
\tightlist
\item
  Some information may look nearly identical in pie chart. But if the
  data is presented with bar charts, the story is different. See figure
  \ref{fig:hickey-before} and \ref{fig:hickey-after} for examples.
\end{enumerate}

\begin{figure}
\includegraphics[width=0.7\linewidth]{images/hickey-before} \caption{Are there any differences among the pollings at points A, B and C? (Source: [@hickey-pie-worst])}\label{fig:hickey-before}
\end{figure}

\begin{figure}
\includegraphics[width=0.7\linewidth]{images/hickey-after} \caption{The differences can be clearly told from the bar charts. (Source: [@hickey-pie-worst])}\label{fig:hickey-after}
\end{figure}

\begin{enumerate}
\def\labelenumi{\arabic{enumi}.}
\setcounter{enumi}{1}
\tightlist
\item
  It is difficult to compare the slices of a circle to figure out the
  distinctions in size between each pie slice, especially when there are
  a lot of categories. See figure \ref{fig:hickey-breakdown} for
  example.
\end{enumerate}

\begin{figure}
\includegraphics[width=0.7\linewidth]{images/hickey-breakdown} \caption{It is hard to compare the size of the slides. (Source: [@hickey-pie-worst])}\label{fig:hickey-breakdown}
\end{figure}

\begin{enumerate}
\def\labelenumi{\arabic{enumi}.}
\setcounter{enumi}{2}
\tightlist
\item
  Pie chart is easy to be manipulated (e.g.~using a 3D pie chart). See
  figure \ref{fig:hickey-3D} for example.
\end{enumerate}

\begin{figure}
\includegraphics[width=0.7\linewidth]{images/hickey-3D} \caption{S and D (red) appears to be roughly even with EPP (teal) in a 3D pie chart. (Source: [@hickey-pie-worst])}\label{fig:hickey-3D}
\end{figure}

\begin{enumerate}
\def\labelenumi{\arabic{enumi}.}
\setcounter{enumi}{3}
\tightlist
\item
  Pie chart may be useful when comparing 2 different categories with
  different amounts of information. Specifically, it does a better job
  to distinguish two parts with a 25:75 split or one that is not 50:50
  as people are sensitive to a right angle or a dividing line that is
  not straight. But this could be simply done by showing two numbers!
  See figure \ref{fig:henry-quarter} and \ref{fig:henry-half} for
  examples.
\end{enumerate}

\section{Chose the right baseline in data
visualization}\label{chose-the-right-baseline-in-data-visualization}

Baseline is very important to data visualization. If baseline is
different, the meanning will change a lot. Now here is a case study to
show the importance of baseline and how to use it in different ways.

Here I use the same method for a new dataset to .

\begin{Shaded}
\begin{Highlighting}[]
\CommentTok{# Create the data.}
\NormalTok{a <-}\KeywordTok{rep}\NormalTok{(}\KeywordTok{c}\NormalTok{(}\DecValTok{2010}\NormalTok{,}\DecValTok{2011}\NormalTok{,}\DecValTok{2012}\NormalTok{,}\DecValTok{2013}\NormalTok{,}\DecValTok{2014}\NormalTok{,}\DecValTok{2015}\NormalTok{),}\DataTypeTok{each =} \DecValTok{4}\NormalTok{)}
\NormalTok{b <-}\StringTok{ }\KeywordTok{seq}\NormalTok{(}\DecValTok{1}\OperatorTok{:}\DecValTok{24}\NormalTok{)}
\NormalTok{c <-}\StringTok{ }\KeywordTok{c}\NormalTok{(}\FloatTok{64.9}\NormalTok{,}\FloatTok{65.33}\NormalTok{,}\FloatTok{71.67}\NormalTok{,}\FloatTok{79.17}\NormalTok{,}\FloatTok{68.78}\NormalTok{,}\FloatTok{69.83}\NormalTok{,}\FloatTok{78.61}\NormalTok{,}\FloatTok{92.68}\NormalTok{,}\FloatTok{89.28}\NormalTok{,}\FloatTok{90.43}\NormalTok{,}\FloatTok{97.96}\NormalTok{,}\FloatTok{106.96}\NormalTok{,}\FloatTok{100.66}\NormalTok{,}\FloatTok{107.53}\NormalTok{,}\FloatTok{117.06}\NormalTok{,}\FloatTok{119.21}\NormalTok{,}\FloatTok{110.05}\NormalTok{,}\FloatTok{97.42}\NormalTok{,}\FloatTok{93.62}\NormalTok{,}\FloatTok{97.99}\NormalTok{,}\DecValTok{80}\NormalTok{,}\FloatTok{88.74}\NormalTok{,}\FloatTok{102.06}\NormalTok{,}\DecValTok{83}\NormalTok{)}
\NormalTok{data <-}\StringTok{ }\KeywordTok{as.data.frame}\NormalTok{(}\KeywordTok{cbind}\NormalTok{(a,b,c))}
\KeywordTok{colnames}\NormalTok{(data) <-}\KeywordTok{c}\NormalTok{(}\StringTok{"year"}\NormalTok{,}\StringTok{"quater"}\NormalTok{,}\StringTok{"sales"}\NormalTok{)}
\end{Highlighting}
\end{Shaded}

\begin{enumerate}
\def\labelenumi{\arabic{enumi}.}
\tightlist
\item
  Regular quaterly sales. We can see sales decreased a lot around 2014.
  \textbf{The baseline here is historical sales.}
\end{enumerate}

\begin{Shaded}
\begin{Highlighting}[]
\CommentTok{# Regular time series for sales}
\KeywordTok{par}\NormalTok{(}\DataTypeTok{cex.axis=}\FloatTok{0.7}\NormalTok{)}
\NormalTok{data.ts <-}\StringTok{ }\KeywordTok{ts}\NormalTok{(data}\OperatorTok{$}\NormalTok{sales, }\DataTypeTok{start=}\KeywordTok{c}\NormalTok{(}\DecValTok{2010}\NormalTok{, }\DecValTok{1}\NormalTok{), }\DataTypeTok{frequency=}\DecValTok{4}\NormalTok{)}
\KeywordTok{plot}\NormalTok{(data.ts, }\DataTypeTok{xlab=}\StringTok{""}\NormalTok{, }\DataTypeTok{ylab=}\StringTok{""}\NormalTok{, }\DataTypeTok{main=}\StringTok{"sales per quater"}\NormalTok{, }\DataTypeTok{las=}\DecValTok{1}\NormalTok{, }\DataTypeTok{bty=}\StringTok{"n"}\NormalTok{)}
\end{Highlighting}
\end{Shaded}

\begin{enumerate}
\def\labelenumi{\arabic{enumi}.}
\setcounter{enumi}{1}
\tightlist
\item
  Quaterly and yearly change sales. \textbf{The baseline here is zero
  and look at the percentage changes.}
\end{enumerate}

\begin{Shaded}
\begin{Highlighting}[]
 \CommentTok{# Quaterly change}
\NormalTok{curr <-}\StringTok{ }\KeywordTok{as.numeric}\NormalTok{(data}\OperatorTok{$}\NormalTok{sales[}\OperatorTok{-}\DecValTok{1}\NormalTok{])}
\NormalTok{prev <-}\StringTok{ }\KeywordTok{as.numeric}\NormalTok{(data}\OperatorTok{$}\NormalTok{sales[}\DecValTok{1}\OperatorTok{:}\NormalTok{(}\KeywordTok{length}\NormalTok{(data}\OperatorTok{$}\NormalTok{sales)}\OperatorTok{-}\DecValTok{1}\NormalTok{)])}
\NormalTok{quaChange <-}\StringTok{ }\DecValTok{100} \OperatorTok{*}\StringTok{ }\KeywordTok{round}\NormalTok{( (curr}\OperatorTok{-}\NormalTok{prev) }\OperatorTok{/}\StringTok{ }\NormalTok{prev, }\DecValTok{2}\NormalTok{ )}
\NormalTok{barCols <-}\StringTok{ }\KeywordTok{sapply}\NormalTok{(quaChange, }
    \ControlFlowTok{function}\NormalTok{(x) \{ }
        \ControlFlowTok{if}\NormalTok{ (x }\OperatorTok{<}\StringTok{ }\DecValTok{0}\NormalTok{) \{}
            \KeywordTok{return}\NormalTok{(}\StringTok{"#2cbd25"}\NormalTok{)}
\NormalTok{        \} }\ControlFlowTok{else}\NormalTok{ \{}
            \KeywordTok{return}\NormalTok{(}\StringTok{"gray"}\NormalTok{)}
\NormalTok{        \}}
\NormalTok{    \})}
\CommentTok{#monChange.ts <- ts(monChange, start=c(1976, 2), frequency=12)}
\KeywordTok{barplot}\NormalTok{(quaChange, }\DataTypeTok{border=}\OtherTok{NA}\NormalTok{, }\DataTypeTok{space=}\DecValTok{0}\NormalTok{, }\DataTypeTok{las=}\DecValTok{1}\NormalTok{, }\DataTypeTok{col=}\NormalTok{barCols, }\DataTypeTok{main=}\StringTok{"% change, quaterly"}\NormalTok{)}
\end{Highlighting}
\end{Shaded}

\begin{Shaded}
\begin{Highlighting}[]
\CommentTok{# Year-over-year change}
\NormalTok{curr <-}\StringTok{ }\KeywordTok{as.numeric}\NormalTok{(data}\OperatorTok{$}\NormalTok{sales[}\OperatorTok{-}\NormalTok{(}\DecValTok{1}\OperatorTok{:}\DecValTok{4}\NormalTok{)])}
\NormalTok{prev <-}\StringTok{ }\KeywordTok{as.numeric}\NormalTok{(data}\OperatorTok{$}\NormalTok{sales[}\DecValTok{1}\OperatorTok{:}\NormalTok{(}\KeywordTok{length}\NormalTok{(data}\OperatorTok{$}\NormalTok{sales)}\OperatorTok{-}\DecValTok{4}\NormalTok{)])}
\NormalTok{annChange <-}\StringTok{ }\DecValTok{100} \OperatorTok{*}\StringTok{ }\KeywordTok{round}\NormalTok{( (curr}\OperatorTok{-}\NormalTok{prev) }\OperatorTok{/}\StringTok{ }\NormalTok{prev, }\DecValTok{2}\NormalTok{ )}
\NormalTok{barCols <-}\StringTok{ }\KeywordTok{sapply}\NormalTok{(annChange, }
    \ControlFlowTok{function}\NormalTok{(x) \{ }
        \ControlFlowTok{if}\NormalTok{ (x }\OperatorTok{<}\StringTok{ }\DecValTok{0}\NormalTok{) \{}
            \KeywordTok{return}\NormalTok{(}\StringTok{"#2cbd25"}\NormalTok{)}
\NormalTok{        \} }\ControlFlowTok{else}\NormalTok{ \{}
            \KeywordTok{return}\NormalTok{(}\StringTok{"gray"}\NormalTok{)}
\NormalTok{        \}}
\NormalTok{    \})}
\KeywordTok{barplot}\NormalTok{(annChange, }\DataTypeTok{border=}\OtherTok{NA}\NormalTok{, }\DataTypeTok{space=}\DecValTok{0}\NormalTok{, }\DataTypeTok{las=}\DecValTok{1}\NormalTok{, }\DataTypeTok{col=}\NormalTok{barCols, }\DataTypeTok{main=}\StringTok{"% change, annual"}\NormalTok{)}
\end{Highlighting}
\end{Shaded}

From this plot, it is very clear that the magnitude of drops in sales
for some quaters.

\begin{enumerate}
\def\labelenumi{\arabic{enumi}.}
\setcounter{enumi}{2}
\tightlist
\item
  The sales difference compare to now. \textbf{The baseline here is the
  current sales.}
\end{enumerate}

\begin{Shaded}
\begin{Highlighting}[]
\CommentTok{# Relative to current 2015}
\NormalTok{curr <-}\StringTok{ }\KeywordTok{as.numeric}\NormalTok{(data}\OperatorTok{$}\NormalTok{sales[}\KeywordTok{length}\NormalTok{(data}\OperatorTok{$}\NormalTok{sales)])}
\NormalTok{salesDiff <-}\StringTok{ }\KeywordTok{as.numeric}\NormalTok{(data}\OperatorTok{$}\NormalTok{sales) }\OperatorTok{-}\StringTok{ }\NormalTok{curr}
\NormalTok{barCols.diff <-}\StringTok{ }\KeywordTok{sapply}\NormalTok{(salesDiff,}
    \ControlFlowTok{function}\NormalTok{(x) \{}
        \ControlFlowTok{if}\NormalTok{ (x }\OperatorTok{<}\StringTok{ }\DecValTok{0}\NormalTok{) \{}
            \KeywordTok{return}\NormalTok{(}\StringTok{"gray"}\NormalTok{)}
\NormalTok{        \} }\ControlFlowTok{else}\NormalTok{ \{}
            \KeywordTok{return}\NormalTok{(}\StringTok{"black"}\NormalTok{)}
\NormalTok{        \}}
\NormalTok{    \}}
\NormalTok{)}
\KeywordTok{barplot}\NormalTok{(salesDiff, }\DataTypeTok{border=}\OtherTok{NA}\NormalTok{, }\DataTypeTok{space=}\DecValTok{0}\NormalTok{, }\DataTypeTok{las=}\DecValTok{1}\NormalTok{, }\DataTypeTok{col=}\NormalTok{barCols.diff, }\DataTypeTok{main=}\StringTok{"Sales difference from last quater 2015"}\NormalTok{)}
\end{Highlighting}
\end{Shaded}

\begin{enumerate}
\def\labelenumi{\arabic{enumi}.}
\setcounter{enumi}{3}
\tightlist
\item
  Sales difference compared to the first quater. ** The baseline here is
  the first quater sales.**
\end{enumerate}

\begin{Shaded}
\begin{Highlighting}[]
\CommentTok{# Relative to first quater}
\NormalTok{ori <-}\StringTok{ }\KeywordTok{as.numeric}\NormalTok{(data}\OperatorTok{$}\NormalTok{sales[}\DecValTok{1}\NormalTok{])}
\NormalTok{salesDiff <-}\StringTok{ }\KeywordTok{as.numeric}\NormalTok{(data}\OperatorTok{$}\NormalTok{sales) }\OperatorTok{-}\StringTok{ }\NormalTok{ori}
\NormalTok{barCols.diff <-}\StringTok{ }\KeywordTok{sapply}\NormalTok{(salesDiff,}
    \ControlFlowTok{function}\NormalTok{(x) \{}
        \ControlFlowTok{if}\NormalTok{ (x }\OperatorTok{<}\StringTok{ }\DecValTok{0}\NormalTok{) \{}
            \KeywordTok{return}\NormalTok{(}\StringTok{"gray"}\NormalTok{)}
\NormalTok{        \} }\ControlFlowTok{else}\NormalTok{ \{}
            \KeywordTok{return}\NormalTok{(}\StringTok{"black"}\NormalTok{)}
\NormalTok{        \}}
\NormalTok{    \}}
\NormalTok{)}
\KeywordTok{barplot}\NormalTok{(salesDiff, }\DataTypeTok{border=}\OtherTok{NA}\NormalTok{, }\DataTypeTok{space=}\DecValTok{0}\NormalTok{, }\DataTypeTok{las=}\DecValTok{1}\NormalTok{, }\DataTypeTok{col=}\NormalTok{barCols.diff, }\DataTypeTok{main=}\StringTok{"Sales difference from first quater 2010"}\NormalTok{)}
\end{Highlighting}
\end{Shaded}

\begin{enumerate}
\def\labelenumi{\arabic{enumi}.}
\setcounter{enumi}{4}
\tightlist
\item
  The difference between quater sales and mean. ** The baseline is mean
  now.**
\end{enumerate}

\begin{Shaded}
\begin{Highlighting}[]
\CommentTok{# difference from the mean}
\NormalTok{mean <-}\StringTok{ }\KeywordTok{mean}\NormalTok{(}\KeywordTok{as.numeric}\NormalTok{(data}\OperatorTok{$}\NormalTok{sales))}
\NormalTok{salesDiff <-}\StringTok{ }\KeywordTok{as.numeric}\NormalTok{(data}\OperatorTok{$}\NormalTok{sales) }\OperatorTok{-}\StringTok{ }\NormalTok{mean}
\NormalTok{barCols.diff <-}\StringTok{ }\KeywordTok{sapply}\NormalTok{(salesDiff,}
    \ControlFlowTok{function}\NormalTok{(x) \{}
        \ControlFlowTok{if}\NormalTok{ (x }\OperatorTok{<}\StringTok{ }\DecValTok{0}\NormalTok{) \{}
            \KeywordTok{return}\NormalTok{(}\StringTok{"gray"}\NormalTok{)}
\NormalTok{        \} }\ControlFlowTok{else}\NormalTok{ \{}
            \KeywordTok{return}\NormalTok{(}\StringTok{"black"}\NormalTok{)}
\NormalTok{        \}}
\NormalTok{    \}}
\NormalTok{)}
\KeywordTok{barplot}\NormalTok{(salesDiff, }\DataTypeTok{border=}\OtherTok{NA}\NormalTok{, }\DataTypeTok{space=}\DecValTok{0}\NormalTok{, }\DataTypeTok{las=}\DecValTok{1}\NormalTok{, }\DataTypeTok{col=}\NormalTok{barCols.diff, }\DataTypeTok{main=}\StringTok{"Sales difference from mean"}\NormalTok{)}
\end{Highlighting}
\end{Shaded}

So before we start to plot, we should decide the baseline we want to
use. Different baseline will lead to totally different graphs.

Reference: \url{https://flowingdata.com/2013/11/26/the-baseline/}

\section{Tips to improve Data
Visualization}\label{tips-to-improve-data-visualization}

\subsection{1.Comparison}\label{comparison}

Include a zero baseline if possibleAlthough a line chart does not have
to start at a zero baseline, it should be included if it gives more
context for comparison. If relatively small fluctuations in data are
meaningful (e.g., in stock market data), you may truncate the scale to
showcase these variances; Always choose the most efficient
visualization; Watch your placement You may have two nice stacked bar
charts that are meant to let your reader compare points, but if they're
placed too far apart to ``get'' the comparison, you've already lost;
Tell the whole story Maybe you had a 30\% sales increase in Q4.
Exciting! But what's more exciting? Showing that you've actually had a
100\% sales increase since Q1. \#\#\# 2.Copy Don't over explain If the
copy already mentions a fact, the subhead, callout, and chart header
don't have to reiterate it; Keep chart and graph headers simple and to
the point There's no need to get clever, verbose, or pun-tastic. Keep
any descriptive text above the chart brief and directly related to the
chart underneath. Remember: Focus on the quickest path to comprehension;
Use callouts wisely Callouts are not there to fill space. They should be
used intentionally to highlight relevant information or provide
additional context; Don't use distracting fonts or elements Sometimes
you do need to emphasize a point. If so, only use bold or italic text to
emphasize a point---and don't use them both at the same time. \#\#\#
3.Color Use a single color to represent the same type of data; Watch out
for positive and negative numbers Don't use red for positive numbers or
green for negative numbers. Those color associations are so strong it
will automatically flip the meaning in the viewer's mind; Make sure
there is sufficient contrast between colors; Avoid patterns Stripes and
polka dots sound fun, but they can be incredibly distracting. If you are
trying to differentiate, say, on a map, use different saturations of the
same color. On that note, only use solid-colored lines (not dashes);
Select colors appropriately; Don't use more than 6 colors in a single
layout. \#\#\# 4.Ordering Order data intuitively There should be a
logical hierarchy. Order categories alphabetically, sequentially, or by
value; Order consistently; Order evenly Use natural increments on your
axes (0, 5, 10, 15, 20) instead of awkward or uneven increments (0, 3,
5, 16, 50). \#\#\# 5.Audience perspective Let the users lead;Know your
audience,Designers should consider the way users prefer to understand
information, even in choosing basic analytic approaches. For users to
feel comfortable adopting and sharing insights from analytics, they must
be able to explain and defend the data. \#\#\# 6.Use layers to tell a
story While style is one form of customization, layering unique data
sets on a single visualization can tell a richer narrative and connect
users to the data without getting too crowded. On a map, this can be as
simple as zooming in and out, but it can also involve drill-downs
(choosing a data point and expanding it to show more detail), links and
other shortcuts. \#\#\# 7.Keep it simple Analytic results shouldn't be
presented to 10 decimal places when the user doesn't need that level of
precision to make a decision or understand a concept. Effective visual
interfaces avoid 3-D effects or ornate gauge designs (a.k.a. ``chart
junk'') when simple numbers, maps or graphs will do.

References:
\url{https://www.columnfivemedia.com/25-tips-to-upgrade-your-data-visualization-design}

\section{Tips for Tableau}\label{tips-for-tableau}

Running totals

Common Baseline

Weighted averages

Moving average

Grouping by aggregates

Different years comparison

Appending excel sheets

Bar chart totals

Fixed axis when re-drawing charts

Auto-fitting screen behavior depending on data selection

References:
\url{http://cdn2.hubspot.net/hubfs/257922/Docs/BlueGranite_whitepaper_10useful.pdf}

\chapter{Tips to improve Data
Visualization}\label{tips-to-improve-data-visualization-1}

1.Comparison

Include a zero baseline if possibleAlthough a line chart does not have
to start at a zero baseline, it should be included if it gives more
context for comparison. If relatively small fluctuations in data are
meaningful (e.g., in stock market data), you may truncate the scale to
showcase these variances; Always choose the most efficient
visualization; Watch your placement You may have two nice stacked bar
charts that are meant to let your reader compare points, but if they're
placed too far apart to ``get'' the comparison, you've already lost;
Tell the whole story Maybe you had a 30\% sales increase in Q4.
Exciting! But what's more exciting? Showing that you've actually had a
100\% sales increase since Q1.

2.Copy

Don't over explain If the copy already mentions a fact, the subhead,
callout, and chart header don't have to reiterate it; Keep chart and
graph headers simple and to the point There's no need to get clever,
verbose, or pun-tastic. Keep any descriptive text above the chart brief
and directly related to the chart underneath. Remember: Focus on the
quickest path to comprehension; Use callouts wisely Callouts are not
there to fill space. They should be used intentionally to highlight
relevant information or provide additional context; Don't use
distracting fonts or elements Sometimes you do need to emphasize a
point. If so, only use bold or italic text to emphasize a point---and
don't use them both at the same time.

3.Color

Use a single color to represent the same type of data; Watch out for
positive and negative numbers Don't use red for positive numbers or
green for negative numbers. Those color associations are so strong it
will automatically flip the meaning in the viewer's mind; Make sure
there is sufficient contrast between colors; Avoid patterns Stripes and
polka dots sound fun, but they can be incredibly distracting. If you are
trying to differentiate, say, on a map, use different saturations of the
same color. On that note, only use solid-colored lines (not dashes);
Select colors appropriately; Don't use more than 6 colors in a single
layout.

4.Ordering

Order data intuitively There should be a logical hierarchy. Order
categories alphabetically, sequentially, or by value; Order
consistently; Order evenly Use natural increments on your axes (0, 5,
10, 15, 20) instead of awkward or uneven increments (0, 3, 5, 16, 50).

5.Audience perspective

Let the users lead;Know your audience,Designers should consider the way
users prefer to understand information, even in choosing basic analytic
approaches. For users to feel comfortable adopting and sharing insights
from analytics, they must be able to explain and defend the data.

6.Use layers to tell a story

While style is one form of customization, layering unique data sets on a
single visualization can tell a richer narrative and connect users to
the data without getting too crowded. On a map, this can be as simple as
zooming in and out, but it can also involve drill-downs (choosing a data
point and expanding it to show more detail), links and other shortcuts.

7.Keep it simple

Analytic results shouldn't be presented to 10 decimal places when the
user doesn't need that level of precision to make a decision or
understand a concept. Effective visual interfaces avoid 3-D effects or
ornate gauge designs (a.k.a. ``chart junk'') when simple numbers, maps
or graphs will do.

References:
\url{https://www.columnfivemedia.com/25-tips-to-upgrade-your-data-visualization-design}
\url{http://www.govtech.com/pcio/10-Tips-for-Data-Visualization.html}

\chapter{Tips for Tableau}\label{tips-for-tableau-1}

\begin{enumerate}
\def\labelenumi{\arabic{enumi}.}
\tightlist
\item
  Running totals
\item
  Common Baseline
\item
  Weighted averages
\item
  Moving average
\item
  Grouping by aggregates
\item
  Different years comparison
\item
  Appending excel sheets
\item
  Bar chart totals
\item
  Fixed axis when re-drawing charts
\item
  Auto-fitting screen behavior depending on data selection References:
  \url{http://cdn2.hubspot.net/hubfs/257922/Docs/BlueGranite_whitepaper_10useful.pdf}
\end{enumerate}

\chapter{Ethics}\label{ethics}

\begin{itemize}
\tightlist
\item
  Implications of (good and bad) data visualization

  \begin{itemize}
  \tightlist
  \item
    The role of data visualization in politics, society, and business
  \end{itemize}
\end{itemize}

Tableau: Viz of the Day

Tableau has a gallery called Viz of the Day
(\url{https://public.tableau.com/en-us/s/gallery} ) that displays great
data visualization examples created by Tableau. It is cool to see how
people are using all kinds of data to create informative yet fun data
visuals. Data being used is also attached so we can try to mimic what
other people did as well.

One interesting example I found is Describe Artists with Emoji
(\url{https://public.tableau.com/en-us/s/gallery/what-emoji-say-about-music?gallery=featured}).
Using the data from Spotify, the author listed the 10 most distinctive
emoji used in the playlists related to popular artists. The table being
used in this visual is very straight forward to link artist to the
emojis and is very easy to compare among artists. When you hover over
the emoji, further information is presented.

\textbf{1. Data visualization in political and social sciences} -
(Reference:
\url{https://github.com/mschermann/data_viz_reader/files/1933699/Zinovyev_Data_Visualization.pdf})

The basic objective of data visualization is to provide an efficient
graphical display for summarizing and reasoning about quantitative
information. And during the last decades, political science has
accumulated a large corpus of various kinds of data, which makes it
gradually become a more quantitative scientific field and requires using
quantitative information in the analysis and reasoning.

Data visualization plays several important roles in it: 1) helps create
informative illustrations of the data, recapitulating large amount of
quantitative information on a diagram; 2) helps formulate new or
supporting existing hypotheses from quantitative data; 3) guides a
statistical analysis of data and checks its validity.

Some useful visualization methods are: 1) \emph{Statistical graphics and
infographics}; 2) \emph{Geographical information systems (GIS)}; 3)
\emph{Graph visualization or network maps}; 4) \emph{Data cartography}.

\textbf{2. Role of data visualization in business}
-(Reference:\url{https://www.iotforall.com/data-visualization-strategy-for-business/})

According to an Experian report, 95\% of U.S. organizations say that
they use data to power business opportunities, and another 84 percent
believe data is an integral part of forming a business strategy.
Visualization helps data impact business in following ways:

1)\emph{Cleansing}

The simplest way to explain the importance of visualization is to look
at visualization as the means to making sense of data. Even the most
basic, widely-used data visualization tools that combine simple pie
charts and bar graphs help people comprehend large amounts of
information fast and easily, compared to paper reports and spreadsheets.

In other words, visualization is the initial filter for the quality of
data streams. Combining data from various sources, visualization tools
perform preliminary standardization, shape data in a unified way and
create easy-to-verify visual objects. As a result, these tools become
indispensable for data cleansing and vetting and help companies prepare
quality assets to derive valuable insights.

2)\emph{Extracting}

Known versatile tools for data visualization and analytics -- Elastic
Stack, Tableau, Highcharts, and more complex database solutions like
Hadoop, Amazon AWS and Teradata, have wide applications in business,
from monitoring performance to improving customer experience on mobile
tools. New generation of data visualization based on AR and VR
technology, however, provides formerly infeasible advantages in terms of
identifying patterns and drawing insights from various data streams.

Building 3D data visualization spaces, companies can create an intuitive
environment that helps data scientists grasp and analyze more data
streams at the same time, observe data points from multiple dimensions,
identify previously unavailable dependencies and manipulate data by
naturally moving objects, zooming, and focusing on more granulated
areas. Moreover, these tools allow us to expand the capabilities of data
visualization by creating collaborative 3D environments for teams. As a
result, new technology helps extract more valuable insights from the
same volume of data.

3)\emph{Strategizing}

As the amount of data grows, it becomes harder to catch up with it.
Therefore, data strategy becomes the necessary part of the success in
applying data to business. Then how data visualization become an
important tool in your strategic kit? First, it helps you cleanse your
data. Secondly, it allows you to identify and extract meaningful
information from it. Finally, data visualization tools enable continuous
real-time monitoring of how your strategy and now data-driven decisions
influence performance and business outcomes. In other words, these tools
visualize not only the data, but also the results, and help correct and
optimize strategy on the go.

Data visualization is one of the initial steps made to derive value from
data. It's also one of the most important steps, as it determines how
efficiently analysts can work with data assets, what insights they are
able to extract and how their data strategy will develop over time.

Therefore, the quality and capabilities of data visualization directly
influence how data impacts your business strategy and what benefits data
applications can bring to the companies and their industries.\_

\textbf{Implications of (good and bad) data visualization}

Raw data is often meaningless or their meaning is not easily concluded.
When people face a large set of measurements they are unable or
unwilling to spend the time required to process it. Our modern living
contributes to an ever-growing pool of ``big data'' and our ability to
collect this type of information becomes easier and easier. Thus
filtering, visualization, and interpretation of data become increasingly
important.

We should understand what to do with data, but first we should
understand why their presentation in graphical format is so powerful.

\begin{itemize}
\item
  EASY RECALL - People can process images more quickly than words.As
  data are transformed into imagery, the readability and cognition of
  the content greatly improve. While people can only seem to remember
  just 10\% of what they hear and only 20\% of what they read, retention
  jumps up to 80\% when they see visual ingormation and do some
  modifications in them.
\item
  PROVIDES WINDOW FOR PERSPECTIVE- With infographics you can pack a lot
  of information into a small space. Colors, shape, movement, contrast
  in scale and weight, and even sound can be used to denote different
  aspects of the data allowing for multi-layered understanding. Below is
  an example for a good graph: Reference: \citep{image_good}
\item
  ENABLES QUALITATIVE ANALYSIS- Color, shape, sounds, and size can make
  evident relationships within data very intuitive. When data points are
  represented as images or components of an entire scene, readers are
  able to see the big picture and understand how the information fits
  within a larger context.
\item
  INCREASE iN USER PARTICIPATION- Interactive infographics can
  substantially increase the amount of time someone will spend with the
  content.
\end{itemize}

Because of their impact, infographics are widely used nowadays. A quick
google will produce a huge array of great examples --- as well as poor
ones. Because while people recognize the value of information graphic
design, and a number of tools are available today that make the creation
of them possible for the layperson, it doesn't mean that they're all
successful or even necessary.

In the example below, the information would be better presented does not
easily answer the simple question: How many airplane seats are left
empty each year? It could have been more clear with the numbers and
comparisons. Reference: \citep{image_bad}

Some useful visualization methods are: 1) \emph{Statistical graphics and
infographics}; 2) \emph{Geographical information systems (GIS)}; 3)
\emph{Graph visualization or network maps}; 4) \emph{Data cartography}.

\textbf{Misrepresentation through Data Visualization} - (Reference:
\url{https://venngage.com/blog/misleading-graphs/})

While the ideal purpose of data visualization is to improve others'
understanding of the data presented, visualization can also be used to
mislead. Some of the main methods of doing so are omitting baselines,
axis manipulation, omitting data, and going against graphing convention.

Omitting baselines is used to imply a greater difference between two
categories, such as in poll results comparing political parties. Axis
manipulation by increasing the highest value on the y-axis affects the
visibility of a slope, making data with an otherwise visible trend
appear flat. Omitting selected data points or narrowing the window of a
graph is used to hide an overall trend, such as a graph of a stock only
showing a current trend and hiding previous bubbles. Graphs can also be
designed to subvert convention so that at first glance the graph is
conveying the opposite message, for example, by using the reader's
associations of colors and temperature to create a graph where hot is
blue and cold is red.

\textbf{A basis for why we should pursue ethical data visualization}
Reference: Cairo, Alberto. ``Ethical Infographics: In data
visualization, journalism meets engineering.'' The IRE Journal, Spring
2014.
\url{https://www.scribd.com/document/230474170/Ethical-Info-Graphics-In-data-visualization-journalism-meets-engineering}
Cairo, Alberto. The Functional Art weblog. 19 June 2016.
\url{http://www.thefunctionalart.com/2014/06/infographics-data-and-visualization.html}

Alberto Cairo addresses into the ethical `why' of data visualization in
this article, while still grounding the discussion in straightforward
analysis of what to do and what not to do. He emphasizes that the
effectiveness of the communicative display is as important as the
information itself. This makes intuitive sense because useful
information is rendered utterly useless if no one can understand it.

Cairo sees data visualization as a harmonization of journalism and
engineering. From these two disciplines, he takes the journalist ethos
of truth-telling and honesty and combines this with an engineering focus
on efficacy and efficiency. The result is a data visualization that
contains accurate and relevant information which is clearly and
concisely conveyed. Cairo describes himself as a ``rule utilitarian''
and uses this to explain why it is ethical or, in his words, ``morally
right,'' to create graphics in this way. Here, it very useful to review
his blogpost introducing the article.

Essentially, the goal is to create the most good while doing the least
harm. As such, conveying truthful and honest relevant information
increases a persons understanding. Increased understanding and knowledge
positively correlates with personal well-being.

So, the information presented must be accurate and relevant. Cairo
briefly addresses guidelines for this which are applicable in all
information gathering fields: beware of selection bias when choosing
preexisting datasets, validate the data, and include important context.
False or irrelevant information doesn't improve anyone's decision-making
capacity, so it cannot enhance well-being.

Even if the information is both accurate and relevant, moral engineering
pitfalls may remain. To avoid the unethical trap of inscrutable (or
misleading) graphics, Cairo exhorts us to take an evidenced based
approach when possible. The purpose of the graphic dictates the form it
takes; aesthetic preferences should never override clarity.

Again, since the ethical purpose is to improve well-being through
understanding, a graphic which is confusing or misleading is unethical,
regardless of intent, since it actually creates misunderstanding for the
audience. While it can be a bit jarring to think of an poorly designed
graphic as ``morally wrong'', it is important to think of the unintended
consequences of visuals which have a powerful impact on their viewers.

\section{Importance of ethics in
visualization}\label{importance-of-ethics-in-visualization}

Ref:

\url{https://backspace.com/notes/2016/01/ethics-in-data-visualization.php}

Over the years, researchers and lawyers have come up with rules and
practices for proper data collection and utilization, with particular
attention on human subject research.

Consent of the subjects to use their data, evaluating any risk with use
or collection of data or protecting anonymity of data are some of the
rules that must be considered for ethical research methods. Under
U.S.law for research institutions receiving federal funding, ethical
aspects must be considered. These rules continue to evolve.

Research have found that even if viewers do not support an idea, data
presented in charts can persuade viewers on the subject matter. It means
that visualization can also be used for deception and there are lot of
techniques that can produce dangerous visualization.Techniques such as
truncated axis (where the y-axis does not start at zero) or using area
to represent quantity (for instance comparing the size of two adjacent
circles) were found to lead to wrong conclusions.

Misleading, incomprehensible, or incredible data visualization can
jeopardize people's trust, goodwill, or faith in research and advocacy
on vital human rights issues. Its ethical responsibility to create
visualizations to give correct and faithful representation of data and
subjects.

\chapter{Conclusion}\label{conclusion}

Reflection, Key Learnings, Outlook

\chapter*{References}\label{references-1}
\addcontentsline{toc}{chapter}{References}

\textbf{1. 3 Expert Data Visualization Tips for Grabbing Readers'
Attention}

URL:
\url{https://towardsdatascience.com/3-expert-data-visualization-tips-for-grabbing-readers-attention-206d8c4621bf}

\emph{Summary}: This article found on Medium explores three important
aspects to focus on when creating a data visualization. The importance
of each aspect is explained along with helpful questions to ask and to
help you evaluate your visualization to ensure it caters to your
audience. Although it primarily focuses on the appearance of visuals, it
also discusses the psychology of the reader as they're looking at a data
visual, which offers a unique and useful perspective.

Here is an outline of each of the 3 aspects: 1. Know what you really
want to say. We want to share patterns, trends, anomalies, etc. with
others through data visuals but we must find the right things to
represent.

\begin{enumerate}
\def\labelenumi{\arabic{enumi}.}
\setcounter{enumi}{1}
\item
  Design. Visuals should be kept as simple as possible without leaving
  out key points. This makes sense because then the audience can focus
  on what's really important.
\item
  Labeling. This section of the article shows a nice comparison of
  before and after removing labels from a chart, and the `after' chart
  looks much cleaner and easier to interpret.
\end{enumerate}

I think often when working with data, we tend to gravitate toward
including more information in a visual, so an important takeaway for me
is that less is more, and not everything we want to show has to be
crammed into one big-picture visual.

\textbf{2. Choose best colors for cartography visualization in a
professional manner}

URL: \url{http://colorbrewer.org}

\emph{Summary}: It has been carefully designed to be a diagnostic tool
for evaluating the robustness of individual color schemes. Full use of
this tool will benefit your map designs because colors (even very
similar colors) are easy to differentiate when they appear in a nicely
ordered sequence (such as a legend). The task of differentiating the
colors, however, becomes much harder when the patterns on the map are
complex, such as in the lower left corner of the diagnostic map.

It will automatically recommend the color schemes in the following
aspects:

1: Can you easily distinguish every color in the random section of the
map (the lower left)? If you have a ten-class map, you should be able to
see clearly ten unique colors.

2: Within each large band of color on the map, we placed several
polygons filled with each map color (`outliers'). For example, if you
have a seven-class map, there will be six outlier colors per band,
demonstrating the appearance of all map colors with each as a
surrounding color. Can you see each outlier clearly? Do all pairs of
outliers in the band look different? If not, perhaps you should choose a
different scheme or fewer classes.

3: You can also change the settings to colorblind-friendly on this site.

\textbf{3. Visualization Tools: An introduction to tools for creating
infographics, timelines and other data visualizations.}

URL:\url{https://guides.library.harvard.edu/c.php?g=310952\&p=2073191}

This website lists lots of tools to do different type of visualizations,
check it out.

\textbf{4. Visual Capitalist}

URL:\url{http://www.visualcapitalist.com/category/politics/}

This company/website creates visual contents in the field of business
and marketing.

\textbf{5. Misleading Graph}

As a student to learn how to be a good data scientis or business
analytics professional, it is important to learn how to read the chart
and interpret the statistic. Graphs can be one of the best ways to
present statistical information, but they can also be one of the most
misleading, even when they are completely accurate. Here, I would like
to share how to detect misleading graphs.Furthermore, we can learn how
to improve our data visualization skills.

\begin{enumerate}
\def\labelenumi{\arabic{enumi}.}
\tightlist
\item
  Omitting Baselines
\end{enumerate}

In the data visulization terms, we call it truncated graph. A truncated
graph (also known as a torn graph) has a y axis that does not start at
0. These graphs can create the impression of important change where
there is relatively little change.Truncated graphs are useful in
illustrating small differences.{[}16{]} Graphs may also be truncated to
save space. Commercial software such as MS Excel will tend to truncate
graphs by default if the values are all within a narrow range.
Truncating graphs make the readers to change their judement for
something that is not significant looks like a huge differece.

A example of using good data in a misleading graph to fool readers comes
from Fox News. \includegraphics{images/1.png}

\begin{enumerate}
\def\labelenumi{\arabic{enumi}.}
\setcounter{enumi}{1}
\tightlist
\item
  Axis Manipulation
\end{enumerate}

Another trick of misleading graph is axis change: Changing thy y-axis
maximum afftect how the graph look like. A higher maximum will make the
grpha to appear less volatiliy ,less strrp than a lower maximum. The
other way of axis change is changing the ratio of a graph's dimensions.
This way will affect how the graph appears. We demostrate chaning the
ratio of graph dimension for below graphs.
\includegraphics{images/Line_graph1.svg.png}
\includegraphics{images/175px-Line_graph1-3.svg.png}
\includegraphics{images/200px-Line_graph1-4.svg.png}

Axis manipulation is the opposite of truncating data, because they
include the axis and baselines but change them so much that they lose
meaning. This type of graph manipulation can be used to push a false
narrative.

3.Cherry Picking Data This is to pick the data that shows a typical
viewpoint. For example, we know house price in Bay area kept increasing
since 2011. However, for those house agencies who want convince buyers
that house prices has decreased , they might select some areas in
typical months that house prices happend to decreased.

It is not technically wrong but it is definitely misleading.This is
often called improper extraction or tactic omitting data, when only a
certain chunk of data is included.This is more common in graphs that
have time as one of their axis. Here is the graph to show what it is.
\includegraphics{images/Bad_graph_extraction.png}
\includegraphics{images/Good_graph_extraction.png}

Reference: How Writers Use Misleading Graphs To Manipulate You BY RYAN
MCCREADY, AUG 10, 2017
\url{https://venngage.com/blog/misleading-graphs/} Misleading graph,
wikipedia
\url{https://en.wikipedia.org/wiki/Misleading_graph\#Truncated_graph}
Data Analysis: Displaying Data - Deception with Graphs
\url{https://web.archive.org/web/20030402093134/http://www.sao.state.tx.us/Resources/Manuals/Method/data/12DECEPD.pdf}

\begin{enumerate}
\def\labelenumi{\arabic{enumi}.}
\setcounter{enumi}{3}
\tightlist
\item
  The Year in Visual Stories and Graphics: New York Times
\end{enumerate}

\url{http://www.nytimes.com/newsgraphics/2013/12/30/year-in-interactive-storytelling/index.html\#dataviz}

Every year, New York Times will select a collection of the year's best
storyteller visualizations, which includes different forms of
state-of-the-art news visualizations. Personally, I think it can be a
good inspiration when we feel like ``I don't know how to be creative on
this''.

\bibliography{book.bib,packages.bib}


\end{document}
