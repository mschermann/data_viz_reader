\documentclass[]{book}
\usepackage{lmodern}
\usepackage{amssymb,amsmath}
\usepackage{ifxetex,ifluatex}
\usepackage{fixltx2e} % provides \textsubscript
\ifnum 0\ifxetex 1\fi\ifluatex 1\fi=0 % if pdftex
  \usepackage[T1]{fontenc}
  \usepackage[utf8]{inputenc}
\else % if luatex or xelatex
  \ifxetex
    \usepackage{mathspec}
  \else
    \usepackage{fontspec}
  \fi
  \defaultfontfeatures{Ligatures=TeX,Scale=MatchLowercase}
\fi
% use upquote if available, for straight quotes in verbatim environments
\IfFileExists{upquote.sty}{\usepackage{upquote}}{}
% use microtype if available
\IfFileExists{microtype.sty}{%
\usepackage{microtype}
\UseMicrotypeSet[protrusion]{basicmath} % disable protrusion for tt fonts
}{}
\usepackage[margin=1in]{geometry}
\usepackage{hyperref}
\hypersetup{unicode=true,
            pdftitle={A Reader on Data Visualization},
            pdfauthor={MSIS 2629 Spring 2018},
            pdfborder={0 0 0},
            breaklinks=true}
\urlstyle{same}  % don't use monospace font for urls
\usepackage{natbib}
\bibliographystyle{apalike}
\usepackage{color}
\usepackage{fancyvrb}
\newcommand{\VerbBar}{|}
\newcommand{\VERB}{\Verb[commandchars=\\\{\}]}
\DefineVerbatimEnvironment{Highlighting}{Verbatim}{commandchars=\\\{\}}
% Add ',fontsize=\small' for more characters per line
\usepackage{framed}
\definecolor{shadecolor}{RGB}{248,248,248}
\newenvironment{Shaded}{\begin{snugshade}}{\end{snugshade}}
\newcommand{\KeywordTok}[1]{\textcolor[rgb]{0.13,0.29,0.53}{\textbf{#1}}}
\newcommand{\DataTypeTok}[1]{\textcolor[rgb]{0.13,0.29,0.53}{#1}}
\newcommand{\DecValTok}[1]{\textcolor[rgb]{0.00,0.00,0.81}{#1}}
\newcommand{\BaseNTok}[1]{\textcolor[rgb]{0.00,0.00,0.81}{#1}}
\newcommand{\FloatTok}[1]{\textcolor[rgb]{0.00,0.00,0.81}{#1}}
\newcommand{\ConstantTok}[1]{\textcolor[rgb]{0.00,0.00,0.00}{#1}}
\newcommand{\CharTok}[1]{\textcolor[rgb]{0.31,0.60,0.02}{#1}}
\newcommand{\SpecialCharTok}[1]{\textcolor[rgb]{0.00,0.00,0.00}{#1}}
\newcommand{\StringTok}[1]{\textcolor[rgb]{0.31,0.60,0.02}{#1}}
\newcommand{\VerbatimStringTok}[1]{\textcolor[rgb]{0.31,0.60,0.02}{#1}}
\newcommand{\SpecialStringTok}[1]{\textcolor[rgb]{0.31,0.60,0.02}{#1}}
\newcommand{\ImportTok}[1]{#1}
\newcommand{\CommentTok}[1]{\textcolor[rgb]{0.56,0.35,0.01}{\textit{#1}}}
\newcommand{\DocumentationTok}[1]{\textcolor[rgb]{0.56,0.35,0.01}{\textbf{\textit{#1}}}}
\newcommand{\AnnotationTok}[1]{\textcolor[rgb]{0.56,0.35,0.01}{\textbf{\textit{#1}}}}
\newcommand{\CommentVarTok}[1]{\textcolor[rgb]{0.56,0.35,0.01}{\textbf{\textit{#1}}}}
\newcommand{\OtherTok}[1]{\textcolor[rgb]{0.56,0.35,0.01}{#1}}
\newcommand{\FunctionTok}[1]{\textcolor[rgb]{0.00,0.00,0.00}{#1}}
\newcommand{\VariableTok}[1]{\textcolor[rgb]{0.00,0.00,0.00}{#1}}
\newcommand{\ControlFlowTok}[1]{\textcolor[rgb]{0.13,0.29,0.53}{\textbf{#1}}}
\newcommand{\OperatorTok}[1]{\textcolor[rgb]{0.81,0.36,0.00}{\textbf{#1}}}
\newcommand{\BuiltInTok}[1]{#1}
\newcommand{\ExtensionTok}[1]{#1}
\newcommand{\PreprocessorTok}[1]{\textcolor[rgb]{0.56,0.35,0.01}{\textit{#1}}}
\newcommand{\AttributeTok}[1]{\textcolor[rgb]{0.77,0.63,0.00}{#1}}
\newcommand{\RegionMarkerTok}[1]{#1}
\newcommand{\InformationTok}[1]{\textcolor[rgb]{0.56,0.35,0.01}{\textbf{\textit{#1}}}}
\newcommand{\WarningTok}[1]{\textcolor[rgb]{0.56,0.35,0.01}{\textbf{\textit{#1}}}}
\newcommand{\AlertTok}[1]{\textcolor[rgb]{0.94,0.16,0.16}{#1}}
\newcommand{\ErrorTok}[1]{\textcolor[rgb]{0.64,0.00,0.00}{\textbf{#1}}}
\newcommand{\NormalTok}[1]{#1}
\usepackage{longtable,booktabs}
\usepackage{graphicx,grffile}
\makeatletter
\def\maxwidth{\ifdim\Gin@nat@width>\linewidth\linewidth\else\Gin@nat@width\fi}
\def\maxheight{\ifdim\Gin@nat@height>\textheight\textheight\else\Gin@nat@height\fi}
\makeatother
% Scale images if necessary, so that they will not overflow the page
% margins by default, and it is still possible to overwrite the defaults
% using explicit options in \includegraphics[width, height, ...]{}
\setkeys{Gin}{width=\maxwidth,height=\maxheight,keepaspectratio}
\IfFileExists{parskip.sty}{%
\usepackage{parskip}
}{% else
\setlength{\parindent}{0pt}
\setlength{\parskip}{6pt plus 2pt minus 1pt}
}
\setlength{\emergencystretch}{3em}  % prevent overfull lines
\providecommand{\tightlist}{%
  \setlength{\itemsep}{0pt}\setlength{\parskip}{0pt}}
\setcounter{secnumdepth}{5}
% Redefines (sub)paragraphs to behave more like sections
\ifx\paragraph\undefined\else
\let\oldparagraph\paragraph
\renewcommand{\paragraph}[1]{\oldparagraph{#1}\mbox{}}
\fi
\ifx\subparagraph\undefined\else
\let\oldsubparagraph\subparagraph
\renewcommand{\subparagraph}[1]{\oldsubparagraph{#1}\mbox{}}
\fi

%%% Use protect on footnotes to avoid problems with footnotes in titles
\let\rmarkdownfootnote\footnote%
\def\footnote{\protect\rmarkdownfootnote}

%%% Change title format to be more compact
\usepackage{titling}

% Create subtitle command for use in maketitle
\newcommand{\subtitle}[1]{
  \posttitle{
    \begin{center}\large#1\end{center}
    }
}

\setlength{\droptitle}{-2em}
  \title{A Reader on Data Visualization}
  \pretitle{\vspace{\droptitle}\centering\huge}
  \posttitle{\par}
  \author{MSIS 2629 Spring 2018}
  \preauthor{\centering\large\emph}
  \postauthor{\par}
  \predate{\centering\large\emph}
  \postdate{\par}
  \date{2018-05-21}

\usepackage{booktabs}
\usepackage{amsthm}
\makeatletter
\def\thm@space@setup{%
  \thm@preskip=8pt plus 2pt minus 4pt
  \thm@postskip=\thm@preskip
}
\makeatother

\usepackage{amsthm}
\newtheorem{theorem}{Theorem}[chapter]
\newtheorem{lemma}{Lemma}[chapter]
\theoremstyle{definition}
\newtheorem{definition}{Definition}[chapter]
\newtheorem{corollary}{Corollary}[chapter]
\newtheorem{proposition}{Proposition}[chapter]
\theoremstyle{definition}
\newtheorem{example}{Example}[chapter]
\theoremstyle{definition}
\newtheorem{exercise}{Exercise}[chapter]
\theoremstyle{remark}
\newtheorem*{remark}{Remark}
\newtheorem*{solution}{Solution}
\begin{document}
\maketitle

{
\setcounter{tocdepth}{1}
\tableofcontents
}
\chapter{Preface}\label{preface}

This is a collaborative writing project as part of the course MSIS 2629
``Data Visualization'' at \href{http://www.scu.edu}{Santa Clara
University}. The purpose of the class reader is to collaboratively
engage with and reflect on data visualizations, to establish a solid
theoretical background, and to collect useful practices and showcases.
More information on the background of this project is available in the
\href{https://mschermann.github.io/msis2629spring2018}{syllabus}.

The following text serves explains how we organize ourselves.

\section{References}\label{references}

\textbf{EVERY} references must be included in the \texttt{book.bib}
file. This file uses the BibTeX notation (Learn how to use BibTeX
\href{http://www.bibtex.org/Using/}{here}.). Most literature search
engines allow you to export the reference information in BibTeX. For
websites we use the following minimal notation (you may add further
information - usually the more the better is a good strategy):

\begin{verbatim}
@misc{great_viz,
  author = {{A great visualizer}},
  year = {1982},
  title = {A ficticious web page title},
  howpublished = {\url{http://great_viz_org/}},
  note = {Accessed: 2018-04-26}
}
\end{verbatim}

Particularly important is the \texttt{note} field. Websites change
frequently, so links will break. If we do this correctly,
\texttt{{[}@great\_viz{]}} will produce \citep{great_viz}.

\section{Images}\label{images}

Images should not be loaded from external website because the links may
change. Instead download a version of the image and create a reference
that contains the link to the image. For example the following image is
a deceptive visualization (the bars do start at zero).

\begin{figure}
\centering
\includegraphics{images/halper_welfare.jpg}
\caption{An Example of a deceptive visualization}
\end{figure}

Source: \citep{halper_2012} referenced in \citep{andale_2014}

The citation for the image looks like this.

\begin{verbatim}
@misc{halper_2012,
  author={Halper, Daniel},
  year={2012},
  title = {Over 100 Million Now Receiving Federal Welfare},
  url={https://www.weeklystandard.com/daniel-halper/over-100-million-now-receiving-federal-welfare},
  note = {Accessed: 2018-04-26}
}
\end{verbatim}

You have probably found this image through a different website that
explains the visualization. For example the following website explains
some problematic aspects of this visualization:

\begin{verbatim}
@misc{andale_2014,
  author={Andalde, Stephanie},
  year={2014},
  title = {Misleading Graphs: Real Life Examples},
  url={http://www.statisticshowto.com/misleading-graphs/},
  note = {Accessed: 2018-04-26}
}
\end{verbatim}

\section{Basic Guidelines}\label{basic-guidelines}

Figures and tables with captions will be placed in \texttt{figure} and
\texttt{table} environments, respectively.--\textgreater{}

\begin{Shaded}
\begin{Highlighting}[]
\KeywordTok{par}\NormalTok{(}\DataTypeTok{mar =} \KeywordTok{c}\NormalTok{(}\DecValTok{4}\NormalTok{, }\DecValTok{4}\NormalTok{, .}\DecValTok{1}\NormalTok{, .}\DecValTok{1}\NormalTok{))}
\KeywordTok{plot}\NormalTok{(pressure, }\DataTypeTok{type =} \StringTok{'b'}\NormalTok{, }\DataTypeTok{pch =} \DecValTok{19}\NormalTok{)}
\end{Highlighting}
\end{Shaded}

\begin{figure}

{\centering \includegraphics[width=0.8\linewidth]{Data_Viz_Reader_files/figure-latex/nice-fig-1} 

}

\caption{Here is a nice figure!}\label{fig:nice-fig}
\end{figure}

Reference a figure by its code chunk label with the \texttt{fig:}
prefix, e.g., see Figure \ref{fig:nice-fig}. Similarly, you can
reference tables generated from \texttt{knitr::kable()}, e.g., see Table
\ref{tab:nice-tab}.

\begin{Shaded}
\begin{Highlighting}[]
\NormalTok{knitr}\OperatorTok{::}\KeywordTok{kable}\NormalTok{(}
 \KeywordTok{head}\NormalTok{(iris, }\DecValTok{20}\NormalTok{), }\DataTypeTok{caption =} \StringTok{'Here is a nice table!'}\NormalTok{,}
 \DataTypeTok{booktabs =} \OtherTok{TRUE}
\NormalTok{)}
\end{Highlighting}
\end{Shaded}

\begin{table}

\caption{\label{tab:nice-tab}Here is a nice table!}
\centering
\begin{tabular}[t]{rrrrl}
\toprule
Sepal.Length & Sepal.Width & Petal.Length & Petal.Width & Species\\
\midrule
5.1 & 3.5 & 1.4 & 0.2 & setosa\\
4.9 & 3.0 & 1.4 & 0.2 & setosa\\
4.7 & 3.2 & 1.3 & 0.2 & setosa\\
4.6 & 3.1 & 1.5 & 0.2 & setosa\\
5.0 & 3.6 & 1.4 & 0.2 & setosa\\
\addlinespace
5.4 & 3.9 & 1.7 & 0.4 & setosa\\
4.6 & 3.4 & 1.4 & 0.3 & setosa\\
5.0 & 3.4 & 1.5 & 0.2 & setosa\\
4.4 & 2.9 & 1.4 & 0.2 & setosa\\
4.9 & 3.1 & 1.5 & 0.1 & setosa\\
\addlinespace
5.4 & 3.7 & 1.5 & 0.2 & setosa\\
4.8 & 3.4 & 1.6 & 0.2 & setosa\\
4.8 & 3.0 & 1.4 & 0.1 & setosa\\
4.3 & 3.0 & 1.1 & 0.1 & setosa\\
5.8 & 4.0 & 1.2 & 0.2 & setosa\\
\addlinespace
5.7 & 4.4 & 1.5 & 0.4 & setosa\\
5.4 & 3.9 & 1.3 & 0.4 & setosa\\
5.1 & 3.5 & 1.4 & 0.3 & setosa\\
5.7 & 3.8 & 1.7 & 0.3 & setosa\\
5.1 & 3.8 & 1.5 & 0.3 & setosa\\
\bottomrule
\end{tabular}
\end{table}

You can write citations, too. For example, we are using the
\textbf{bookdown} package \citep{R-bookdown} in this sample book, which
was built on top of R Markdown and \textbf{knitr} \citep{xie2015}.

\chapter{Introduction}\label{introduction}

\section{What is Data Visualization?}\label{what-is-data-visualization}

\textbf{Data Visualization helps reveal insights and patterns that are
not immediately visible in the raw data.}

\textbf{Data visualization} refers to representing data in a visual
context to help people understand the significance of that data. A way
so that information, numbers and measurements makes sense is a form of
art -- the art of data visualization. Graphs do that for us. According
to Friedman (2008) \citep{viz} the ``main goal of data visualization is
to communicate information clearly and effectively through graphical
means. It doesn't mean that data visualization needs to look boring to
be functional or extremely sophisticated to look beautiful. To convey
ideas effectively, both aesthetic form and functionality need to go hand
in hand, providing insights into a rather sparse and complex data set by
communicating its key-aspects in a more intuitive way.''

\textbf{Information visualization}, the art of representing data in a
way that it is easy to understand and manipulate, can help us make sense
of information and thus make it useful. From business decision making to
simple route navigation -- there is a huge (and growing) need for data
to be presented so that it delivers value.

This article is a brief introduction to information visualization. It
explains briefly how information visualization helps to make sense of
data as well as find relationships between data and confirm ideas. Some
examples and common uses of information visualization are discussed
below \citep{info_viz}.

\section{Why is Data Visualization
Important?}\label{why-is-data-visualization-important}

Today more than ever, data visualization represents a simple,
user-friendly approach to understanding data and making business
decisions quickly. Here are some references on why it is important
today:

\subsection{Chris Pittenturf's article on the importance of data
visualization to
businesses}\label{chris-pittenturfs-article-on-the-importance-of-data-visualization-to-businesses}

The article \citep{viz_importance}, written by Chris Pittenturf, VP-Data
\& Analytics, Palace Sports \& Entertainment, explores data
visualization and its importance to businesses today.

The article begins with a definition of data visualization in simple
terms and goes on to explain how a good data visualization should be
visually engaging to the reader. Chris goes on to explain the basic
criteria that a data visualization should satisfy, to be an effective
visualization. These criteria and their brief meanings are as follows:

\begin{enumerate}
\def\labelenumi{\arabic{enumi}.}
\tightlist
\item
  Informative: The visualization should be able to convey the desired
  information from the data to the reader.
\item
  Efficient: The visualization should not be ambiguous.
\item
  Appealing: The visualization should be captivating and visually
  pleasing.
\item
  (Optional) Interactive and Predictive: The visualizations can contain
  variables and filters with which the users may interact in order to
  predict results of different scenarios.
\end{enumerate}

Pittenturf goes on to give various day-to-day examples where
visualization gives a better understanding of the data. One extremely
simple example used by Pittenturf is that of an energy bill. Pittenturf
states that as a consumer, when we receive an energy bill, we normally
look at the graph in the bill first before proceeding to read the text
in the bill. Pittenturf states that consumers are more likely to analyze
and understand the visualizations before reading further along. The
article ends with Pittenturf emphasizing the importance of data
visualizations in our businesses as well as in our daily lives.It gives
a simple, short and crisp understanding of what data visualization is
and how it is relevant to everyone. Data visualization is an aid to get
a better understanding of the complex insights that any business data
provides. Most of the data used by the businesses is highly unstructured
and these businesses can get a better understanding of their businesses
by visualizing their data.

\subsection{David McCandless's TED talk on data
visualization}\label{david-mccandlesss-ted-talk-on-data-visualization}

Visuals help us understand concepts that would otherwise be difficult to
contextualize---for example, expenditures or valuations of extremely
large amounts of money are represented in the billion dollar-o-gram by
color-coded, relatively-sized boxes. Furthermore, it allows synthesis of
a breadth of information to be delivered in a small, easily-digestible,
aesthetically pleasing way. Visuals serve as a sort of map for a vast
landscape of information---they direct your eyes to the important places
and details. And the eye, as McCandless notes, is uniquely suited among
our senses to process large amounts of information and detect patterns.
The billion dollar-o-gram is extremely readable and rather pretty, but
it seems a bit dubious to compare the predicted Iraq War cost to the
``mushroomed'' actual cost of Iraq and Afghanistan wars, since its
purpose seems only to conflate two wars for dramatic effect
\citep{viz_ted}.

Beyond its ability to make information from several different sources
and in large amounts more quickly and easily understood, data
visualization can also reveal smaller interesting patterns---allowing us
to play the ``data detective'' as McCandless calls it. In other words,
as we have already discussed, data visualization can not only be
extremely effective in a declarative manner, but can also be used as an
exploratory tool \citep{viz_ted}.

McCandless also postulates that we all have a latent ``design literacy''
that is being developed every day as we are constantly bombarded with
visuals, and that our minds and our eyes are taking in this information
and processing it so that we all have an intuitive sense of design, and
have actually begun to demand a visual aspect to our information. This
is an interesting perspective, since everyone does seem to have a sense
of visual aspects---space, color, etc., but of course the time-honored
adage tells us that beauty is in the eye of the beholder. So while it
might be whimsical to claim that we are all designers, there is still,
of course, great value in learning formal principles of design
\citep{viz_ted}.

\section{A Brief History of Data
Visualization}\label{a-brief-history-of-data-visualization}

``The only new thing in the world is the history you don't know.'' ---
Harry S Truman

\subsection{Data Visualization: modern
product?}\label{data-visualization-modern-product}

It is common to think of statistical graphics and data visualization as
relatively modern developments in statistics, but in fact, the graphic
representation of quantitative information has deep roots. These roots
reach into the histories of the earliest map-making and visual
depiction, and later into thematic cartography, statistics and
statistical graphics, medicine, and other fields.

Developments in technologies (printing, reproduction) mathematical
theory and practice, and empirical observation and recording, enabled
the wider use of graphics and new advances in form and content. This
paper provides an overview of the intellectual history of data
visualization from medieval to modern times, as well as describes and
illustrates some significant advances along the
way.\citep{data_viz_history}.

\subsection{\texorpdfstring{Milestones Tour
\citep{data_viz_history}}{Milestones Tour {[}@data\_viz\_history{]}}}\label{milestones-tour-data_viz_history}

\begin{longtable}[]{@{}ll@{}}
\toprule
\begin{minipage}[b]{0.16\columnwidth}\raggedright\strut
\textbf{Phase}\strut
\end{minipage} & \begin{minipage}[b]{0.78\columnwidth}\raggedright\strut
\textbf{Description}\strut
\end{minipage}\tabularnewline
\midrule
\endhead
\begin{minipage}[t]{0.16\columnwidth}\raggedright\strut
Pre-17th Century: Early Maps and Diagrams\strut
\end{minipage} & \begin{minipage}[t]{0.78\columnwidth}\raggedright\strut
Data visualization has come a long way. Prior to the 17th century, data
visualization already existed. Though displayed in other format such as
maps, the content is much similar to today's visualizations, which
mostly presented geologic, economic, and medical data. The earliest
seeds of visualization arose in geometric diagrams, in tables of the
positions of stars and other celestial bodies, and in the making of maps
to aid in navigation and exploration.\strut
\end{minipage}\tabularnewline
\begin{minipage}[t]{0.16\columnwidth}\raggedright\strut
1600-1699: Measurement and Theory\strut
\end{minipage} & \begin{minipage}[t]{0.78\columnwidth}\raggedright\strut
Among the most important problems of the 17th century were those
concerned with physical measurement of time, distance, and space for
astronomy, surveying, map making, navigation and territorial expansion.
This century also saw considerable new growth in theory as well as the
dawn of practical application.\strut
\end{minipage}\tabularnewline
\begin{minipage}[t]{0.16\columnwidth}\raggedright\strut
1700-1799: New Graphic Forms\strut
\end{minipage} & \begin{minipage}[t]{0.78\columnwidth}\raggedright\strut
With some rudiments of statistical theory, data of interest and
importance, and the idea of graphic representation somewhat established,
the 18th century witnessed the expansion of these aspects to new domains
and new graphic forms.\strut
\end{minipage}\tabularnewline
\begin{minipage}[t]{0.16\columnwidth}\raggedright\strut
1800-1850: Beginnings of Modern Graphics\strut
\end{minipage} & \begin{minipage}[t]{0.78\columnwidth}\raggedright\strut
With the foundation provided by the previous innovations of design and
technique, the first half of the 19th century witnessed explosive growth
in statistical graphics and thematic mapping at a rate which would not
be equaled until modern times.\strut
\end{minipage}\tabularnewline
\begin{minipage}[t]{0.16\columnwidth}\raggedright\strut
1850--1900: The Golden Age of Statistical Graphics\strut
\end{minipage} & \begin{minipage}[t]{0.78\columnwidth}\raggedright\strut
By the mid-1800s, all the conditions for the rapid growth of
visualization had generated a ``perfect storm'' for data graphics.
Official state statistical offices were established throughout Europe,
in recognition of the growing importance of numerical information for
social planning,industrialization, commerce, and transportation.\strut
\end{minipage}\tabularnewline
\begin{minipage}[t]{0.16\columnwidth}\raggedright\strut
1900-1950: The Modern Dark Ages\strut
\end{minipage} & \begin{minipage}[t]{0.78\columnwidth}\raggedright\strut
If the late 1800s were the ``golden age'' of statistical graphics and
thematic cartography, the early 1900s can be called the ``modern dark
ages'' of visualization. There were few graphical innovations, and, by
the mid-1930s, the enthusiasm for visualization which characterized the
late 1800s had been supplanted by the rise of quantification and formal,
often statistical, models in the social sciences.\strut
\end{minipage}\tabularnewline
\begin{minipage}[t]{0.16\columnwidth}\raggedright\strut
1950--1975: Rebirth of Data Disualization\strut
\end{minipage} & \begin{minipage}[t]{0.78\columnwidth}\raggedright\strut
Still under the influence of the formal and numerical zeitgeist from the
mid-1930s on, data visualization began to rise from dormancy in the mid
1960s.\strut
\end{minipage}\tabularnewline
\begin{minipage}[t]{0.16\columnwidth}\raggedright\strut
1975--present: High-D, Interactive and Dynamic Data Visualization\strut
\end{minipage} & \begin{minipage}[t]{0.78\columnwidth}\raggedright\strut
During the last quarter of the 20th century, data visualization has
blossomed into a mature, vibrant and multidisciplinary area of research,
as seen in this handbook, and software tools for a wide range of
visualization methods and data types are available for every
computer.\strut
\end{minipage}\tabularnewline
\bottomrule
\end{longtable}

\section{Key Figures in the History of Data
Visualization}\label{key-figures-in-the-history-of-data-visualization}

The history of data visualization is full of incredible stories marked
by major events, led by a few key players. The article
\citep{history_viz} introduces some of the amazing men and women who
paved the way by combining art, science, and statistics.And one of them
is Charles Joseph Minard, whose most famous work is the map of
Napoleon's Russian campaign of 1812 which could be used as a data
product for Data Visualization. Below we have some figures names with
their famous works, and other stories in the article
\citep{history_viz}.

\subsection{William Playfair
(1759--1823)}\label{william-playfair-17591823}

William Playfair is considered the father of statistical graphics,
having invented the line and bar chart we use so often today. He is also
credited with having created the area and pie chart. Playfair was a
Scottish engineer and political economist who published ``The Commercial
and Political Atlas''" in 1786. This book featured a variety of graphs
including the image below. In this famous example, he compares exports
from England with imports into England from Denmark and Norway from 1700
to 1780.

\begin{figure}
\centering
\includegraphics{images/Playfair.png}
\caption{}
\end{figure}

\subsection{John Snow (1813--1858)}\label{john-snow-18131858}

In 1854, a cholera epidemic spread quickly through Soho in London. The
Broad Street area had seen over 600 dead, and the surviving residents
and business owners had largely fled the terrible disease. Physician
John Snow plotted the locations of cholera deaths on a map. The
surviving maps of his work show a method of tallying the death counts,
drawn as lines parallel to the street, at the appropriate addresses.
Snow's research revealed a pattern. He saw a clear concentration around
the water pump on Broad Street, which helped find the cause of the
infection.

\begin{figure}
\centering
\includegraphics{images/Snow.png}
\caption{}
\end{figure}

\subsection{Charles Joseph Minard
(1781--1870)}\label{charles-joseph-minard-17811870}

Charles Joseph Minard was a French civil engineer famous for his
representation of numerical data on maps. His most famous work is the
map of Napoleon's Russian campaign of 1812 illustrating the dramatic
loss of his army over the advance on Moscow and the following retreat.
The map displays how many soldiers are still marching and how many have
already died at different points in the army's march. Drawn in 1869,
many consider it the best statistical graphic ever created. It
represents the earliest beginnings of data journalism.

\begin{figure}
\centering
\includegraphics{images/Minard.png}
\caption{}
\end{figure}

This classic lithograph dates back to 1869, displaying the number of men
in Napoleon's 1812 Russian army, their movements, and the temperatures
they encountered along their way. It's been called one of the ``best
statistical drawings ever created.'' The work is an important reminder
that the fundamentals of data visualization lie in a nuanced
understanding of the many dimensions of data. Tools like D3.js and HTML
are no good without a firm grasp of your dataset and sharp communication
skills.

\section{Contemporary Visualists}\label{contemporary-visualists}

\subsection{Hans Rosling}\label{hans-rosling}

Hans Rosling took his interest in Global Health and developed stunning
visualizations about it using statistical methods and data from the UN.
He was a noted TED speaker and one of his most interesting TED talks is
\emph{``Asia's Rise: How and When''} \citep{hans}. In this, Hans shows
trends of the Western countries vs Developing countries like India and
China and makes predictions using stunning visualizations like the
Bubble chart. In this video, he also predicts the exact date on which
India and China will move ahead of USA as strong economic forces. Hans
was the co-founder and developer of the foundation
``Gapminder''\citep{gapminder} which develops tools to help the people
make sense of global data. One of the most important goals of Gapminder
foundation is to end ignorance in the world by developing fact-based
visualizations to show how the world really is.

\subsection{David McCandless}\label{david-mccandless}

David McCandless is a British data-journalist and his blog
\emph{``Information is Beautiful''} \citep{info_beautiful} hosts some of
the most visually stunning graphs, charts, and maps on a wide range of
topics like science, food, dogs and countries.

One such chart, ``International Number Ones: Because every country is
good at something (according to data),'' is a captivating work that
displays something at which each country is the best.
\citep{country_chart}

Some of the interesting findings are as follows:

\begin{longtable}[]{@{}ll@{}}
\toprule
\textbf{Country } & ** No.1 in **\tabularnewline
\midrule
\endhead
Canada & Doughnuts\tabularnewline
USA & Spam Emails\tabularnewline
India & Bananas\tabularnewline
Norway & Pizza Eaters\tabularnewline
Togo & Unhappiness\tabularnewline
Colombia & Happiness\tabularnewline
\bottomrule
\end{longtable}

The visualizations on this website are updated and revised whenever new
data is available. The original version of the above-mentioned graph can
be seen here: \citep{country_original}

\chapter{Fundamentals}\label{fundamentals}

\begin{longtable}[]{@{}lll@{}}
\toprule
\begin{minipage}[b]{0.04\columnwidth}\raggedright\strut
\textbf{Step}\strut
\end{minipage} & \begin{minipage}[b]{0.11\columnwidth}\raggedright\strut
\textbf{Name}\strut
\end{minipage} & \begin{minipage}[b]{0.76\columnwidth}\raggedright\strut
\textbf{Description}\strut
\end{minipage}\tabularnewline
\midrule
\endhead
\begin{minipage}[t]{0.04\columnwidth}\raggedright\strut
1\strut
\end{minipage} & \begin{minipage}[t]{0.11\columnwidth}\raggedright\strut
Perform Data Discovery and Determine The Story\strut
\end{minipage} & \begin{minipage}[t]{0.76\columnwidth}\raggedright\strut
Before this step it is easy to underestimate the effort level it takes
to pull the best insights from the data. Data manipulation products like
Tableau, Domo, Pentaho, IBM's Many Eyes, and R, among others, make
insight extraction that much easier to gain understanding of data using
a visual medium. The key is to start with a simple portion of your data
and to start pulling basic insights to visualize and correlate with each
other. This process leads towards a compound series of questions, which
helps provide an overall vision to the end product. We see the effect
during our discovery process, which leads to unforeseen avenues for data
intelligence.\strut
\end{minipage}\tabularnewline
\begin{minipage}[t]{0.04\columnwidth}\raggedright\strut
2\strut
\end{minipage} & \begin{minipage}[t]{0.11\columnwidth}\raggedright\strut
Data Infrastructure Setup\strut
\end{minipage} & \begin{minipage}[t]{0.76\columnwidth}\raggedright\strut
Data infrastructures can be simple or complex depending what the end
goal is. Many clients prefer to go the route of complete data
integration in order to centralize their data repositories. Technologies
such as Hadoop have helped by unifying disparate data sources, but other
options such as data cloud environments can help produce API's for
future product deployments. Why is this important? Accessibility of data
is an important foundation not only within the context of dashboards,
but also the possibility of branching out to other products.\strut
\end{minipage}\tabularnewline
\begin{minipage}[t]{0.04\columnwidth}\raggedright\strut
3\strut
\end{minipage} & \begin{minipage}[t]{0.11\columnwidth}\raggedright\strut
Product Design \& Development\strut
\end{minipage} & \begin{minipage}[t]{0.76\columnwidth}\raggedright\strut
Wireframing, prototyping, and application development are the main
engines to transform an idea into a final product. Products can range
from static presentations/reports to full interactive applications.
Mobile, tablet, TV, and workstation platforms can all be mediums to help
deliver the final product. The secret to a great end product is how well
the data story is conceptualized. If the story is weak then the end
product will also suffer.\strut
\end{minipage}\tabularnewline
\begin{minipage}[t]{0.04\columnwidth}\raggedright\strut
4\strut
\end{minipage} & \begin{minipage}[t]{0.11\columnwidth}\raggedright\strut
QA \& Product Release\strut
\end{minipage} & \begin{minipage}[t]{0.76\columnwidth}\raggedright\strut
The best part of any project is to get it finalized and released for all
to see. All data gets verified for accuracy, functionality testing (if
applicable), application flow (if applicable), design testing, and
remaining items are all completed. The end result is an engaging visual
product for all intended audiences to see and use.\strut
\end{minipage}\tabularnewline
\bottomrule
\end{longtable}

In other words, visualization is the initial filter for the quality of
data streams. Combining data from various sources, visualization tools
perform preliminary standardization, shape data in a unified way and
create easy-to-verify visual objects. As a result, these tools become
indispensable for data cleansing and vetting and help companies prepare
quality assets to derive valuable insights.

\section{Storytelling}\label{storytelling}

Storytelling is an essential part of data visualization. It is extremely
important to effectively communicate information through the
visualization. Stikeleather's article (2013) discussed the way in which
a visual designer tells a story with a visualization.

\begin{enumerate}
\def\labelenumi{\arabic{enumi}.}
\tightlist
\item
  Find the compelling narrative
\item
  Think about the audience (e.g., novice, generalist, managerial,
  export, executive)
\item
  Be objective and offer balance
\item
  Don't censor
\end{enumerate}

Data visualization will not always unleash a ready-made story on its
own. There are no rules, no `protocol' that will guarantee us a story.
Instead, it makes more sense to look for `insights,' which can be
artfully woven into stories in the hands of a good journalist.

Here is a process that may be followed for finding insights to tell a
story:

\citep{VisualizeToInsights}

\begin{figure}
\centering
\includegraphics{images/DataInsights.JPG}
\caption{}
\end{figure}

\textbf{References}

\citep{data_journ} \citep{design_principles} \citep{DataVizTips}
\citep{practitioners_guide}

\subsection{Explore}\label{explore}

Loading any data set into a spreadsheet can also be a form of
visualization as the data becomes visible in a table. Hence the focus
should not be whether we need data visualization or not but should be on
which form of data visualization is best for the situation.

\subsection{5 Second Rule}\label{second-rule}

Research shows that the average modern attention span for viewing
anything online is less than 5 seconds, so if you can't grab attention
within 5 minutes, you've likely lost your viewer. Include clear titles
and instructions, and tell people succinctly what the visualization
shows and how to interact with it.

\subsection{Design and layout matter}\label{design-and-layout-matter}

The design and layout should facilitate ease of understanding to convey
your message to the viewer.

Artists use design principles as the foundation of any visual work. If
you want to take your data visualization from an everyday dashboard to a
compelling data story, incorporate graphic designer Melissa Anderson's
principles of design: balance, emphasis, movement, pattern, repetition,
proportion, rhythm, variety, and unity, discussed in more detail in the
design principles section \citep{design_principles}.

\subsection{Keep it simple}\label{keep-it-simple}

Keep charts simple and easy to interpret. Instead of overloading
viewers' brains with lots of information, keep only necessary elements
in the chart and help the audience understand quickly what is going on.

\subsection{Pretty doesn't mean
effective}\label{pretty-doesnt-mean-effective}

There is a misconception that aesthetically pleasing visualization is
more effective. To draw attention, sometimes we want them to be pretty
and eye-catching. But if it fails to communicate the data properly,
you'll lose your audience's interest as quickly as you gained it.

\subsection{Use color purposely and
effectively}\label{use-color-purposely-and-effectively}

Use of color may be prettier and attractive but can be distracting too.
Thus, the color should be used only if it assists in conveying your
message.

\begin{longtable}[]{@{}ll@{}}
\toprule
\begin{minipage}[b]{0.18\columnwidth}\raggedright\strut
\textbf{Criteria}\strut
\end{minipage} & \begin{minipage}[b]{0.68\columnwidth}\raggedright\strut
\textbf{Description}\strut
\end{minipage}\tabularnewline
\midrule
\endhead
\begin{minipage}[t]{0.18\columnwidth}\raggedright\strut
Color for Numerical Scales\strut
\end{minipage} & \begin{minipage}[t]{0.68\columnwidth}\raggedright\strut
Color for numerical scales should be used with caution. The way you
interpret a shade depends on the colors around it and sometimes it can
lead to false conclusions.\strut
\end{minipage}\tabularnewline
\begin{minipage}[t]{0.18\columnwidth}\raggedright\strut
Leverage Color Associations\strut
\end{minipage} & \begin{minipage}[t]{0.68\columnwidth}\raggedright\strut
When we say strawberries we associate red color with it. If we can
leverage the how people associate different colors for different things,
we will not even need a legend to interpret things. Color can be used to
leverage long-term memory very quickly.\strut
\end{minipage}\tabularnewline
\begin{minipage}[t]{0.18\columnwidth}\raggedright\strut
Use Bright Colors to Highlight\strut
\end{minipage} & \begin{minipage}[t]{0.68\columnwidth}\raggedright\strut
To attract attention to a certain part of data, bright colors can be
used. Alarm colors draw the eye quickly to areas that need attention and
help get that message across.\strut
\end{minipage}\tabularnewline
\bottomrule
\end{longtable}

\subsection{Maps}\label{maps}

Use of maps can be tricky. Geographical data doesn't imply a map. Maps
can be useful for application where proximity matters, but for straight
``what is higher'' type comparisons, they're not very effective as large
regions will draw attention easier than smaller regions due to more
concentrated color.

\section{Gestalt Principles}\label{gestalt-principles}

Data is simply a collection of many individual elements (i.e.,
observations, typically represented as rows in a data table). In data
viz, our goal is usually to group these elements together in a
meaningful way to highlight patterns and anomalies. Described this way,
it makes sense that Gestalt Principles are a good set of guidelines to
assemble different elements into groups.

\begin{longtable}[]{@{}ll@{}}
\toprule
\begin{minipage}[b]{0.18\columnwidth}\raggedright\strut
\textbf{Criteria}\strut
\end{minipage} & \begin{minipage}[b]{0.68\columnwidth}\raggedright\strut
\textbf{Description}\strut
\end{minipage}\tabularnewline
\midrule
\endhead
\begin{minipage}[t]{0.18\columnwidth}\raggedright\strut
Proximity\strut
\end{minipage} & \begin{minipage}[t]{0.68\columnwidth}\raggedright\strut
White space can be used to group elements together and separate
others\strut
\end{minipage}\tabularnewline
\begin{minipage}[t]{0.18\columnwidth}\raggedright\strut
Similarity\strut
\end{minipage} & \begin{minipage}[t]{0.68\columnwidth}\raggedright\strut
Objects that look similar are instinctively grouped together in our
minds\strut
\end{minipage}\tabularnewline
\begin{minipage}[t]{0.18\columnwidth}\raggedright\strut
Enclosure\strut
\end{minipage} & \begin{minipage}[t]{0.68\columnwidth}\raggedright\strut
Helps distinguish between groups\strut
\end{minipage}\tabularnewline
\begin{minipage}[t]{0.18\columnwidth}\raggedright\strut
Symmetry\strut
\end{minipage} & \begin{minipage}[t]{0.68\columnwidth}\raggedright\strut
Objects should not be out of balance, or missing, or wrong. If an object
is asymmetrical, the viewer will waste time trying to find the problem
instead of concentrating on the instruction.\strut
\end{minipage}\tabularnewline
\begin{minipage}[t]{0.18\columnwidth}\raggedright\strut
Closure\strut
\end{minipage} & \begin{minipage}[t]{0.68\columnwidth}\raggedright\strut
We tend to complete shapes and paths even if part of them is
missing\strut
\end{minipage}\tabularnewline
\begin{minipage}[t]{0.18\columnwidth}\raggedright\strut
Continuity\strut
\end{minipage} & \begin{minipage}[t]{0.68\columnwidth}\raggedright\strut
We tend to continue shapes beyond their ending points (similar to
closure)\strut
\end{minipage}\tabularnewline
\begin{minipage}[t]{0.18\columnwidth}\raggedright\strut
Connection\strut
\end{minipage} & \begin{minipage}[t]{0.68\columnwidth}\raggedright\strut
Helps group elements together\strut
\end{minipage}\tabularnewline
\begin{minipage}[t]{0.18\columnwidth}\raggedright\strut
Figure and ground\strut
\end{minipage} & \begin{minipage}[t]{0.68\columnwidth}\raggedright\strut
We typically notice only one of several main visual aspects of a graph;
what we do notice becomes the figure, and everything else becomes the
``background''. This one is especially interesting because it is not as
obvious as some of the others, but is really important in matching a
data viz design to its purpose. \citep{principles-fusioncharts}\strut
\end{minipage}\tabularnewline
\bottomrule
\end{longtable}

\subsection{Analyze and Interpret}\label{analyze-and-interpret}

Once the data is visualized, the next step is to learn something from
the picture that is created. Questions that can be asked based on the
picture can be:

\begin{itemize}
\tightlist
\item
  What can be seen in this image? Is it what that was expected?
\item
  Are there any interesting patterns?
\item
  What does this mean in the context of the data?
\end{itemize}

Sometimes we might end up with visualization that, in spite of its
beauty, might seem to tell that nothing of interest can be found from
data. But there is almost always something that we can learn from any
visualization, however trivial.

\subsection{Document Your Insights and
Steps}\label{document-your-insights-and-steps}

If you think of this process as a journey through the dataset, the
documentation is your travel diary. It will tell you where you have
traveled to, what you have seen there and how you made your decisions
for your next steps. You can even start your documentation before taking
your first look at the data.

In most cases when we start to work with a previously unseen dataset, we
are already full of expectations and assumptions about the data.
Usually, there is a reason why we are interested in that dataset that we
are looking at. It's a good idea to start the documentation by writing
down these initial thoughts. This helps us to identify our bias and
reduces the risk of misinterpretation of the data by just finding what
we originally wanted to find.

I really think that the documentation is the most important step of the
process, and it is also the one we're most likely to tend to skip. As
you will see in the example below, the described process involves a lot
of plotting and data wrangling. Looking at a set of 15 charts you
created might be very confusing, especially after some time has passed.
In fact, those charts are only valuable (to you or any other person you
want to communicate your findings) if presented in the context in which
they have been created. Hence you should take the time to make some
notes on things like:

\begin{itemize}
\tightlist
\item
  Why have I created this chart?
\item
  What have I done to the data to create it?
\item
  What does this chart tell me?
\end{itemize}

\subsection{Transform Data}\label{transform-data}

Naturally, with the insights that you have gathered from the last
visualization, you might have an idea of what you want to see next. You
might have found some interesting pattern in the dataset which you now
want to inspect in more detail. Possible transformations are:

\begin{longtable}[]{@{}ll@{}}
\toprule
\begin{minipage}[b]{0.18\columnwidth}\raggedright\strut
\textbf{Transformation}\strut
\end{minipage} & \begin{minipage}[b]{0.68\columnwidth}\raggedright\strut
\textbf{Description}\strut
\end{minipage}\tabularnewline
\midrule
\endhead
\begin{minipage}[t]{0.18\columnwidth}\raggedright\strut
Zooming\strut
\end{minipage} & \begin{minipage}[t]{0.68\columnwidth}\raggedright\strut
This allows us to have look at a certain detail in the visualization
Aggregation To combine many data points into a single group\strut
\end{minipage}\tabularnewline
\begin{minipage}[t]{0.18\columnwidth}\raggedright\strut
Filtering\strut
\end{minipage} & \begin{minipage}[t]{0.68\columnwidth}\raggedright\strut
This helps us to (temporarily) remove data points that are not in our
major focus\strut
\end{minipage}\tabularnewline
\begin{minipage}[t]{0.18\columnwidth}\raggedright\strut
Outlier removal\strut
\end{minipage} & \begin{minipage}[t]{0.68\columnwidth}\raggedright\strut
This allows us to get rid of single points that are not representative
of99\% of the dataset.\strut
\end{minipage}\tabularnewline
\bottomrule
\end{longtable}

Let's consider that you have visualized a graph and what came out of
this was nothing but a mess of nodes connected through hundreds of edges
(a very common result when visualizing so-called densely connected
networks), one common transformation step would be to filter some of the
edges. If, for instance, the edges represent money flows from donor
countries to recipient countries we could remove all flows below a
certain amount \citep{DataVizBestPrac}.

\subsection{Examples of Best Practices in Visual
Analysis}\label{examples-of-best-practices-in-visual-analysis}

Referenced below is a free pdf with some examples of best practices in
visual analysis. It discusses the most effective charts for various
kinds of analysis. It is a helpful and relevant resource for data
science students interested in presenting analyses using simple and
effective visualizations that tell the complete story.

Some of the key areas the author highlights are visualizing trends over
time, comparison and ranking, correlation, distribution, geographical
data etc. The author gives examples of how simple graphs can also be
made more effective simply by adding a few more elements or making
simple adjustments.

This is a great starting point for creating effective charts and we may
use these principles also when we start doing advanced analytics.

\section{Three Rules to Follow in order to Develop Intuitive
Dashboards}\label{three-rules-to-follow-in-order-to-develop-intuitive-dashboards}

Often a designer can become too concerned with coming up with a visual
that is too intricate and overly complicated. A dashboard should be
appealing but also easy to understand. Following these rules will lead
to the effective presentation of the data \citep{intuitive-dash}.

\subsection{The dashboard should read left to
right}\label{the-dashboard-should-read-left-to-right}

Because we read from top to bottom and left to right, a reader's eyes
will naturally look in the upper left of a page. The content should
therefore flow like words in a book. It is important to note that the
information at the top of the page does not always have to be the most
important. Annual data is usually more important to a business but daily
or weekly data could be used more often for day to day work. This should
be kept in mind when designing a dashboard as dashboards are often used
as a quick convenient way to look up data.

\subsection{Group related information
together}\label{group-related-information-together}

Grouping related data together is an intuitive way to help the flow of
the visual. It does not make sense for a user to have to search in
different areas to find the information they need.

\subsection{Find relationships between seemingly unrelated areas and
display visuals together to show the
relationship.}\label{find-relationships-between-seemingly-unrelated-areas-and-display-visuals-together-to-show-the-relationship.}

Grouping unrelated data seems contradictory to the second rule, but the
important thing is to tell a story not previously observed. Data
analytics is all about finding stories the data are trying to tell. Once
they are discovered, the stories need to be presented in an effective
manner. Grouping unrelated data together makes it easier to see how they
change together.

\section{Design Principles}\label{design-principles}

\subsection{\texorpdfstring{Melissa Anderson's Principles of Design
\citep{design_principles}}{Melissa Anderson's Principles of Design {[}@design\_principles{]}}}\label{melissa-andersons-principles-of-design-design_principles}

\begin{longtable}[]{@{}ll@{}}
\toprule
\begin{minipage}[b]{0.16\columnwidth}\raggedright\strut
\textbf{Criteria}\strut
\end{minipage} & \begin{minipage}[b]{0.78\columnwidth}\raggedright\strut
\textbf{Description}\strut
\end{minipage}\tabularnewline
\midrule
\endhead
\begin{minipage}[t]{0.16\columnwidth}\raggedright\strut
Balance\strut
\end{minipage} & \begin{minipage}[t]{0.78\columnwidth}\raggedright\strut
A design is said to be balanced if key visual elements such as color,
shape, texture, and negative space are uniformly distributed. Balance
doesn't mean that each side of the visualization needs perfect symmetry,
but it is important to have the elements of the dashboard/visualization
distributed evenly. And it is important to remember the non-data
elements, such as a logo, title, caption, etc. that can affect the
balance of the display.\strut
\end{minipage}\tabularnewline
\begin{minipage}[t]{0.16\columnwidth}\raggedright\strut
Variety\strut
\end{minipage} & \begin{minipage}[t]{0.78\columnwidth}\raggedright\strut
Variety in color, shape, and chart-type draws and keeps users engaged
with data. Including more variety can increase information retention by
the viewer. But when there is too much variety, important details can be
overlooked. Variety may affect balance, but when done correctly, variety
can help increase the recall of information. However if overdone, too
much variety can feel cluttered and blur together the images and data in
the mind of the viewer.\strut
\end{minipage}\tabularnewline
\begin{minipage}[t]{0.16\columnwidth}\raggedright\strut
Emphasis\strut
\end{minipage} & \begin{minipage}[t]{0.78\columnwidth}\raggedright\strut
Draw viewers' attention towards important data by using key visual
elements. Emphasis is the component that is most related to when reading
through the nine principles of design. It is the key to be conscious of
what is drawing the viewers attention to the art. When thinking about
the art design of data visualization it is also very important to remain
keen on the main point of your story and how the entire visualization is
either drawing the viewer to that point of emphasis or how they are
being distracted or drawn elsewhere.\strut
\end{minipage}\tabularnewline
\begin{minipage}[t]{0.16\columnwidth}\raggedright\strut
Movement\strut
\end{minipage} & \begin{minipage}[t]{0.78\columnwidth}\raggedright\strut
Ideally movement should mimic the way people usually read, starting at
the top of the page, moving across it, and then down. Movement can also
be created by using complementary colors to pull the user's attention
across the page.\strut
\end{minipage}\tabularnewline
\begin{minipage}[t]{0.16\columnwidth}\raggedright\strut
Pattern\strut
\end{minipage} & \begin{minipage}[t]{0.78\columnwidth}\raggedright\strut
patterns are ideal for displaying similar sets of information, or for
sets of data that equal in value. Disrupting the pattern can also be
effective in drawing viewers' attention; it naturally draws
curiosity.\strut
\end{minipage}\tabularnewline
\begin{minipage}[t]{0.16\columnwidth}\raggedright\strut
Repetition\strut
\end{minipage} & \begin{minipage}[t]{0.78\columnwidth}\raggedright\strut
Relationships between sets of data can be communicated by repeating
chart types, shapes, or colors.\strut
\end{minipage}\tabularnewline
\begin{minipage}[t]{0.16\columnwidth}\raggedright\strut
Proportion\strut
\end{minipage} & \begin{minipage}[t]{0.78\columnwidth}\raggedright\strut
Proportion can be subtle but it can go a long way to enhancing a
viewer's experience and understanding of the data. The danger of
proportion though is that it can be easy to deceive people
subconsciously.Naturally images will have a greater impact on how our
brains perceive the dashboard or visualization. For example, someone can
change the scale of a graph or images to inflate their results and even
if they write the numbers next to it, the shortcut many people will take
is to interpret the data based on the image. This is why it is important
we take care to accurately reflect proportion in our data visualization
and remain critical of how others use proportion in their visualization.
If a person is portrayed next to a house, the house is going to look
bigger. In data visualization, the proportion can indicate the
importance of data sets, along with the actual relationship between
numbers.\strut
\end{minipage}\tabularnewline
\begin{minipage}[t]{0.16\columnwidth}\raggedright\strut
Rhythm\strut
\end{minipage} & \begin{minipage}[t]{0.78\columnwidth}\raggedright\strut
A design has proper rhythm when the design elements create the movement
that is pleasing to the eye. If the design is not able to do so,
rearranging visual elements may help.\strut
\end{minipage}\tabularnewline
\begin{minipage}[t]{0.16\columnwidth}\raggedright\strut
Unity\strut
\end{minipage} & \begin{minipage}[t]{0.78\columnwidth}\raggedright\strut
Unity across design will happen naturally if all other design principles
are implemented.\strut
\end{minipage}\tabularnewline
\bottomrule
\end{longtable}

\subsection{Tufte's Design Principles of Graphical
Excellence}\label{tuftes-design-principles-of-graphical-excellence}

A graph should be impressive and can obtain audience's attention. How
can we achieve this? We must consider several aspects:
\textbf{efficiency, complexity, structure, density and beauty}. We also
should consider the audience whether they will be confused about the
design.

\subsubsection{Principle 1: Maximizing the data-ink ratio, within
reason.}\label{principle-1-maximizing-the-data-ink-ratio-within-reason.}

Data-ink is the non-erasable core of a graphic, the non-redundant ink
arranged in response to variation in the numbers represented. It is also
the proportion of graphic's ink devoted to the non-redundant display of
data-information.

\[{Data \ Ink \ Ratio} = \frac{{Data \ Ink}}{{Total \ Ink}}\]

This basic idea follows three principles:

\begin{enumerate}
\def\labelenumi{\arabic{enumi}.}
\tightlist
\item
  Erase non-data-ink, within reason.
\item
  Erase redundant data-ink, within reason.
\item
  Always revise and edit.
\end{enumerate}

Examples: 1.Erase non-data-ink and redundant data-ink.
\includegraphics{images/Tufte_figure1.png} (source:\citep{Tufte_2001})

\begin{enumerate}
\def\labelenumi{\arabic{enumi}.}
\setcounter{enumi}{1}
\item
  Erase non-data-ink and redundant data-ink.
  \includegraphics{images/Tufte_figure2.png} (source:
  \citep{appli_2017}) \includegraphics{images/Tufte_figure3.png}
  (source: \citep{appli_2017})
\item
  Always revise and edit. \includegraphics{images/Tufte_figure4.png}
  (source:\citep{Tufte_2001})
\end{enumerate}

The graphs will be better for more information per unit of space and per
unit of ink is displayed. Graphics are almost always going to improve as
they go through editing ,revision, and testing against differernt design
options. Try to figure out whehter the audience looking at the new
designs be confused? Nothing is lost to those puzzled by the frame of
dashes,and something isgained by those who do understand. We can also
assume that if you understand the statistical graphics, most other
readers will, too because it is a frequent mistake in thinking about
statistical graphics to underestimate the audience. Some of the new
designs may appear odd, but this is probably because we have not seen
them before.

\subsubsection{Principle 2: Mobilize every graphical element, perhaps
several times over, to show the
data.}\label{principle-2-mobilize-every-graphical-element-perhaps-several-times-over-to-show-the-data.}

The danger of multifunctioning elements is that they tend to generate
graphical puzzles, with encodings that can only be broken by their
inventor.Thus design techniques for enhancing graphical clarity in the
face of complexity must be developed along with multifunctioning
elements.

In other words, we should try to make all present graphical elements
data encoding elements. We must make every graphical element effective.

Example: \includegraphics{images/Tufte_figure6.png}

\begin{figure}
\centering
\includegraphics{images/Tufte_figure6.png}
\caption{}
\end{figure}

(source:\citep{Tufte_2001})

\subsubsection{Principle 3: Maximize data density and the size of the
data matrix, within
reason.}\label{principle-3-maximize-data-density-and-the-size-of-the-data-matrix-within-reason.}

High performation graphics should be designed with special care. As
volume of data increases, data measures must shrink (smaller dots for
scatters,thinner lines for busy time-series).

\[{Data \ Density} = \frac{{Entries \ in \ the \ Data \ Matrix}}{{Area \ of \ Chart}}\]

\subsection{Composition of design
principles}\label{composition-of-design-principles}

\subsubsection{Escape flatland -- small multiples, parallel
sequencing.}\label{escape-flatland-small-multiples-parallel-sequencing.}

Data is multivariate. Doesn't necessarily mean 3D projection. How can we
enhance mulitvariate data on inherently 2D surfaces?

\begin{enumerate}
\def\labelenumi{\arabic{enumi}.}
\item
  Example for small multiples.
  \includegraphics{images/Tufte_figure8.png} (source:\citep{Tufte_2001})
\item
  Example for parallel sequencing
  \includegraphics{images/Tufte_figure7.png} (source:\citep{Tufte_2001})
\end{enumerate}

\subsubsection{Macro/Micro: Provide the user with both views (overview
and
detail).}\label{macromicro-provide-the-user-with-both-views-overview-and-detail.}

Carefully designed view can show a macro structure (overview) as well as
micro structure (detail) in one space.

Example: \includegraphics{images/Tufte_figure9.png}
(source:\citep{Tufte_2001})

\subsubsection{Utilize Layering \&
Separation.}\label{utilize-layering-separation.}

Supported by Gestalt laws (The principles of grouping):

\begin{enumerate}
\def\labelenumi{\arabic{enumi}.}
\tightlist
\item
  Grouping with colors
\item
  Using Color to separate
\item
  1+1 = 3 (clutter)
\end{enumerate}

Example: \includegraphics{images/Tufte_figure10.png}
(source:\citep{Tufte_2001})

\subsubsection{Utilize narratives of space and
time.}\label{utilize-narratives-of-space-and-time.}

Tell a story of position and chronology through visual elements.

Example: \includegraphics{images/Tufte_figure11.png} (source:
\citep{narratives_2017}) \includegraphics{images/Tufte_figure12.png}
(source: \citep{narratives_2017})

\subsection{Adapting your story to a different set of
audiences}\label{adapting-your-story-to-a-different-set-of-audiences}

Jonathon Corum is a graphics designer for The New York Times and he
provided a very informative talk to a strictly scientific audience on
how to create and design visualizations that explain material originally
created for a certain audience, i.e.~the scientific community, but now
is to be related to a different audience, (in his case, the readership
of the Times or maybe the public at large). The talk is filled with
examples and break downs of how he has moved from his base content to
the final product, all of which are illuminating examples by themselves.
There is also great power in the broader themes that he is trying to
convey.

First, of course is knowing the audience that you are producing the work
for, but even in this step, do not lose sight of the ultimate goal of
conveying understanding, of explaining a concept. You are searching for
a visual idea in your content that can be communicated to your audience.
Some of the main highlights to help you make this connection with your
audience involve:

\subsubsection{Focusing the attention}\label{focusing-the-attention}

What can be removed? Realize that consistency can help eliminate
unnecessary distractions. There may be a trade off between losing
information but conveying the ultimate meaning more clearly. Label
important things rather than relying on a legend, which requires the
viewer to hold on to too much information at once.

\subsubsection{Involving your audience}\label{involving-your-audience}

Give them opportunities to connect their own general knowledge on the
topic. Use real world comparisons or examples to help build and relate
context. Encourage comparisons and make this easy for the viewer to
process and see.

\subsubsection{Explaining why}\label{explaining-why}

Providing context, adding time sequence details, showing movement,
change and mechanism will all guide your audience in connecting the dots
and understanding the significance of what you are trying to
communicate.

\subsection{Three Rules to Follow in order to Develop Intuitive
Dashboards:}\label{three-rules-to-follow-in-order-to-develop-intuitive-dashboards-1}

Often a designer can become too concerned with coming up with a visual
that is too intricate and overly complicated. A dashboard should be
appealing but also easy to understand. Following these rules will lead
to effective presentation of the data \citep{intuitive-dash}.

\subsubsection{The dashboard should read left to
right}\label{the-dashboard-should-read-left-to-right-1}

Because we read from top to bottom and left to right, a reader's eyes
will naturally look in the upper left of a page. The content should
therefore flow like words in a book. It is important to note that the
information at the top of the page does not always have to be the most
important. Annual data is usually more important to a business but daily
or weekly data could be used more often for day to day work. This should
be kept in mind when designing a dashboard as dashboards are often used
as a quick convenient way to look up data.

\subsubsection{Group related information
together}\label{group-related-information-together-1}

Grouping related data together is an intuitive way to help the flow of
the visual. It does not make sense for a user to have to search in
different areas to find the information they need.

\subsubsection{Find relationships between seemingly unrelated areas and
display visuals together to show the
relationship}\label{find-relationships-between-seemingly-unrelated-areas-and-display-visuals-together-to-show-the-relationship}

Grouping unrelated data seems contradictory to the second rule, but the
important thing is to tell a story not previously observed. Data
analytics is all about finding stories the data are trying to tell. Once
they are discovered, the stories need to be presented in an effective
manner. Grouping unrelated data together makes it easier to see how they
change together.

\subsection{What I learned recreating one chart using 24
tools}\label{what-i-learned-recreating-one-chart-using-24-tools}

Lisa Rost's article ``What I learned recreating one chart using 24
tools'' describes lessons learned from recreating one chart using many
different data visualization tools \citep{different_tools}. The author
used apps Excel, Plotly, Easycharts, Google Sheets, Lyra, Highcharts,
Tableau, Polestar, Quadrigram, Illustrator, RAW, and NodeBox, as well as
charting libraries ggvis, Bokeh, Highcharts, ggplot2, Processing, NVD3,
Seaborn, Vega, D3, matplotlib, Vega-Lite, and R. She links her github
page on the project which details the data set she used, containing the
health expectancy in years as well as GDP per capita and population for
about 200 countries in the year 2015, as well has her process and
results of visualizing the data using each tool. However, in the
article, she focuses on the main takeaways from the exercise, which was
especially interesting in the context of our class discussion on
different types of tools and their respective strengths. She also
provides her own graphics to help illustrate her lessons learned.

\subsubsection{There Are No Perfect Tools, Just Good Tools for People
with Certain
Goals}\label{there-are-no-perfect-tools-just-good-tools-for-people-with-certain-goals}

Since data visualization is necessary in many spheres, from science to
journalism, data visualization projects will often have quite disparate
objectives, and the people working on them will have different
requirements. And as the author aptly points out, it is impossible for
one tool to satisfy the needs of every data visualizer; so there will
necessarily be tools better suited to specific situations. For example,
does the user need a tool for exploratory visualization of the data, or
does the user seek to create graphs and charts to show the public or a
specific audience something?

\begin{figure}
\centering
\includegraphics{images/analysis_spectrum.png}
\caption{}
\end{figure}

The author also notes that the flexibility of a tool is a sticking point
as well---if you need to change your data while developing a data
visualization, certain apps like Illustrator will not be ideal because
changing the data even slightly requires you to build the graph again
from scratch. Another thing to think about is the type of chart you are
trying to create---is a basic, canned bar or line graph all you need (in
which case something like Excel will do the trick), or does your project
necessitate a more innovative or custom chart (like something possible
in D3.js)? Interactivity is another big question---only certain tools
will make this possible.

\begin{figure}
\centering
\includegraphics{images/interactivity.png}
\caption{}
\end{figure}

\subsubsection{There Are No Perfect Tools, Just Good Tools for People
with Certain
Mindsets}\label{there-are-no-perfect-tools-just-good-tools-for-people-with-certain-mindsets}

This section of the article is all about the difference in people's
preferences and opinions; from the people who build the tools to the
users, everyone thinks differently. Therefore, certain tools will be
inherently more intuitive to use for different people.

\subsubsection{\texorpdfstring{We Still Live in an `Apps Are for the
Easy Stuff, Code Is for the Good Stuff' World''** Basically, writing
code can be scary for anyone without a coding background, but it
provides more flexibility, and, as mentioned in class, code is perfectly
reproducible. On the other hand, apps are much more user-friendly for
the less computer
science-savvy.}{We Still Live in an Apps Are for the Easy Stuff, Code Is for the Good Stuff World''** Basically, writing code can be scary for anyone without a coding background, but it provides more flexibility, and, as mentioned in class, code is perfectly reproducible. On the other hand, apps are much more user-friendly for the less computer science-savvy.}}\label{we-still-live-in-an-apps-are-for-the-easy-stuff-code-is-for-the-good-stuff-world-basically-writing-code-can-be-scary-for-anyone-without-a-coding-background-but-it-provides-more-flexibility-and-as-mentioned-in-class-code-is-perfectly-reproducible.-on-the-other-hand-apps-are-much-more-user-friendly-for-the-less-computer-science-savvy.}

\begin{figure}
\centering
\includegraphics{images/apps_vs_code.png}
\caption{}
\end{figure}

\subsubsection{Every Tool Forces You Down a
Path}\label{every-tool-forces-you-down-a-path}

Rost quotes her former NPR Visuals teammate for the final lesson header,
pointing out that tools themselves influence the development of a data
visualization with their respective features, strengths, and
limitations.

\begin{figure}
\centering
\includegraphics{images/tools_force_paths.png}
\caption{}
\end{figure}

\subsection{Typography and Data
Visualization}\label{typography-and-data-visualization}

This article discusses less common applications of typography in data
visualization. While data components such as quantitative or categorical
data are commonly represented by visual features like colors, sizes or
shapes, utilization of boldface, font variation, and other typographic
elements in data visualization are less prevalent.

\textbf{Highlighted in the article are preattentive visual attributes.}
Preattentive attributes are those that perceptual psychologists have
determined to be easily recognized by the human brain irrespective of
how many items are displayed. Therefore, ``preattentive visual
attributes are desirable in data visualization as they can demand
attention only when a target is present, can be difficult to ignore, and
are virtually unaffected by load.'' Examples of preattentive attributes
are size/area, hue, and curvature.

This brings us to the disparate situation of the popularity of visual
aspects like color and size and typographic aspects such as font
variation, capitalization and bold. The authors present several possible
reasons for this, beginning with the preattentiveness of visual
attributes like size and hue.However, some typographic attributes such
as line width or size, intensity, or font weight (a combination of the
two) are considered preattentive as well.

Furthermore, these visual attributes are inherently more viscerally
powerful, and they are easy to code in a variety of programming
languages. Technology has also perhaps previously limited the use of
typographic attributes, for only recently have fine details such as
serifs, italics, etc. been made readily visible to the audiences of data
visualizations by technological advances.

Lastly, the authors remark that it is possible the lack of variety of
typographic elements used in data visualizations is due to the limited
knowledge of computer scientists and other individuals pursuing data
visualization in how to apply these elements effectively. While the
first few proposed explanations make sense from personal experience with
technology and exposure to data visualizations and design in general,
the hypothesis that lack of knowledge of typographic elements in data
visualization seems more plausible if it was being applied to a small
group of people rather than all of the data visualization design
community. I would say that it is more likely that the use of
typographic elements in data visualization is less popular because there
are fewer instances in which it can be used appropriately, or a status
quo bias---if current visual attributes are received well, the
prevailing attitude may be not to fix what is not broken. However, the
authors also point out that despite the dearth of typographic attributes
in data visualization, other spheres like typography, cartography,
mathematics, chemistry, and programming ``have a rich history with type
and font attributes that informs the scope of the parameter space.''

The authors continue by pointing out some tips for using typographic
attributes to encode different data types, since certain attributes may
be suited to particular purposes. For example, font weight (size and
intensity) is ideal for representing quantitative or ordered data, and
font type (shape) is better suited to denote categories in the data.

Furthermore, as in typography and cartography, use of typographic
attributes in data visualization raises concerns of legibility, the
ability to understand both individual characters and commonalities that
identify a font family, and readability, the ability to read lines and
blocks of words. Often, interactivity of a visualization will not only
improve functionality, but also provide a solution to readability issues
by providing a means to zoom in on small text.

There are a few examples of unusual/innovative use of typography for
data visualization in the article, not all of which I agree are made
more effective by the interesting utilization of typographic attributes,
but the ``Who Survived the Titanic'' visualization's use of typographic
attributes allowed it to not only answer macro-questions very quickly,
such as if women and children were actually first to be evacuated across
classes, but also to provide answers to micro-questions, like whether or
not the Astors survived. It used common visual elements like color and
area to indicate whether or not a person survived and number/proportion
of people, as well as typographic aspects like italic and simple text
replacement to indicate gender and the passengers names.

\begin{figure}
\centering
\includegraphics{images/TypographicTitanic.jpg}
\caption{}
\end{figure}

The authors round out the article by addressing the most common
criticisms of typography in data visualization, the foremost one being
whether or not text should even be considered an element of data
visualization, since visualization connotes preattentive visual encoding
of information, and text or sequential information necessitates more
investment of attention to understand. Another criticism is that textual
representations are not as visually appealing even when used
effectively. However, the authors counter that ``this criticism
indicates both the strength and weakness of type? that while text may
not be suited for adding style or drama to a visualization, it can be
particularly powerful in situations where a finer level of detail is
needed, without sacrificing representation of higher level patterns.
Lastly, a label length problem is common when using text in
visualizations; differing lengths of names or labels may skew perception
so that longer labels seem more important than shorter labels. This
problem was encountered in the Titanic visualization with the varying
lengths representations of passengers' names, and was corrected by only
including a given name and a surname, the length of which could only
vary so much.

\subsection{Data visualization in
Business}\label{data-visualization-in-business}

\citep{biz_strategy} According to an Experian report, 95\% of U.S.
organizations say that they use data to power business opportunities,
and another 84 percent believe data is an integral part of forming a
business strategy. Visualization helps data impact business in following
ways:

\subsubsection{Cleaning}\label{cleaning}

The simplest way to explain the importance of visualization is to look
at visualization as the means to making sense of data. Even the most
basic, widely-used data visualization tools that combine simple pie
charts and bar graphs help people comprehend large amounts of
information fast and easily, compared to paper reports and spreadsheets.
In other words, visualization is the initial filter for the quality of
data streams. Combining data from various sources, visualization tools
perform preliminary standardization, shape data in a unified way and
create easy-to-verify visual objects. As a result, these tools become
indispensable for data cleansing and vetting and help companies prepare
quality assets to derive valuable insights.

\subsubsection{Extracting}\label{extracting}

Known versatile tools for data visualization and analytics -- Elastic
Stack, Tableau, Highcharts, and more complex database solutions like
Hadoop, Amazon AWS and Teradata, have wide applications in business,
from monitoring performance to improving customer experience on mobile
tools. New generation of data visualization based on AR and VR
technology, however, provides formerly unfeasible advantages in terms of
identifying patterns and drawing insights from various data streams.

Building 3D data visualization spaces, companies can create an intuitive
environment that helps data scientists grasp and analyze more data
streams at the same time, observe data points from multiple dimensions,
identify previously unavailable dependencies and manipulate data by
naturally moving objects, zooming, and focusing on more granulated
areas. Moreover, these tools allow us to expand the capabilities of data
visualization by creating collaborative 3D environments for teams. As a
result, new technology helps extract more valuable insights from the
same volume of data.

\subsubsection{Strategizing}\label{strategizing}

As the amount of data grows, it becomes harder to catch up with it.
Therefore, data strategy becomes the necessary part of the success in
applying data to business. Then how data visualization become an
important tool in your strategic kit? First, it helps you cleanse your
data. Secondly, it allows you to identify and extract meaningful
information from it. Finally, data visualization tools enable continuous
real-time monitoring of how your strategy and now data-driven decisions
influence performance and business outcomes. In other words, these tools
visualize not only the data, but also the results, and help correct and
optimize strategy on the go.

Data visualization is one of the initial steps made to derive value from
data. It's also one of the most important steps, as it determines how
efficiently analysts can work with data assets, what insights they are
able to extract and how their data strategy will develop over time.

Therefore, the quality and capabilities of data visualization directly
influence how data impacts your business strategy and what benefits data
applications can bring to the companies and their industries.

\section{Data Visualization Tools}\label{data-visualization-tools}

Due to the rise of big data analytics, there has been an increased need
for data visualization tools to help understand the data. Besides
Tableau, there are several other software tools one can use for data
visualization like Sisense, Plotly, FusionCharts, Highcharts,
Datawrapper, and Qlikview. This article is from Forbes and has a brief,
clear introduction about these 7 powerful software options for data
visualization. This could be helpful for future reference because for
different purposes I may need to use different tools. Each option has
its advantages and disadvantages and this article helps highlight them.

\begin{longtable}[]{@{}ll@{}}
\toprule
\begin{minipage}[b]{0.16\columnwidth}\raggedright\strut
\textbf{Tool}\strut
\end{minipage} & \begin{minipage}[b]{0.78\columnwidth}\raggedright\strut
\textbf{Description}\strut
\end{minipage}\tabularnewline
\midrule
\endhead
\begin{minipage}[t]{0.16\columnwidth}\raggedright\strut
\textbf{Tableau}\strut
\end{minipage} & \begin{minipage}[t]{0.78\columnwidth}\raggedright\strut
The most popular in the group and has many users. It is simple to use,
making it easy to learn and can handle large data sets. Tableau can
handle big data thanks to integration with database handling
applications such as MySQL, Hadoop, and Amazon AWS.\strut
\end{minipage}\tabularnewline
\begin{minipage}[t]{0.16\columnwidth}\raggedright\strut
\textbf{Qlikview}\strut
\end{minipage} & \begin{minipage}[t]{0.78\columnwidth}\raggedright\strut
The the main competitor to Tableau and is also quite popular. Qlikview
is customizable and has a wide range of features which can be a
double-edged sword. These features take more time to learn and get
acquainted with. However, once one gets past the learning curve, they
have a powerful tool at their disposal.\strut
\end{minipage}\tabularnewline
\begin{minipage}[t]{0.16\columnwidth}\raggedright\strut
\textbf{FusionCharts}\strut
\end{minipage} & \begin{minipage}[t]{0.78\columnwidth}\raggedright\strut
The distinctive aspect of FusionCharts is that graphics do not have to
be created from scratch. Users can start with a template and insert
their own data from their project.\strut
\end{minipage}\tabularnewline
\begin{minipage}[t]{0.16\columnwidth}\raggedright\strut
\textbf{Highcharts:}\strut
\end{minipage} & \begin{minipage}[t]{0.78\columnwidth}\raggedright\strut
It proudly claims to be used by 72\% of the 100 biggest companies in the
world. It is a simple tool that does not require specialized training
and quickly generates the desired output. Unlike some tools, Highcharts
focuses on cross-browser support, allowing for greater access and
use.\strut
\end{minipage}\tabularnewline
\begin{minipage}[t]{0.16\columnwidth}\raggedright\strut
\textbf{Datawrapper:}\strut
\end{minipage} & \begin{minipage}[t]{0.78\columnwidth}\raggedright\strut
It is making a name for itself in the media industry. It has a simple
user interface making it easy to generate charts and embed into
reports.\strut
\end{minipage}\tabularnewline
\begin{minipage}[t]{0.16\columnwidth}\raggedright\strut
\textbf{Plotly:}\strut
\end{minipage} & \begin{minipage}[t]{0.78\columnwidth}\raggedright\strut
It can create more sophisticated visuals thanks to integration with
programming languages such as Python and R. The danger is creating
something more complicated than necessary. The whole point of data
visualization is to quickly and clearly convey information.\strut
\end{minipage}\tabularnewline
\begin{minipage}[t]{0.16\columnwidth}\raggedright\strut
\textbf{Sisense:}\strut
\end{minipage} & \begin{minipage}[t]{0.78\columnwidth}\raggedright\strut
It can bring together multiple sources of data for easier access. It can
even work with large data sets. Sisense makes it easy to share finished
products across departments, ensuring everyone can get the information
they need.\strut
\end{minipage}\tabularnewline
\bottomrule
\end{longtable}

\subsection{Data Mining vs.Data
Visualization}\label{data-mining-vs.data-visualization}

In \textbf{Data Mining}, there are different processes involve carrying
out the data mining process such as data extraction, data management,
data transformations, data pre-processing, etc. In \textbf{Data
Visualization}, the primary goal is to convey the information
efficiently and clearly without any deviations or complexities in the
form of statistical graphs, information graphs, and plots. Also, the
author listed the top 7 comparisons between data mining and data
visualization, and 12 key differences between data mining and data
visualization. After reading the article, you will have a very clear
understanding of what are data mining and data visualization and the
characters for those two techniques.

\subsection{Interactive Data
Visualization}\label{interactive-data-visualization}

Interactive or Dynamic data visualization delivers today's complex sea
of data in a graphically compelling and an easy-to-understand way. It
enables direct actions on a plot to change elements and link between
multiple plots. It enables users to accomplish traditional data
exploration tasks by making charts
interactive\citep{benefits_interactive_viz}.

Interactive Data Visualization Software has the following benefits:

\begin{enumerate}
\def\labelenumi{\arabic{enumi}.}
\tightlist
\item
  Absorb information in constructive ways: With the volume and velocity
  of data created everyday, dynamic data viz enables enhanced process
  optimization, insight discovery and decision making.
\item
  Visualize relationships and patterns: Helps in better understanding of
  correlations among operational data and business performance.
\item
  Identify and act on emerging trends faster: Helps decision makers to
  grasp shifts in behaviors and trends across multiple data sets much
  more quickly.
\item
  Manipulate and interact directly with data: Enables users to engage
  data more frequently.
\item
  Foster a new business language : Ability to tell a story through data
  that instantly relates the performance of a business and its assets.
\end{enumerate}

\subsection{D3.js}\label{d3.js}

D3.js stands for Data Driven Document, a JS library for interactive Big
Data visualization in literally ANY way required
real-time\citep{d3_interactive_viz}. This is not a tool, mind you, so a
user should have a solid understanding of JavaScript to work with the
data and present it in a humanly-understandable form. To say more, this
library renders the data into SVG and HTML5 formats, so older browsers
like IE7 and 8 cannot leverage D3.js capabilities.

The data gathered from disparate sources like huge-scale data sets is
bind in real-time with DOM to produce interactive animations ( 2D and 3D
alike) in an extremely rapid way. The D3 architecture allows the users
to intensively reuse the codes across a variety of add-ons and plug-ins.
Some of the key advantages are: It is dynamic, free and open source and
very flexible with all web technologies, the ability to handle big data
and the functional style allows to reuse the codes.

\textbf{The Hitchhiker' Guide to d3.js} is a wonderful guide for
self-teaching d3.js. This guide is meant to prepare readers mentally as
well as give readers some fruitful directions to pursue. There is a lot
to learn besides the d3.js API, both technical knowledge around web
standards like HTML, SVG, CSS and JavaScript as well as communication
concepts and data visualization principles. Chances are you know
something about some of those things, so this guide will attempt to give
you good starting points for the things you want to learn more about.

It starts from the insights of learning d3.js by showing interviews with
those top visualization practitioners. Then the author gives key
concepts and useful features for learning visualization like d3-shape,
d3 selection, d3-collection, ds-hierarchy, ds-zoom as well as d3-force.

My favorite part of this guide is it lists a lot of useful resources
links for learning d3.js. For example, it recommends d3 API Reference,
2000+ d3 case studies and tutorials for d3. I did my exploratory
analysis version of group project on d3. And I found this guide helpful
during the progress. It also includes some meetup groups here in the bay
area. So, maybe we can meet data friends through the group.

\subsection{Tableau}\label{tableau}

Tableau is amid the market leaders for the Big Data visualization,
especially efficient for delivering interactive data visualization for
the results derived from Big Data operations, deep learning algorithms
and multiple types of AI-driven apps \citep{tableau_interactive_viz}.
Tableau can be integrated with Amazon AWS, MySQL, Hadoop, Teradata and
SAP, making this solution a versatile tool for creating detailed graphs
and intuitive data representation. This way the C-suite and middle-chain
managers are able to make grounded decisions based on informative and
easily-readable Tableau graphs. Tableau is business intelligence (BI)
and analytics platform created for the purposes of helping people see,
understand, and make decisions with data. It is the industry leader in
interactive data visualization tools, offering a broad range of maps,
charts, graphs, and more graphical data presentations. It is a painless
option when cost is not a concern and you do not need advanced and
complex analysis.The application is very handy for quickly visualizing
trends in data, connecting to a variety of data sources, and mapping
cities/regions and their associated data.

\textbf{The key advantages} are: It provides non technical user the
ability to build complex reports and dashboard with zero coding skills.
Using drag-n-drop functionalities of Tableau, user can create a very
interactive visuals within minutes. It can handle millions of rows of
data with ease and users can make live to connections to different data
sources like SQL etc.

Tips for Tableau:

\begin{enumerate}
\def\labelenumi{\arabic{enumi}.}
\tightlist
\item
  Running totals
\item
  Common Baseline
\item
  Weighted averages
\item
  Moving average
\item
  Grouping by aggregates
\item
  Different years comparison
\item
  Appending excel sheets
\item
  Bar chart totals
\item
  Fixed axis when re-drawing charts
\item
  Auto-fitting screen behavior depending on data selection
\end{enumerate}

\citep{VizBP} \citep{ExtremePre}

\subsection{Building advanced analytics application with
TabPy}\label{building-advanced-analytics-application-with-tabpy}

Imagine a scenario where we can just enter some x values in a dashboard
form, and the visualization would predict the y variable! \citep{TabPy}
shows how to integrate and visualize data from Python in Tableau. This
is especially relevant to all data science students, as this is one of
the tools used for visualizing advanced analytics. The author here has
given an example using data from Seattle's police department's 911 calls
and he tries to identify criminal hotspots in the area.The author uses
machine learning (spatial clustering) and creates a great interactive
visualization, where you can click on the type of criminal activity and
the graph will show various clusters. There are other examples and use
cases that may be downloaded, and the scripts are also given by the
author for anyone who is interested in trying it out.

\subsection{R Shiny}\label{r-shiny}

R Shiny enables us to produce interactive data visualizations with a
minimum knowledge of HTML, CSS, or Java using a simple web application
framework that runs under the R statistical platform
\citep{shiny_interactive_viz}. Standalone apps can be hosted on a
webpage or embedded in R Markdown documents and dashboards can be built
using R shiny. It combines the computational power of R with the
interactivity of the modern web. The main advantages of using R Shiny
are : Its flexibility of pulling in whatever package in R that you want
to solve your problem, reaping the benefits of an open source ecosystem
for R and JavaScript visualization libraries, thereby allowing to create
highly custom applications and enabling timely, high quality interactive
data experience without (or with much less) web development and without
the limitations or cost of proprietary BI tools.

\subsection{Jupyter}\label{jupyter}

\subsection{Google chart}\label{google-chart}

A free and powerful integration of all Google power. The tool is
rendering the resulting charts to HTML5/SVG, so they are compatible with
any browser. Support for VML ensures compatibility with older IE
versions, and the charts can be ported to the latest releases of Android
and iOS. What's even more important, Google chart combines the data from
multiple Google services like Google Maps. This results in producing
interactive charts that absorb data real-time and can be controlled
using an interactive dashboard.
\url{https://towardsdatascience.com/top-4-popular-big-data-visualization-tools-4ee945fe207d}

The tool is rendering the resulting charts to HTML5/SVG, so they are
compatible with any browser. Support for VML ensures compatibility with
older IE versions, and the charts can be ported to the latest releases
of Android and iOS. What's even more important, Google chart combines
the data from multiple Google services like Google Maps. This results in
producing interactive charts that absorb data real-time and can be
controlled using an interactive dashboard. \citep{Top4VizTools}

\subsection{Corporate Scorecards and Data
Visualization}\label{corporate-scorecards-and-data-visualization}

Corporate transparency, flat organizations, open book policies, etc. are
terms executives and entrepreneurs learn about all the time
\citep{SCORECARDS}. As the corporate world shifts towards a more open
culture, the demand for open data and insights have increased
dramatically. This shift has helped the overall corporate strategic
planning and management process--easing the alignment of business
activities towards a series of goals. Being transparent top down aligns
the culture to sail towards the same North Star. The growth of corporate
transparency is not only important internally, but externally as well.
Corporate certifications like B Corporations certifications (B Corp),
require companies to provide a transparent view on their social
conscious efforts to the general public. Achieving the certification is
one step of the process; the true goal is to show the world how and why
the certification is truly deserved.

\subsection{Data Augmentation}\label{data-augmentation}

There are ways to use data visualization at every level of an
organization. These applications lets us quickly create insightful
visualizations, in minutes. It allows users to visualize data and
explore the vast domain interactively. Ref: \citep{app1} Some of them
are mentioned below:

\citep{ref_pdf_ar}

Because computer interfacing is changing every day, it is important for
our clients to adapt the technology. The language of communicating data
in 3D is explored to understand ways to take advantage of all dimensions
in augmented reality and virtual reality to deliver information based on
the user's perspective, interest, and urgency.

Creating a mechanism to become aware of the user's intention by
analyzing the gaze through reactive design, we achieved developing a
complex system for demonstrating massive amount of data and organizing
it in a spatial system. The user could walk through and explore the data
and interact with different data visualizations. Moving through space is
used to provide different levels of detail for specific data through Z
axis.

Analytical engineer Steluta Iordache states that virtual reality is
changing the environment of data analysis. It has long been predicted
that augmented reality (AR) and virtual reality (VR) will, sooner rather
than later, dive head first into the mainstream of public consciousness.
Now, expectations are beginning to meet reality, and as tech giants such
as Facebook, Samsung, and Google place heavy investments in these
sectors, this seems inevitable. However, placing the headsets and gaming
to one side (most experts believe AR and VR will most dynamically
disrupt the gaming industry), these nascent technologies can be used by
corporate organizations, too.

By using proper visualization, it is possible to simplify understanding
of a problem and discover a solution more easily. Recently, we have seen
data integrated in the real world and users have been able to interact
with that data, which is not possible with traditional methods such as
plots and charts. We believe AR and VR can build the presentation of the
data and show more information at the same time, as well as allow the
viewer to explore the data by interacting with it. However, when we
analyze data it can be difficult to see the big picture while also
having access to finer details. So the question is: how can AR and VR be
used to understand complex data by interacting with it within a virtual
environment? You can find the answer here\citep{vr_education}

\section{Research Results \& What's
Next}\label{research-results-whats-next}

With the development, studies and new tools applied in data
visualization, more people understand it matters \citep{next_steps} .
But given its youth and interdisciplinary nature, research methods and
training in the field of data visualization are still developing. So, we
asked ourselves: what steps might help accelerate the development of the
field? Based on a group brainstorm and discussion, this article shares
some of the proposals of ongoing discussion and experiment with new
approaches \citep{next_steps}:

\begin{itemize}
\tightlist
\item
  \textbf{Adapting the Publication and Review Process:} As the article
  states, ``both `good' and `bad' reviews could serve as valuable
  guides,'' so providing reviewer guidelines could be helpful for
  fledgling practitioners in the field.
\item
  \textbf{Promoting Discussion and Accretion:} Discussion of research
  papers actively occurs at conferences, on social media, and within
  research groups. Much of this discussion is either ephemeral or
  non-public. So ongoing discussion might explicitly transition to the
  online forum.
\item
  \textbf{Research Methods Training:} Developing a core curriculum for
  data visualization research might help both cases, guiding students
  and instructors alike. For example, recognizing that empirical methods
  were critical to multiple areas of computer science, Stanford CS
  faculty organized a new course on
  \href{http://sing.stanford.edu/cs303-sp11/}{Designing Computer Science
  Experiments}. Also, online resources could be reinforced with a
  catalog of learning resources, ranging from tutorials and self-guided
  study to online courses. Useful examples include Jake Wobbrock's
  Practical Statistics for HCI and Pierre Dragicevic's resources for
  reforming statistical practice.
\end{itemize}

\chapter{Case Studies}\label{case-studies}

Case studies contain valuable information about development records. The
examination and evaluation of case studies helps show that new designs
are just as usable as existing techniques, demonstrating that the field
is suitable for future development. This chapter contains some useful
case studies. Many of the case studies below come from the following
articles:

\begin{longtable}[]{@{}ll@{}}
\toprule
\begin{minipage}[b]{0.15\columnwidth}\raggedright\strut
\textbf{Source}\strut
\end{minipage} & \begin{minipage}[b]{0.79\columnwidth}\raggedright\strut
\textbf{Description}\strut
\end{minipage}\tabularnewline
\midrule
\endhead
\begin{minipage}[t]{0.15\columnwidth}\raggedright\strut
\citep{10_BEST}\strut
\end{minipage} & \begin{minipage}[t]{0.79\columnwidth}\raggedright\strut
This source presents picks the top 10 best data visualizations of 2015.
For each pick, the author displays the project plot and also describes
his reasoning for choosing that chart as an exemplary visualization.
This article is useful for getting a basic understanding of what
characteristics a good visualization should include.\strut
\end{minipage}\tabularnewline
\begin{minipage}[t]{0.15\columnwidth}\raggedright\strut
\citep{cool_data}\strut
\end{minipage} & \begin{minipage}[t]{0.79\columnwidth}\raggedright\strut
The author has chosen fifteen of the best infographics and data
visualizations from 2016 and explained the reasoning behind these
choices. The following six examples are from the articles:\strut
\end{minipage}\tabularnewline
\begin{minipage}[t]{0.15\columnwidth}\raggedright\strut
\citep{int_viz_capt}\strut
\end{minipage} & \begin{minipage}[t]{0.79\columnwidth}\raggedright\strut
16 Captivating Data Visualization Examples\strut
\end{minipage}\tabularnewline
\begin{minipage}[t]{0.15\columnwidth}\raggedright\strut
\citep{15_mindblowing}\strut
\end{minipage} & \begin{minipage}[t]{0.79\columnwidth}\raggedright\strut
15 Data Visualizations That Will Blow Your Mind: ``If a picture is worth
a thousand words, a data visualization is worth at least a million. As
inspiration for your own work with data, check out these 15 data
visualizations that will wow you. Taken together, this roundup is an
at-a-glance representation of the range of uses data analysis has, from
pop culture to public good.''\strut
\end{minipage}\tabularnewline
\begin{minipage}[t]{0.15\columnwidth}\raggedright\strut
\citep{int_viz_2}\strut
\end{minipage} & \begin{minipage}[t]{0.79\columnwidth}\raggedright\strut
15 Data Visualizations That Explain Trump, the White Oscars and Other
Crazy Current Events\strut
\end{minipage}\tabularnewline
\begin{minipage}[t]{0.15\columnwidth}\raggedright\strut
\citep{vizwiz}\strut
\end{minipage} & \begin{minipage}[t]{0.79\columnwidth}\raggedright\strut
Vizwiz is a blog about Tableau-based data visualization. Case studies
about how to improve your visualizations written by Andy Kriebel, a
famous Tableau Zen Master. I would like to recommend this blog because
it is not only practical but also full of insights. My favorite part of
this blog is the ``Makeover Monday,'' which develops a new visualization
based on an original one. This blog also includes great tips for and
examples of Tableau.\strut
\end{minipage}\tabularnewline
\bottomrule
\end{longtable}

Visualization is like art; it speaks where words fail. There are
phenomena like the Syrian war, the number flights during Thanksgiving in
the USA, the understanding of depths for developing a perspective about
the range of the issue, the controversy of `\#OscarsSoWhite', etc. on
which we can write endless paragraphs, but which might still fail to
convince readers. The links show some intricate visualizations of some
of these important and relevant topics--visualizations that speak
volumes and are much more persuasive than an essay, with a tiny fraction
of the text. The usefulness of data visualizations is not just limited
to business and analytics; almost anything in the world can be explained
by visualizations. Wars, rescue operations, social issues etc. can be
visualized to get a clear idea of all the details of the issues.

\section{Geographic Visualizations}\label{geographic-visualizations}

\subsection{Spies in the Skies}\label{spies-in-the-skies}

\citep{spies_sky} referenced in \citep{cool_data}

The map is filled with red and blue lines (representing FBI and DHS
aircraft, respectively) which illustrate the flight paths of the planes.
When planes circle an area more than once, the circles become darker.
The circles change in accordance to day and time, and individual cities
can be typed into a search bar to see the flight patterns over them. The
visualization rather creatively looks almost like a hand-drawn map.
While presenting a normally uncomfortable topic, this allows individuals
to check things for themselves, hopefully providing some peace of mind.

\begin{figure}
\centering
\includegraphics{images/NYCflights.png}
\caption{New York Flight Patterns}
\end{figure}

\subsection{Two Centuries of U.S.
Immigration}\label{two-centuries-of-u.s.-immigration}

\citep{Immigration} referenced in \citep{cool_data}

The interactive map shows the rate of immigration into the U.S. from
other countries over the last 200 years in 10-year segments. Colored
dots represent 10,000 people coming from the specified country.
Countries then light up when they have one of the highest rates of
migration. What makes this a good visualization is that it is engaging
and easy to read and interpret. The movement of the dots draws the
reader's attention while the brightly lit countries make it easy to pick
out the highest total migrations.

\begin{figure}
\centering
\includegraphics{images/immigration.png}
\caption{US Immigration}
\end{figure}

\subsection{Uber: Crafting Data-Driven
Maps}\label{uber-crafting-data-driven-maps}

\citep{uber_maps} Map visualization is very important for companies like
Uber that need to track metrics using geo-space points. In this article,
the designer from Uber talks about the challenges of designing such
visualizations and the possible solutions. While a lot of the problems
are related to the large scale of the data, there are some insights on
scatter plots and hex bins, adding trip lines and making custom tools to
help make decisions. The visualization in this article is beneficial for
developing geospatial graphics.

\subsection{Every Satellite Orbiting
Earth}\label{every-satellite-orbiting-earth}

By David Yanofsky and Tim Fernholz, Published:Nov17,2014 Reference:
\citep{Satellite} This interactive graph, built using a database from
the Union of Concerned Scientists, displays the trajectories of the
1,300 active satellites currently orbiting the Earth. Each satellite is
represented by a circular icon, color-coded by country and sized
according to launch mass.

\subsection{An Interactive Visualization of NYC Street
Trees}\label{an-interactive-visualization-of-nyc-street-trees}

\citep{trees} Using data from NYC Open Data, this interactive
visualization shows the variety and quantity of street trees planted
across the five New York City boroughs.

\subsection{U.S. Migration Patterns}\label{u.s.-migration-patterns}

The New York Times data team mapped out Americans' moving patterns from
1900 to present, and the results are fascinating to interact with. You
can see where people living in each state were born, and where people
are moving to and from. \citep{migration}

\subsection{Selfie City}\label{selfie-city}

\citep{selfie} Selfie City, a detailed multi-component visual
exploration of 3,200 selfies from five major cities around the world,
offers a close look at the demographics and trends of selfies. The team
behind the project collected and filtered the data using Instagram and
Mechanical Turk. Explore the differences between selfies snapped in New
York and Berlin for example, as well as those between men and women
across the world.

\section{Time-Based Visualizations}\label{time-based-visualizations}

\subsection{How People Like You Spend Their
Time}\label{how-people-like-you-spend-their-time}

\citep{spendingtime} referenced in \citep{cool_data} This visual lists
several categories such as ``personal care'' and ``work'' along one side
of a graph with a line illustrating the amount of time the average
person in a certain demographic spends on each subject. Entering
different statistics at the top, such as changing gender or age, causes
the lines to shift to feature that demographic. The simplicity of this
visualization helps the information get across and avoids bogging down
the statistics. Sometimes, less is more.

\begin{figure}
\centering
\includegraphics{images/SpendingTime.png}
\caption{}
\end{figure}

\section{Animated Data Visualization}\label{animated-data-visualization}

\subsection{A Day in the Life of
Americans}\label{a-day-in-the-life-of-americans}

\citep{American_life}

The page linked above includes a great example of an animated data
visualization showing the time people spend on daily activities
throughout the day. The plot is simple and easy to interpret, but it
also includes a good number of variables including time, activity type,
number of people doing each activity, and the order in which activities
are done.

One of the plot's biggest strengths is that by using one dot to
represent each person in the study and using animation, we can actually
drill down to the level of an individual and follow him or her
throughout the day. The accumulation of dots for each particular
activity also gives us an aggregate-level view of the same data, so we
get both individual and aggregate insights.

A drawback of the plot is that it is hard for our eyes to keep track of
1000 simultaneously moving dots. The author of the post addresses this
by creating subsequent plots with stationary lines at key times of the
day. This represents people's movements from one activity to another
without overwhelming the reader.

Overall, this is an engaging, informative, and fun animated plot that
has relevance and tells a story.

\section{Demographics}\label{demographics}

\subsection{The American Workday}\label{the-american-workday}

NPR tapped into American Time Use Survey data to ascertain the share of
workers in a wide range of industries who are at work at any given time.
The chart overlays the traditional 9 AM-5 PM standard over the graph for
a reference point, helping the audience draw interesting conclusions.

\subsection{An Aging Nation: Projected Number of Children and Older
Adults}\label{an-aging-nation-projected-number-of-children-and-older-adults}

\citep{aging_nation} Aging population is always a hot topic in social
economics and politics. I collected several different data
visualizations that show the aging population in the world.

\begin{figure}
\centering
\includegraphics{images/aging_nation.jpg}
\caption{}
\end{figure}

This one includes a bar chart and a line graph to demonstrate the aging
population compared with population of children. The good things about
this visualization are that it is simple to see and compare, employs
color to differentiate the categories, and highlights the intersection
point.

\subsection{From Pyramid to Pillar: A Century of Change, Population of
the
U.S.}\label{from-pyramid-to-pillar-a-century-of-change-population-of-the-u.s.}

\citep{population_pyramid}

\begin{figure}
\centering
\includegraphics{images/Pyramid.jpg}
\caption{}
\end{figure}

This is a \textbf{population pyramid}. ``A \textbf{population pyramid}
is a pair of back-to to histograms for each sex that displays the
distribution of a population in all age groups and in gender.'' It is a
good candidate to visualizing changes in population distributions (sex,
age, year). The shape of a pyramid is also used to represent other
characteristics of a population. To illustrate, A pyramid with a very
wide base and a narrow top section suggests a population with both high
fertility and death rates. It is a useful tool for making sence of
census data. \citep{animated_pyramid} offers an animated pyramid.

\includegraphics{images/3_1.png} This is an animated and
multiple-population pyramid. It used to compare different patterns
across countries. One additional benefit for the interactive population
pyramid is that it shows the shape changes by year, which is useful for
continuous time-series comparison. A similar project with R code is
\href{https://www.r-bloggers.com/who-is-old-visualizing-the-concept-of-prospective-ageing-with-animated-population-pyramids/}{here}.

\subsection{A Guide to Who is Fighting Whom in
Syria}\label{a-guide-to-who-is-fighting-whom-in-syria}

\subsection{Young voters, class and turnout: how Britain voted in
2017}\label{young-voters-class-and-turnout-how-britain-voted-in-2017}

\citep{UKvotes2017} This article's goal is to convey the change in party
votes in the 2017 UK general election compared to votes in 2015. The
change in party votes was shown with regards to three demographic
factors: age, class, and ethnicity. For each factor, there are four
graphs (one per political party), each illustrated in the party's
standard color. The change in percent of votes is shown as an arrow
where the arrow's shaft is the length of the difference from 2015 to
2017 while the x-axis is the demographic factor split into different
bins.

What makes this a good visualization is that it is very easy to read and
interpret. The color-coding of the arrows and party name makes it easy
to pick out the different parties and the arrow lengths highlight just
how large of a change happened. For example, in the Age section, it is
easy to see the pattern between the Labour party gaining many voters
ages 18 to 44 and the Conservative party gaining voters ages 45 and up.

\begin{figure}
\centering
\includegraphics{images/Party_Votes_by_Age.png}
\caption{UK Party Votes by Age}
\end{figure}

\subsection{Simpson's Paradox}\label{simpsons-paradox}

\url{http://vudlab.com/simpsons/}

The Visualizing Urban Data Idealab (VUDlab) out of the University of
California-Berkeley put together this visual representation of data that
disproves the claim in a 1973 suit that charged the school with sex
discrimination. Though the graduate schools had accepted 44\% of male
applicants but only 35\% of female applicants, researchers later
uncovered that if the data were properly pooled, there was actually a
small but statistically significant bias in favor of women. This is
called a Simpson's Paradox.

\subsection{Millennial Generation
Diversity}\label{millennial-generation-diversity}

The millennial generation is bigger, more diverse than boomers
\citep{age_groups}. CNNMoney's interactive chart showing the size and
diversity of the millennial generation compared to baby boomers was
built using U.S. Census Data. It turns dry numbers into an intriguing
story, illustrating the racial makeup of different age groups from 1913
to present.

\section{Uncategorized}\label{uncategorized}

\subsection{Connecting the Dots Behind the
Election}\label{connecting-the-dots-behind-the-election}

This article by the New York Times lists several different candidates
and creates compelling visuals that link their campaigns to previous
ones \citep{campaign_staff}\citep{cool_data}. Each visual contains
several different sized dots that represent a specific campaign,
administration, or other governmental organization related to the
candidate's current campaign, which are then connected by arrows.
Hovering over a specific dot highlights the connections between the
groups. The visual is a great way to summarize what would otherwise
require a long slog through years of information into an easily
accessible, easily viewable format so that voters can figure out where
the candidates' experiences lie.

\begin{figure}
\centering
\includegraphics{images/clinton_campaign.png}
\caption{Clinton 2016 Campaign Staff}
\end{figure}

Source: \citep{campaign_staff} referenced in \citep{cool_data}

\subsection{Malaria}\label{malaria}

For example, an author re-designed ``The Seasonality of Confirmed
Malaria Cases in Zambia Southern Province'' by pointing out what works
well and what could be improved and also explains his goals for the new
visualization (ref:
\url{http://www.vizwiz.com/2018/04/malaria.html})\citep{vizwiz_malaria}

\subsection{Green Honey}\label{green-honey}

\citep{green_honey} referenced in \citep{cool_data}

The visualization spans a webpage. As you scroll down, the text changes,
as do many colored dots that move over the white background. The dots
are used to represent not only each colors' hue, but the numbers that
fall into each category---for example, what colors are the most popular
``base'' colors for English and Chinese. The continuous flow of this
visualization helps bring it together, allowing users to scroll through
the information at their own pace, but also creating a seamless,
creative work.

\begin{figure}
\centering
\includegraphics{images/colorwords.png}
\caption{}
\end{figure}

\subsection{Linguistic Concepts}\label{linguistic-concepts}

\citep{lingui_data} This case study is about usage of linguistic
concepts; it discusses how the data is being used and how visual
graphics are used to deliver the main insights. It presents an
educational tool that integrates computational linguistics resources for
use in non-technical undergraduate language science courses. By using
the tool in conjunction with case studies, it provides opportunities for
students to gain an understanding of linguistic concepts and analysis
through the lens of realistic problems in feasible ways.

Reference: \citep{lingui_data}

\subsection{Is it Better to Rent or
Buy?}\label{is-it-better-to-rent-or-buy}

reference: \citep{rent_or_buy}

The calculator includes several sloping charts. Each chart includes a
factor that will affect how much you'll have to pay, such as the
individual cost of your home and your mortgage rates. A movable scale
along the bottom of each chart allows you to enter different data, such
as changing the ``cost of rent per month'' on the side. This can be
useful in price comparision: if you can find a similar house to rent for
that much per month or less, it's more cost effective to just rent the
home. This visualization is incredibly thorough and a useful tool for
homeowners of any age and status.

\begin{figure}
\centering
\includegraphics{images/rentcalc.png}
\caption{}
\end{figure}

\subsection{What's really warming the
world?}\label{whats-really-warming-the-world}

\citep{world_warming} referenced in \citep{int_viz_1}

This case study begins by clearly explaining necessary background
information and the analytic questions it seeks to answer. Next it
analyzes each factor separately using both verbal explanations and
dynamic graphics to compare the observed temperature movements, and then
categorizes related factors into ``natural factors'' or ``human
factors''. After that, it combines all the dynamic graphics into one,
which makes the results easier and more straightforward to compare.
Lastly, the authors provide more detailed explanations with data set
sources to support their results. Overall, this case study is
straightforward, easy to understand, but also informative.

\subsection{Hans Rosling's 200 Countries, 200 Years, 4
Minutes}\label{hans-roslings-200-countries-200-years-4-minutes}

Global health data expert Hans Rosling's famous statistical documentary
The Joy of Stats aired on BBC in 2010, but it's still turning heads. In
the remarkable segment ``200 Countries, 200 Years, 4 Minutes,'' Rosling
uses augmented reality to explore public health data in 200 countries
over 200 years using 120,000 numbers, in just four minutes
\citep{hans_rosling}.

\subsection{A Guide to Who is Fighting Whom in
Syria}\label{a-guide-to-who-is-fighting-whom-in-syria-1}

One of the charts shown in the link \citep{int_viz_1}, the visualization
of `A Guide to Who is Fighting Whom in Syria' is an interesting graphic
to study. The visualization and its report can be seen at
\citep{syria_chart}.

\begin{figure}
\centering
\includegraphics{images/img_syria_summary.PNG}
\caption{Who is Fighting Whom in Syria}
\end{figure}

This visualization helps elucidate an extremely complicated topic like
the Syrian War. It consists of 3 different emojis in three different
colors, with each (color+facial expression) combination showing the ties
and conflicts between the various groups involved in the Syrian War.
When you click on each of the emoji, a small dialogue box pops up that
explains the relationships between the various countries and rebel
groups involved in the war. This is not only easy to understand, but it
is also pleasing to the eyes.

\begin{figure}
\centering
\includegraphics{images/img_syria_friendly.PNG}
\caption{Green emoji shows `Friendly' relationship}
\end{figure}

\begin{figure}
\centering
\includegraphics{images/img_syria_enemies.PNG}
\caption{Red emoji shows the `Enemies' relationship}
\end{figure}

\begin{figure}
\centering
\includegraphics{images/img_syria_complicated.PNG}
\caption{Yellow emoji shows `Complicated' relationship}
\end{figure}

\subsection{Adding up the White Oscars
Winners}\label{adding-up-the-white-oscars-winners}

\citep{oscars_sowhite_chart} referenced in \citep{int_viz_2}

A visualization of all previous winners of the Best Actor/Actress Oscar
winners can be seen here \citep{oscars_sowhite_chart} in an article by
Bloomberg. From the attributes of past Ocsars winners, the authors have
developed a set of attributes that they believe will continue to be
prevalent in future Oscar winners. It is extremely interesting to see
how the article shows the features of the Best Actress, Actor, movies,
etc. in a simple and captivating visual. The visualization is
interactive and we can click on each attribute like `Hair Color', `Eye
Color', etc. to see the features of the actors and actresses who are
likely to win the Oscars.

\begin{figure}
\centering
\includegraphics{images/img_oscars_actors.PNG}
\caption{Best Actor and Best Actress}
\end{figure}

Source: \citep{oscars_sowhite_chart} referenced in \citep{int_viz_2}

Similarly, the visualization gives also information about the different
aspects of movies that are more likely to win, like `Length,' `Month,'
`Budget,' etc.

\begin{figure}
\centering
\includegraphics{images/img_oscars_pic.PNG}
\caption{Best Picture}
\end{figure}

\subsection{Kissmetrics blog: visualization of
metrics}\label{kissmetrics-blog-visualization-of-metrics}

\citep{facebook_organic} Kissmetrics blog is a place where people talk
about analytics, marketing, and testing through narratives and
visualization of metrics. Metrics are important in the real world,
especially when developing/promoting products. Visualization of metrics
is also essential so that stakeholders can monitor performance, identify
problems and deep dive into potential issues.

A good example from the Kissmetrics blog is about Facebook's organic
reach. One important point discussed in the blog is whether the
Facebook's organic reach is decreasing drastically. The general trend
shows that there is a huge decline in Facebook's page organic reach.

\begin{figure}
\centering
\includegraphics{images/facebook-page-organic-reach-decline-over-years.jpg}
\caption{}
\end{figure}

The following graphs show that the engagement is increasing; that is,
while the quantity of content is decreasing, the quantity is increasing.

\begin{figure}
\centering
\includegraphics{images/average-facebook-reach.png}
\caption{}
\end{figure}

\begin{figure}
\centering
\includegraphics{images/average-facebook-daily-reach.png}
\caption{}
\end{figure}

This resonates with what we have learned at class in terms of how
different perspectives of interpreting data can lead to different
conclusions.

\subsection{How the Recession Reshaped the Economy, in 255
Charts}\label{how-the-recession-reshaped-the-economy-in-255-charts}

The first large graph contains 255 lines to show how the number of jobs
has changed for every industry in America, using color to highlight the
lines and let viewers see the specifics for each industry. By hovering
over a line, viewers can the detailed information of that industry's job
trend. Keeping this extra data hidden until needed makes it easier for
readers to absorb the bigger picture from this huge data visualization.
Below the overall chart are subsets categorized by job sector and
sub-industries. Readers can choose the industry or sector they are
interested in and, like in the first graph, view the more detailed
information by hovering over a line. \citep{recession_economy}

\subsection{Music Timeline}\label{music-timeline}

\url{https://research.google.com/bigpicture/music/}

Google's Music Timeline illustrates a variety of music genres waxing and
waning in popularity from 2010 to present day, based on how many Google
Play Music users have an artist or album in their library, and other
data such as album release dates.

\subsection{State of the Union 2014 Minute by Minute on
Twitter}\label{state-of-the-union-2014-minute-by-minute-on-twitter}

\url{http://twitter.github.io/interactive/sotu2014/\#p1}

Twitter's data team assembled an impressive interactive data hub that
depicts how Twitter users across the globe reacted to each paragraph of
President Obama's 2014 State of the Union address. You can slice and
dice the data by topic hash tag (for example, \#budget, \#defense, or
\#education) and state, resulting in a pretty powerful visualization.

\subsection{Goldilocks Exoplanets}\label{goldilocks-exoplanets}

\url{https://news.nationalgeographic.com/news/2014/04/140417-exoplanet-interactive/}

Using data from the Planetary Habitability Laboratory at the University
of Puerto Rico, the interactive graph plots planetary mass, atmospheric
pressure, and temperature to determine what exoplanets might be home, or
have been home at one point, to living beings.

\subsection{Washington Wizards' Shooting
Stars}\label{washington-wizards-shooting-stars}

\citep{basketball}

This detailed data visualization demonstrates D.C.'s basketball team's
shooting success during the 2013 season. Using statistics released by
the NBA, the visualization allows viewers to examine data for each of 15
players. For example, viewersare able to see how successful each player
was at a variety of types of shots from a range of spots on the court,
compared to others in the league.

\subsection{Global Carbon Emissions}\label{global-carbon-emissions}

\citep{CO2_emission}
\url{https://www.theguardian.com/environment/ng-interactive/2014/dec/01/carbon-emissions-past-present-and-future-interactive}

This data visualization, based on data from the World Resource
Institute's Climate Analysis Indicators Tool and the Intergovernmental
Panel on Climate Change, shows how national CO₂ emissions have
transformed over the last 150 years and what the future might hold. It
also allows the audience to explore emissions by country for a range of
different scenarios.

\subsection{Britain's diet in data}\label{britains-diet-in-data}

This is a good example about how to present a large amount of
comprehensive data - distributed across different categories and
measured in different metrics - in a simple yet effective manner, while
still maintaining interest and aesthetics. The data product attempts to
show how the average Briton's diet has changed over the last 4 decades
for the better \citep{britain_diet_2016}. It does this by displaying
simple trend lines that show that more harmful and rich foods are being
consumed less and the healthier and leaner foods are being consumed
more. It further breaks down every major food category into tens of its
constituent products, and in both the overview and deep-dive versions,
provides further levers to massage more meaning out of the data. It also
shows how the contribution of different foods to the typical diet has
changed over the years. Here, we can toggle the year to see exactly how
much of each food was consumed, again with another deep-dive into the
constituents of every major food group.

\includegraphics{images/britain-diet-data-trends.PNG} Source:
\citep{britain-diet-data-trends} referenced in \citep{britain_diet_2016}
\includegraphics{images/britain-diet-data-typical_diet.png}

Such a visualization is ideal for the layman wanting to walk away with a
basic but accurate understanding of the dietary changes, and also
provides plenty for the more discerning viewer who might have more time
and inclination to dissect and parse through the graphs. It is difficult
to use the same data product to cater to both types of viewers in such a
satisfactory capacity, which is what makes this particular data product
so impressive and effective. It satisfies the principles of graphical
excellence as stated by Edward Tufte \citep{visual_display}

\begin{quote}
Graphical excellence is that which gives to the viewer the greatest
number of ideas in the shortest time with the least ink in the smallest
space.
\end{quote}

\subsection{Visualization of big data security: a case study on the
KDD99 cup data
set}\label{visualization-of-big-data-security-a-case-study-on-the-kdd99-cup-data-set}

This paper utilized a visualization algorithm together with big data
analysis in order to gain better insights into the KDD99 dataset:

\textbf{Abstract}\\
Cybersecurity has been thrust into the limelight in the modern
technological era because of an array of attacks often bypassing
untrained intrusion detection systems (IDSs). Therefore, deciphering
better methods for identifying attack types to train IDSs more
effectively has become a field of great interest. Key cyber-attack
insights exist in big data; however, an efficient approach is required
to determine strong attack types to train IDSs to become more effective
in key areas. Despite the rising growth in IDS research, there is a lack
of studies involving big data visualization, which is key. The KDD99
dataset has served as a strong benchmark since 1999; therefore, we
utilized this data set in our experiment. In this study, we utilized
hash algorithm, a weight table, and sampling method to deal with the
inherent problems caused by analyzing big data: volume, variety, and
velocity. By utilizing a visualization algorithm, we were able to gain
insights into the KDD99 data set with a clear identification of
``normal'' clusters and described distinct clusters of effective
attacks.

To read the full paper, please follow the reference link: \citep{KDD99}

\subsection{\texorpdfstring{How Data Visualization is Helping in
Healthcare Decision
Making?\href{https://marksmanhealthcare.com/data-visualization-helping-healthcare-decision-making/}{link}}{How Data Visualization is Helping in Healthcare Decision Making?link}}\label{how-data-visualization-is-helping-in-healthcare-decision-makinglink}

Healthcare service providers increasingly investigate different visual
and interactive methods in creating and examining large graphs and
charts, interactive visualizations, and 2D/3D visualization of discrete
event simulation (DES) to comprehend complex and large datasets,
recognize connections and trends, model and simulate healthcare events,
and communicate and interpret the findings. Expected results include
more efficient and effective clinical performance monitoring and
improvement, patient flow modeling and management, better patient care
quality, security and effectiveness, better support for clinical costing
and resource coordination, better-planned development and competitive
advantage.

Traditional visualization strategies often require significant
processing time, which restrains high-throughput analysis. Interactive
visualization frameworks maintain a closed loop between the user and the
system and, thus, need to be very fast. Building such a framework
requires the development of new visualization methods, and there exists
the need to design new and effective interaction techniques which are
being developed by researchers.

Informatics for Integrating Biology and the Bedside (i2b2), an
initiative sponsored by the NIH Roadmap National Centers for Biomedical
Computing is another such program that provides a query tool that
supplies aggregate counts and basic analyses of patient populations from
clinical data warehouses (CDWs). i2b2 (i2b2 to Tableau) is effective at
estimating patient cohort sizes and has an extendable design where
modules with additional features can be developed. Other tools such as
R, Python etc also helping healthcare a lot.

Today, data visualization solutions can be found everywhere in
healthcare systems from hospital operations monitoring and patient
profiling to demand projection and capacity planning. Moreover, health
informatics databases and networks have amplified benefits with
information visualization as it dramatically expands the capacity of
patients, clinicians, and public health policy makers to make better
decisions.

\subsection{Tableau: Viz of the Day}\label{tableau-viz-of-the-day}

Tableau has a gallery called
\href{https://public.tableau.com/en-us/s/gallery}{Viz of the Day} that
displays great data visualization examples created by Tableau. It is a
great opportunity to see how others are using many different kinds of
data to create informative yet fun data visuals. Source data is also
attached, so recreation of the visualization is possible as well.

\href{https://public.tableau.com/en-us/s/gallery/what-emoji-say-about-music?gallery=featured}{Describe
Artists with Emoji}. Using the data from Spotify, the author listed the
10 most distinctive emoji used in the playlists related to popular
artists. The table being used in this visual is very straightforward to
link artist to the emojis and is very easy to compare among artists.
When you hover over the emoji, further information is presented.

\chapter{Patterns}\label{patterns}

\section{Using Shapes as Filters in Tableau When Your Fields Are
Measures}\label{using-shapes-as-filters-in-tableau-when-your-fields-are-measures}

Reference: \citep{interworks}

This article is useful to analyze and redesign the different graphs
presented in the article ``America's unique gun violence problem,
explained in 17 maps and charts'' \citep{gunviolence}.This article
introduces the methodologies on how to use shapes as filters in Tableau
when your fields Are Measures. Basically, it teaches you how to load
custom shapes as action filters and use them for showing different
graphs with those filters, which can make your visualization more
interesting and interactive. You can also download the tableau file for
practice.

Case studies document the development record of a project. They provide
the user with an insight into what occurred and relevant details of the
process. A person can gain valuable knowledge that can be reused in
their own projects and improve his or her own methodologies simply by
learning from what others have done.

This article explains how data visualization can enhance awareness of
the data available and its importance in business decisions. The author
explains a situation where poor data visualization led to bad decisions
and the negative impact of these decisions.

\section{Outlier Detection}\label{outlier-detection}

\citep{outliar}

We can use data visualization for outlier detection in a data set.
Different methods for outlier detection in functional data have been
developed during the years. Several of these methods rely on different
notions of functional depth, robust principal components, or random
projections of infinite-dimensional data into R. Some distributional
approaches have also been considered (Gervini, 2009). In functional data
analysis, we observe curves defined over a given real interval and shape
outliers may be defined as those curves that exhibit a different shape
from the rest of the sample. Other types of outliers include:

\textbf{Type 1:} Global outliers (also called ``point anomalies''): A
data point is considered a global outlier if its value is far outside
the entirety of the data set in which it is found.

\textbf{Type 2:} Contextual (conditional) outliers: A data point is
considered a contextual outlier if its value significantly deviates from
the rest the data points in the same context. Note that this means that
same value may not be considered an outlier if it occurred in a
different context. If we limit our discussion to time series data, the
``context'' is almost always temporal, because time series data are
records of a specific quantity over time. Contextual outliers are common
in time series data.

\textbf{Type 3:} Collective outliers: A subset of data points within a
data set is considered anomalous if those values as a collection deviate
significantly from the entire data set, but the values of the individual
data points are not themselves anomalous in either a contextual or
global sense. In time series data, one way this can manifest is as
normal peaks and valleys occurring outside of a time frame when that
seasonal sequence is normal or as a combination of time series data that
is in an outlier as a group.

Ref:\citep{outliar}. Below is a simple example. Outlier treatment is
important because it can drastically bias/change the fit estimates and
predictions. Illustration:

\begin{Shaded}
\begin{Highlighting}[]
\CommentTok{# Inject outliers into data.}
\NormalTok{cars1 <-}\StringTok{ }\NormalTok{cars[}\DecValTok{1}\OperatorTok{:}\DecValTok{30}\NormalTok{, ]  }\CommentTok{# original data}
\NormalTok{cars_outliers <-}\StringTok{ }\KeywordTok{data.frame}\NormalTok{(}\DataTypeTok{speed=}\KeywordTok{c}\NormalTok{(}\DecValTok{19}\NormalTok{,}\DecValTok{19}\NormalTok{,}\DecValTok{20}\NormalTok{,}\DecValTok{20}\NormalTok{,}\DecValTok{20}\NormalTok{), }\DataTypeTok{dist=}\KeywordTok{c}\NormalTok{(}\DecValTok{190}\NormalTok{, }\DecValTok{186}\NormalTok{, }\DecValTok{210}\NormalTok{, }\DecValTok{220}\NormalTok{, }\DecValTok{218}\NormalTok{))  }\CommentTok{# introduce outliers.}
\NormalTok{cars2 <-}\StringTok{ }\KeywordTok{rbind}\NormalTok{(cars1, cars_outliers)  }\CommentTok{# data with outliers.}

\CommentTok{# Plot of data with outliers.}
\KeywordTok{par}\NormalTok{(}\DataTypeTok{mfrow=}\KeywordTok{c}\NormalTok{(}\DecValTok{1}\NormalTok{, }\DecValTok{2}\NormalTok{))}
\KeywordTok{plot}\NormalTok{(cars2}\OperatorTok{$}\NormalTok{speed, cars2}\OperatorTok{$}\NormalTok{dist, }\DataTypeTok{xlim=}\KeywordTok{c}\NormalTok{(}\DecValTok{0}\NormalTok{, }\DecValTok{28}\NormalTok{), }\DataTypeTok{ylim=}\KeywordTok{c}\NormalTok{(}\DecValTok{0}\NormalTok{, }\DecValTok{230}\NormalTok{), }\DataTypeTok{main=}\StringTok{"With Outliers"}\NormalTok{, }\DataTypeTok{xlab=}\StringTok{"speed"}\NormalTok{, }\DataTypeTok{ylab=}\StringTok{"dist"}\NormalTok{, }\DataTypeTok{pch=}\StringTok{"*"}\NormalTok{, }\DataTypeTok{col=}\StringTok{"red"}\NormalTok{, }\DataTypeTok{cex=}\DecValTok{2}\NormalTok{)}
\KeywordTok{plot}\NormalTok{(cars2}\OperatorTok{$}\NormalTok{dist,cars2}\OperatorTok{$}\NormalTok{speed)}
\end{Highlighting}
\end{Shaded}

\includegraphics{Data_Viz_Reader_files/figure-latex/Outliar Detection-1.pdf}

\begin{Shaded}
\begin{Highlighting}[]
\CommentTok{# Plot of original data without outliers. Note the change in slope (angle) of best fit line.}
\KeywordTok{plot}\NormalTok{(cars1}\OperatorTok{$}\NormalTok{speed, cars1}\OperatorTok{$}\NormalTok{dist, }\DataTypeTok{xlim=}\KeywordTok{c}\NormalTok{(}\DecValTok{0}\NormalTok{, }\DecValTok{28}\NormalTok{), }\DataTypeTok{ylim=}\KeywordTok{c}\NormalTok{(}\DecValTok{0}\NormalTok{, }\DecValTok{230}\NormalTok{), }\DataTypeTok{main=}\StringTok{"Outliers removed }\CharTok{\textbackslash{}n}\StringTok{ A much better fit!"}\NormalTok{, }\DataTypeTok{xlab=}\StringTok{"speed"}\NormalTok{, }\DataTypeTok{ylab=}\StringTok{"dist"}\NormalTok{, }\DataTypeTok{pch=}\StringTok{"*"}\NormalTok{, }\DataTypeTok{col=}\StringTok{"red"}\NormalTok{, }\DataTypeTok{cex=}\DecValTok{2}\NormalTok{)}
\end{Highlighting}
\end{Shaded}

\includegraphics{Data_Viz_Reader_files/figure-latex/Outliar Detection-2.pdf}

Detection of Outliers is performed using:

\begin{itemize}
\tightlist
\item
  Univariate Approach
\item
  Multivariate Approach
\item
  Multivariate Model Approach
\end{itemize}

\section{Genetic Network
Reconstruction}\label{genetic-network-reconstruction}

Data visualization techniques are used to reconstruct genetic networks
from genomics data. Reconstructed genetic networks are predicted
interactions among genes of interest and these interactions are inferred
from genomics data, microarray data or DNA sequences. Genomics data are
generally contaminated and high-dimensional. It is important to examine
and clean data carefully to attain meaningful inferences. Thus,
visualization tools that are used in the preprocessing of data
associated with genetic network reconstruction are also reviewed and
chosen wisely.

\section{Tips to Improve Data
Visualization}\label{tips-to-improve-data-visualization}

\citep{French} \citep{Steier}

\subsection{Comparison}\label{comparison}

Include a zero baseline if possible. Although a line chart does not have
to start at a zero baseline, it should be included if it gives more
context for comparison. If relatively small fluctuations in data are
meaningful (e.g., in stock market data), you may truncate the scale to
showcase these variances; Always choose the most efficient
visualization; Watch your placement You may have two nice stacked bar
charts that are meant to let your reader compare points, but if they're
placed too far apart to ``get'' the comparison, you've already lost;
Tell the whole story. Maybe you had a 30\% sales increase in Q4.
Exciting! But what's more exciting? Showing that you've actually had a
100\% sales increase since Q1.

\subsection{Copy}\label{copy}

Don't over explain if the copy already mentions a fact, the subhead,
callout, and chart header don't have to reiterate it; Keep chart and
graph headers simple and to the point. There's no need to get clever,
verbose, or pun-tastic. Keep any descriptive text above the chart brief
and directly related to the chart underneath. Remember: Focus on the
quickest path to comprehension; Use callouts wisely Callouts are not
there to fill space. They should be used intentionally to highlight
relevant information or provide additional context; Don't use
distracting fonts or elements Sometimes you do need to emphasize a
point. If so, only use bold or italic text to emphasize a point---and
don't use them both at the same time.

\subsection{Color}\label{color}

Use a single color to represent the same type of data; Watch out for
positive and negative numbers Don't use red for positive numbers or
green for negative numbers. Those color associations are so strong it
will automatically flip the meaning in the viewer's mind; Make sure
there is sufficient contrast between colors; Avoid patterns Stripes and
polka dots sound fun, but they can be incredibly distracting. If you are
trying to differentiate, say, on a map, use different saturation of the
same color. On that note, only use solid-colored lines (not dashes);
Select colors appropriately; Don't use more than 6 colors in a single
layout.

\subsection{Ordering}\label{ordering}

Order data intuitively There should be a logical hierarchy. Order
categories alphabetically, sequentially, or by value; Order
consistently; Order evenly Use natural increments on your axes (0, 5,
10, 15, 20) instead of awkward or uneven increments (0, 3, 5, 16, 50).

\subsection{Audience perspective}\label{audience-perspective}

Let the users lead; Know your audience, designers should consider the
way users prefer to understand information, even in choosing basic
analytic approaches. For users to feel comfortable adopting and sharing
insights from analytics, they must be able to explain and defend the
data.

\subsection{Use layers to tell a
story}\label{use-layers-to-tell-a-story}

While style is one form of customization, layering unique data sets on a
single visualization can tell a richer narrative and connect users to
the data without getting too crowded. On a map, this can be as simple as
zooming in and out, but it can also involve drill-downs (choosing a data
point and expanding it to show more detail), links and other shortcuts.

\subsection{Keep it simple}\label{keep-it-simple-1}

Analytic results shouldn't be presented to 10 decimal places when the
user doesn't need that level of precision to make a decision or
understand a concept. Effective visual interfaces avoid 3-D effects or
ornate gauge designs (a.k.a. ``chart junk'') when simple numbers, maps
or graphs will do.

\section{Building Advanced Analytics Application with
TabPy}\label{building-advanced-analytics-application-with-tabpy-1}

\citep{TabPy}

Imagine a scenario where we can just enter some x values in a dashboard
form, and the visualization would predict the y variable!!! Here is a
link that shows how to integrate and visualize data from Python in
Tableau. This is especially relevant to all data science students, as
this is one of the tools used for visualizing advanced analytics. The
author here has given an example using data from Seattle's police
department's 911 calls and he tries to identify criminal hotspots in the
area. The author uses machine learning (spatial clustering) and creates
a great interactive visualization, where you can click on the type of
criminal activity and the graph will show various clusters. There are
other examples and use cases that may be downloaded, and the scripts are
also given by the author to anyone who is interested in trying it out.

\section{Pick the Right Chart Type}\label{pick-the-right-chart-type}

Data visualization is a combination of art and science. When it comes to
the artistic aspect, there are no correct answers for doing the
visualization. There are many ways to present the data. However, when
making sense of facts, numbers, and measurement, the better
understanding is promoted by a logical path to follow. To determine the
best type of chart is hard for those new to data visualization. Most
people learn it by referring to other people's work without
understanding the underlying logic, so they don't have the theory in
their mind to make the judgment.

When we are choosing the type of chart, we need to answer some
questions:

\begin{itemize}
\tightlist
\item
  How many features would you like to show in a chart?
\item
  How many data points do you want to display for each variable?
\item
  Will you display time serious data or among items or groups.
\end{itemize}

After answering these questions, you should able to get a better
imagination of your ideal graph. The simple guidance for using the
different type of chart is line charts for tracking trends over time,
bar charts to compare quantities, scatter plots for a joint variation of
two data items, bubble charts showing joint variation of three data
items, and pie charts to compare parts of a whole.

\section{Why pie chart is bad: a comparison with the bar
chart}\label{why-pie-chart-is-bad-a-comparison-with-the-bar-chart}

Using pie chart is usually considered as a bad idea when it comes to
data visualization. But why? Here, we explore some cons of using the pie
chart to convey information and compare its effectiveness to bar chart
\citep{hickey-pie-worst} \citep{henry-defense-pie} \citep{quach-penny}.

\begin{enumerate}
\def\labelenumi{\arabic{enumi}.}
\tightlist
\item
  Some information may look nearly identical in pie chart. But if the
  data is presented with bar charts, the story is different. See figure
  \ref{fig:hickey-before} and \ref{fig:hickey-after} for examples.
\end{enumerate}

\includegraphics{images/hickey-before.jpg} Source:
\citep{hickey-pie-worst}

\includegraphics{images/hickey-after.jpg} Source:
\citep{hickey-pie-worst}

\begin{enumerate}
\def\labelenumi{\arabic{enumi}.}
\setcounter{enumi}{1}
\tightlist
\item
  It is difficult to compare the slices of a circle to figure out the
  distinctions in size between each pie slice, especially when there are
  a lot of categories. See figure \ref{fig:hickey-breakdown} for
  example.
\end{enumerate}

\begin{figure}
\centering
\includegraphics{images/hickey-breakdown.jpg}
\caption{}
\end{figure}

(Source: \citep{hickey-pie-worst})

\begin{enumerate}
\def\labelenumi{\arabic{enumi}.}
\setcounter{enumi}{2}
\tightlist
\item
  A Pie chart is easy to be manipulated (e.g.~using a 3D pie chart). See
  figure \ref{fig:hickey-3D} for example.
\end{enumerate}

\includegraphics{images/hickey-3D.jpg} Source: \citep{hickey-pie-worst}

\begin{enumerate}
\def\labelenumi{\arabic{enumi}.}
\setcounter{enumi}{3}
\tightlist
\item
  A Pie chart may be useful when comparing 2 different categories with
  different amounts of information. Specifically, it does a better job
  to distinguish two parts with a 25:75 split or one that is not 50:50
  as people are sensitive to a right angle or a dividing line that is
  not straight. But this could be simply done by showing two numbers!
  See figure \ref{fig:henry-quarter} and \ref{fig:henry-half} for
  examples.
\end{enumerate}

\includegraphics{images/henry-quarter.png} (Source:
\citep{henry-defense-pie})

\includegraphics{images/henry-half.png} (Source:
\citep{henry-defense-pie})

\section{Chose the right baseline in data
visualization}\label{chose-the-right-baseline-in-data-visualization}

The baseline is very important to data visualization. If the baseline is
different, the meaning will change a lot. Now here is a case study to
show the importance of baseline and how to use it in different ways.

Here I use the same method for a new data set to . (missing\ldots{}?)

\begin{Shaded}
\begin{Highlighting}[]
\CommentTok{# Create the data.}
\NormalTok{a <-}\KeywordTok{rep}\NormalTok{(}\KeywordTok{c}\NormalTok{(}\DecValTok{2010}\NormalTok{,}\DecValTok{2011}\NormalTok{,}\DecValTok{2012}\NormalTok{,}\DecValTok{2013}\NormalTok{,}\DecValTok{2014}\NormalTok{,}\DecValTok{2015}\NormalTok{),}\DataTypeTok{each =} \DecValTok{4}\NormalTok{)}
\NormalTok{b <-}\StringTok{ }\KeywordTok{seq}\NormalTok{(}\DecValTok{1}\OperatorTok{:}\DecValTok{24}\NormalTok{)}
\NormalTok{c <-}\StringTok{ }\KeywordTok{c}\NormalTok{(}\FloatTok{64.9}\NormalTok{,}\FloatTok{65.33}\NormalTok{,}\FloatTok{71.67}\NormalTok{,}\FloatTok{79.17}\NormalTok{,}\FloatTok{68.78}\NormalTok{,}\FloatTok{69.83}\NormalTok{,}\FloatTok{78.61}\NormalTok{,}\FloatTok{92.68}\NormalTok{,}\FloatTok{89.28}\NormalTok{,}\FloatTok{90.43}\NormalTok{,}\FloatTok{97.96}\NormalTok{,}\FloatTok{106.96}\NormalTok{,}\FloatTok{100.66}\NormalTok{,}\FloatTok{107.53}\NormalTok{,}\FloatTok{117.06}\NormalTok{,}\FloatTok{119.21}\NormalTok{,}\FloatTok{110.05}\NormalTok{,}\FloatTok{97.42}\NormalTok{,}\FloatTok{93.62}\NormalTok{,}\FloatTok{97.99}\NormalTok{,}\DecValTok{80}\NormalTok{,}\FloatTok{88.74}\NormalTok{,}\FloatTok{102.06}\NormalTok{,}\DecValTok{83}\NormalTok{)}
\NormalTok{data <-}\StringTok{ }\KeywordTok{as.data.frame}\NormalTok{(}\KeywordTok{cbind}\NormalTok{(a,b,c))}
\KeywordTok{colnames}\NormalTok{(data) <-}\KeywordTok{c}\NormalTok{(}\StringTok{"year"}\NormalTok{,}\StringTok{"quater"}\NormalTok{,}\StringTok{"sales"}\NormalTok{)}
\end{Highlighting}
\end{Shaded}

\begin{enumerate}
\def\labelenumi{\arabic{enumi}.}
\tightlist
\item
  Regular quarterly sales. We can see sales decreased a lot around 2014.
  \textbf{The baseline here is historical sales.}
\end{enumerate}

\begin{Shaded}
\begin{Highlighting}[]
\CommentTok{# Regular time series for sales}
\KeywordTok{par}\NormalTok{(}\DataTypeTok{cex.axis=}\FloatTok{0.7}\NormalTok{)}
\NormalTok{data.ts <-}\StringTok{ }\KeywordTok{ts}\NormalTok{(data}\OperatorTok{$}\NormalTok{sales, }\DataTypeTok{start=}\KeywordTok{c}\NormalTok{(}\DecValTok{2010}\NormalTok{, }\DecValTok{1}\NormalTok{), }\DataTypeTok{frequency=}\DecValTok{4}\NormalTok{)}
\KeywordTok{plot}\NormalTok{(data.ts, }\DataTypeTok{xlab=}\StringTok{""}\NormalTok{, }\DataTypeTok{ylab=}\StringTok{""}\NormalTok{, }\DataTypeTok{main=}\StringTok{"sales per quater"}\NormalTok{, }\DataTypeTok{las=}\DecValTok{1}\NormalTok{, }\DataTypeTok{bty=}\StringTok{"n"}\NormalTok{)}
\end{Highlighting}
\end{Shaded}

\includegraphics{Data_Viz_Reader_files/figure-latex/unnamed-chunk-2-1.pdf}

\begin{enumerate}
\def\labelenumi{\arabic{enumi}.}
\setcounter{enumi}{1}
\tightlist
\item
  Quarterly and yearly change sales. \textbf{The baseline here is zero
  and look at the percentage changes.}
\end{enumerate}

\begin{Shaded}
\begin{Highlighting}[]
\CommentTok{# Quaterly change}
\NormalTok{curr <-}\StringTok{ }\KeywordTok{as.numeric}\NormalTok{(data}\OperatorTok{$}\NormalTok{sales[}\OperatorTok{-}\DecValTok{1}\NormalTok{])}
\NormalTok{prev <-}\StringTok{ }\KeywordTok{as.numeric}\NormalTok{(data}\OperatorTok{$}\NormalTok{sales[}\DecValTok{1}\OperatorTok{:}\NormalTok{(}\KeywordTok{length}\NormalTok{(data}\OperatorTok{$}\NormalTok{sales)}\OperatorTok{-}\DecValTok{1}\NormalTok{)])}
\NormalTok{quaChange <-}\StringTok{ }\DecValTok{100} \OperatorTok{*}\StringTok{ }\KeywordTok{round}\NormalTok{( (curr}\OperatorTok{-}\NormalTok{prev) }\OperatorTok{/}\StringTok{ }\NormalTok{prev, }\DecValTok{2}\NormalTok{ )}
\NormalTok{barCols <-}\StringTok{ }\KeywordTok{sapply}\NormalTok{(quaChange, }
    \ControlFlowTok{function}\NormalTok{(x) \{ }
        \ControlFlowTok{if}\NormalTok{ (x }\OperatorTok{<}\StringTok{ }\DecValTok{0}\NormalTok{) \{}
            \KeywordTok{return}\NormalTok{(}\StringTok{"#2cbd25"}\NormalTok{)}
\NormalTok{        \} }\ControlFlowTok{else}\NormalTok{ \{}
            \KeywordTok{return}\NormalTok{(}\StringTok{"gray"}\NormalTok{)}
\NormalTok{        \}}
\NormalTok{    \})}

\KeywordTok{barplot}\NormalTok{(quaChange, }\DataTypeTok{border=}\OtherTok{NA}\NormalTok{, }\DataTypeTok{space=}\DecValTok{0}\NormalTok{, }\DataTypeTok{las=}\DecValTok{1}\NormalTok{, }\DataTypeTok{col=}\NormalTok{barCols, }\DataTypeTok{main=}\StringTok{"% sales change, quaterly"}\NormalTok{)}
\end{Highlighting}
\end{Shaded}

\includegraphics{Data_Viz_Reader_files/figure-latex/unnamed-chunk-3-1.pdf}

\begin{Shaded}
\begin{Highlighting}[]
\CommentTok{# Year-over-year change}
\NormalTok{curr <-}\StringTok{ }\KeywordTok{as.numeric}\NormalTok{(data}\OperatorTok{$}\NormalTok{sales[}\OperatorTok{-}\NormalTok{(}\DecValTok{1}\OperatorTok{:}\DecValTok{4}\NormalTok{)])}
\NormalTok{prev <-}\StringTok{ }\KeywordTok{as.numeric}\NormalTok{(data}\OperatorTok{$}\NormalTok{sales[}\DecValTok{1}\OperatorTok{:}\NormalTok{(}\KeywordTok{length}\NormalTok{(data}\OperatorTok{$}\NormalTok{sales)}\OperatorTok{-}\DecValTok{4}\NormalTok{)])}
\NormalTok{annChange <-}\StringTok{ }\DecValTok{100} \OperatorTok{*}\StringTok{ }\KeywordTok{round}\NormalTok{( (curr}\OperatorTok{-}\NormalTok{prev) }\OperatorTok{/}\StringTok{ }\NormalTok{prev, }\DecValTok{2}\NormalTok{ )}
\NormalTok{barCols <-}\StringTok{ }\KeywordTok{sapply}\NormalTok{(annChange, }
    \ControlFlowTok{function}\NormalTok{(x) \{ }
        \ControlFlowTok{if}\NormalTok{ (x }\OperatorTok{<}\StringTok{ }\DecValTok{0}\NormalTok{) \{}
            \KeywordTok{return}\NormalTok{(}\StringTok{"#2cbd25"}\NormalTok{)}
\NormalTok{        \} }\ControlFlowTok{else}\NormalTok{ \{}
            \KeywordTok{return}\NormalTok{(}\StringTok{"gray"}\NormalTok{)}
\NormalTok{        \}}
\NormalTok{    \})}

\KeywordTok{barplot}\NormalTok{(annChange, }\DataTypeTok{border=}\OtherTok{NA}\NormalTok{, }\DataTypeTok{space=}\DecValTok{0}\NormalTok{, }\DataTypeTok{las=}\DecValTok{1}\NormalTok{, }\DataTypeTok{col=}\NormalTok{barCols, }\DataTypeTok{main=}\StringTok{"% sales change, annual"}\NormalTok{)}
\end{Highlighting}
\end{Shaded}

\includegraphics{Data_Viz_Reader_files/figure-latex/unnamed-chunk-4-1.pdf}
From this plot, it is very clear that the magnitude of drops in sales
for some quarters.

\begin{enumerate}
\def\labelenumi{\arabic{enumi}.}
\setcounter{enumi}{2}
\tightlist
\item
  The sales difference compare to now. \textbf{The baseline here is the
  current sales.}
\end{enumerate}

\begin{Shaded}
\begin{Highlighting}[]
\CommentTok{# Relative to current 2015}
\NormalTok{curr <-}\StringTok{ }\KeywordTok{as.numeric}\NormalTok{(data}\OperatorTok{$}\NormalTok{sales[}\KeywordTok{length}\NormalTok{(data}\OperatorTok{$}\NormalTok{sales)])}
\NormalTok{salesDiff <-}\StringTok{ }\KeywordTok{as.numeric}\NormalTok{(data}\OperatorTok{$}\NormalTok{sales) }\OperatorTok{-}\StringTok{ }\NormalTok{curr}
\NormalTok{barCols.diff <-}\StringTok{ }\KeywordTok{sapply}\NormalTok{(salesDiff,}
    \ControlFlowTok{function}\NormalTok{(x) \{}
        \ControlFlowTok{if}\NormalTok{ (x }\OperatorTok{<}\StringTok{ }\DecValTok{0}\NormalTok{) \{}
            \KeywordTok{return}\NormalTok{(}\StringTok{"gray"}\NormalTok{)}
\NormalTok{        \} }\ControlFlowTok{else}\NormalTok{ \{}
            \KeywordTok{return}\NormalTok{(}\StringTok{"black"}\NormalTok{)}
\NormalTok{        \}}
\NormalTok{    \}}
\NormalTok{)}
\KeywordTok{barplot}\NormalTok{(salesDiff, }\DataTypeTok{border=}\OtherTok{NA}\NormalTok{, }\DataTypeTok{space=}\DecValTok{0}\NormalTok{, }\DataTypeTok{las=}\DecValTok{1}\NormalTok{, }\DataTypeTok{col=}\NormalTok{barCols.diff, }\DataTypeTok{main=}\StringTok{"Sales difference from last quater 2015"}\NormalTok{)}
\end{Highlighting}
\end{Shaded}

\includegraphics{Data_Viz_Reader_files/figure-latex/unnamed-chunk-5-1.pdf}

\begin{enumerate}
\def\labelenumi{\arabic{enumi}.}
\setcounter{enumi}{3}
\tightlist
\item
  Sales difference compared to the first quarter. ** The baseline here
  is the first quater sales.**
\end{enumerate}

\begin{Shaded}
\begin{Highlighting}[]
\CommentTok{# Relative to first quater}
\NormalTok{ori <-}\StringTok{ }\KeywordTok{as.numeric}\NormalTok{(data}\OperatorTok{$}\NormalTok{sales[}\DecValTok{1}\NormalTok{])}
\NormalTok{salesDiff <-}\StringTok{ }\KeywordTok{as.numeric}\NormalTok{(data}\OperatorTok{$}\NormalTok{sales) }\OperatorTok{-}\StringTok{ }\NormalTok{ori}
\NormalTok{barCols.diff <-}\StringTok{ }\KeywordTok{sapply}\NormalTok{(salesDiff,}
    \ControlFlowTok{function}\NormalTok{(x) \{}
        \ControlFlowTok{if}\NormalTok{ (x }\OperatorTok{<}\StringTok{ }\DecValTok{0}\NormalTok{) \{}
            \KeywordTok{return}\NormalTok{(}\StringTok{"gray"}\NormalTok{)}
\NormalTok{        \} }\ControlFlowTok{else}\NormalTok{ \{}
            \KeywordTok{return}\NormalTok{(}\StringTok{"black"}\NormalTok{)}
\NormalTok{        \}}
\NormalTok{    \}}
\NormalTok{)}
\KeywordTok{barplot}\NormalTok{(salesDiff, }\DataTypeTok{border=}\OtherTok{NA}\NormalTok{, }\DataTypeTok{space=}\DecValTok{0}\NormalTok{, }\DataTypeTok{las=}\DecValTok{1}\NormalTok{, }\DataTypeTok{col=}\NormalTok{barCols.diff, }\DataTypeTok{main=}\StringTok{"Sales difference from first quater 2010"}\NormalTok{)}
\end{Highlighting}
\end{Shaded}

\includegraphics{Data_Viz_Reader_files/figure-latex/unnamed-chunk-6-1.pdf}

\begin{enumerate}
\def\labelenumi{\arabic{enumi}.}
\setcounter{enumi}{4}
\tightlist
\item
  The difference between quarter sales and mean. ** The baseline is mean
  now.**
\end{enumerate}

\begin{Shaded}
\begin{Highlighting}[]
\CommentTok{# difference from the mean}
\NormalTok{mean <-}\StringTok{ }\KeywordTok{mean}\NormalTok{(}\KeywordTok{as.numeric}\NormalTok{(data}\OperatorTok{$}\NormalTok{sales))}
\NormalTok{salesDiff <-}\StringTok{ }\KeywordTok{as.numeric}\NormalTok{(data}\OperatorTok{$}\NormalTok{sales) }\OperatorTok{-}\StringTok{ }\NormalTok{mean}
\NormalTok{barCols.diff <-}\StringTok{ }\KeywordTok{sapply}\NormalTok{(salesDiff,}
    \ControlFlowTok{function}\NormalTok{(x) \{}
        \ControlFlowTok{if}\NormalTok{ (x }\OperatorTok{<}\StringTok{ }\DecValTok{0}\NormalTok{) \{}
            \KeywordTok{return}\NormalTok{(}\StringTok{"gray"}\NormalTok{)}
\NormalTok{        \} }\ControlFlowTok{else}\NormalTok{ \{}
            \KeywordTok{return}\NormalTok{(}\StringTok{"black"}\NormalTok{)}
\NormalTok{        \}}
\NormalTok{    \}}
\NormalTok{)}
\KeywordTok{barplot}\NormalTok{(salesDiff, }\DataTypeTok{border=}\OtherTok{NA}\NormalTok{, }\DataTypeTok{space=}\DecValTok{0}\NormalTok{, }\DataTypeTok{las=}\DecValTok{1}\NormalTok{, }\DataTypeTok{col=}\NormalTok{barCols.diff, }\DataTypeTok{main=}\StringTok{"Sales difference from mean"}\NormalTok{)}
\end{Highlighting}
\end{Shaded}

\includegraphics{Data_Viz_Reader_files/figure-latex/unnamed-chunk-7-1.pdf}

So before we start to plot, we should decide the baseline we want to
use. Different baseline will lead to totally different graphs.
reference: \citep{Baseline_2013}

\section{Using design patterns to find greater meaning in your
data}\label{using-design-patterns-to-find-greater-meaning-in-your-data}

Visualizations that show comparisons, connections, and conclusions offer
analytical clarity.

Patterns based on function can help you see differences and similarities
more clearly, understand relationships and behaviors more intimately,
and predict future results with a greater level of certainty. When these
patterns are presented as visualizations, they help you 1) see
comparisons, 2) make connections, and 3) draw conclusions from your data
sets. The major functions can be described as:

\subsection{Comparisons}\label{comparisons}

\includegraphics{images/patten-1.jpg} As shown in Figure 1, the bar
chart with sparkline enables you to review the data at two different
levels: a high-level assessment of the short-term three-month returns is
represented with the bar chart, while the sparkline (the line chart
below the bar) provides the details of the historical returns. Quickly
and concisely, the sparkline shows you the path that has led up to the
most recent returns. You can then assess that a narrow path provides
consistent returns across the years while a wide path provides varied
returns. Side-by-side comparisons of funds organized into two
columns---\% Returns and \% Ahead of Benchmark---enables peer
comparisons and fund-specific benchmark comparisons. Hence, you can see
that not only has Global Large Cap Core provided positive returns, it
has also provided the best and most consistent returns when compared to
the benchmark.

\subsection{Connections}\label{connections}

The string of charts in Figure 2 shows 10-year to year-to-date (YTD)
performance returns, which can be interpreted as individual charts or a
group of category charts. \includegraphics{images/pattern-2.jpg} Similar
to sounds waves, the symmetrical area charts grow equidistant from the
source (the zero line) at each time interval to accentuate the returns
even further. Here, the y-axis is shown in percentage. Instead of using
the zero line to indicate positive or negative returns, it uses color to
denote if the category returns are positive (black) or negative (red).
For example, Multi-Cap Russell 3000 Growth produced 20\% positive
returns within the one-year time period and is shown with color fill in
both directions from the zero line to purposefully duplicate the large
gains and specifically uses black color fill to indicate the returns are
positive. As evident from the name, the symmetrical chart doubles the
returns to emphasize the amount of color fill.

What else can you derive from organizing the information in a spectrum
of negative to positive returns? Based on this organization, three
groups of categories have resulted in straight losses (red), heavy gains
(black), or a mix of gains and losses across a decade of returns. The
string of charts makes it easier for you to see these three groups of
categories to assess their distribution. Just like sound waves, each
chart is a sound bite that streams the returns for each category with a
``scream'' announcing a huge gain (e.g., Multi-Cap Russel 3000 Growth)
or loss (e.g., Mid-Cap Russel Mid Cap Growth). In some cases (e.g.,
Large Cap S\&P 500), the chart quietly announces mixed returns to
adequately demand less attention.

Next, you might wonder how you would have fared if you had invested in
certain funds. You might ask: if I had purchased this fund five years
ago, what would my return be? And what about the YTD returns? Since
market timing is key to investment choices, the following presentation
of hypothetical investments represents a range of results.

\subsection{Conclusions}\label{conclusions}

\includegraphics{images/pattern-3.png} In Figure 3, varied performance
results become clear with a layered approach to show five potential
entry points (10-year, 5-year, 3-year, 1-year, YTD) into an investment.
For example, the International Large Cap Core fund provided 27\% YTD
returns, which contrast the negative returns you would have received had
you invested in the fund 1, 5, or 10 years ago. Here, conclusions are
derived based on known inputs with a divided review of positive or
negative outcomes (shown on the y-axis).

The line weights help to identify each entry point and show the range of
differences between the entry points. Accordingly so, resulting returns
are shown with simplified curves that connect the inputs and outputs. In
this case, the chart has been customized to show an instance in which
the user has opted to see the YTD return values as percentages listed to
the right of each resulting output.\citep{data_meaning}

\section{Chart types}\label{chart-types}

Let's review the most commonly used chart types and explain what
circumstance should better use typical chart and the pros and cons of
each type of chart. Before introducing different types of charts, you
can use the following website to familiar with different types of charts
\citep{charts_viz}. \#\#\# Time Series Data

Reference: \citep{aya-time-series} What are some of the most common data
visualizations seen in newspapers, textbooks, and corporate annual
reports? Graphs showing a country's GDP growth trends or charts
capturing a company's sales growth in the last 4 quarters would be high
up on the list. \textbf{Essentially, these are visualizations that track
time series data -- the performance of an indicator over a period of
time -- also known as temporal visualizations.}

Temporal visualizations are one of the simplest, quickest ways to
represent important time series data. There are 7 handy temporal
visualization styles for your time series data.

\subsection{Line Graph}\label{line-graph}

A line graph is the simplest way to represent time series data. It is
intuitive, easy to create, and helps the viewer get a quick sense of how
something has changed over time.

\subsection{Stacked Area Chart}\label{stacked-area-chart}

Stacked area charts are area charts similar to a line chart. In an area
chart, multiple variables are ``stacked'' on top of each other, and the
area below each line is colored to represent each variable. Stacked area
charts are useful to show how both a cumulative total and individual
components of that total changed over time. The order in which we stack
the variables is crucial because there can sometimes be a difference in
the actual plot versus human perception.

The figure below is a stacked area chart showing time series data:
\includegraphics{images/aya-stacked.png} (Source:
\citep{aya-time-series})

\subsection{Bar Charts}\label{bar-charts}

represent data as horizontal or vertical bars. The length of each bar is
proportional to the value of the variable at that point in time. A bar
chart is the right choice for you when you wish to look at how the
variable moved over time or when you wish to compare variables versus
each other. Grouped or stacked bar charts help you combine both these
purposes in one chart while keeping your visualization simple and
intuitive. The chart plots the value vertically whereas we perceive the
value to be at right angles to the general direction of the chart. For
instance, in the figure below, a bar graph would be a cleaner
alternative. \includegraphics{images/aya-stacked-perception.jpg}
(Source: \citep{aya-time-series})

For instance, this grouped bar chart in this interactive visualization
of a number of deaths by disease type in India not only lets you compare
the deaths due to diarrhea, malaria, and acute respiratory disease
across time, but also lets you compare the number of deaths by these
three diseases in a given year. By switching to the stacked bar chart
view, you get an intuitive sense of the proportion of deaths caused by
each disease. We can use two different bar charts to represent time
series data.

\subsection{Column Charts}\label{column-charts}

This should be the most popular chart type. This chart is good to do a
comparison between different values when specific values are important.
TBD

Still have hard time to choose? There are many resources on-line can
help you do the decision. For example, Dr.~Andre Abela creates a chart
selection diagram that is helpful to pick the right chart depends on the
data type. The link of website is{]}**

\begin{figure}
\centering
\includegraphics{images/aya-bar1.png}
\caption{}
\end{figure}

(Source: \citep{aya-time-series}) \includegraphics{images/aya-bar2.png}

(Source: \citep{aya-time-series})

To avoid clutter and confusion, make sure to not use more than 3
variables in a stacked or group bar chart. It is also a good practice to
use consistent bold colors and leave appropriate space between two bars
in a bar chart. Also, check out our blog on 5 common mistakes that lead
to bad data visualization to learn why the base axis for your bar charts
should start from zero.

\subsection{Gantt Chart}\label{gantt-chart}

\textbf{Gantt charts are a popular project management tool since they
present a concise snapshot of various tasks spread across various phases
of the project.} You can show additional information such as the
correlation between individual tasks, resources used in each task,
overlapping resources, etc., by the use of colors and placement of bars
in a Gantt chart.

is a horizontal bar chart showing work completed in a certain period of
time with respect to the time allocated for that particular task. It is
named after the American engineer and management consultant Henry Gantt
who extensively used this framework for project management. Assume
you're planning the logistics for a dance concert. There are lots of
activities to be completed, some of which will take place simultaneously
while some can be done only after another activity has been completed.
For instance, the choreographers, soundtrack, and dancers need to be
finalized before the choreography can begin. However, the costumes,
props, and stage decor can be planned at the same time as the
choreography. With careful preparation, Gantt charts can help you plan
for complex, long-term projects that are likely to undergo several
revisions and have various resource and task dependencies. Gantt charts
are a popular project management tool since they present a concise
snapshot of various tasks spread across various phases of the project.
You can show additional information such as the correlation between
individual tasks, resources used in each task, overlapping resources,
etc., by the use of colors and placement of bars in a Gantt chart.

\begin{figure}
\centering
\includegraphics{images/aya-gantt.png}
\caption{}
\end{figure}

\subsection{Stream Graph}\label{stream-graph}

\textbf{Stream graphs are great to represent and compare time series
data for multiple variables.} Stream graphs are, thus, apt for large
data sets. Remember that choice of colors is very important, especially
when there are lots of variables. Variables that do not have
significantly high values might tend to get drowned out in the
visualization if the colors are not chosen well.

(Source: \citep{aya-time-series})is essentially a stacked area graph,
but displaced around a central horizontal axis. The stream graph looks
like flowing liquid, hence the name. They are great to represent and
compare time series data for multiple variables. Stream graphs are,
thus, apt for large data sets. Remember that choice of colors is very
important, especially when there are lots of variables. Variables that
do not have significantly high values might tend to get drowned out in
the visualization if the colors are not chosen well.

A stream graph showing a randomly chosen listener's last.fm
music-listening habits over time.

\subsection{Heat Map}\label{heat-map}

\textbf{Heat maps are perfect for a two-tiered time frame} -- for
instance, 7 days of the week spread across 52 weeks in the year, or 24
hours in a day spread across 30 days of the month, and so on. The
limitation, though, is that only one variable can be visualized in a
heat map. Comparison between two or more variables is very difficult to
represent. \includegraphics{images/aya-stream.png} (Source:
\citep{aya-time-series})

Geo-spatial visualizations often use heat maps since they quickly help
identify ``Hot spots'' or regions of high concentrations of a given
variable. When adapted to temporal visualizations, heat maps can help us
explore two levels of time in a 2D array. Heat maps are perfect for a
two-tiered time frame -- for instance, 7 days of the week spread across
52 weeks in the year, or 24 hours in a day spread across 30 days of the
month, and so on. The limitation, though, is that only one variable can
be visualized in a heat map. Comparison between two or more variables is
very difficult to represent.

This heat map visualizes birthdays for babies born in the United States
between 1973 and 1999. The vertical axis represents the 31 days in a
month while the horizontal axis represents the 12 months in a year. This
chart quickly helps us identify that a large number of babies were born
in the later half of July, August, and September.

\includegraphics{images/aya-heat-map.png} (Source:
\citep{aya-time-series})

\subsection{Polar Area Diagram}\label{polar-area-diagram}

Think beyond the straight line! Sometimes, time series data can be
cyclical -- a season in a year, time of the day, and so on. Polar area
diagrams help represent the cyclical nature time series data cleanly. A
polar diagram looks like a traditional pie chart, but the sectors differ
from each other not by the size of their angles but by how far they
extend out from the center of the circle.

\textbf{Polar area diagrams are useful for representing seasonal or
cyclical time series data, such as climate or seasonal crop data.
Multiple variables can be neatly stacked in the various sectors of the
pie.}

It is crucial to clarify whether the variable is proportional to the
area or radius of the sector. It is a good practice to have the area of
the sectors proportional to the value being represented. In that case,
the radius should be proportional to the square root of the value of the
variable (since area of a circle is proportional to the square of the
radius).

This popular polar area diagram created by Florence Nightingale shows
causes of mortality among British troops in the Crimean War. Each color
in the diagram represents a different cause of death. (Check out the the
text legend for more details.)

\includegraphics{images/aya-polar.jpg} \citep{aya-time-series}

\citep{TimeSeries} This article explains how time series data
visualization can sometimes be deceptive.

It first takes an example of two random time series data and plots them
on a graph which gives an impression that the two are strongly
correlated. But if we do some statistical testing the two do not show
any relationship, this is an example of \textbf{``correlation does not
necessary mean causation''}.

In another set of examples author has taken trending two random time
series data and shown how even statistical tests can give a wrong
interpretation. The article then explains using visualization how a
general trended time series can be different than a more controlled and
measured trending time series.

\section{5 Tips to improve Data
Visualization}\label{tips-to-improve-data-visualization-1}

\subsection{Comparison}\label{comparison-1}

\textbf{Include a zero baseline if possible.} Although a line chart does
not have to start at a zero baseline, it should be included if it gives
more context for comparison. If relatively small fluctuations in data
are meaningful (e.g., in stock market data), you may truncate the scale
to showcase these variances; Always choose the most efficient
visualization; Watch your placement You may have two nice stacked bar
charts that are meant to let your reader compare points, but if they're
placed too far apart to ``get'' the comparison, you've already lost;
Tell the whole story. Maybe you had a 30\% sales increase in Q4.
Exciting! But what's more exciting? Showing that you've actually had a
100\% sales increase since Q1.

\subsection{Copy}\label{copy-1}

\textbf{Don't over explain.} If the copy already mentions a fact, the
subhead, callout, and chart header don't have to reiterate it; Keep
chart and graph headers simple and to the point There's no need to get
clever, verbose, or pun-tastic. Keep any descriptive text above the
chart brief and directly related to the chart underneath.
\textbf{Remember: Focus on the quickest path to comprehension; Use
callouts wisely Callouts are not there to fill space.} They should be
used intentionally to highlight relevant information or provide
additional context; Don't use distracting fonts or elements Sometimes
you do need to emphasize a point. If so, only use bold or italic text to
emphasize a point---and don't use them both at the same time.

\subsection{Color}\label{color-1}

\textbf{Use a single color to represent the same type of data; Watch out
for positive and negative numbers.} Don't use red for positive numbers
or green for negative numbers. Those color associations are so strong it
will automatically flip the meaning in the viewer's mind; Make sure
there is sufficient contrast between colors; Avoid patterns Stripes and
polka dots sound fun, but they can be incredibly distracting. If you are
trying to differentiate, say, on a map, use different saturation of the
same color. On that note, only use solid-colored lines (not dashes);
Select colors appropriately; Don't use more than 6 colors in a single
layout.

\subsection{Ordering}\label{ordering-1}

\textbf{Order data intuitively should have a logical hierarchy.} Order
categories alphabetically, sequentially, or by value; Order
consistently; Order evenly Use natural increments on your axes (0, 5,
10, 15, 20) instead of awkward or uneven increments (0, 3, 5, 16, 50).

\subsection{Audience perspective}\label{audience-perspective-1}

\textbf{Let the users lead; Know your audience.} Designers should
consider the way users prefer to understand information, even in
choosing basic analytic approaches. For users to feel comfortable
adopting and sharing insights from analytics, they must be able to
explain and defend the data.

\subsection{Use layers to tell a
story}\label{use-layers-to-tell-a-story-1}

\textbf{While style is one form of customization, layering unique data
sets on a single visualization can tell a richer narrative and connect
users to the data without getting too crowded.} On a map, this can be as
simple as zooming in and out, but it can also involve drill-downs
(choosing a data point and expanding it to show more detail), links and
other shortcuts.

\subsection{Keep it simple}\label{keep-it-simple-2}

\textbf{Keep it as simple as possible!}Analytic results shouldn't be
presented to 10 decimal places when the user doesn't need that level of
precision to make a decision or understand a concept. Effective visual
interfaces avoid 3-D effects or ornate gauge designs (a.k.a. ``chart
junk'') when simple numbers, maps or graphs will do.

+reference: \citep{French} +reference: \citep{Steier}

\section{Word Cloud}\label{word-cloud}

A Word Cloud or Tag Cloud is a visual representation of text data in the
form of tags, which are typically single words whose importance is
visualized by way of their size and color. It displays how frequently
words appear in a given body of text, by making the size of each word
proportional to its frequency.

Word clouds can add clarity during text analysis in order to effectively
communicate your data results.It is an effective tool for Q researchers,
marketers, Non-profits, Human resources ,Educators, Politicians and
journalists.

** Pros of Word Clouds **

\begin{enumerate}
\def\labelenumi{\arabic{enumi}.}
\tightlist
\item
  It is easy to understand and make an impact.
\item
  It can easily be shared.
\item
  It is visually engaging than a table data.
\item
  It is fast and reveals the essential.
\item
  They delight and provide emotional connection.
\end{enumerate}

** Cons of Word Clouds **

\begin{enumerate}
\def\labelenumi{\arabic{enumi}.}
\tightlist
\item
  Emphasis based on length of the words.
\item
  Words whose letters contain many ascenders and descenders may receive
  more attention.
\item
  They're not very accurate.
\item
  A lot of data cleaning required before generating the word cloud.
\item
  Context is lost.
\end{enumerate}

\citep{wordcloud}

** Ways of generating a word cloud **

\textbf{R:} The procedure of creating word clouds is very simple in R
with text mining package (TM) and the word cloud generator package. The
major steps involved are: text mining which involves text cleaning and
transformation, building term-document matrix and generating word cloud
\citep{r}.

\textbf{Wordle:} Wordle is a toy for generating ``word clouds'' from
text that you provide.It is very popular, free and easy to use. You do
need Java though Chrome. In Wordle, you generate word clouds from text
you input. Clouds can be tweaked with different color schemes, layouts,
and fonts. Images created from this tool can be saved and reused
\citep{wordle}.

Other popular tools include ABCya, Tagul, Tag Crowd, CloudArt.

\subsection{Calendar View}\label{calendar-view}

(Reproducible code for reference)

\citep{Calendar_Layout}

We have all seen the calendar views in the various data products that we
worked on. Please find an open source code that I found, this will help
you replicate and create your own calendar:
\includegraphics{images/CalendarView.jpg}

\citep{CalendarView}

This example demonstrates loading of CSV data, which is then quantized
into a diverging color scale. The values are visualized as colored cells
per day. Days are arranged into columns by week, then grouped by month
and years.

\begin{Shaded}
\begin{Highlighting}[]
\DataTypeTok{<!DOCTYPE }\NormalTok{html}\DataTypeTok{>}
\KeywordTok{<body>}
\KeywordTok{<script}\OtherTok{ src=}\StringTok{"https://d3js.org/d3.v4.min.js"}\KeywordTok{></script>}
\KeywordTok{<script>}

\KeywordTok{var}\NormalTok{ width }\OperatorTok{=} \DecValTok{960}\OperatorTok{,}
\NormalTok{    height }\OperatorTok{=} \DecValTok{136}\OperatorTok{,}
\NormalTok{    cellSize }\OperatorTok{=} \DecValTok{17}\OperatorTok{;}

\KeywordTok{var}\NormalTok{ formatPercent }\OperatorTok{=} \VariableTok{d3}\NormalTok{.}\AttributeTok{format}\NormalTok{(}\StringTok{".1%"}\NormalTok{)}\OperatorTok{;}

\KeywordTok{var}\NormalTok{ color }\OperatorTok{=} \VariableTok{d3}\NormalTok{.}\AttributeTok{scaleQuantize}\NormalTok{()}
\NormalTok{    .}\AttributeTok{domain}\NormalTok{([}\OperatorTok{-}\FloatTok{0.05}\OperatorTok{,} \FloatTok{0.05}\NormalTok{])}
\NormalTok{    .}\AttributeTok{range}\NormalTok{([}\StringTok{"#a50026"}\OperatorTok{,} \StringTok{"#d73027"}\OperatorTok{,} \StringTok{"#f46d43"}\OperatorTok{,} \StringTok{"#fdae61"}\OperatorTok{,} \StringTok{"#fee08b"}\OperatorTok{,} \StringTok{"#ffffbf"}\OperatorTok{,} \StringTok{"#d9ef8b"}\OperatorTok{,} \StringTok{"#a6d96a"}\OperatorTok{,} \StringTok{"#66bd63"}\OperatorTok{,} \StringTok{"#1a9850"}\OperatorTok{,} \StringTok{"#006837"}\NormalTok{])}\OperatorTok{;}

\KeywordTok{var}\NormalTok{ svg }\OperatorTok{=} \VariableTok{d3}\NormalTok{.}\AttributeTok{select}\NormalTok{(}\StringTok{"body"}\NormalTok{)}
\NormalTok{  .}\AttributeTok{selectAll}\NormalTok{(}\StringTok{"svg"}\NormalTok{)}
\NormalTok{  .}\AttributeTok{data}\NormalTok{(}\VariableTok{d3}\NormalTok{.}\AttributeTok{range}\NormalTok{(}\DecValTok{1990}\OperatorTok{,} \DecValTok{2011}\NormalTok{))}
\NormalTok{  .}\AttributeTok{enter}\NormalTok{().}\AttributeTok{append}\NormalTok{(}\StringTok{"svg"}\NormalTok{)}
\NormalTok{    .}\AttributeTok{attr}\NormalTok{(}\StringTok{"width"}\OperatorTok{,}\NormalTok{ width)}
\NormalTok{    .}\AttributeTok{attr}\NormalTok{(}\StringTok{"height"}\OperatorTok{,}\NormalTok{ height)}
\NormalTok{  .}\AttributeTok{append}\NormalTok{(}\StringTok{"g"}\NormalTok{)}
\NormalTok{    .}\AttributeTok{attr}\NormalTok{(}\StringTok{"transform"}\OperatorTok{,} \StringTok{"translate("} \OperatorTok{+}\NormalTok{ ((width }\OperatorTok{-}\NormalTok{ cellSize }\OperatorTok{*} \DecValTok{53}\NormalTok{) / }\DecValTok{2}\NormalTok{) }\OperatorTok{+} \StringTok{","} \OperatorTok{+}\NormalTok{ (height }\OperatorTok{-}\NormalTok{ cellSize }\OperatorTok{*} \DecValTok{7} \OperatorTok{-} \DecValTok{1}\NormalTok{) }\OperatorTok{+} \StringTok{")"}\NormalTok{)}\OperatorTok{;}

\VariableTok{svg}\NormalTok{.}\AttributeTok{append}\NormalTok{(}\StringTok{"text"}\NormalTok{)}
\NormalTok{    .}\AttributeTok{attr}\NormalTok{(}\StringTok{"transform"}\OperatorTok{,} \StringTok{"translate(-6,"} \OperatorTok{+}\NormalTok{ cellSize }\OperatorTok{*} \FloatTok{3.5} \OperatorTok{+} \StringTok{")rotate(-90)"}\NormalTok{)}
\NormalTok{    .}\AttributeTok{attr}\NormalTok{(}\StringTok{"font-family"}\OperatorTok{,} \StringTok{"sans-serif"}\NormalTok{)}
\NormalTok{    .}\AttributeTok{attr}\NormalTok{(}\StringTok{"font-size"}\OperatorTok{,} \DecValTok{10}\NormalTok{)}
\NormalTok{    .}\AttributeTok{attr}\NormalTok{(}\StringTok{"text-anchor"}\OperatorTok{,} \StringTok{"middle"}\NormalTok{)}
\NormalTok{    .}\AttributeTok{text}\NormalTok{(}\KeywordTok{function}\NormalTok{(d) }\OperatorTok{\{} \ControlFlowTok{return}\NormalTok{ d}\OperatorTok{;} \OperatorTok{\}}\NormalTok{)}\OperatorTok{;}

\KeywordTok{var}\NormalTok{ rect }\OperatorTok{=} \VariableTok{svg}\NormalTok{.}\AttributeTok{append}\NormalTok{(}\StringTok{"g"}\NormalTok{)}
\NormalTok{    .}\AttributeTok{attr}\NormalTok{(}\StringTok{"fill"}\OperatorTok{,} \StringTok{"none"}\NormalTok{)}
\NormalTok{    .}\AttributeTok{attr}\NormalTok{(}\StringTok{"stroke"}\OperatorTok{,} \StringTok{"#ccc"}\NormalTok{)}
\NormalTok{  .}\AttributeTok{selectAll}\NormalTok{(}\StringTok{"rect"}\NormalTok{)}
\NormalTok{  .}\AttributeTok{data}\NormalTok{(}\KeywordTok{function}\NormalTok{(d) }\OperatorTok{\{} \ControlFlowTok{return} \VariableTok{d3}\NormalTok{.}\AttributeTok{timeDays}\NormalTok{(}\KeywordTok{new} \AttributeTok{Date}\NormalTok{(d}\OperatorTok{,} \DecValTok{0}\OperatorTok{,} \DecValTok{1}\NormalTok{)}\OperatorTok{,} \KeywordTok{new} \AttributeTok{Date}\NormalTok{(d }\OperatorTok{+} \DecValTok{1}\OperatorTok{,} \DecValTok{0}\OperatorTok{,} \DecValTok{1}\NormalTok{))}\OperatorTok{;} \OperatorTok{\}}\NormalTok{)}
\NormalTok{  .}\AttributeTok{enter}\NormalTok{().}\AttributeTok{append}\NormalTok{(}\StringTok{"rect"}\NormalTok{)}
\NormalTok{    .}\AttributeTok{attr}\NormalTok{(}\StringTok{"width"}\OperatorTok{,}\NormalTok{ cellSize)}
\NormalTok{    .}\AttributeTok{attr}\NormalTok{(}\StringTok{"height"}\OperatorTok{,}\NormalTok{ cellSize)}
\NormalTok{    .}\AttributeTok{attr}\NormalTok{(}\StringTok{"x"}\OperatorTok{,} \KeywordTok{function}\NormalTok{(d) }\OperatorTok{\{} \ControlFlowTok{return} \VariableTok{d3}\NormalTok{.}\VariableTok{timeWeek}\NormalTok{.}\AttributeTok{count}\NormalTok{(}\VariableTok{d3}\NormalTok{.}\AttributeTok{timeYear}\NormalTok{(d)}\OperatorTok{,}\NormalTok{ d) }\OperatorTok{*}\NormalTok{ cellSize}\OperatorTok{;} \OperatorTok{\}}\NormalTok{)}
\NormalTok{    .}\AttributeTok{attr}\NormalTok{(}\StringTok{"y"}\OperatorTok{,} \KeywordTok{function}\NormalTok{(d) }\OperatorTok{\{} \ControlFlowTok{return} \VariableTok{d}\NormalTok{.}\AttributeTok{getDay}\NormalTok{() }\OperatorTok{*}\NormalTok{ cellSize}\OperatorTok{;} \OperatorTok{\}}\NormalTok{)}
\NormalTok{    .}\AttributeTok{datum}\NormalTok{(}\VariableTok{d3}\NormalTok{.}\AttributeTok{timeFormat}\NormalTok{(}\StringTok{"%Y-%m-%d"}\NormalTok{))}\OperatorTok{;}

\VariableTok{svg}\NormalTok{.}\AttributeTok{append}\NormalTok{(}\StringTok{"g"}\NormalTok{)}
\NormalTok{    .}\AttributeTok{attr}\NormalTok{(}\StringTok{"fill"}\OperatorTok{,} \StringTok{"none"}\NormalTok{)}
\NormalTok{    .}\AttributeTok{attr}\NormalTok{(}\StringTok{"stroke"}\OperatorTok{,} \StringTok{"#000"}\NormalTok{)}
\NormalTok{  .}\AttributeTok{selectAll}\NormalTok{(}\StringTok{"path"}\NormalTok{)}
\NormalTok{  .}\AttributeTok{data}\NormalTok{(}\KeywordTok{function}\NormalTok{(d) }\OperatorTok{\{} \ControlFlowTok{return} \VariableTok{d3}\NormalTok{.}\AttributeTok{timeMonths}\NormalTok{(}\KeywordTok{new} \AttributeTok{Date}\NormalTok{(d}\OperatorTok{,} \DecValTok{0}\OperatorTok{,} \DecValTok{1}\NormalTok{)}\OperatorTok{,} \KeywordTok{new} \AttributeTok{Date}\NormalTok{(d }\OperatorTok{+} \DecValTok{1}\OperatorTok{,} \DecValTok{0}\OperatorTok{,} \DecValTok{1}\NormalTok{))}\OperatorTok{;} \OperatorTok{\}}\NormalTok{)}
\NormalTok{  .}\AttributeTok{enter}\NormalTok{().}\AttributeTok{append}\NormalTok{(}\StringTok{"path"}\NormalTok{)}
\NormalTok{    .}\AttributeTok{attr}\NormalTok{(}\StringTok{"d"}\OperatorTok{,}\NormalTok{ pathMonth)}\OperatorTok{;}

\VariableTok{d3}\NormalTok{.}\AttributeTok{csv}\NormalTok{(}\StringTok{"dji.csv"}\OperatorTok{,} \KeywordTok{function}\NormalTok{(error}\OperatorTok{,}\NormalTok{ csv) }\OperatorTok{\{}
  \ControlFlowTok{if}\NormalTok{ (error) }\ControlFlowTok{throw}\NormalTok{ error}\OperatorTok{;}

  \KeywordTok{var}\NormalTok{ data }\OperatorTok{=} \VariableTok{d3}\NormalTok{.}\AttributeTok{nest}\NormalTok{()}
\NormalTok{      .}\AttributeTok{key}\NormalTok{(}\KeywordTok{function}\NormalTok{(d) }\OperatorTok{\{} \ControlFlowTok{return} \VariableTok{d}\NormalTok{.}\AttributeTok{Date}\OperatorTok{;} \OperatorTok{\}}\NormalTok{)}
\NormalTok{      .}\AttributeTok{rollup}\NormalTok{(}\KeywordTok{function}\NormalTok{(d) }\OperatorTok{\{} \ControlFlowTok{return}\NormalTok{ (d[}\DecValTok{0}\NormalTok{].}\AttributeTok{Close} \OperatorTok{-}\NormalTok{ d[}\DecValTok{0}\NormalTok{].}\AttributeTok{Open}\NormalTok{) / d[}\DecValTok{0}\NormalTok{].}\AttributeTok{Open}\OperatorTok{;} \OperatorTok{\}}\NormalTok{)}
\NormalTok{    .}\AttributeTok{object}\NormalTok{(csv)}\OperatorTok{;}

  \VariableTok{rect}\NormalTok{.}\AttributeTok{filter}\NormalTok{(}\KeywordTok{function}\NormalTok{(d) }\OperatorTok{\{} \ControlFlowTok{return}\NormalTok{ d }\KeywordTok{in}\NormalTok{ data}\OperatorTok{;} \OperatorTok{\}}\NormalTok{)}
\NormalTok{      .}\AttributeTok{attr}\NormalTok{(}\StringTok{"fill"}\OperatorTok{,} \KeywordTok{function}\NormalTok{(d) }\OperatorTok{\{} \ControlFlowTok{return} \AttributeTok{color}\NormalTok{(data[d])}\OperatorTok{;} \OperatorTok{\}}\NormalTok{)}
\NormalTok{    .}\AttributeTok{append}\NormalTok{(}\StringTok{"title"}\NormalTok{)}
\NormalTok{      .}\AttributeTok{text}\NormalTok{(}\KeywordTok{function}\NormalTok{(d) }\OperatorTok{\{} \ControlFlowTok{return}\NormalTok{ d }\OperatorTok{+} \StringTok{": "} \OperatorTok{+} \AttributeTok{formatPercent}\NormalTok{(data[d])}\OperatorTok{;} \OperatorTok{\}}\NormalTok{)}\OperatorTok{;}
\OperatorTok{\}}\NormalTok{)}\OperatorTok{;}

\KeywordTok{function} \AttributeTok{pathMonth}\NormalTok{(t0) }\OperatorTok{\{}
  \KeywordTok{var}\NormalTok{ t1 }\OperatorTok{=} \KeywordTok{new} \AttributeTok{Date}\NormalTok{(}\VariableTok{t0}\NormalTok{.}\AttributeTok{getFullYear}\NormalTok{()}\OperatorTok{,} \VariableTok{t0}\NormalTok{.}\AttributeTok{getMonth}\NormalTok{() }\OperatorTok{+} \DecValTok{1}\OperatorTok{,} \DecValTok{0}\NormalTok{)}\OperatorTok{,}
\NormalTok{      d0 }\OperatorTok{=} \VariableTok{t0}\NormalTok{.}\AttributeTok{getDay}\NormalTok{()}\OperatorTok{,}\NormalTok{ w0 }\OperatorTok{=} \VariableTok{d3}\NormalTok{.}\VariableTok{timeWeek}\NormalTok{.}\AttributeTok{count}\NormalTok{(}\VariableTok{d3}\NormalTok{.}\AttributeTok{timeYear}\NormalTok{(t0)}\OperatorTok{,}\NormalTok{ t0)}\OperatorTok{,}
\NormalTok{      d1 }\OperatorTok{=} \VariableTok{t1}\NormalTok{.}\AttributeTok{getDay}\NormalTok{()}\OperatorTok{,}\NormalTok{ w1 }\OperatorTok{=} \VariableTok{d3}\NormalTok{.}\VariableTok{timeWeek}\NormalTok{.}\AttributeTok{count}\NormalTok{(}\VariableTok{d3}\NormalTok{.}\AttributeTok{timeYear}\NormalTok{(t1)}\OperatorTok{,}\NormalTok{ t1)}\OperatorTok{;}
  \ControlFlowTok{return} \StringTok{"M"} \OperatorTok{+}\NormalTok{ (w0 }\OperatorTok{+} \DecValTok{1}\NormalTok{) }\OperatorTok{*}\NormalTok{ cellSize }\OperatorTok{+} \StringTok{","} \OperatorTok{+}\NormalTok{ d0 }\OperatorTok{*}\NormalTok{ cellSize}
      \OperatorTok{+} \StringTok{"H"} \OperatorTok{+}\NormalTok{ w0 }\OperatorTok{*}\NormalTok{ cellSize }\OperatorTok{+} \StringTok{"V"} \OperatorTok{+} \DecValTok{7} \OperatorTok{*}\NormalTok{ cellSize}
      \OperatorTok{+} \StringTok{"H"} \OperatorTok{+}\NormalTok{ w1 }\OperatorTok{*}\NormalTok{ cellSize }\OperatorTok{+} \StringTok{"V"} \OperatorTok{+}\NormalTok{ (d1 }\OperatorTok{+} \DecValTok{1}\NormalTok{) }\OperatorTok{*}\NormalTok{ cellSize}
      \OperatorTok{+} \StringTok{"H"} \OperatorTok{+}\NormalTok{ (w1 }\OperatorTok{+} \DecValTok{1}\NormalTok{) }\OperatorTok{*}\NormalTok{ cellSize }\OperatorTok{+} \StringTok{"V"} \OperatorTok{+} \DecValTok{0}
      \OperatorTok{+} \StringTok{"H"} \OperatorTok{+}\NormalTok{ (w0 }\OperatorTok{+} \DecValTok{1}\NormalTok{) }\OperatorTok{*}\NormalTok{ cellSize }\OperatorTok{+} \StringTok{"Z"}\OperatorTok{;}
\OperatorTok{\}}
\KeywordTok{</script>}
\end{Highlighting}
\end{Shaded}

\section{\texorpdfstring{An example to back some of our theories on `how
to tell stories using data visualization' / `exploratory data
visualization'}{An example to back some of our theories on how to tell stories using data visualization / exploratory data visualization}}\label{an-example-to-back-some-of-our-theories-on-how-to-tell-stories-using-data-visualization-exploratory-data-visualization}

\citep{DataUSA} MIT Media Lab in collaboration with Deloitte has created
a new visualization tool, that aggregates US government open source data
from various sources and mines information to generate trends and
stories about cities, jobs, industries etc. to the common man. Just
looking at any of the open data sources would give us an idea about the
vastness (breadth and depth) of the available data. It is amazing to see
how they have brought it all together on a single platform in a very
easy to decipher format. What caught my attention here is the
categorization of Information on the website that enables the following:

\begin{enumerate}
\def\labelenumi{\arabic{enumi}.}
\tightlist
\item
  Easy browsing of various categories of information available at a
  single glance
\item
  An Easy search on any topic of interest and get deeper information on
  each
\item
  Logical construction of information using data and visuals under each
  category
\item
  Comparative Analysis of cities
\item
  Variety of exploratory visualizations to learn from
\item
  Most important - Stories that these data tell, e.g., Evolution of the
  American Worker, Poverty is bad for your health, Men still dominate in
  the highest paying industries, Opioid addiction damage and so many
  others.
\end{enumerate}

We think of a topic, and its possible it's there! Value add to students,
organizations, governments etc. is better understanding of your
consumers, talent pool, jobs, climate, and what not, that just improves
our decision-making ability manifold by spending just a few seconds on
the website. And for this class, the best part is that the data is also
available for download. So, we can easily download this data, replicate
the visuals and try to redesign and tell our own stories with this data.
There is also other similar websites, that has some good visualizations
on census data: \citep{CensusDataViz}

\textbf{Automatic visualization is a bad idea, generally speaking.} Some
(many?) might argue that automated visualization is a worthwhile
pursuit. And I would agree that some parts of visualization certainly
should be automatic, such as standard chart types and recurring
geometries. Pieces of visualization, such as annotation and axis
construction can be automatic. There are plenty of tools to make our
lives easier.

But full-on automation where insight fountains out from any dataset are
farfetched at this point, because this requires automatic analysis. The
analysis is context-specific and requires more than boilerplate
statistics. The most interesting visualization is context-specific.

In 2016, Arden Manning believed Workplace Automation making Data
Visualization Smarter. According to him, the goal of data visualization
tools was to make understanding data easier, but more often than not it
doesn't quite go to plan. We've all been there. The software is able to
analyze huge amounts of data and incredible speeds, but how can it
explain the results of that analysis? Today, the only means of doing
this is with graphics. However, data visualization can't explain data,
leaving room for interpretation. The thought behind graphics makes sense
-- turn data into something easy to understand -- but the reality is
more complex, and we are left working late writing reports explaining
how many trades were canceled and by what desks or why sales fell in
August of 2015. Whenever I am stuck doing a repetitive task, I always
think `why can't we automate this'? And now, finally, technology has
caught up. Narrative generation software can run as a plug-in to your
dashboards. Tools like Savvy actually install on your server and allow
every dashboard user to get a written summary anytime they want. This
software is fully plugged and play, it takes seconds to install, and
it's easy to use. Today, it runs with Microsoft Excel, Qlik Sense 3.0,
and is available as an API. In fact, the software even lets you copy and
paste the text it generates so that you can use it in emails and reports
explaining data. Yet again automation is making our working lives easier
by automating repetitive tasks and allowing us to fully leverage
existing data reserves.

According to Alysson Ferreira, a UI Engineer, he published his idea of
Data Visualization tools in 2017. In this new era of information,
there's an increasing need to understand the latest trends quickly and
efficiently, which means there's also a need for meaningful sources of
trustworthy information.

This is where data visualization comes in. Data visualization is the art
of displaying information by combining the beauty of imagery with the
conciseness of statistics, which allows us to organize complex data into
convenient graphical representations. In simple terms, data
visualization is the art of translating complex data into meaningful
information.

We can also find the topic of data visualization's future on Quora.
Amalie Sharma thought in future trend of data visualization are better
tools, open for all, Increase in Interactivity/Animation and Portable
Data.

Plug in any data set into a magic box and it spits out a lovely
visualization you can show all of your co-workers, friends, and family.
That's the promise of a lot of startups, but it doesn't quite work that
way. The goal of data visualization tools was to make understanding data
easier, but more often than not it doesn't quite go to plan. The problem
is that graphics alone don't fully explain data, and so we are inundated
with queries: why did the numbers fall in whatever month? Data
visualization can't explain data, leaving room for interpretation.

Although simple visualizations such as standard chart types (bar chart,
line chart etc.) are already automated to a certain extend in Microsoft
Office tools and other software available in the market, but full on
automation where insight fountains out from any data set is far-fetched
at this point, because this requires automatic analysis. The automated
analysis here means that the tool or algorithm has to understand the
context and also select the best visualization.

The focus in today's world has been on open source tools and
technologies and these tools although being free for the most part
require more effort to seamlessly integrate to the current visualization
workflow. As mentioned in one of the articles about D3.js:

D3.js is one of the first data visualization tools that comes to mind
when talking about free, open-source alternatives. It's a
JavaScript-based library for creating web visualization and displays the
results on the web page. However, with great power comes great
responsibility. D3.js is extremely powerful and flexible, because it
allows you to build amazing things with it, but as a trade-off, it's not
the easiest tool to use, so you might need to spend some time going
through the helpful library documentation.

At the end, it's not only about the tool its more about what you are
trying to do; what your professor, client, business or whatever needs.

\section{Reusable Data Visualization Code in
R}\label{reusable-data-visualization-code-in-r}

\citep{viz_R}

This site includes full sets of R code to generate specific types of
graphs in ggplot2. Plots in ggplot2 are created by using ``layering''.
There is a base plot and then other aspects of the plot such as
aesthetics, titles ,and labels are added to using extra code. For those
who favor Python for data viz, this layering approach in R is actually
similar to the syntax in Python's matplotlib library, in which
set\_style and specifying the axes labels and title are done separately
from the code that generates the plot itself.

To provide an example the ``layering'' mentioned above, here is a
generic snippet of code for creating a scatterplot with ggplot2 and the
mtcars data set in R base, using this website's code as a template:

\begin{Shaded}
\begin{Highlighting}[]
\KeywordTok{library}\NormalTok{(ggplot2)}

\KeywordTok{theme_set}\NormalTok{(}\KeywordTok{theme_bw}\NormalTok{())  }\CommentTok{#set background theme}

\NormalTok{plot1 <-}\StringTok{ }\KeywordTok{ggplot}\NormalTok{(mtcars, }\KeywordTok{aes}\NormalTok{(}\DataTypeTok{x =}\NormalTok{ hp, }\DataTypeTok{y =}\NormalTok{ mpg)) }\OperatorTok{+}\StringTok{ }\KeywordTok{geom_point}\NormalTok{(}\KeywordTok{aes}\NormalTok{(}\DataTypeTok{col=}\KeywordTok{factor}\NormalTok{(vs), }\DataTypeTok{size =} \DecValTok{2}\NormalTok{)) }\OperatorTok{+}\StringTok{ }\KeywordTok{geom_smooth}\NormalTok{(}\DataTypeTok{method =} \StringTok{"loess"}\NormalTok{, }\DataTypeTok{se =}\NormalTok{ F) }\OperatorTok{+}\StringTok{ }\KeywordTok{xlim}\NormalTok{(}\KeywordTok{c}\NormalTok{(}\DecValTok{0}\NormalTok{, }\DecValTok{400}\NormalTok{)) }\OperatorTok{+}\StringTok{ }\KeywordTok{ylim}\NormalTok{(}\KeywordTok{c}\NormalTok{(}\DecValTok{0}\NormalTok{, }\DecValTok{40}\NormalTok{)) }\OperatorTok{+}\StringTok{ }\KeywordTok{labs}\NormalTok{(}\DataTypeTok{title =} \StringTok{"Horsepower vs. MPG"}\NormalTok{, }\DataTypeTok{y =} \StringTok{"Miles Per Gallon"}\NormalTok{, }\DataTypeTok{x =} \StringTok{"Horsepower"}\NormalTok{)}

\KeywordTok{plot}\NormalTok{(plot1)  }\CommentTok{#we have to actually call the plot() function on the plot object we created}
\end{Highlighting}
\end{Shaded}

\includegraphics{Data_Viz_Reader_files/figure-latex/unnamed-chunk-8-1.pdf}
The ggplot2 package allows R users to go beyond the simple and often
rudimentary-looking graphs in R and offers many ways of customizing data
visualizations. In a way, the layering technique also makes it easier to
remember the code to generate these plots, since geom functions for the
layers remain constant and they are all included in a single line of
code.

\section{Data Mining and Data
Visualization}\label{data-mining-and-data-visualization}

According to a paper in 2018, we can tell the difference of data mining
from data visualizations. Here is a chart that helps us understand this
better.

\begin{longtable}[]{@{}lll@{}}
\toprule
\begin{minipage}[b]{0.08\columnwidth}\raggedright\strut
\textbf{BASIS FOR COMPARISON}\strut
\end{minipage} & \begin{minipage}[b]{0.41\columnwidth}\raggedright\strut
\textbf{Data Mining}\strut
\end{minipage} & \begin{minipage}[b]{0.41\columnwidth}\raggedright\strut
\textbf{Data Visualization}\strut
\end{minipage}\tabularnewline
\midrule
\endhead
\begin{minipage}[t]{0.08\columnwidth}\raggedright\strut
Definition\strut
\end{minipage} & \begin{minipage}[t]{0.41\columnwidth}\raggedright\strut
Searches and produces a suitable result from large data chunks\strut
\end{minipage} & \begin{minipage}[t]{0.41\columnwidth}\raggedright\strut
Gives a simple overview of complex data\strut
\end{minipage}\tabularnewline
\begin{minipage}[t]{0.08\columnwidth}\raggedright\strut
Preference\strut
\end{minipage} & \begin{minipage}[t]{0.41\columnwidth}\raggedright\strut
This is has different applications and preferred for web search
engines\strut
\end{minipage} & \begin{minipage}[t]{0.41\columnwidth}\raggedright\strut
Preferred for data forecasting and predictions\strut
\end{minipage}\tabularnewline
\begin{minipage}[t]{0.08\columnwidth}\raggedright\strut
Area\strut
\end{minipage} & \begin{minipage}[t]{0.41\columnwidth}\raggedright\strut
Comes under data science\strut
\end{minipage} & \begin{minipage}[t]{0.41\columnwidth}\raggedright\strut
Comes under the area of data science\strut
\end{minipage}\tabularnewline
\begin{minipage}[t]{0.08\columnwidth}\raggedright\strut
Platform\strut
\end{minipage} & \begin{minipage}[t]{0.41\columnwidth}\raggedright\strut
Operated with web software systems or applications\strut
\end{minipage} & \begin{minipage}[t]{0.41\columnwidth}\raggedright\strut
Supports and works better in complex data analyses and
applications\strut
\end{minipage}\tabularnewline
\begin{minipage}[t]{0.08\columnwidth}\raggedright\strut
Generality\strut
\end{minipage} & \begin{minipage}[t]{0.41\columnwidth}\raggedright\strut
New technology but underdeveloped\strut
\end{minipage} & \begin{minipage}[t]{0.41\columnwidth}\raggedright\strut
More useful in real time data forecasting\strut
\end{minipage}\tabularnewline
\begin{minipage}[t]{0.08\columnwidth}\raggedright\strut
Algorithm\strut
\end{minipage} & \begin{minipage}[t]{0.41\columnwidth}\raggedright\strut
Many algorithms exist in using data mining\strut
\end{minipage} & \begin{minipage}[t]{0.41\columnwidth}\raggedright\strut
No need of using any algorithms\strut
\end{minipage}\tabularnewline
\begin{minipage}[t]{0.08\columnwidth}\raggedright\strut
Integration\strut
\end{minipage} & \begin{minipage}[t]{0.41\columnwidth}\raggedright\strut
Runs on any web-enabled platform or with any applications\strut
\end{minipage} & \begin{minipage}[t]{0.41\columnwidth}\raggedright\strut
Irrespective of hardware or software, it provides visual
information\strut
\end{minipage}\tabularnewline
\bottomrule
\end{longtable}

In Data Mining, there are different processes involve carrying out the
data mining process such as data extraction, data management, data
transformations, data pre-processing, etc.

In Data Visualization, the primary goal is to convey the information
efficiently and clearly without any deviations or complexities in the
form of statistical graphs, information graphs, and plots.

Also, the author listed the top 7 comparisons between data mining and
data visualization, and 12 key differences between data mining and data
visualization. After reading the article, you will have a very clear
understanding of what are data mining and data visualization and the
characters for those two techniques.

\chapter{Ethics}\label{ethics}

\section{Ethical Theory and Practice from Journalism and
Engineering}\label{ethical-theory-and-practice-from-journalism-and-engineering}

\citep{poli_social_science} Over the years, researchers and lawyers have
come up with rules and practices for proper data collection and
utilization, with particular attention on human subject research.
Consent of the subjects to use their data, evaluation of any risk with
use or collection of data, and protecting the anonymity of data are some
of the rules that must be considered for ethical research methods. Under
U.S. law, research institutions receiving federal funding must consider
ethical aspects of their research. These rules continue to evolve.

Data presented in charts can persuade viewers on the subject matter,
even if viewers do not support the idea presented. This means that
visualizations can also be used to deceive and there are many techniques
for this leading viewers to wrong conclusions. Misleading,
incomprehensible, or incredible data visualization can jeopardize
people's trust, goodwill, or faith in research and advocacy on vital
human rights issues. Its ethical responsibility to create visualizations
to give the correct and faithful representation of data and subjects.

The basic objective of data visualization is to provide an efficient
graphical display for summarizing and reasoning about quantitative
information. And during the last decades, political science has
accumulated a large corpus of various kinds of data, which gradually
become a more scientific and requires using quantitative information in
the analysis and reasoning.

Under U.S. law, research institutions receiving federal funding must
consider ethical aspects of their research. Over the years, researchers
and lawyers have established rules and practices for proper data
collection and utilization, with particular attention to human subject
research. Some of the most important of these rules for ethical research
methods include consent of the subjects to use their data, evaluation of
any risk with use or collection of data, and protecting anonymity of
data. However, these rules continue to evolve.

Data presented in charts can persuade viewers on the subject matter,
even if viewers do not support the idea presented. This means that
visualizations can also be used to present misleading arguments and
deceive viewers. Misleading, incomprehensible, or incredible data
visualization can jeopardize people's trust, goodwill, or faith in
research and advocacy on vital human rights issues. There is no shortage
of techniques for deception through data visualization, and researchers
have an ethical responsibility to give correct and faithful
representation of data and subjects.

The basic objective of data visualization is to provide an efficient
graphical display for summarizing and posing a claim about quantitative
information. However, the value of data visualization is not limited to
business and the hard sciences. During the last decades, political
science has accumulated a large corpus of various kinds of data, and has
gradually become a quantitative and scientific field that requires the
use of quantitative information in analysis and reasoning.

Data visualization plays several important roles:

Obtaining the consent of the subjects to use their data, evaluating any
risk with use or collection of data or protecting the anonymity of data
are some of the rules that must be considered for ethical research
methods. Under U.S.law for research institutions receiving federal
funding, ethical aspects must be considered. These rules continue to
evolve.

Research has found that even if viewers do not support an idea, data
presented in charts can persuade viewers on the subject matter. It means
that visualization can also be used for deception and there are lot of
techniques that can produce dangerous visualization. Techniques such as
truncated axis (where the y-axis does not start at zero) or using the
area to represent a quantity (for instance comparing the size of two
adjacent circles) were found to lead to wrong conclusions.

Misleading, incomprehensible, or incredible data visualization can
jeopardize people's trust, goodwill, or faith in research and advocacy
on vital human rights issues. Its ethical responsibility to create
visualizations to give a correct and faithful representation of data and
subjects. Data visualization plays several important roles in it:

\begin{enumerate}
\def\labelenumi{\arabic{enumi}.}
\tightlist
\item
  it helps create informative illustrations of the data, recapitulating
  a large amount of quantitative information on a diagram;
\item
  it helps formulate new or support existing hypotheses from
  quantitative data;
\item
  it guides a statistical analysis of data and checks its validity.
\end{enumerate}

Some useful visualization methods are:

\begin{enumerate}
\def\labelenumi{\arabic{enumi}.}
\tightlist
\item
  Statistical graphics and infographics
\item
  Geographical information systems (GIS)
\item
  Graph visualization or network maps
\item
  Data cartography
\end{enumerate}

\section{Importance of Ethics in
Visualization}\label{importance-of-ethics-in-visualization}

\citep{ethical_infographics}

Alberto Cairo addresses the ethical `why' of data visualization in this
article, while still grounding the discussion in a straightforward
analysis of what to do and what not to do. He emphasizes that the
effectiveness of the communicative display is as important as the
information itself. This makes intuitive sense because useful
information is rendered utterly useless if no one can understand it.
Cairo sees data visualization as a harmonization of journalism and
engineering. From these two disciplines, he takes the journalist ethos
of truth-telling and honesty and combines this with an engineering focus
on efficacy and efficiency. The result is a data visualization that
contains accurate and relevant information which is clearly and
concisely conveyed. Cairo describes himself as a ``rule utilitarian''
and uses this to explain why it is ethical or, in his words, ``morally
right,'' to create graphics in this way. Here, it very useful to review
his post on the blog introducing the article. Essentially, the goal is
to create the most good while doing the least harm. As such, conveying
truthful and honest relevant information increases a person's
understanding. Increased understanding and knowledge positively
correlates with personal well-being. The information presented must be
accurate and relevant. Cairo briefly addresses guidelines for this which
are applicable in all information gathering fields: beware of selection
bias when choosing preexisting datasets, validate the data, and include
important context. False or irrelevant information doesn't improve
anyone's decision-making capacity, so it cannot enhance well-being. Even
if the information is both accurate and relevant, moral engineering
pitfalls may remain. To avoid the unethical trap of inscrutable (or
misleading) graphics, Cairo exhorts us to take an evidenced-based
approach when possible. The purpose of the graphic dictates the form it
takes; aesthetic preferences should never override clarity. Again, since
the ethical purpose is to improve well-being through understanding, a
graphic which is confusing or misleading is unethical, regardless of
intent, since it actually creates misunderstanding for the audience.
While it can be a bit jarring to think of a poorly designed graphic as
``morally wrong'', it is important to think of the unintended
consequences of visuals which have a powerful impact on their viewers.
Cairo sees data visualization as a harmonization of journalism and
engineering. From these two disciplines, he takes the journalist ethos
of truth-telling and combines this with an engineering focus on efficacy
and efficiency. The result is a data visualization that contains
accurate and relevant information which is clearly and concisely
conveyed. Cairo describes himself as a ``rule utilitarian'' and uses
this to explain why it is ``morally right'' to create graphics in this
way. Here, it is useful to review his blogpost introducing the article.
Essentially, the goal is to create the most good while doing the least
harm. As such, conveying honest and relevant information increases a
person's understanding. Increased understanding and knowledge positively
correlates with personal well-being. The information presented must be
accurate and relevant.

Cairo briefly addresses guidelines that are applicable in all
information gathering fields: beware of selection bias when choosing
preexisting datasets, validate the data, and include important context.
False or irrelevant information doesn't improve anyone's decision-making
capacity, so it cannot enhance well-being. Even if the information is
both accurate and relevant, moral pitfalls may remain. To avoid the
unethical trap of inscrutable or misleading graphics, Cairo exhorts us
to take an evidenced-based approach when possible. The purpose of the
graphic dictates the form it takes; aesthetic preferences should never
override clarity.

Again, since the ethical purpose is to improve well-being through
understanding, a graphic which is confusing or misleading is unethical,
regardless of intent, since it actually creates misunderstanding for the
audience. While it can be a bit jarring to think of a poorly designed
graphic as ``morally wrong,'' it is important to think of the unintended
consequences of visuals which have a powerful impact on their viewers.

The basic objective of data visualization is to provide an efficient
graphical display for summarizing and reasoning about quantitative
information. And during the last decades, political science has
accumulated a large corpus of various kinds of data, which makes it
gradually become a more quantitative scientific field and requires using
quantitative information in the analysis and reasoning.

Data visualization plays several important roles in it: 1) helps create
informative illustrations of the data, recapitulating a large amount of
quantitative information on a diagram; 2) helps formulate new or
supporting existing hypotheses from quantitative data; 3) guides a
statistical analysis of data and checks its validity.

\section{Implications of (Good/Bad) Data
Visualization}\label{implications-of-goodbad-data-visualization}

Raw data is often meaningless or their meaning is not easily understood.
When people face a large set of measurements, they are unable or
unwilling to spend the time required to process it. Technological
advances of the Digital Age contribute to an ever-growing pool of ``big
data'' and our ability to collect this type of information becomes
easier and easier. Thus filtering, visualization, and interpretation of
data become increasingly important.

Raw data is often meaningless or their meaning is not easily concluded.
When people face a large set of measurements they are unable or
unwilling to spend the time required to process it. Our modern living
contributes to an ever-growing pool of ``big data'' and our ability to
collect this type of information becomes easier and easier. Thus
filtering, visualization, and interpretation of data become increasingly
important.

We should understand how best to derive meaning from data, but first we
should understand why its presentation in graphical format is so
powerful.

While the ideal purpose of data visualization is to facilitate
understanding of data, visualization can also be used to mislead. Some
of the main methods of doing so are omitting baselines, axis
manipulation, omitting data, and ignoring graphing convention. Omitting
baselines is used to imply a greater difference between two categories,
such as in poll results comparing political parties. Axis manipulation
by increasing the highest value on the y-axis affects the visibility of
a slope, making data with an otherwise visible trend appear flat.
Omitting selected data points or narrowing the window of a graph is used
to hide an overall trend, such as a graph of a stock only showing a
current trend and hiding previous bubbles. Graphs can also be designed
to subvert convention so that at first glance the graph is conveying the
opposite message, for example, by using the reader's associations of
colors and temperature to create a graph where hot is blue and cold is
red.

\begin{longtable}[]{@{}ll@{}}
\toprule
\begin{minipage}[b]{0.16\columnwidth}\raggedright\strut
\textbf{Principle}\strut
\end{minipage} & \begin{minipage}[b]{0.78\columnwidth}\raggedright\strut
\textbf{Description}\strut
\end{minipage}\tabularnewline
\midrule
\endhead
\begin{minipage}[t]{0.16\columnwidth}\raggedright\strut
\textbf{1. Easy Recall}\strut
\end{minipage} & \begin{minipage}[t]{0.78\columnwidth}\raggedright\strut
People can process images more quickly than words. When data is
transformed into images, the readability and cognition of the content
greatly improve. While people can only remember just 10\% of what they
hear and 20\% of what they read, retention jumps up to 80\% when they
for visual information with interaction.\strut
\end{minipage}\tabularnewline
\begin{minipage}[t]{0.16\columnwidth}\raggedright\strut
\textbf{2. Providing Window for Perspective}\strut
\end{minipage} & \begin{minipage}[t]{0.78\columnwidth}\raggedright\strut
With infographics you can pack a lot of information into a small space.
Colors, shape, movement, the contrast in scale and weight, and even
sound can be used to denote different aspects of the data allowing for
multi-layered understanding \citep{image_good}.\strut
\end{minipage}\tabularnewline
\begin{minipage}[t]{0.16\columnwidth}\raggedright\strut
\textbf{3. Enable Qualitative Analysis}\strut
\end{minipage} & \begin{minipage}[t]{0.78\columnwidth}\raggedright\strut
Color, shape, sounds, and size can make evident relationships within
data very intuitive. When data points are represented as images or
components of an entire scene, readers are able to see the big picture
and understand how the information fits within a larger context.\strut
\end{minipage}\tabularnewline
\begin{minipage}[t]{0.16\columnwidth}\raggedright\strut
\textbf{4. Increase in User Participation}\strut
\end{minipage} & \begin{minipage}[t]{0.78\columnwidth}\raggedright\strut
Interactive infographics can substantially increase the amount of time
someone will spend with the content. Because of their impact,
infographics are widely used nowadays. A quick google will produce a
huge array of great examples --- as well as poor ones. Because while
people recognize the value of information graphic design, and a number
of tools are available today that make the creation of them possible for
the layperson, it doesn't mean that they're all successful or even
necessary.\strut
\end{minipage}\tabularnewline
\bottomrule
\end{longtable}

\section{General Guidelines for Ethical
Visuals}\label{general-guidelines-for-ethical-visuals}

\citep{ethics_code}

Data visualization is an up-and-coming field that currently does not
have many established regulations. This makes it easy to manipulate
readers without technically reporting false information. However,
certain standards should be followed in order to generate meaningful and
accurate visuals. The process can be broken down into three steps, each
with its own set of guiding rules.

\subsection{Data Collection}\label{data-collection}

The first step in any project is gathering the data. This is relatively
simple and does not offer much of an opportunity to introduce confusion.
The one thing to remember is to always get data from a reliable source.
The data provides the foundation for the entire project and must
,therefore, be trustworthy and verifiable.

\subsection{Data Analysis}\label{data-analysis}

This is the stage where the discoveries are made and provided the first
opportunity to manipulate the story for good or bad. There is usually a
lot of data cleaning to do before creating a visual representation, but
all manipulation should make sense. Code should be shared so anyone can
follow the entire process. It is also important to explicitly state any
assumptions taken, though these should be kept to a minimum. Here it is
important to look at what the source data actually shows its ethical
responsibility of presenters for careful analysis of the data and find
true stories from them.

As the amount of data grows, it becomes harder to catch up with it.
Therefore, data strategy becomes the necessary part of the success in
applying data to the business. Then how data visualization become an
important tool in your strategic kit? First, it helps you cleanse your
data. Secondly, it allows you to identify and extract meaningful
information from it. Finally, data visualization tools enable continuous
real-time monitoring of how your strategy and now data-driven decisions
influence performance and business outcomes. In other words, these tools
visualize not only the data, but also the results, and help correct and
optimize strategy on the go.

\subsection{Design}\label{design}

Once a story is found, it must be presented in an honest way. This is
where deceptive techniques could be tempting to make a stronger
argument. An experienced individual will know how to spot these
deceptions and disregard any findings. This ultimately hurts the
credibility of the author and anyone else involved in the publication.

Visualization should not be used to intentionally hide or confuse the
truth, it should not seek to mislead the uninformed. Visualization has
great power, and as they say, with great power comes great
responsibility.

\section{Defintions of Data Deception and Graphic
Integrity}\label{defintions-of-data-deception-and-graphic-integrity}

Data visualization is a powerful communication tool to support arguments
with numbers in a way that is accessible and engaging. It is becoming
more and more popular to communicate and support arguments nowadays.
More people than ever before are making their own charts and
infographics, which is presenting a unique problem. Despite the
availability of some great charting resources and resources online to
create and design amazing data products, we are witnessing an influx of
poorly-designed misleading or downright deceptive data visualizations
\citep{decept_study},\citep{rose_tint}.

So what does \textbf{data deception} mean? Data deception, defined by
School of Law at the New York University, is ``a graphical depiction of
information, designed with or without an intent to deceive, that may
create a belief about the message and/or its components, which varies
from the actual message.'' Deceptive, misleading, or distorted graphs
are those that intentionally or unintentionally skew the data, and
result in a representation of incorrect conclusions.

Edward Tufte already introduced the concept of graphical integrity in
his book and presented six principles of graphic integrity. Here are the
principles from the book:

\begin{itemize}
\tightlist
\item
  The representation of numbers, as physically measured on the surface
  of the graphic itself, should be directly proportional to the
  numerical quantities measured.
\item
  Clear, detailed, and thorough labeling should be used to defeat
  graphical distortion and ambiguity. Write out explanations of the data
  on the graphic itself. Label important events in the data.
\item
  In time-series displays of money, deflated and standardized units of
  monetary measurement are nearly always better than nominal units.
\item
  The number of information-carrying (variable) dimensions depicted
  should not exceed the number of dimensions in the data.
\item
  Show data variation, not design variation.
\item
  Graphics must not quote data out of context.
\end{itemize}

There are some ways in which distorted graphs can be created
\citep{evil_axes},\citep{mislead_graph_ex}:

\begin{longtable}[]{@{}ll@{}}
\toprule
\begin{minipage}[b]{0.16\columnwidth}\raggedright\strut
\textbf{Tool}\strut
\end{minipage} & \begin{minipage}[b]{0.78\columnwidth}\raggedright\strut
\textbf{Description}\strut
\end{minipage}\tabularnewline
\midrule
\endhead
\begin{minipage}[t]{0.16\columnwidth}\raggedright\strut
\textbf{Improper scaling of y-axis}\strut
\end{minipage} & \begin{minipage}[t]{0.78\columnwidth}\raggedright\strut
This is one of the classic misleading graphs. Instead of scale starting
from zero or a baseline, y-axis is scaled conveniently to highlight the
differences among bins.\strut
\end{minipage}\tabularnewline
\begin{minipage}[t]{0.16\columnwidth}\raggedright\strut
\textbf{Improper labeling of graphs}\strut
\end{minipage} & \begin{minipage}[t]{0.78\columnwidth}\raggedright\strut
Lack of labels make the graph hard to interpret for the reader and lead
to wrong conclusions.\strut
\end{minipage}\tabularnewline
\begin{minipage}[t]{0.16\columnwidth}\raggedright\strut
\textbf{Paired graphs on different scale}\strut
\end{minipage} & \begin{minipage}[t]{0.78\columnwidth}\raggedright\strut
It is not a fair comparison if two elements are plotted side-by-side, on
a different scale and compared. This makes one graph look better than
the other, even when it is not.\strut
\end{minipage}\tabularnewline
\begin{minipage}[t]{0.16\columnwidth}\raggedright\strut
\textbf{Dual axis with different scales}\strut
\end{minipage} & \begin{minipage}[t]{0.78\columnwidth}\raggedright\strut
If we are plotting two elements on the same graph with different scales,
even if the axes are properly labeled, it is assumed that both axes are
on the same scale.\strut
\end{minipage}\tabularnewline
\begin{minipage}[t]{0.16\columnwidth}\raggedright\strut
\textbf{Incomplete data}\strut
\end{minipage} & \begin{minipage}[t]{0.78\columnwidth}\raggedright\strut
Short-term graphs are made to manipulate the trend, which will not be
seen otherwise. Time-series data are cut intentionally to show a trend
within a particular period to create a more favorable visual
impression.\strut
\end{minipage}\tabularnewline
\bottomrule
\end{longtable}

\section{Visual Lies}\label{visual-lies}

\citep{visual-lies} focuses on a few methods that data visualizers
utilize to mislead users about research findings. For each method, the
author has highlighted the signifiers that are manipulated to promote an
unrealistic understanding of the visualized data. The author has
concentrated on examples of three areas to create deceptive data
visualization: size, segmentation, and graph type.

\subsection{Size}\label{size}

Size signifies quantity, volume or degree of variables within a data. In
first figure, the y-axis from the graph to the right is cut when
transcribed onto the graph on the left. Here both the graphs show the
same data but the one on the left represents the data in a misleading
fashion because of the way the axis is cut, and the result is that
interest rates have increased drastically from 2008 to 2012 -- a
misinterpretation that is avoided in the graph on the right.

\begin{figure}
\centering
\includegraphics{images/Size1.png}
\caption{}
\end{figure}

\subsection{Quantity}\label{quantity}

Quantity is measures size. When depicting points on a scatter plot, the
author suggests that it is helpful to manipulate the size the points to
represent differing values of a variable that is not represented on the
x and y axes. The following graph shows quantity as a two completely
different measures. One chart uses quantity as area and other uses it as
radius. The result is that the differences in quantity between points on
such a scatter plot would appear more dramatic than they should be.

\begin{figure}
\centering
\includegraphics{images/Quantity1.png}
\caption{}
\end{figure}

\subsection{Segmentation}\label{segmentation}

Figure shows an example of segmentation with a deceptive instance of
binning given in the legend on the left. Segmentation can be used to
show category, parts, domains or ranges within a chart. The author
states that correct use of segmentation can be a powerful tool to
enhance understanding, but can be deceptive if used incorrectly. This
example shows how binning can be misleading; in the left figure, binning
is not done appropriately, and it is therefore difficult to come up with
actual values of the data.

Figure 3: \includegraphics{images/Segmentation 1.png}

\subsection{Graph}\label{graph}

Two graphs that are often misrepresented are pie-charts and maps. In the
following figure, the author explains that pie-charts cannot be compared
accurately to one another. When striving for an accurate portrayal of
values, they should be avoided. The author further states that it would
be difficult to understand the pie-charts had the numbers not been
given.

Alberto Cairo addresses into the ethical `why' of data visualization in
this article, while still grounding the discussion in a straightforward
analysis of what to do and what not to do. He emphasizes that the
effectiveness of the communicative display is as important as the
information itself. This makes intuitive sense because useful
information is rendered utterly useless if no one can understand it.

\begin{figure}
\centering
\includegraphics{images/PieCharts.png}
\caption{}
\end{figure}

The author also asserts that when showing spatial data analysis, always
show population density when visualizing values that are
person-dependent. On a heat map where color signifies quantity, The
author suggests that a user will be drawn to the colors that a legend
indicates are most extreme.

In following figure, areas that are darkest are simply the most
population-dense regions of the United States. Without accounting for
population density, the newly created map may look the same as hundreds
of maps bearing a striking resemblance to the figure, which are falsely
considered informative and are regularly shared across social media
sites.

\begin{figure}
\centering
\includegraphics{images/Maps1.png}
\caption{}
\end{figure}

The above pointers are helpful when analyzing a deceptive version of a
data product. However, data visualizers need to carefully draw the line
between creating misleading graphs that tell a different story and
developing deceptive versions that intend to exaggerate. This should be
applied in our projects and can also be used to enhance our
understanding of data visualization products.

Misleading graphs are sometimes deliberately misleading and sometimes
it's just a case of people not understanding the data behind the graph
they create \citep{andale_2014}. But some real life misleading graphs go
above and beyond the classic types. Some are intended to mislead, others
are intended to shock. The ``classic'' types of misleading graphs
include cases where:

\subsection{The Missing Baseline}\label{the-missing-baseline}

For example, the vertical scale is too big or too small, skips numbers,
or does not start at zero. For example, in the graph below, you might be
thinking that the graph on the right shows that The Times makes double
the sales of The Daily Telegraph. However, a closer look at the scale
reveals that although The Times does make more sales, it is only beating
the competition by about 10\%.

\subsection{The graph is not labeled
properly}\label{the-graph-is-not-labeled-properly}

A graph may have the correct figures but still mislead its audience.
This one used a BIG HEADLINE that suggests to its audience that 5.3\% of
children get spinal cord injuries, which is a pretty scary statistic for
parents. But the real figure is about .0000003\% (based on 2000 injuries
per year out of a population of around 74,000,000). And for the figure 1
used in this article, ``Misleading Graphs: Displaying a Change in One
Variable Using Area or Volume'' \citep{scaling_issues}, the label for
the smaller triangle in this graph says \$26.4 while the label for the
larger triangle says \$114.6. \$114.6 is 4.34 times \$26.4. It certainly
looks to me as if more than 4.34 smaller triangles will fit in the
larger triangle. It is the altitudes of the triangles that are
proportional to the numbers in the labels.

\subsection{Data is left out}\label{data-is-left-out}

Only including part of the data is also an easy opportunity to mislead.
The following graph only uses temperatures of the first half of the year
to prove it was rising dramatically. For more examples and inspirations
on misleading or deceptive graphs refer the following articles:

\begin{itemize}
\tightlist
\item
  Bar charts without zero \& evenly spaced tick marks for uneven
  intervals: \citep{whats_wrong}
\item
  Graphs not drawn to scale:\citep{scaling_issues}
\end{itemize}

\subsection{Treating correlation as
causation}\label{treating-correlation-as-causation}

Even if the labels and data in your graph are correct, the conclusion is
not necessarily logically correct. A correlation between X and Y does
not automatically indicate that the change in one variable is caused by
the change in the values of the other one, i.e.~correlation does not
imply causation. Viewers should bear in mind that such visuals only
present the correlation between ice cream sold and murders, not than
causation.

\begin{figure}
\includegraphics[width=0.7\linewidth]{images/harlin-ice-cream} \caption{A strange correlation between ice cream sales and murders (Source: [@harlin-coorelation])}\label{fig:harlin-ice-cream}
\end{figure}

Another trick for creating misleading graphs is an axis change: Changing
the y-axis maximum affects how the information in the graph is
perceived. A higher maximum will make the graph to appear less volatile
or steep than a lower maximum. The axis can also be altered to deceive
by changing the ratio of a graph's dimensions, as demonstrated in the
below graphs.

\begin{figure}
\centering
\includegraphics{images/Line_graph1.svg.png}
\caption{}
\end{figure}

\begin{figure}
\centering
\includegraphics{images/175px-Line_graph1-3.svg.png}
\caption{}
\end{figure}

\begin{figure}
\centering
\includegraphics{images/200px-Line_graph1-4.svg.png}
\caption{}
\end{figure}

While not technically wrong, improper extraction or tactic omitting
data, when only a certain chunk of data is included, is certainly
misleading. This is more common in graphs that have time as one of their
axes.

\begin{figure}
\centering
\includegraphics{images/Bad_graph_extraction.png}
\caption{}
\end{figure}

Visualizations should be simple and easy to remember ,but at the same
time it should contain the essence of responsible visualization. The
make final results pure, ethical procedures need to be practiced
throughout all the steps of visualization.
\includegraphics{images/Good_graph_extraction.png}

In the data visualization terms, we call it truncated graph. A truncated
graph (also known as a torn graph) has a y-axis that does not start at
0. These graphs can create the impression of important change where
there is relatively little change.Truncated graphs are useful in
illustrating small differences.{[}16{]} Graphs may also be truncated to
save space. Commercial software such as MS Excel will tend to truncate
graphs by default if the values are all within a narrow range.
Truncating graphs make the readers to change their judgement for
something that is not significant looks like a huge difference.

A example of using good data in a misleading graph to fool readers comes
from Fox News. \includegraphics{images/1.png} \citep{DataMiningVsViz}

\chapter{Conclusion}\label{conclusion}

Reflection, Key Learnings, Outlook

\chapter*{References}\label{references-1}
\addcontentsline{toc}{chapter}{References}

\textbf{2.1 The History of Data Visualization} \textbf{Author: Dashboard
Insight, Dashboard Insight, 2013} URL:
\url{http://www.dashboardinsight.com/news/news-articles/the-history-of-data-visualization.aspx}

\textbf{2.2 Current research: Deceptive visualizations} \textbf{Author:
Infogram, 2016}
URL:\url{https://medium.com/@Infogram/study-asks-how-deceptive-are-deceptive-visualizations-8ff52fd81239}

\textbf{Author: Agata Kwapien in Data Visualization, 2015} URL:
\url{https://www.datapine.com/blog/misleading-data-visualization-examples/}

\textbf{2.3 A Brief History of Data Visualization,York University.}
\textbf{Auhtor: Michael Friendly, 2006}
URL:\url{http://www.datavis.ca/papers/hbook.pdf} \emph{Summary: } This
paper provides an overview of the intellectual history of data
visualization from medieval to modern times,describing and illustrating
some significant advances along the way.

\textbf{2.4 Data Visualization and the 9 Fundamental Design Principles}
\textbf{Auhthor: Melissa Anderson, 2017}
URL:\url{https://www.idashboards.com/blog/2017/07/26/data-visualization-and-the-9-fundamental-design-principles/}

\textbf{2.5 A Practitioner Guide to Best Practices in Data
Visualization.Interfaces 47(6):473-488.} \textbf{Auhtor: Jeffrey D.
Camm, Michael J. Fry, Jeffrey Shaffer, 2017 } URL:
\url{https://doi.org/10.1287/inte.2017.0916}

\textbf{2.6 The 7 Best Data Visualization Tools In 2017} \textbf{Author:
Bernard Marr, 2017} URL:
\url{https://www.forbes.com/sites/bernardmarr/2017/07/20/the-7-best-data-visualization-tools-in-2017/\#3a12b8ea6c30}

\textbf{2.7 The Data Visualisation Catalogue} URL:
\url{https://datavizcatalogue.com}

\textbf{2.8 The Extreme Presentation(tm) Method } \textbf{Aurthor:
Dr.~Abela, 2015 } URL:
\url{http://extremepresentation.typepad.com/blog/2015/01/announcing-the-slide-chooser.html}

\textbf{2.9 Data Visualization: How to Pick the Right Chart Type? }
\textbf{Author: J��nis Gulbis, 2016 } URL:
\url{https://eazybi.com/blog/data_visualization_and_chart_types/}

\textbf{2.10 Data Visualization Best Practices} \textbf{Author:
melindasantos, 2017} URL:
\url{http://paristech.com/blog/data-visualization-best-practices/}
\url{http://paristech.com/blog/data-visualization-best-practices/}
\url{http://extremepresentation.typepad.com/blog/2015/01/announcing-the-slide-chooser.html}

\textbf{2.11 3 simple rules for intuitive dashboard design}
\textbf{Author: Happy Dashboarding, 2017 } URL:
\url{https://www.klipfolio.com/blog/intuitive-dashboard-design}

\textbf{2.12 How deceptive are deceptive visualizations?}
\textbf{Author: Pandey, A. V., Rall, K., Satterthwaite, M. L., Nov, O.,
\& Bertini, E. ,2015 }

\textbf{2.13 An empirical analysis of common distortion techniques}
\textbf{Author: Anshul Vikram Pandey, 2015 }

\textbf{2.14 Factors in Computing Systems: Crossings (Vol. 2015-April,
pp.~1469-1478). Association for Computing Machinery. DOI:
10.1145/2702123.2702608 (2) Tufte, E. R., and Graves-Morris, P. The
visual display of quantitative information, vol.~2. Graphics press
Cheshire, CT,1983.}

\textbf{2.15 Axes of evil: How to lie with graphs} \textbf{Author:
ANDREA ROBERTSON} URL: \url{http://hypsypops.com/axes-evil-lie-graphs/}

\textbf{2.16 Misleading Graphs: Real Life Examples} \textbf{Author:
Stephanie, February 28th, 2016} URL:
\url{http://www.statisticshowto.com/misleading-graphs/}

\textbf{2.17 Next Steps for Data Visualization Research} \textbf{Author:
UW Interactive Data Lab, 2015} URL:
\url{https://medium.com/@uwdata/next-steps-for-data-visualization-research-3ef5e1a5e349}

\textbf{2.18 Using Typography to Expand the Design Space of Data
Visualization. She Ji: The Journal of Design, Economics and Innovation,
2(1), pp 59-87.}

\textbf{2.19 Using Typography to Expand the Design Space of Data
Visualization} \textbf{Banissi, Ebad, \& Brath, Richard. (2016). } URL:
\url{https://www.sciencedirect.com/science/article/pii/S2405872616300107}.

\textbf{2.20 Using Data Visualization to Find Insights in Data} URL:
\url{http://datajournalismhandbook.org/1.0/en/understanding_data_7.html}

\textbf{2.21 Building advanced analytics application with TabPy} URL:
\url{https://www.tableau.com/about/blog/2017/1/building-advanced-analytics-applications-tabpy-64916}

\textbf{2.22 Some best practices for visualization:} URL:
\url{http://www.dataplusscience.com/files/visual-analysis-guidebook.pdf}

** 2.23 Avoiding Common Mistakes with Time Series\textbf{ }Author: TOM
FAWCETT,2015** URL:
\url{https://www.svds.com/avoiding-common-mistakes-with-time-series/}

\textbf{4.1 The Baseline and Working with Time Series in R}
\textbf{Author: Nathan Yau, 2013} URL:
\url{https://flowingdata.com/2013/11/26/the-baseline/}

\textbf{4.2 Using design patterns to find greater meaning in your data}
\textbf{Author: Julie RodriguezPiotr Kaczmarek May 11, 2016} URL:
\url{https://www.oreilly.com/ideas/using-design-patterns-to-find-greater-meaning-in-your-data}

\textbf{4.3 Design Iron Fist} \textbf{Author: Jarrod Drysdale} URL:
\url{https://studiofellow.com/newsletter/}

\textbf{4.4 The Creative Aid Handbook} URL:
\url{https://issuu.com/koorookooroo/docs/kooroo_kooroo_creative_aid}

\section{More ways to improve your visualization
design}\label{more-ways-to-improve-your-visualization-design}

From online surveys to beefed-up analytics, we're able to gather and
analyze more data than ever before. But how do you turn your findings
from a dense spreadsheet into something that really makes your point?
Good information design is the key.

There's a wealth of free resources out there in the form of handy little
design ebooks.

\begin{itemize}
\tightlist
\item
  \textbf{Design's Iron Fist} --- Jarrod Drysdale
\end{itemize}

The free ebook, Design's Iron Fist, is a collection of Drysdale's
previous work all wrapped up in one neat little package. Aside from
practical tutorials and processes, this book also offers help on how to
get into the mindset of being a truly great designer.

\begin{itemize}
\tightlist
\item
  \textbf{The Creative Aid Handbook} --- Kooroo Kooroo
\end{itemize}

Creativity doesn't just happen overnight. It's something that each and
every designer has to work at on a day-to-day basis. If you find that
your innovative juices are running dry, The Creative Aid Handbook could
be the answer. The helpful guide looks at how you can boost your
intellect, foster your well-being, and, most importantly, become more
creative.

\begin{itemize}
\tightlist
\item
  \textbf{Designbetter.co} --- InVision
\end{itemize}

InVision released three fantastic design books that are available for
free. Each book discusses various aspects of design like design process,
management, and business. Moreover, some of the materials are available
in audio format.

\textbf{5.1 Importance of Ethics in Visualization: Data visualization in
political and social sciences} \textbf{Author: Andrei Zinovyev, year:
N/A} URL:
\url{https://github.com/mschermann/data_viz_reader/files/1933699/Zinovyev_Data_Visualization.pdf}
* \textbf{Type Classification}

Type Classification is a helpful beginner's guide to typography. It
should give you the foundations you need to not only start classifying
various forms of type but also understanding when and how to use them to
alarmingly great effect. It covers a history of each of the type forms
and the basic facts you need know about them.\citep{design_ebooks}

\textbf{5.2 Data Visualization in Political and Social Sciences: Data
visualization in political and social sciences} \textbf{Author: Andrei
Zinovyev, year: N/A} URL:
\url{https://github.com/mschermann/data_viz_reader/files/1933699/Zinovyev_Data_Visualization.pdf}

\textbf{5.3 Data Visualization in Business: How Data Visualization
Impacts Your Business Strategy,} \textbf{Author: Katherine Lazarevich,
2018} URL:
\url{https://www.iotforall.com/data-visualization-strategy-for-business/}

\textbf{5.4 Implications of (Good/Bad) Data Visualization: How Writers
Use Misleading Graphs to Manipulate You} \textbf{Author: Ryan McCready,
2017} URL: \url{https://venngage.com/blog/misleading-graphs/}

\textbf{5.5 General Guidelines to Ethical Visuals: A Code of Ethics for
Data Visualization Professionals} \textbf{Author: Drew Skau, 2012} URL:
\url{https://visual.ly/blog/a-code-of-ethics-for-data-visualization-professionals/}

\textbf{1. Expert Data Visualization Tips for Grabbing Readers'
Attention} \textbf{Author: Payman Taei,2017} URL:
\url{https://towardsdatascience.com/3-expert-data-visualization-tips-for-grabbing-readers-attention-206d8c4621bf}

\section{Useful Links on Data Visualization Trends, Tutorials and
Research
Papers}\label{useful-links-on-data-visualization-trends-tutorials-and-research-papers}

\begin{itemize}
\tightlist
\item
  \citep{charts_viz} - You can find different types of plots used in
  data visualization at
  \href{https://datavizcatalogue.com/search.html}{Data Catalogue}.
\item
  \citep{eagereyes_viz} - Robert Kosara's website which contains recent
  developments happening in visualization and are likely to have an
  impact.
\item
  \citep{research_viz} - About Robert Kosara and his research papers.
\item
  \citep{twitter_Kosara} - Robert Kosara's twitter handle.
\item
  \citep{flowingdata} - Website which offers courses, tutorials and
  happenings in viz.
\item
  \citep{infogram} - An infogram helps a user making different types of
  plots and learning the art of visualization. Engaging infographics,
  reports, charts, dashboards and maps can be easily created in minutes
  with it.
\end{itemize}

\section{Resources for Aspiring Data
Visualists}\label{resources-for-aspiring-data-visualists}

\subsection{Tableau Community}\label{tableau-community}

The following groups or communities help you to explore Tableau further
\citep{Tableau_Community}: * It will help us enhance our learning * Get
answers for most of your doubts In tableau * Post new questions and
crowd source answers * Attend events, seminars and join conferences
conducted locally/ globally * Give back to the community once you become
an expert in that field

There are very active Tableau Social Media Groups
\citep{LinkedIn_Groups}:

\begin{itemize}
\tightlist
\item
  Tableau Enthusiasts: LinkedIn Group (19K members)
\item
  Tableau Software Fans \& Friends: LinkedIn Group (45k members)
\end{itemize}

\subsection{Blogs}\label{blogs}

Here is a list of the top 10 blogs that Tableau itself suggests
following \citep{Top_10_Blogs}:

\begin{enumerate}
\def\labelenumi{\arabic{enumi}.}
\tightlist
\item
  \href{http://www.storytellingwithdata.com/}{Storytelling with Data}
\item
  \href{https://informationisbeautiful.net/}{Information is Beautiful}
\item
  \href{https://flowingdata.com/}{Flowing Data}
\item
  \href{http://www.visualisingdata.com/}{Visualizing Data}
\item
  \href{http://junkcharts.typepad.com/}{Junk Charts}
\item
  \href{https://pudding.cool/}{The Pudding}
\item
  \href{https://www.theatlas.com/}{The Atlas}
\item
  \href{https://www.economist.com/blogs/graphicdetail}{Graphic Detail}
\item
  \href{https://www.census.gov/dataviz/}{US
  Census};\href{https://www.fema.gov/data-visualization}{FEMA}
\item
  \href{https://www.tableau.com/about/blog}{Tableau Blog}
\end{enumerate}

\textbf{2. Choose best colors for cartography visualization in a
professional manner} \textbf{Author: Cynthia Brewer, Mark Harrower and
The Pennsylvania State University} URL: \url{http://colorbrewer.org}

\bibliography{book.bib,packages.bib}


\end{document}
